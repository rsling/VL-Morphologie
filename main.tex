\def\GRAPHPATH{localgraphics}

\ifdefined\HANDOUT
  \documentclass[handout,aspectratio=1610,dvipsnames]{beamer}
  \def\GRAPHPATH{graphics}
\else
  \documentclass[aspectratio=1610,dvipsnames]{beamer}
\fi

\usepackage[margin=2cm]{geometry}
\usepackage[ngerman]{babel}
\usepackage{setspace}
\usepackage{booktabs}
\usepackage{array,graphics}
\usepackage{color}
\usepackage{hyperref}
\usepackage{soul}
\usepackage[linecolor=gray,backgroundcolor=yellow!50,textsize=tiny]{todonotes}
\usepackage[linguistics]{forest}
\usepackage{multirow}
\usepackage{pifont}
\usepackage{wasysym}
\usepackage{langsci-gb4e}
\usepackage{soul}
\usepackage{enumitem}
\usepackage{marginnote}
\usepackage{fontspec}
\usepackage[headings]{fancyhdr}
\usepackage{tikz}
\usepackage{ulem}
\usepackage{longtable}
\usepackage{booktabs}
\usepackage{movement-arrows}

\author{Prof.\ Dr.\ Roland Schäfer | Germanistische Linguistik FSU Jena}
\title{Morphologie | \TITLE%
  \ifdefined\SOLUTIONS \gruen{\\Musterlösung} \fi
}
\date{Version 2024}

\definecolor{rot}{rgb}{0.7,0.2,0.0}
\newcommand{\rot}[1]{\textcolor{rot}{#1}}
\definecolor{blau}{rgb}{0.1,0.2,0.7}
\newcommand{\blau}[1]{\textcolor{blau}{#1}}
\definecolor{gruen}{rgb}{0.0,0.7,0.2}
\newcommand{\gruen}[1]{\textcolor{gruen}{#1}}
\definecolor{grau}{rgb}{0.6,0.6,0.6}
\newcommand{\grau}[1]{\textcolor{grau}{#1}}
\definecolor{orongsch}{RGB}{255,165,0}
\newcommand{\orongsch}[1]{\textcolor{orongsch}{#1}}
\definecolor{tuerkis}{RGB}{63,136,143}
\definecolor{braun}{RGB}{108,71,65}
\newcommand{\tuerkis}[1]{\textcolor{tuerkis}{#1}}
\newcommand{\braun}[1]{\textcolor{braun}{#1}}

\newcommand*\Rot{\rotatebox{75}}
\newcommand*\RotRec{\rotatebox{90}}

\newcommand{\zB}{z.\,B.\ }
\newcommand{\ZB}{Z.\,B.\ }
\newcommand{\Sub}[1]{\ensuremath{_{\text{#1}}}}
\newcommand{\Up}[1]{\ensuremath{^{\text{#1}}}}
\newcommand{\UpSub}[2]{\ensuremath{^{\text{#1}}_{\text{#2}}}}
\newcommand{\Doppelzeile}{\vspace{2\baselineskip}}
\newcommand{\Zeile}{\vspace{\baselineskip}}
\newcommand{\Halbzeile}{\vspace{0.5\baselineskip}}
\newcommand{\Viertelzeile}{\vspace{0.25\baselineskip}}

\newcommand{\whyte}[1]{\textcolor{white}{#1}}

\newcommand{\Spur}[1]{t\Sub{#1}}
\newcommand{\Ti}{\Spur{1}}
\newcommand{\Tii}{\Spur{2}}
\newcommand{\Tiii}{\Spur{3}}
\newcommand{\Tiv}{\Spur{4}}
\newcommand*{\mybox}[1]{\framebox{#1}}
\newcommand\ol[1]{{\setul{-0.9em}{}\ul{#1}}}

\newcommand{\Lf}{
  \setlength{\itemsep}{1pt}
  \setlength{\parskip}{0pt}
  \setlength{\parsep}{0pt}
}

\forestset{
  Ephr/.style={draw, ellipse, thick, inner sep=2pt},
  Eobl/.style={draw, rounded corners, inner sep=5pt},
  Eopt/.style={draw, rounded corners, densely dashed, inner sep=5pt},
  Erec/.style={draw, rounded corners, double, inner sep=5pt},
  Eoptrec/.style={draw, rounded corners, densely dashed, double, inner sep=5pt},
  Ehd/.style={rounded corners, fill=gray, inner sep=5pt,
    delay={content=\whyte{##1}}
  },
  Emult/.style={for children={no edge}, for tree={l sep=0pt}},
  phrasenschema/.style={for tree={l sep=2em, s sep=2em}},
  sake/.style={tier=preterminal},
  ake/.style={
    tier=preterminal
    },
}

\forestset{
  decide/.style={draw, chamfered rectangle, inner sep=2pt},
  finall/.style={rounded corners, fill=gray, text=white},
  intrme/.style={draw, rounded corners},
  yes/.style={edge label={node[near end, above, sloped, font=\scriptsize]{Ja}}},
  no/.style={edge label={node[near end, above, sloped, font=\scriptsize]{Nein}}}
}

\usetikzlibrary{arrows,positioning} 

\defaultfontfeatures{Ligatures=TeX,Numbers=OldStyle, Scale=MatchLowercase}
\setmainfont{Linux Libertine O}
\setsansfont{Linux Biolinum O}

\setlength{\parindent}{0pt}

\fancyhead[E,O]{}
\fancyfoot[E,O]{}
\renewcommand{\headrulewidth}{0pt}
\pagestyle{fancy}
\setlength{\headsep}{50pt}
\setlength{\textheight}{\textheight-25pt}

\newenvironment{nohyphens}{%
  \par
  \hyphenpenalty=10000
  \exhyphenpenalty=10000
  \sloppy
}{\par}

\newenvironment{spread}
{%
  \newdimen\origiwspc%
  \newdimen\origiwstr%
  \origiwspc=\fontdimen2\font%
  \origiwstr=\fontdimen3\font%
  \fontdimen2\font=1em%
  \doublespacing%
}{%
  \fontdimen2\font=\origiwspc%
  \fontdimen3\font=\origiwstr%
}

\newcommand{\Sol}[1]{
  \ifdefined\SOLUTIONS
    \gruen{#1}
  \fi
}

\newcommand{\Solalt}[2]{
  \ifdefined\SOLUTIONS
    \gruen{#1}
  \else
    #2
  \fi
}





\ifdefined\TITLE
  \title[Morphologie | \StrSubstitute{\TITLE}{+}{ }]{Einführung in die Morphologie und Lexikologie\\\StrSubstitute{\TITLE}{+}{ }}
\else
  \title[Morphologie]{Einführung in die Morphologie und Lexikologie}
\fi

\author{Roland Schäfer}
\institute[FSU Jena]{Institut für Germanistische Sprachwissenschaft\\Friedrich-Schiller-Universität Jena}
\date[EGBD3]{\scriptsize \grau{stets aktuelle Fassungen: \url{https://github.com/rsling/VL-Morphologie}}}

\begin{document}

\begingroup
  \setbeamertemplate{navigation symbols}{}
  
  \begin{frame}[noframenumbering,plain]
   \titlepage
  \end{frame}

  \ifdefined\TITLE
    \begin{frame}[noframenumbering,plain]
      \centering 
      \begin{minipage}[c]{0.975\textwidth}
      \begin{block}
        {\rot{Hinweise für diejenigen, die die Klausur bestehen möchten}}
        \begin{enumerate}
          \item Folien sind niemals selbsterklärend und nicht zum Selbststudium geeignet.\\
            Sie müssen sich die Videos ansehen und regelmäßig das Seminar besuchen.
          \item Ohne eine gründliche Lektüre der angegebenen Abschnitte des Buchs\\
            bestehen Sie die Klausur nicht.
            Das Buch definiert den Klausurstoff.
          \item Arbeiten Sie die entsprechenden Übungen im Buch durch.
            Nichts hilft\\
            Ihnen besser, um sich auf die Klausur vorzubereiten.
          \item \rot{Beginnen Sie spätestens jetzt mit dem Lernen.}
            \Zeile
          \item \rot{Langjähriger Erfahrungswert:
            Wenn Sie diese Hinweise nicht berücksichtigen, bestehen Sie die Klausur wahrscheinlich nicht.}
        \end{enumerate}
      \end{block}
      \end{minipage}
    \end{frame}
  \else
  \begin{frame}{Inhalt}
    \centering 
    \scalebox{0.9}{\begin{minipage}{\textwidth}
      \begin{multicols}{2}
        \tableofcontents
      \end{multicols}
    \end{minipage}}
    \end{frame}
  \fi
\endgroup

\ifdefined\TITLE
  \input{includes/\TITLE}
\else
  \section{Grammatik und Grammatik im Lehramt}
  \let\woopsi\section\let\section\subsection\let\subsection\subsubsection
  \section{Überblick}

\begin{frame}
  {Grammatik und Grammatikunterricht}
  \onslide<+->
  \begin{itemize}[<+->]
    \item \blau{Grammatik}
      \begin{itemize}
        \item Grammatik und Grammatikalität
        \item Häufigkeiten und typische Muster
        \item Sprachrichtigkeit
      \end{itemize}
      \Zeile
    \item \blau{Grammatik und Lehramtsstudium}
      \begin{itemize}
        \item Wozu Deutschunterricht?
        \item Bildungssprache und Sprachbetrachtung
        \item Aufgaben von Lehrpersonal in Deutsch
      \end{itemize}
  \end{itemize}
\end{frame}


\section{Grammatik}

\begin{frame}
  {Deutsche Sätze erkennen und interpretieren}
  \onslide<+->
  \onslide<+->
  \begin{exe}
    \ex[ ]{Dies ist ein Satz.}
    \onslide<+->
    \ex[*]{Satz dies ein ist.}
    \onslide<+->
    \ex[*]{Kno kna knu.}
    \onslide<+->
    \ex[*]{This is a sentence.}
    \onslide<+->
    \Zeile
    \ex[*]{Dies ist ein Satz}
  \end{exe}
\end{frame}


\begin{frame}
  {Form und Bedeutung: Kompositionalität}
  \onslide<+->
  \onslide<+->
  \begin{exe}
    \ex Das ist ein Kneck.
    \onslide<+->
    \Zeile
  \ex Jede Farbe ist ein Kurzwellenradio.
  \ex Der dichte Tank leckt.
\end{exe}
    \Zeile
  \onslide<+->

  \Large\begin{block}{Kompositionalität (Kompositionalitäts- oder Frege-Prinzip}
    Die Bedeutung komplexer sprachlicher Ausdrücke ergibt sich\\
    aus der Bedeutung ihrer Teile und der Art ihrer Kombination.\\

    Die Eigenschaft von Sprache, die dieses Prinzip beschreibt,\\
    nennt man Kompositionalität.
  \end{block}
\end{frame}

\begin{frame}
  {Grammatik als System und Grammatikalität}
  \onslide<+->
  \onslide<+->
  \Large\begin{block}{Grammatik}
    Eine Grammatik ist ein \alert{System von Regularitäten}, nach denen\\
    aus einfachen Einheiten komplexe Einheiten einer Sprache\\
    gebildet werden.
  \end{block}
  \Zeile

  \onslide<+->

  \begin{block}{Grammatikalität}
    Jede von einer bestimmten Grammatik beschriebene Symbolfolge\\
    ist \alert{grammatisch} relativ zu dieser Grammatik, alle anderen\\
    sind \alert{ungrammatisch}.
  \end{block}
\end{frame}

\begin{frame}
  {(Un)grammatisch ist nicht gleich (in)akzeptabel}
  \onslide<+->
  \onslide<+->
  \begin{exe}
    \ex\begin{xlist}
      \ex Bäume wachsen werden hier so schnell nicht wieder.
      \onslide<+->
      \ex Touristen übernachten sollen dort schon im nächsten Sommer.
      \onslide<+->
      \ex Schweine sterben müssen hier nicht.
      \onslide<+->
      \ex Der letzte Zug vorbeigekommen ist hier 1957.
      \onslide<+->
      \ex Das Telefon geklingelt hat hier schon lange nicht mehr.
      \onslide<+->
      \ex Häuser gestanden haben hier schon immer.
      \onslide<+->
      \ex Ein Abstiegskandidat gewinnen konnte hier noch kein einziges Mal.
      \onslide<+->
      \ex Ein Außenseiter gewonnen hat hier erst letzte Woche.
      \onslide<+->
      \ex Die Heimmannschaft zu gewinnen scheint dort fast jedes Mal.
      \onslide<+->
      \ex Ein Außenseiter gewonnen zu haben scheint hier noch nie.
      \onslide<+->
      \ex Ein Außenseiter zu gewinnen versucht hat dort schon oft.
      \onslide<+->
      \ex Einige Außenseiter gewonnen haben dort schon im Laufe der Jahre.
    \end{xlist}
  \end{exe}
\end{frame}

\begin{frame}
  {Grammatikalität und Inakzeptabilität}
  \onslide<+->
  \onslide<+->
  \alert{Grammatikalität}\\
  \Halbzeile
  \begin{itemize}[<+->]
    \item grammatisch | Strukturen, die von einer Grammatik beschrieben werden
    \item ungrammatische Strukturen markiert mit Asterisk *
  \end{itemize}
  \Zeile
  \alert{Akzeptabilität}
  \Halbzeile
  \begin{itemize}[<+->]
    \item akzeptabel | Strukturen, die Menschen als ihre Sprache akzeptieren
    \item mögliche Gründe für Unterschiede zwischen Grammatikalität und Akzeptabilität
      \begin{itemize}[<+->]
          \Halbzeile
        \item kognitive Grammatik | nicht unbedingt eindeutig kodiert (probabilistisch)
        \item Performanz | Störeinflüsse \slash\ eingeschränkte kognitive Verarbeitungsfähigkeit
        \item Individualgrammatik | unterschiedliche Grammatiken auf Basis individuellen Inputs
      \end{itemize}
  \end{itemize}
\end{frame}

\begin{frame}
  {Kern und Peripherie}
  \onslide<+->
  \onslide<+->
  Manche grammatischen Strukturen sind \alert{typischer} als \rot{andere}.\\
  \Zeile
  \onslide<+->
  \begin{exe}
    \ex\label{ex:kernundperipherie022}
      \begin{xlist}
        \ex \alert{Baum, Haus, Matte, Döner, Angst, Öl, Kutsche, \ldots}
        \ex \rot{System, Kapuze, Bovist, Schlamassel, Marmelade, Melodie, \ldots}
      \end{xlist}
      \onslide<+->
      \ex
      \begin{xlist}
        \ex \alert{geht, läuft, lacht, schwimmt, liest, \ldots}
        \ex \rot{kann, muss, will, darf, soll, mag}
      \end{xlist}
      \onslide<+->
      \ex
      \begin{xlist}
        \ex \alert{des Hundes, des Geistes, des Tisches, des Fußes, \ldots}
        \ex \rot{des Schweden, des Bären, des Prokuristen, des Phantasten, \ldots}
      \end{xlist}
  \end{exe}
  \onslide<+->
  \Zeile
  \Large
  \centering
  \alert{Hohe Typenhäufigkeit} vs.\ \rot{niedrige Typenhäufigkeit}.  
\end{frame}

\begin{frame}
  {Zwei verschiedene Häufigkeiten}
  \onslide<+->
  \onslide<+->
  \Large\begin{block}{Typenhäufigkeit}
    Wie viele \alert{verschiedene} Realisierungen (=~Typen)\\
    einer Sorte linguistischer Einheiten gibt es?
  \end{block}

  \onslide<+->
  \Zeile
  
  \begin{block}{Tokenhäufigkeit}
    Wie häufig sind die \alert{ggf.\ identischen} Realisierungen\\
    (=~Tokens) einer Sorte linguistischer Einheiten?
  \end{block}
\end{frame}

\begin{frame}
  {Aussagen über das Deutsche}
  \onslide<+->
  \begin{itemize}[<+->]
    \item Relativsätze und eingebettete \textit{w}-Sätze werden nicht\\
      durch Komplementierer (\textit{dass}) eingeleitet.
      \Viertelzeile
    \item \textit{fragen} ist ein schwaches Verb.
      \Viertelzeile
    \item \textit{zurückschrecken} bildet das Perfekt mit dem Hilfsverb \textit{sein}.
      \Viertelzeile
    \item Im Aussagesatz steht vor dem finiten Verb genau ein Satzglied.
      \Viertelzeile
    \item In Kausalsätzen mit \textit{weil} steht das finite Verb an letzter Stelle.
      \Viertelzeile
    \item \textit{zwecks} ist eine Präposition, die den Genitiv regiert und\\
      nur mit Ereignissubstantiven kombiniert werden kann.
  \end{itemize}
\end{frame}


\begin{frame}
  {Normkonform? Regularitätenkonform?}
  \onslide<+->
  \onslide<+->
  \begin{exe}
    \ex
    \begin{xlist}
      \ex Dann sieht man auf der ersten Seite \alert{wer} \rot{dass} kommt.
      \onslide<+->
      \Halbzeile
      \ex Er \rot{frägt} nach der Uhrzeit.
      \onslide<+->
      \Halbzeile
      \ex Man \rot{habe} zu jener Zeit nicht vor Morden \alert{zurückgeschreckt}.
      \onslide<+->
      \Halbzeile
      \ex \rot{Der Universität} \alert{zum Jubiläum} gratulierte auch Bundesministerin Wilms.
      \onslide<+->
      \Halbzeile
      \ex Er ist noch im Büro, \alert{weil} das Licht \rot{brennt} noch.
      \onslide<+->
      \Halbzeile
      \ex Ich schreibe Ihnen \alert{zwecks} \rot{Platz} im Seminar.
    \end{xlist}
  \end{exe}
\end{frame}


\begin{frame}
  {Regel, Regularität und Generalisierung}
  \onslide<+->
  \onslide<+->
  \begin{block}{Regularität}
    Eine grammatische Regularität innerhalb eines Sprachsystems liegt dann vor,\\
    wenn sich Klassen von Symbolen unter vergleichbaren Bedingungen gleich\\
    (und damit vorhersagbar) verhalten.
  \end{block}

  \onslide<+->
  \Halbzeile

  \begin{block}{Regel}
    Eine grammatische Regel ist die Beschreibung einer Regularität, die\\
    in einem normativen Kontext geäußert wird.
  \end{block}

  \onslide<+->
  \Halbzeile
  
  \begin{block}{Generalisierung}
    Eine grammatische Generalisierung ist eine durch Beobachtung zustandegekommene\\
    Beschreibung einer Regularität.
  \end{block}
\end{frame}

\begin{frame}
  {Regel vs.\ Regularität bzw.\ Generalisierung}
  \onslide<+->
  \onslide<+->
  Was ist dann der Status dieser Aussagen?\\
  \Zeile 
  \onslide<+->
  \begin{itemize}
    \item[?] \grau{Relativsätze und eingebettete \textit{w}-Sätze werden nicht\\
      durch Komplementierer wie \textit{dass} eingeleitet.}
    \item[?] \grau{\textit{Fragen} ist ein schwaches Verb.}
    \item[?] \grau{\textit{Zurückschrecken} bildet das Perfekt mit dem Hilfsverb \textit{sein}.}
    \item[?] \grau{Im Aussagesatz steht vor dem finiten Verb genau ein Satzglied.}
    \item[?] \grau{In Kausalsätzen mit \textit{weil} steht das finite Verb an letzter Stelle.}
    \item[?] \grau{\textit{zwecks} ist eine Präposition, die den Genitiv regiert und\\
    nur mit Ereignissubstantiven kombiniert werden kann.}
  \end{itemize}
  \Zeile
  \onslide<+->
  → Entweder \alert{Generalisierungen} über die Grammatik von \alert{Varietäten des Deutschen} \\
  oder \rot{normative Regeln}, die die gegebenen Sätze als \rot{falsch} kennzeichnen.
\end{frame}

\begin{frame}
  {Norm ist Beschreibung}
  \onslide<+->
  \begin{itemize}[<+->]
    \item Norm als Grundkonsens
    \item Sprache und Norm im Wandel
    \item Norm und Situation (Register, Stil, \dots)
    \item Variation in der Norm
      \Zeile
    \item Trotzdem \alert{Relevanz der Norm, insbesondere im schulischen Deutschunterricht}
    \item \alert{Normabweichungen erklären} | Warum passt der Fehler nicht ins System?
    \item \alert{das System erklären} | Wie hängt "`richtig"' mit Generalisierungen zusammen?
    \item \rot{schwarze Grammatikdidaktik} | "`Das ist falsch, merk dir das!"'
  \end{itemize}
\end{frame}


\section{Grammatik im Lehramtsstudium}

\begin{frame}
  {Bildungssprache in der siebten Jahrgangsstufe}
  \onslide<+->
  \onslide<+->
  \alert{"`Gib in eigenen Worten die Aufgabenstellung wieder."'} \grau{(\citealt{GogolinLange2011,Feilke2012})}\\
  \Halbzeile
  \onslide<+->
  \begin{quote}\footnotesize
    \orongsch{(Textaufgabe)}\\
    Im Salzbergwerk Bad Friedrichshall wird Steinsalz abgebaut. Das Salz lagert 40 m unter Meereshöhe, während Bad Friedrichshall 155 m über Meereshöhe liegt. Welche Strecke legt der Förderkorb zurück? \grau{(aus: mathe live, 7. Sj, 2000, S. 19)}\\
    \Halbzeile
    \onslide<+->
    \alert{(Schülerantwort A)}\\
    es steht also m m h- die wollen Steinsalz abbauen und das ist zwar in Salzbergwerk Bad Frieshalle -- oder wie das hier steht -- Friedrichshall -- ja und mmh das das liegt aber vier\slash vierzig Millimeter unter des Meeres -- ja vierzig Meter unter Meereshöhe -- und aber die wollen während ähm aber die wollen bei Fried\slash Friedrichshall 155 Meter über das Meereshöhe Meereshöhe liegt -- obwohldas da ober liegt und jetzt wissen sie nicht welche Strecke sie nehmen sollen undjetzt wollen sie wissen -- wie viel Strecken Strecken es eigentlich ist -- m m h weil so ein För\slash Förderkorb bis zur Erdoberfläche zurück\\
    \Halbzeile
    \onslide<+->
    \alert{(Schülerantwort B)}\\
    also -- ähm (\ldots) -- da das\slash der\slash das Bergwerk Bergwerk 40 Meter unter der Meereshöhe liegt und und Friedrichshall 155 über der Meereshöhe--  muss man 155 plus 40 machen -- weil- dieser -- ähmähm (\ldots) Förderkorb muss ja von 40 Meter 40 Meter unter Meeres\slash unter der Meereshöhe nach oben -- das alles transportieren
  \end{quote}
\end{frame}

\begin{frame}
  {Sprachbetrachtung und Literatur im Deutsch-Abitur I}
  \onslide<+->
  \onslide<+->
  Sprachlich-grammatische Betrachtung zur Literatur in Abiturarbeiten \grau{\citep{Haecker2009}}\\
  \Zeile
  \onslide<+->
  \begin{quote}
    Bsp. 4: Diese Verknüpfung durch Kommas oder Gedankenstriche zeigen (G), dass Ferdinand und sein Vater eine gehobene Sprache sprechen.\\
    \Zeile
    \onslide<+->
    Bsp. 5: Die (\ldots) rhetorischen Fragen deuten darauf hin, dass sich der Präsident irgendwo versucht für sein Handeln zu rechtfertigen und seinem Sohn weiterhin Vorwürfe zu machen (Sb).\\
    \Zeile
    \onslide<+->
    Bsp. 6: Ferdinands und Luisens Persönlichkeiten wurden sehr durch Sprache und die szenische Gestaltung der Szene unterstützt. Ferdinand, der Stürmer und Dränger, bedient sich einer sehr bildhaften Sprache durch Metaphern, Personifikationen und Vergleiche. Luises Sprache ist dagegen durch viele Pausen und Satzstücken (G) geprägt, was ihre Verzweiflung und Unruhe deutlich macht.
  \end{quote}
\end{frame}

\begin{frame}
  {Sprachbetrachtung und Literatur im Deutsch-Abitur II}
  \onslide<+->
  \onslide<+->
  Sprachlich-grammatische Betrachtung zur Literatur in Abiturarbeiten \grau{\citep{Haecker2009}}\\
  \Zeile
  \onslide<+->
  \begin{quote}
    Bsp. 10: <Kirsch> \ldots\ durch den Wegfall des Verbs <soll> nur das Wesentliche, in diesem Fall die Landschaft in ihrer Schönheit, beachtet werden \ldots\ Die Konjunktion \textit{und} (V.\ 16) führt alles zusammen. Das Adverb \textit{ganz} (V.\ 17) verstärkt das Ideal: Ruhe und Schönheit. Der Konsekutivsatz \textit{dass man weiß} (V.\ 19), eingeleitet durch \textit{so} (V.\ 18) stellt den Zusammenhang der Idylle mit der lyrischen Person her. Dieser wird erweitert durch den Kausalsatz \textit{weil man glauben kann} (V.\ 21). Der Zusammenhang wird weiter auch betont mit dem Demonstrativpronomen \textit{dieses} (V.\ 20) und dem bestimmten Artikel \textit{das} (V.\ 22). 
  \end{quote}
\end{frame}


\begin{frame}
  {Bildungssprache}
  \onslide<+->
  \onslide<+->
  \alert{\textit{Der Deutschunterricht führt zu einem kompletten Umbau\\
  der Grammatik des Kindes.}} \grau{\citep{Bredel2013,Eisenberg2004}}\\
  \Zeile
  \begin{itemize}[<+->]
    \item Anforderungen:
    \begin{itemize}[<+->]
      \item Darstellung komplexer Sachverhalte
      \item \dots\ und nicht-faktischer (z.\,B.\ hypothetischer) Sachverhalte
      \item Intensionalität (Abstraktion statt Beispielen)
      \item Registerbewusstsein
    \end{itemize}
       \Halbzeile 
      \item Eigenschaften
    \begin{itemize}[<+->]
      \item dekontextualisiert
      \item schriftorientiert
      \item normorientiert
    \end{itemize}
        \Halbzeile
      \item \alert{Das alles ist verknüpft mit spezifischen grammatischen Formen!}
      \item[→] \alert{Bildungssprache}
  \end{itemize}
\end{frame}

\begin{frame}
  {Sprachbetrachtung}
  \onslide<+->
  \begin{itemize}[<+->]
    \item \alert{Sprachbetrachtung ist der Schlüssel zur Beherrschung der Bildungssprache!}
     \Zeile 
    \item Bewusstsein über richtige und angemessene Form
     \Zeile 
    \item explizite Sprachbetrachtung im Alltag
      \Halbzeile
      \begin{itemize}[<+->]
        \item Selbst- oder Fremdkorrektur
        \item Suche nach richtigen Ausdruck
        \item Orthographie optimieren
        \item Texte optimieren
        \item Begriffe definieren
        \item Grammatikalität beurteilen
      \end{itemize}
  \end{itemize}
\end{frame}

\begin{frame}
  {Ausgangsbasis | vorliterate Kinder und Sprachbetrachtung}
  \onslide<+->
  \onslide<+->
  Klassische Studien \grau{(\citealt{Bredel2013}, auch \citealt[57--58]{Schaefer2018})}\\
  \Halbzeile
  \onslide<+->
  \begin{itemize}[<+->]
    \item \alert{bedeutungsbezogene} bzw.\ \alert{holistische} Betrachtung
      \Viertelzeile
      \begin{itemize}
        \item \textit{Welches Wort ist länger, Haus oder Streichholzschächtelchen?} →\ \textit{Haus.}
          \Viertelzeile
        \item Assoziationen zu Substantiven wie \textit{Bett} →\ \alert{Ereignisse} wie \textit{Schlafengehen} usw.\\
          Erwachsene assoziieren \alert{taxonomisch verwandte Gegenstände} (Nachttisch, Sofa usw.)
          \Viertelzeile
        \item \textit{Warum heißt der Geburtstag  "`Geburtstag"'?} →\ \textit{Weil es Geschenke und Kuchen gibt.}
          \Viertelzeile
        \item \textit{Wieviele Wörter enthält der Satz "`Im alten Haus lebt eine junge Frau."'} →\ \textit{Zwei.}
      \end{itemize}
      \Halbzeile
    \item \textit{Benenne das letzte Wort des Satzes.} → Funktioniert!
      \Halbzeile
    \item[→] Die mentale Grammatik basiert auf Wörtern,\\
      der sprachbetrachtende Zugriff allerdings noch nicht.
  \end{itemize}
\end{frame}

\begin{frame}
  {Schulunterricht}
  \onslide<+->
  \begin{itemize}[<+->]
    \item \alert{systematisch}
      \begin{itemize}
        \item in knapper Zeit das Ganze im Blick
      \end{itemize}
      \Zeile
    \item funktional im Sinn von \alert{Form-Funktion-Beziehung}
      \begin{itemize}
        \item Formen systematisieren
        \item erst dann auf Funktionen beziehen
      \end{itemize}
      \Zeile
    \item \alert{induktiv}
      \begin{itemize}
        \item keine rein deduktive Anwendung vorgegebener Begriffe
        \item Erkenntnisprozesse über sprachliche Formen und Funktionen
        \item \alert{\textit{Grammatik machen}} (Eisenberg)
      \end{itemize}
  \end{itemize}
\end{frame}

\begin{frame}
  {Aufgaben von Lehrkräften}
  \onslide<+->
  \onslide<+->
  \alert{\textit{Lehrkräften wird die Sprache der Lernenden anvertraut.}} \grau{\citep{Eisenberg2004}}\\
  \Zeile
  \begin{itemize}[<+->]
    \item Unterrichten der Schrift, Orthographie und Schreibung
    \item Unterweisung in Bildungssprache\slash Sprachbetrachtung
    \item Erkennen und \alert{Einordnen} von \alert{sprachlichen Defiziten}
    \item Erkennen von \alert{Interferenz mit Dialekt bzw.\ anderen Erstsprachen}
    \item \alert{Bewerten} von sprachlichen Leistungen
    \item \alert{Erklären} der Bewertung (auch gegenüber Eltern)
      \Zeile
    \item[→] Anforderung | vertieftes Wissen über Sprache, vor allem Grammatik
    \item[→] Methode der sprachlichen Analyse über Faktenwissen hinaus
    \item[→] \rot{Studierende des Lehramts müssen ein erheblich tieferes Grammatikwissen\\
    als die Schulkinder und Jugendlichen haben, die sie später unterrichten!}
  \end{itemize}
\end{frame}


\begin{frame}
  {"`Wie war das?"'}
  Ich wiederhole zur Sicherheit nochmal\ldots\\
  \Zeile
  \onslide<+->
  \begin{center}
    \Large\rot{Studierende des Lehramts müssen\\
               ein erheblich tieferes Grammatikwissen\\
               als die Schulkinder und Jugendlichen haben,\\
               die sie später unterrichten!}
  \end{center}
\end{frame}

\begin{frame}
  {"`Wozu brauchen wir das denn?"'}
  \onslide<+->
  \begin{itemize}[<+->]
    \item Diese Frage gilt hiermit als nachhaltig beantwortet.
      \Zeile
    \item \rot{Linguistik | keine praktische Anleitungen für erfolgreiche Schulstunden}
    \item sondern Grundausbildung im \alert{Umgang mit Sprache} 
      \Zeile
    \item[→] Minimalforderung | \alert{Examinierte Lehrkräfte müssen die Aufgaben\\
         für Schüler selber lösen und in den Gesamtkontext einordnen können.}
    \item Nichtmal das klappt zuverlässig \citep{SchaeferSayatz2017a}.
  \end{itemize}
\end{frame}

\section{Zur nächsten Woche | Überblick}

\begin{frame}
  {Morphologie und Lexikon des Deutschen | Plan}
  \rot{Alle} angegebenen Kapitel\slash Abschnitte aus \rot{\citet{Schaefer2018b}} sind \rot{Klausurstoff}!\\
  \Halbzeile
  \begin{enumerate}
    \item Grammatik und Grammatik im Lehramt \rot{(Kapitel 1 und 3)}
    \item \alert{Morphologie und Grundbegriffe} \rot{(Kapitel 2, Kapitel 7 und Abschnitte 11.1--11.2)}
    \item Wortklassen als Grundlage der Grammatik \rot{(Kapitel 6)}
    \item Wortbildung | Komposition \rot{(Abschnitt 8.1)}
    \item Wortbildung | Derivation und Konversion \rot{(Abschnitte 8.2--8.3)}
    \item Flexion | Nomina außer Adjektiven \rot{(Abschnitte 9.1--9.3)}
    \item Flexion | Adjektive und Verben \rot{(Abschnitt 9.4 und Kapitel 10)}
    \item Valenz \rot{(Abschnitte 2.3, 14.1 und 14.3)}
    \item Verbtypen als Valenztypen \rot{(Abschnitte 14.4--14.5, 14.7--14.9)} 
    \item Kernwortschatz und Fremdwort \grau{(vorwiegend Folien)}
  \end{enumerate}
  \Halbzeile
  \centering 
  \url{https://langsci-press.org/catalog/book/224}
\end{frame}

  \let\subsection\section\let\section\woopsi

  \section{Morphologie und Grundbegriffe}
  \let\woopsi\section\let\section\subsection\let\subsection\subsubsection
  \section{Überblick}

\begin{frame}
  {Morphologie | Flexion und Wortbildung}
  \pause
  \begin{itemize}[<+->]
    \item \alert{Formveränderungen} und \alert{Merkmalsänderungen}
      \begin{itemize}[<+->]
        \item Veränderungen von Werten
        \item Veränderungen von Merkmalsaustattungen
      \end{itemize}
      \Halbzeile
    \item Morphe (= Wortbestandteile) und ihre Funktionen
    \item Morphe | alle Stämme und alle nicht-lexikalischen Morphe
      \Halbzeile
    \item statische und volatile Merkmale
    \item Wortbildung vs.\ Flexion, definiert anhand von Merkmalen
      \Halbzeile
    \item Syntax und Morphologie
    \item Phrasenbestimmung
    \item Köpfe
    \item Nominalphrasen und Präpositionalphrasen
  \end{itemize}
\end{frame}


\section{Stämme und Affixe}

\begin{frame}
  {Form und Funktion | Flexion}
  \pause
  \begin{exe}
    \ex
    \begin{xlist}
      \ex \alert{Den Präsidenten} begrüßte \alert{der Dekan} äußerst respektlos.
      \pause
      \ex \alert{Der Dekan} begrüßte \alert{den Präsidenten} äußerst respektlos.
    \end{xlist}
    \pause
    \ex
    \begin{xlist}
      \ex \alert{Die Präsidentin} begrüßte \alert{die Dekanin} äußerst respektlos.
      \pause
      \ex \alert{Die Dekanin} begrüßte \alert{die Präsidentin} äußerst respektlos.
    \end{xlist}
  \end{exe}
  \pause
  \Zeile
  Formveränderungen lexikalischer Wörter \alert{schränken ihre möglichen grammatischen Funktionen und Relationen im Satz ein}\dots\\
  \pause
  \Halbzeile
  \dots und sie haben semantische und systemexterne Folgen.

\end{frame}

\begin{frame}
  {Form und Funktion | Wortbildung}
  \pause
  \begin{exe}
    \ex grün\alert{lich}, röt\alert{lich}, gelb\alert{lich}
    \pause
    \ex Neu\alert{igkeit}, Blöd\alert{heit}, Tauch\alert{er}, Heb\alert{ung}
    \pause
    \ex Fenster\alert{rahmen}, Tücher\alert{spender}, Glas\alert{korken}, Unter\alert{schrank}
  \end{exe}
  \pause
  \Zeile
  Formveränderungen von einem zu einem anderen lexikalischen Wort führen zu Bedeutungs- und kategorialen Veränderungen.
\end{frame}

\begin{frame}
  {Markierungsfunktionen von Morphen I}
  \pause
  \begin{exe}
    \ex
    \begin{xlist}
      \ex{(der) \alert<4>{Berg}}
      \ex{(den) \alert<4>{Berg}}
      \ex{(dem) \alert<4>{Berg}}
      \ex{(des) \alert<5>{Berg}\rot<5>{-es}}
      \ex{(die) \alert<6>{Berg}\rot<6>{-e}}
      \ex{(der) \alert<6>{Berg}\rot<6>{-e}}
    \end{xlist}
    \pause
    \ex
    \begin{xlist}
      \ex{(der) \alert<8>{Mensch}}
      \ex{(den) \alert<9>{Mensch}\rot<9>{-en}}
      \ex{(dem) \alert<9>{Mensch}\rot<9>{-en}}
      \ex{(des) \alert<9>{Mensch}\rot<9>{-en}}
      \ex{(die) \alert<9>{Mensch}\rot<9>{-en}}
      \ex{(der) \alert<9>{Mensch}\rot<9>{-en}}
    \end{xlist}
  \end{exe}
\end{frame}

\begin{frame}
  {Markierungsfunktionen von Morphen II}
  \pause
  \begin{exe}
    \ex
    \begin{xlist}
      \ex{(ich) \alert<3>{kauf}\rot<3>{-e}}
      \ex{(du) \alert<4>{kauf}\rot<4>{-st}}
      \ex{(wir) \alert<5>{kauf}\rot<5>{-en}}
      \ex{(sie) \alert<5>{kauf}\rot<5>{-en}}
    \end{xlist}
  \end{exe}
\end{frame}

\begin{frame}
  {Morphe und Markierungsfunktionen}
  \pause
  \begin{itemize}[<+->]
    \item Formveränderungen
      \begin{itemize}[<+->]
        \item oft nicht \alert{eine} Funktion
        \item \rot{Einschränkung} der möglichen Funktionen
      \end{itemize}
   \Halbzeile 
    \item \alert{Markierungsfunktion} | eine \alert{Einschränkung}\\
      der möglichen Merkmale oder Werte einer Wortform
    \item zum Beispiel \textit{-en} bei schw.\ Maskulina | \rot{nicht} Nominativ Singular
    \item oder \textit{-en} bei Verben im Präsens | Plural und nicht adressatbezogen
      \Halbzeile
    \item \alert{Morphe | alle segmentalen Einheiten mit Markierungsfunktion}
    \item \alert{Stämme} und \alert{Affixe}
  \end{itemize}
\end{frame}

\begin{frame}
  {Stämme I}
  \pause
  \begin{exe}
    \ex
    \begin{xlist}
      \ex{(ich) \alert<5->{kauf}-e\\
        (du) \alert<5->{kauf}-st\\
        (ihr) \alert<5->{kauf}-t }
        \pause
        \ex{(ich) \alert<6->{kauf}-te\\
        (du) \alert<6->{kauf}-test\\
        (ihr) \alert<6->{kauf}-tet}
        \pause
        \ex{(ich habe) ge-\alert<7->{kauf}-t\\
        (du hast) ge-\alert<7->{kauf}-t\\
        (ihr habt) ge-\alert<7->{kauf}-t}
    \end{xlist}
  \end{exe}
\end{frame}

\begin{frame}
  {Stämme II}
  \begin{exe}
    \ex
    \begin{xlist}
      \ex{(ich) \alert<4->{nehm}-e\\
        (du) \rot<5->{nimm}-st\\
          (es) \rot<5->{nimm}-t\\
          (ihr) \alert<4->{nehm}-t}
        \pause
        \ex{(ich) \orongsch<6->{nahm}\\
        (du) \orongsch<6->{nahm}-st\\
          (ihr) \orongsch<6->{nahm}-t}
        \pause
        \ex{(ich habe) ge-\gruen<7->{nomm}-en\\
        (du hast) ge-\gruen<7->{nomm}-en\\
        (ihr habt) ge-\gruen<7->{nomm}-en}
    \end{xlist}
  \end{exe}
  \pause
  \pause
  \pause
  \pause
  \pause
  Der \alert{Stamm} kann nicht "`der unveränderliche Wortbestandteil"'\\
  eines lexikalischen Wortes (in einem Paradigma) sein.\\
  \Zeile
  \pause
  \alert{\dots aber der mit der Bedeutung, also der lexikalischen Markierungsfunktion}!
\end{frame}

\begin{frame}
  {Affixe}
  \pause
  \begin{exe}
    \ex
    \begin{xlist}
      \ex (ich) nehm\alert<6->{-e}
      \pause
      \ex (des) Berg\alert<7->{-es}
      \pause
      \ex Schön\alert<8->{-heit}
      \pause
      \ex \alert<9->{Un-}ding
    \end{xlist}
  \end{exe}
  \Zeile
  \pause
  \pause
  \pause
  \pause
  \pause
  \begin{itemize}[<+->]
    \item \alert{keine lexikalische Markierungsfunktion} (= keine eigene Bedeutung)
    \item \alert{nicht wortfähig} = nicht ohne Stamm verwendbar
  \end{itemize}
\end{frame}



\section{Merkmale in Flexion und Wortbildung}

\begin{frame}
  {Statische und volatile Merkmale}
  \pause
  \begin{itemize}[<+->]
    \item Eigenschaften | "`Rotsein"' (Erdbeere), "`325m hoch"' (Eiffelturm) usw.
    \item Merkmale | \alert{\textsc{Farbe}}, \alert{\textsc{Länge}} usw.
    \item Werte
      \begin{itemize}[<+->]
        \item \alert{\textsc{Farbe}}: \rot{\textit{rot}}, \rot{\textit{grau}}, \ldots
        \item \alert{\textsc{Länge}}: \rot{\textit{3cm}}, \rot{\textit{325m}}, \ldots
      \end{itemize}
  \end{itemize}
  \pause
  \Halbzeile 
  \begin{exe}
    \ex
    \begin{xlist}
      \ex{Haus = [\textsc{Bed}: \gruen<12->{\textbf{\textit{haus}}}, \textsc{Klasse}: \gruen<12->{\textbf{\textit{subst}}}, \textsc{Gen}: \gruen<12->{\textbf{\textit{neut}}}, \textsc{Kas}: \orongsch<13->{\textit{nom}}, \textsc{Num}: \orongsch<13->{\textit{sg}}]}
      \pause
      \ex{Haus-es = [\textsc{Bed}: \gruen<12->{\textbf{\textit{haus}}}, \textsc{Klasse}: \gruen<12->{\textbf{\textit{subst}}}, \textsc{Gen}: \gruen<12->{\textbf{\textit{neut}}}, \textsc{Kas}: \orongsch<13->{\textit{gen}}, \textsc{Num}: \orongsch<13->{\textit{sg}}]}
      \pause
      \ex{Häus-er = [\textsc{Bed}: \gruen<12->{\textbf{\textit{haus}}}, \textsc{Klasse}: \gruen<12->{\textbf{\textit{subst}}}, \textsc{Gen}: \gruen<12->{\textbf{\textit{neut}}}, \textsc{Kas}: \orongsch<13->{\textit{nom}}, \textsc{Num}: \orongsch<13->{\textit{pl}}]}
    \end{xlist}
  \end{exe}
  \Halbzeile
  \pause
  \begin{itemize}[<+->]
    \item bei einem lexikalischen Wort
      \begin{itemize}
        \item \gruen{statische Merkmale} wertestabil
        \item \orongsch{volatile Merkmale} werteverändernd im Paradigma
      \end{itemize}
  \end{itemize}
\end{frame}

\begin{frame}
  {Wortbildung in Abgrenzung zur Flexion}
  \pause
  \begin{exe}
    \ex
    \begin{xlist}
      \ex trocken (Adj) → \alert{Trocken}\rot{-heit} (Subst)\label{ex:trocken}
      \ex Kauf (Subst), Rausch (Subst) → \alert{Kauf}\rot{-rausch} (Subst)\label{ex:kauf}
      \ex gehen (V) → \alert{be}\rot{-gehen} (V)\label{ex:gehen}
    \end{xlist}
    \pause
    \ex
    \begin{xlist}
      \ex \alert{lauf}\rot{-en} (1\slash 3 Pl Prs Ind) → \alert{lauf}\rot{-e} (1 Sg Prs Ind)\label{ex:lauf}
      \ex \alert{Münze} (Sg) → \alert{Münze}\rot{-n} (Pl)\label{ex:muenze}
    \end{xlist}
  \end{exe}
  \pause
  \Halbzeile
  \begin{itemize}[<+->]
    \item Wortbildung
      \begin{itemize}[<+->]
        \item statische Merkmale geändert | Wortklasse, Bedeutung \alert{(\ref{ex:trocken})}
        \item \ldots oder gelöscht | alles außer der Bedeutung des Erstglieds bei Komposition \alert{(\ref{ex:kauf})}
        \item \ldots oder umgebaut | Valenz von Verben beim Applikativ \alert{(\ref{ex:gehen})}
        \item \orongsch{produktives Erschaffen neuer lexikalischer Wörter}
      \end{itemize}
  \Halbzeile
    \item Flexion
      \begin{itemize}
        \item Änderung der Werte volatiler Merkmale \alert{(\ref{ex:lauf},\ref{ex:muenze})}
        \item \alert{oft Anpassung an syntaktischen Kontext}
      \end{itemize}
  \end{itemize}
\end{frame}


\section{Konstituenten}


\begin{frame}
  {Es gibt keine \orongsch{reine Morphologie}}
  \onslide<+->
  \onslide<+->
  Ebenen der Grammatik\\
  \Viertelzeile
  \begin{itemize}[<+->]
    \item \alert{Phonologie} | Kombinatorik von Lauten, Silben, Betonung (Akzent) usw.
    \item \alert{Morphologie} | Kombinatorik von Wortbestandteilen und deren Funktionen
    \item \alert{Syntax} | Kombinatorik von Wörtern, Wortgruppen und Sätzen
    \item \alert{Semantik} | Ableitung von Bedeutungen aus der formalen Kombinatorik
  \end{itemize}
  \onslide<+->
  \Zeile
  Einige Interaktionen und Schnittstellen\\
  \Viertelzeile
  \begin{itemize}[<+->]
    \item \orongsch{Lexik} | Klassifikation von Wörtern nach \alert{grammatischen Merkmalen}
    \item \orongsch{Morphophonologie} | Beschränkungen der \alert{Morphologie} aufgrund der \alert{Phonologie} 
    \item \orongsch{Morphosyntax} | Schnittstelle von Morphologie und Syntax (Kasus, Numerus, Valenz)
    \item \orongsch{Syntax-Semantik-Morphologie-Lexik-Schnittstelle} | Passive, Infinitivsyntax usw.
  \end{itemize}
  \onslide<+->
  \Halbzeile
  \rot{→ Wir brauchen ein minimales (Schul-)Wissen über Syntax in der Morphologie.}
\end{frame}

\begin{frame}
  {Sprachliche Einheiten und ihre Bestandteile}
  \onslide<+->
  \onslide<+->
  Wichtig vor allem für die Syntax | \alert{Strukturbildung}\\
  \Zeile
  \begin{itemize}[<+->]
    \item\footnotesize \alert{Satz} \\
      {Nadezhda reißt die Hantel souveräner als andere Gewichtheberinnen.}
      \Halbzeile

    \item\footnotesize \alert{Satzteile} \\
      {Nadezhda | reißt | die Hantel | souveräner als andere Gewichtheberinnen}
      \Halbzeile

    \item\footnotesize \alert{Wörter} \\
      {Nadezhda | reißt | die | Hantel | souveräner | als | andere | Gewichtheberinnen}
      \Halbzeile

    \item\footnotesize \alert{Wortteile} \\
      {Nadezhda | reiß | t | d | ie | Hantel | souverän | er | als | ander | e | Gewicht | heb | er | inn | en}
      \Halbzeile

    \item\footnotesize \alert{Laute\slash Buchstaben} \\
      {N | a | d | e | z | h | d | a \ldots}
  \end{itemize}
\end{frame}


\begin{frame}
  {Syntaktische Strukturen und morphologische Merkmale}
  \onslide<+->
  \onslide<+->
  \begin{center}
  \resizebox{0.8\textwidth}{!}{\begin{forest}
    [Nadezhda reißt die Hantel souveräner als andere Gewichtheberinnen
      [Nadezhda]
      [reißt]
      [die Hantel, alt=<3->{orongsch}{}
        [die, alt=<4->{orongsch}{}]
        [Hantel, alt=<5->{orongsch}{}]
      ]
      [souveräner als andere Gewichtheberinnen
        [souveräner]
        [als andere Gewichtheberinnen
          [als]
          [andere Gewichtheberinnen, alt=<6->{gruen}{}
            [andere, alt=<7->{gruen}{}]
            [Gewichtheberinnen, alt=<8->{gruen}{}]
          ]
        ]
      ]
    ]
  \end{forest}}
  \end{center}

  \Zeile
  \onslide<9->{Übereinstimmung von Merkmalen in syntaktischen Gruppen\\}
  \onslide<10->{\orongsch{Akkusativ Femininum Singular}} \onslide<11->{| \gruen{Nominativ Plural}}
\end{frame}

\begin{frame}
  {Morphologie und Syntax | I}
  \onslide<+->
  \onslide<+->
  \alert{Kongruenz} | Merkmalübereinstimmung in Nominalphrasen\\
  \Zeile
  \Zeile
  \centering 
  \onslide<+->
  \begin{tikzpicture}[node distance=1.5cm, auto]
    \node (context) {Wir möchten};
    \node [right=of context] (diesen) {\alert<5->{diesen}};
    \node[right=of diesen] (schönen) {\alert<4->{schönen}};
    \node[right=of schönen] (Sportwagen) {\alert<4->{Sportwagen}};
    \onslide<4->{\path[<->, trueblue, draw, bend left=30] (schönen) edge node {\scriptsize Akk Mask Sg} (Sportwagen);}
    \onslide<5->{\path[<->, trueblue, draw, bend left=60] (diesen) edge node[above] {\scriptsize Akk Mask Sg} (Sportwagen);}
    \onslide<6->{\path[<->, trueblue, draw, bend left=30] (diesen) edge node[above] {\scriptsize Akk Mask Sg} (schönen);}
  \end{tikzpicture}
\end{frame}

\begin{frame}
  {Morphologie und Syntax | II}
  \onslide<+->
  \onslide<+->
  \alert{Kongruenz} | Merkmalübereinstimmung zwischen Subjekt und finitem Verb\\
  \Zeile
  \Zeile
  \centering 
  \onslide<+->
  \begin{tikzpicture}[node distance=1cm, auto]
   \node                      (context)    {Ich glaube, dass};
   \node[right=of context]    (ihr)        {\alert<4->{ihr}};
   \node[right=of ihr]        (den)        {den};
   \node[right=of den]        (Wagen)      {Wagen};
   \node[right=of Wagen]      (anschieben) {anschieben};
   \node[right=of anschieben] (müsst)      {\alert<4->{müsst}};
   \onslide<4->{\path[<->, trueblue, draw, bend left=30]  (ihr) edge node {\small 2.~Per Pl} (müsst);}
  \end{tikzpicture}
\end{frame}

\begin{frame}
  {Morphologie und Syntax | III}
  \onslide<+->
  \onslide<+->
  \gruen{Rektion} | Präpositionen bestimmen den Kasus von ganzen \alert{Nominalphrasen}\\
  \Zeile
  \Zeile
  \centering 
  \onslide<+->
  \begin{tikzpicture}[node distance=1cm, auto]
   \node                      (context)    {Wir fahren};
   \node[right=of context]    (mit)        {\gruen<4->{mit}};
   \node[right=of mit]        (dem)        {\alert<6->{dem}};
   \node[right=of dem]        (neuen)      {\alert<5->{neuen}};
   \node[right=of neuen]      (Wagen)      {\alert<4->{Wagen}};
   \node[right=of Wagen]      (rest)       {nach hause};
   \onslide<4->{\path[->, gruen, draw, bend right=30]  (mit) edge node[below] {Dat} (Wagen);}
   \onslide<5->{\path[<->, trueblue, draw, bend left=30]  (neuen) edge node {\footnotesize Dat Mask Sg} (Wagen);}
   \onslide<6->{\path[<->, trueblue, draw, bend left=30]  (dem) edge node {\footnotesize Dat Mask Sg} (neuen);}
  \end{tikzpicture}
\end{frame}

\begin{frame}
  {Morphologie und Syntax | IV}
  \onslide<+->
  \onslide<+->
  \gruen{Rektion} | Verben bestimmen den Kasus von ganzen \alert{Nominalphrasen}\\
  \Zeile
  \Zeile
  \centering 
  \onslide<+->
  \begin{tikzpicture}[node distance=1cm, auto]
    \node                      (Nom)        {\alert<4->{Ich}};
    \node[right=of Nom]        (V)          {\gruen<4->{gab}};
    \node[right=of V]          (Dat)        {\alert<5->{dem netten Kollegen}};
    \node[right=of Dat]        (Akk)        {\alert<6->{den Stift}};
    \node[right=of Akk]        (rest)       {zurück};
    \onslide<4->{\path[->, gruen, draw, bend right=-30]  (V) edge node[below] {Nom} (Nom);}
    \onslide<5->{\path[->, gruen, draw, bend right=30]  (V) edge node[below] {Dat} (Dat);}
    \onslide<6->{\path[->, gruen, draw, bend right=60]  (V) edge node[below] {Akk} (Akk);}
  \end{tikzpicture}
\end{frame}

\begin{frame}
  {Phrasenbestimmung}
  \onslide<+->
  \onslide<+->
  \alert{Konstituenten} | Bestandteile irgendeiner Struktur\\
  \Halbzeile
  \onslide<+->
  \alert{Phrasen} | syntaktische Konstituenten mit bestimmten Eigenschaften\\
  \Zeile
  \begin{itemize}[<+->]
    \item Phrasenbestimmung | ähnlich \alert{Satzgliedanalyse} aus der Schule
    \item \alert{Tests} auf Phrasenstatus
    \item Unsicherheiten trotz Tests
  \end{itemize}
\end{frame}

\begin{frame}
  {Pronominalisierungstest}
  \pause
  \begin{exe}
    \ex Mausi isst \alert<3->{den leckeren Marmorkuchen}.\\
    \pause
      \KTArr{PronTest} Mausi isst \alert{ihn}.
    \pause
    \ex{\label{ex:konstituententests025} \rot<5->{Mausi isst} den Marmorkuchen.\\
    \pause
      \KTArr{PronTest} \Ast \rot{Sie} den Marmorkuchen.}
    \pause
    \ex{\label{ex:konstituententests026} Mausi isst \alert<7->{den Marmorkuchen und das Eis mit Multebeeren}.\\
    \pause
    \KTArr{PronTest} Mausi isst \alert{sie}.}
  \end{exe}
  \pause
  \Halbzeile
  Pronominalausdrücke i.\,w.\,S.
  \begin{exe}
    \ex{\label{ex:konstituententests027} Ich treffe euch \alert<9->{am Montag} \gruen<10->{in der Mensa}.\\
    \pause
    \KTArr{PronTest} Ich treffe euch \alert{dann} \gruen<10->{dort}.}
      \pause
      \pause
      \ex{\label{ex:konstituententests028} Er liest den Text \alert<12->{auf eine Art, die ich nicht ausstehen kann}.\\
      \pause
      \KTArr{PronTest} Er liest den Text \alert{so}.}
  \end{exe}
\end{frame}

\begin{frame}
  {Vorfeldtest\slash Bewegungstest}
  \pause
  \begin{exe}
    \ex
    \begin{xlist}
      \ex{Sarah sieht den Kuchen \alert<3->{durch das Fenster}.\\
        \pause
        \KTArr{VfTest} \alert{Durch das Fenster} sieht Sarah den Kuchen.}
      \pause
      \ex{Er versucht \alert{zu essen}.\\
        \pause
        \KTArr{VfTest} \alert<5->{Zu essen} versucht er.}
      \pause
      \ex{Sarah möchte gerne \alert{einen Kuchen backen}.\\
        \pause
        \KTArr{VfTest} \alert<7->{Einen Kuchen backen} möchte Sarah gerne.}
      \pause
      \ex{Sarah möchte \rot<9->{gerne einen} Kuchen backen.\\
        \pause
        \KTArr{VfTest} \Ast \rot{Gerne einen} möchte Sarah Kuchen backen.}
    \end{xlist}
  \end{exe}
  \pause
  \Halbzeile
  verallgemeinerter "`Bewegungstest"'\\
  \begin{exe}
    \ex\label{ex:konstituententests037}
    \begin{xlist}
      \ex{\label{ex:konstituententests038} Gestern hat \alert<11->{Elena} \gruen<11->{im Turmspringen} \orongsch<11->{eine Medaille} gewonnen.}
      \pause
      \ex{\label{ex:konstituententests039} Gestern hat \gruen{im Turmspringen} \alert{Elena} \orongsch{eine Medaille} gewonnen.}
      \pause
      \ex{\label{ex:konstituententests040} Gestern hat \gruen{im Turmspringen} \orongsch{eine Medaille} \alert{Elena} gewonnen.}
    \end{xlist}
  \end{exe}
\end{frame}

\begin{frame}
  {Koordinationstest}
  \pause
  \begin{exe}
    \ex\label{ex:konstituententests041}
    \begin{xlist}
      \ex Wir essen \alert<3->{einen Kuchen}.\\
      \pause
        \KTArr{KoorTest} Wir essen \alert{einen Kuchen} \gruen{und} \alert{ein Eis}.
      \pause
      \ex Wir \alert<5->{essen einen Kuchen}.\\
      \pause
        \KTArr{KoorTest} Wir \alert{essen einen Kuchen} \gruen{und} \alert{lesen ein Buch}.
      \pause
      \ex Sarah hat versucht, \alert<7->{einen Kuchen zu backen}.\\
      \pause
        \KTArr{KoorTest} Sarah hat versucht, \alert{einen Kuchen zu backen} \gruen{und} \\{}\alert{heimlich das Eis aufzuessen}.
      \pause
      \ex Wir sehen, dass \alert<9->{die Sonne scheint}.\\
      \pause
        \KTArr{KoorTest} Wir sehen, dass \alert{die Sonne scheint} \gruen{und} \\{}\alert{Mausi den Rasen mäht}.
    \end{xlist}
  \end{exe}
  \pause
  \begin{exe}
    \ex{\label{ex:konstituententests047} Der Kellner notiert, dass \rot<11->{meine Kollegin einen Salat} möchte.\\
    \pause
    \KTArr{KoorTest} Der Kellner notiert, dass \rot{meine Kollegin einen Salat}\\
    \gruen{und} \rot{mein Kollege einen Sojaburger} möchte.}
    \end{exe}
\end{frame}

\begin{frame}
  {Jede Phrase hat einen Kopf!}
  \onslide<+->
  \onslide<+->
  \orongsch{Der Kopf bestimmt \alert{allein} über die relevanten grammatischen Eigenschaften\\
  der Phrase und kann nie weggelassen werden.}\\
  \onslide<+->
  \Halbzeile
  Phrasen werden daher nach der Kategorie des Kopfes benannt.\\
  \Zeile
  \begin{itemize}[<+->]
    \item \alert{Nominalphrasen} (NPs) haben \orongsch{Nomina} als Köpfe
      \begin{itemize}[<+->]
        \item \alert{[der schöne \orongsch{Baum} vor dem Fenster]}
        \item \grau{Ich kenne} \alert{keinerlei \orongsch{Blumen}, die jetzt schon blühen würden}.
      \end{itemize}
      \Halbzeile
    \item \alert{Adjektivphrasen} (APs) haben \orongsch{Adjektive} als Köpfe
      \begin{itemize}[<+->]
        \item \grau{der} \alert{[überaus \orongsch{schöne}]} \grau{Baum vor dem Fenster}
        \item \grau{Die Kollegin ist} \alert{[\orongsch{stolz} auf ihre Tochter]}.
      \end{itemize}
      \Halbzeile
    \item \alert{Präpositionalphrasen} (PPs) haben \orongsch{Präpositionen} als Köpfe
      \begin{itemize}[<+->]
        \item \grau{der Baum} \alert{[\orongsch{vor} dem Fenster]}
        \item \grau{Der Baum steht} \alert{[\orongsch{vor} dem Fenster]}.
      \end{itemize}
  \end{itemize}
\end{frame}


\begin{frame}
  {Einige typische Muster von Nominalphrasen (NPs)}
  \onslide<+->
  \onslide<+->
  Je nach Art des Kopfs -- Eigenname (Name), Substantiv (Subst), Pronomen (Pro) --\\
  sind die Positionen links vom Kopf nicht besetzbar.\\
  \onslide<+->
 \Halbzeile 
  \begin{center}
    \scalebox{0.8}{\begin{tabular}[h]{lllll}
      \toprule
      \grau{Artikel oder}    & \grau{AP}         & \alert{nominaler}     & \grau{PPs, Adverben}   & \grau{Relativsätze und}  \\
      \grau{Genitiv-NP}      & \grau{} & \alert{Kopf}          & \grau{usw.} & \grau{Komplementsätze}   \\
      \midrule
      &&&& \\
      \textit{die}             & \textit{drei}       & \alert{\textit{Tische}}\Sub{Subst} & \textit{vor der Tafel}    & \textit{die heute fehlen}              \\
      &&&& \\
      \textit{Otjes}           & \textit{intelligente} & \alert{\textit{Kinder}}\Sub{Subst} & & \\
      &&&& \\
      && \alert{\textit{Orangensaft}}\Sub{Subst} && \\
      &&&& \\
      \Dim                     & \Dim                & \alert{\textit{Lemmy}}\Sub{Name} & \textit{von Motörhead}      &                               \\
      &&&& \\
      \Dim                     & \Dim               & \alert{\textit{jener}}\Sub{Pro}  & \textit{dort drüben} & \\
      &&&& \\
      \Dim                     & \Dim               & \alert{\textit{alle}}\Sub{Pro}   & & \textit{die einen Kaffe möchten} \\
    \end{tabular}}
  \end{center}
\end{frame}

\begin{frame}
  {Einige typische Muster von Nominalphrasen (PPs)}
  \onslide<+->
  \onslide<+->
  Die NP rechts ist obligatorisch, ihr Kasus wird von der Präposition bestimmt.
  \onslide<+->
  \Halbzeile
  \begin{center}
    \scalebox{0.8}{\begin{tabular}[h]{lll}
      \toprule
      \grau{Modifizierer} & \alert{Präposition} & \alert{NP (Kasus von} \\
       & \alert{(Kopf)}      & \alert{Präposition bestimmt)} \\
      \midrule
      && \\
      & \textit{\alert{mit}} & \textit{den drei Tischen vor der Tafel, die heute fehlen} \\
      && \\
      & \textit{\alert{von}} & \textit{Otjes intelligenten Kindern} \\
      && \\
      \textit{ganz} & \textit{\alert{ohne}} & \textit{Orangensaft} \\
      && \\
      & \textit{\alert{dank}} & \textit{Lemmy von Motörhead} \\
      && \\
      \textit{genau} & \textit{\alert{neben}} & \textit{jenem dort drüben} \\
      && \\
      & \textit{\alert{für}}   & \textit{alle, die Kaffee möchten} \\
    \end{tabular}}
  \end{center}
\end{frame}

\ifdefined\TITLE
  \section{Zur nächsten Woche | Überblick}

  \begin{frame}
    {Morphologie und Lexikon des Deutschen | Plan}
    \rot{Alle} angegebenen Kapitel\slash Abschnitte aus \rot{\citet{Schaefer2018b}} sind \rot{Klausurstoff}!\\
    \Halbzeile
    \begin{enumerate}
      \item Grammatik und Grammatik im Lehramt (Kapitel 1 und 3)
      \item Morphologie und Grundbegriffe (Kapitel 2, Kapitel 7 und Abschnitte 11.1--11.2)
      \item \rot{Wortklassen als Grundlage der Grammatik (Kapitel 6)}
      \item Wortbildung | Komposition (Abschnitt 8.1)
      \item Wortbildung | Derivation und Konversion (Abschnitte 8.2--8.3)
      \item Flexion | Nomina außer Adjektiven (Abschnitte 9.1--9.3)
      \item Flexion | Adjektive und Verben (Abschnitt 9.4 und Kapitel 10)
      \item Valenz (Abschnitte 2.3, 14.1 und 14.3)
      \item Verbtypen als Valenztypen (Abschnitte 14.4--14.5, 14.7--14.9) 
      \item Kernwortschatz und Fremdwort (vorwiegend Folien)
    \end{enumerate}
    \Halbzeile
    \centering 
    \url{https://langsci-press.org/catalog/book/224}
  \end{frame}

\fi

  \let\subsection\section\let\section\woopsi

  \section{Wortklassen als Grundlage der Grammatik}
  \let\woopsi\section\let\section\subsection\let\subsection\subsubsection
  \input{includes/03.+Wortklassen+als+Grundlage+der+Grammatik.tex}
  \let\subsection\section\let\section\woopsi
  
  \section{Wortbildung | Komposition}
  \let\woopsi\section\let\section\subsection\let\subsection\subsubsection
  \section{Überblick}

\begin{frame}
  {Wortbildung | Komposition}
  \onslide<+->
  \begin{itemize}[<+->]
    \item Wiederholung | statische und volatile Merkmale
    \item Wiederholung | Wortbildung und Flexion
      \Zeile
    \item Produktivität und Transparenz
    \item Köpfe und Typen von Komposita
    \item Kompositionsfugen
  \end{itemize}
\end{frame}

\section{Wortbildung}


\begin{frame}
  {Wiederholung | Statische und volatile Merkmale}
  \pause
  \begin{itemize}[<+->]
    \item Eigenschaften | "`Rotsein"' (Erdbeere), "`325m hoch"' (Eiffelturm) usw.
    \item Merkmale | \alert{\textsc{Farbe}}, \alert{\textsc{Länge}} usw.
    \item Werte
      \begin{itemize}[<+->]
        \item \alert{\textsc{Farbe}}: \rot{\textit{rot}}, \rot{\textit{grau}}, \ldots
        \item \alert{\textsc{Länge}}: \rot{\textit{3cm}}, \rot{\textit{325m}}, \ldots
      \end{itemize}
  \end{itemize}
  \pause
  \Halbzeile 
  \begin{exe}
    \ex
    \begin{xlist}
      \ex{Haus = [\textsc{Bed}: \gruen<12->{\textbf{\textit{haus}}}, \textsc{Klasse}: \gruen<12->{\textbf{\textit{subst}}}, \textsc{Gen}: \gruen<12->{\textbf{\textit{neut}}}, \textsc{Kas}: \orongsch<13->{\textit{nom}}, \textsc{Num}: \orongsch<13->{\textit{sg}}]}
      \pause
      \ex{Haus-es = [\textsc{Bed}: \gruen<12->{\textbf{\textit{haus}}}, \textsc{Klasse}: \gruen<12->{\textbf{\textit{subst}}}, \textsc{Gen}: \gruen<12->{\textbf{\textit{neut}}}, \textsc{Kas}: \orongsch<13->{\textit{gen}}, \textsc{Num}: \orongsch<13->{\textit{sg}}]}
      \pause
      \ex{Häus-er = [\textsc{Bed}: \gruen<12->{\textbf{\textit{haus}}}, \textsc{Klasse}: \gruen<12->{\textbf{\textit{subst}}}, \textsc{Gen}: \gruen<12->{\textbf{\textit{neut}}}, \textsc{Kas}: \orongsch<13->{\textit{nom}}, \textsc{Num}: \orongsch<13->{\textit{pl}}]}
    \end{xlist}
  \end{exe}
  \Halbzeile
  \pause
  \begin{itemize}[<+->]
    \item bei einem lexikalischen Wort
      \begin{itemize}
        \item \gruen{statische Merkmale} wertestabil
        \item \orongsch{volatile Merkmale} werteverändernd im Paradigma
      \end{itemize}
  \end{itemize}
\end{frame}

\begin{frame}
  {Wiederholung | Wortbildung in Abgrenzung zur Flexion}
  \pause
  \begin{exe}
    \ex
    \begin{xlist}
      \ex trocken (Adj) → \alert{Trocken}\rot{-heit} (Subst)\label{ex:trocken}
      \ex Kauf (Subst), Rausch (Subst) → \alert{Kauf}\rot{-rausch} (Subst)\label{ex:kauf}
      \ex gehen (V) → \alert{be}\rot{-gehen} (V)\label{ex:gehen}
    \end{xlist}
    \pause
    \ex
    \begin{xlist}
      \ex \alert{lauf}\rot{-en} (1\slash 3 Pl Prs Ind) → \alert{lauf}\rot{-e} (1 Sg Prs Ind)\label{ex:lauf}
      \ex \alert{Münze} (Sg) → \alert{Münze}\rot{-n} (Pl)\label{ex:muenze}
    \end{xlist}
  \end{exe}
  \pause
  \Halbzeile
  \begin{itemize}[<+->]
    \item Wortbildung
      \begin{itemize}[<+->]
        \item statische Merkmale geändert | Wortklasse, Bedeutung \alert{(\ref{ex:trocken})}
        \item \ldots oder gelöscht | alles außer der Bedeutung des Erstglieds bei Komposition \alert{(\ref{ex:kauf})}
        \item \ldots oder umgebaut | Valenz von Verben beim Applikativ \alert{(\ref{ex:gehen})}
        \item \orongsch{produktives Erschaffen neuer lexikalischer Wörter}
      \end{itemize}
  \Halbzeile
    \item Flexion
      \begin{itemize}
        \item Änderung der Werte volatiler Merkmale \alert{(\ref{ex:lauf},\ref{ex:muenze})}
        \item \alert{oft Anpassung an syntaktischen Kontext}
      \end{itemize}
  \end{itemize}
\end{frame}



\begin{frame}
  {Wortbildung}
  \onslide<+->
  \begin{itemize}[<+->]
    \item virtuell unbegrenzter Wortschatz
      \Zeile
    \item gut durchschaubares und \alert{gut lernbares} System\\
      \grau{trotz vieler Probleme und Einschränkungen im Detail}
      \Zeile
    \item Funktionen der Wortbildung
      \begin{itemize}
        \item Komposition | \alert{komplexe Konzepte} (\textit{Lötzinnschmelztemperatur})
        \item Konversion | \alert{Reifizierung} (z.B.\ eines Ereignisses als Objekt, \textit{der Lauf})
        \item Derivation | \alert{Modifikation von Bedeutungen} (\textit{\alert{un}schön}),\\
          \alert{Bezug auf Teilaspekte von Konzepten} (z.\,B.\ Ereigniskonzepten, \textit{Fahr\alert{er}})
      \end{itemize}
      \Halbzeile
    \item Hauptproblem der Wortbildung\\
      \rot{Welche Bildungen sind wirklich produktiv?}
  \end{itemize}
\end{frame}


\begin{frame}
  {Wortbildung in der Bildungssprache}
  \pause
  \begin{itemize}[<+->]
    \item Wortbildung als einer der Kerne der Bildungssprache
    \item kann sowohl \alert{verdichten} als auch \alert{präzisieren}
    \Halbzeile
    \item komplexe Sachverhalte \alert{optimiert} formulieren
      \begin{itemize}[<+->]
        \item möglichst kurz
        \item maximal verständlich | Wortbildung hochgradig etabliert\\
          im Deutschen → problemlose Verarbeitung durch Hörer
      \end{itemize}
      \Halbzeile
    \item Aber \rot{das Unterrichten externer Funktionsregularitäten ist besonders\\
      im Fall der Wortbildung extrem schwierig.}
      \Halbzeile
      \begin{itemize}[<+->]
        \item "`Wenn du kommunikativ X erreichen willst, nimm eine Derivation auf \textit{-igkeit}."'
        \item So funktioniert das wohl eher nicht.
        \item Eine allgemeine souveräne \alert{Beherrschung des formalen Systems}\\
          führt zu einer globalen \alert{Optimierung der Schrift- und Bildungssprache}
      \end{itemize}
  \end{itemize}
\end{frame}


\section{Komposition}

\begin{frame}
  {Beispiele für Komposition}
  \onslide<+->
  Komposition | \alert{Stamm\Sub{1} + Stamm\Sub{2} → neuer Stamm\Sub{3}}
  \Halbzeile
  \onslide<+->
  \begin{exe}
    \ex
    \begin{xlist}
      \ex{Kopf.\alert{hörer}}
      \onslide<+->
      \ex{Laut.\alert{sprecher}}
      \onslide<+->
      \ex{Kraft.\alert{werk}}
      \onslide<+->
      \ex{Lehr.\alert{veranstaltung}}
      \onslide<+->
      \ex{Rot.\alert{eiche}}
      \onslide<+->
      \ex{Lauf.\alert{schuhe}}
      \onslide<+->
      \ex{Ess.\alert{besteck}}
      \onslide<+->
      \ex{Fertig.\alert{gericht}}
      \onslide<+->
      \ex{feuer.\alert{rot}}
    \end{xlist}
  \end{exe}
\end{frame}

\begin{frame}
  {Produktivität und Transparenz}
  \onslide<+->
  \begin{itemize}[<+->]
    \item \alert{alle} Beispiele auf der vorherigen Folie \alert{lexikalisiert}
      \begin{itemize}[<+->]
        \item vergleichsweise häufig vorkommende Wörter
        \item überwiegend spezifischere Bedeutung, als Bestandteile vermuten lassen
        \item aber Art der Bildung erkennbar
        \item zumindest für erwachsene Sprecher auch bewusst
      \end{itemize}
      \Halbzeile
    \item \alert{transparent} | Rekonstruierbarkeit der Bildung\\
      (auch bei abweichender Gesamtbedeutung)
      \Halbzeile
    \item \alert{produktiv gebildet} | Neubildung durch Sprecher\\
      in einer gegebenen Situation
    \item Produktivität ist \rot{graduell} aufzufassen!
    \item \orongsch{\textit{Buchbutter}} > \textit{Batterieschublade} > \textit{Laufschuhe} > \gruen{\textit{Hundstage}}
  \end{itemize}
\end{frame}

\begin{frame}[fragile,label=hierarchie]
  {Rekursion}
  \onslide<+->
  \onslide<+->
  \begin{center}
    \scalebox{0.7}{
      \begin{forest}
        [Bushaltestellenunterstandsreparatur
          [Bushaltestellenunterstand
            [Bushaltestelle
              [Bus]
              [Haltestelle
                [halten]
                [Stelle]
              ]
            ]
            [Unterstand
              [unter]
              [Stand]
            ]
          ]
          [Reparatur]
        ]
      \end{forest}
    }
  \end{center}
  \begin{itemize}[<+->]
    \item Wortbildung | immer \alert{binär}, also \alert{Wort+Wort} (nicht \rot{Wort+Wort+Wort} usw.)
      \Viertelzeile
    \item \alert{hierarchische Strukturbildung} durch wiederholte lineare Anfügung
      \Viertelzeile
    \item Rekursion allgemein | \alert{Eine Verknüpfung hat als Ergebnis\\
      eine Einheit, die wieder auf dieselbe Art verknüpft werden kann.}
    \item Rekursion in Linguistik | immer eingeschränkt, nicht "`endlos"'
  \end{itemize}
\end{frame}

\begin{frame}
  {Köpfe}
  \onslide<2->
  \begin{exe}
    \ex
    \begin{xlist}
      \ex \orongsch<19->{Laut}.\alert<10->{sprecher} \onslide<19->{\orongsch{(\textit{laut} verliert Wortklasse, \dots)}}
      \onslide<3->
      \ex \orongsch<20->{Kraft}.\alert<11->{werk} \onslide<20->{\orongsch{(\textit{Kraft} verliert Wortklasse, Genus, \dots)}}
      \onslide<4->
      \ex \orongsch<21->{Lauf}.\alert<12->{schuhe} \onslide<21->{\orongsch{(\textit{laufen} verliert Wortklasse? Genus? \dots)}}
      \onslide<5->
      \ex \orongsch<22->{Ess}.\alert<13->{besteck} \onslide<22->{\orongsch{(\textit{essen} verliert Wortklasse, \dots)}}
      \onslide<6->
      \ex \orongsch<23->{feuer}.\alert<14->{rot} \onslide<23->{\orongsch{(\textit{Feuer} verliert Wortklasse, \dots)}}
    \end{xlist}
  \end{exe}
  \onslide<7->
  \begin{itemize}
    \item \alert{Kopf}
      \begin{itemize}
          \onslide<8->
        \item steht immer rechts
          \onslide<9->
        \item bestimmt alle grammatischen Merkmale des Kompositums
      \end{itemize}
      \Halbzeile
      \onslide<15->
    \item \orongsch{Nicht-Kopf}
      \begin{itemize}
          \onslide<16->
        \item immer links
          \onslide<17->
        \item verliert alle grammatischen Merkmale
          \onslide<18->
        \item Bedeutung geht in Gesamtbedeutung ein
      \end{itemize}
  \end{itemize}
\end{frame}

\begin{frame}
  {Relevante Kompositionstypen | Determinativkomposita}
  \onslide<+->
  Determinativkomposita | \textit{Schulheft}, \textit{Regalbrett} usw.
  \Halbzeile
  \begin{itemize}[<+->]
    \item Kopf--Kern-Test
      \begin{itemize}[<+->]
        \item Ein Schulheft ist ein Heft. \gruen{\Ck}
        \item Ein Regalbrett ist ein Brett. \gruen{\Ck}
      \end{itemize}
    \item Nicht-Kopf--Kern-Test
      \begin{itemize}[<+->]
        \item Ein Schulheft ist eine Schule. \rot{\Fl}
        \item Ein Regalbrett ist ein Regal. \rot{\Fl}
      \end{itemize}
      \Halbzeile
    \item Rektionstest
      \begin{itemize}[<+->]
        \item Bei einem Schulheft heftet\slash verheftet\slash beheftet\ldots jemand eine Schule \rot{\Fl}
        \item Bei einem Regalbrett brettert\slash verbrettert\dots jemand ein Regal \rot{\Fl}
      \end{itemize}
  \end{itemize}
\end{frame}


\begin{frame}
  {Relevante Kompositionstypen | Rektionskomposita}
  \onslide<+->
  Objekt-Rektionskomposita | \textit{Hemdenwäsche}, \textit{Geldfälschung} usw.
  \Halbzeile
  \begin{itemize}[<+->]
    \item Kopf--Kern-Test
      \begin{itemize}[<+->]
        \item Eine Hemdenwäsche ist eine Wäsche. \rot{\Ck}
        \item Eine Geldfälschung ist eine Fälschung. \rot{\Ck}
      \end{itemize}
    \item Nicht-Kopf--Kern-Test
      \begin{itemize}[<+->]
        \item Eine Hemdenwäsche ist ein Hemd. \rot{\Fl}
        \item Eine Geldfälschung ist Geld. \rot{\Fl}
      \end{itemize}
      \Halbzeile
    \item Objekt-Rektionstest
      \begin{itemize}[<+->]
        \item Bei einer Hemdenwäsche werden Hemden gewaschen. \gruen{\Ck}
        \item Bei einer Geldfälschung wird Geld gefälscht. \gruen{\Ck}
      \end{itemize}
      \Halbzeile
    \item Kopf | oft mit \alert{-ung} usw. von einem Verb abgeleitet
    \item Nicht-Kopf zu Kopf wie \alert{Objekt} zu Verb
  \end{itemize}
\end{frame}


\begin{frame}
  {Relevante Kompositionstypen | Rektionskomposita}
  \onslide<+->
  Subjekt-Rektionskomposita | \textit{Hemdenwäscher}, \textit{Geldfälscher} usw.
  \Halbzeile
  \begin{itemize}[<+->]
    \item Kopf--Kern-Test
      \begin{itemize}[<+->]
        \item Ein Hemdenwäscher ist eine Wäsche. \gruen{\Ck}
        \item Ein Geldfälscher ist eine Fälschung. \gruen{\Ck}
      \end{itemize}
    \item Nicht-Kopf--Kern-Test
      \begin{itemize}[<+->]
        \item Ein Hemdenwäscher ist ein Hemd. \rot{\Fl}
        \item Ein Geldfälscher ist Geld. \rot{\Fl}
      \end{itemize}
      \Halbzeile
    \item Subjekt-Rektionstest
      \begin{itemize}[<+->]
        \item Ein Hemdenwäscher wäscht Hemden. \gruen{\Ck}
        \item Ein Geldfälscher fälscht Geld. \gruen{\Ck}
      \end{itemize}
      \Halbzeile
    \item Kopf | meistens mit \alert{\textit{-er}} von einem Verb abgeleitet
    \item Nicht-Kopf zu Kopf wie \alert{Subjekt} zu Verb
  \end{itemize}
\end{frame}

\begin{frame}
  {Kompositionsfugen bei Substantiv-Substantiv-Komposita}
  \onslide<+->
  \onslide<+->
  \begin{center}
    \scalebox{0.9}{
      \begin{tabular}{llrr}
        \toprule
        Fuge          & Beispiel                        & Komposita \% & Erstglieder \% \\
        \midrule                                                                                                    
        \alert{$\varnothing$} & \textit{Garten.tür}             & \alert{60.25}        & \alert{41.77}  \\ 
        \alert{-(e)s}         & \textit{Gelegenheit-s.dieb}     & \alert{23.69}        & \alert{45.74}  \\ 
        \alert{-n}            & \textit{Katze-n.pfote}          & \alert{10.38}        &  \rot{5.29}  \\ 
        \rot{-en}             & \textit{Frau-en.stimme}         &  \rot{3.02}          &  \rot{4.19}  \\ 
        \rot{*e}              & \textit{Kirsch.kuchen}          &  \rot{0.78}          &  \rot{0.20}  \\ 
        \rot{-e}              & \textit{Geschenk-e.laden}       &  \rot{0.71}          &  \rot{1.90}  \\ 
        \rot{-er}             & \textit{Kind-er.buch}           &  \rot{0.38}          &  \rot{0.07}  \\ 
        \rot{\char`~er}       & \textit{Büch-er.regal}          &  \rot{0.37}          &  \rot{0.11}  \\ 
        \rot{\char`~e}        & \textit{Händ-e.druck}           &  \rot{0.22}          &  \rot{0.63}  \\ 
        \rot{-ns}             & \textit{Name-ns.schutz}         &  \rot{0.13}          &  \rot{0.04}  \\ 
        \rot{\char`~}         & \textit{Mütter.zentrum}         &  \rot{0.05}          &  \rot{0.06}  \\ 
        \rot{-ens}            & \textit{Herz-ens.angelegenheit} &  \rot{0.03}          &  \rot{0.01}  \\ 
        \bottomrule
      \end{tabular}
    }\\
    \Halbzeile
    \grau{\tiny{(aus \citealt{SchaeferPankratz2018})}}
  \end{center}
\end{frame}

\begin{frame}
  {Steuerung der Fugen durch Erstglied}
  \onslide<+->
  \begin{itemize}[<+->]
    \item Wörter mit s-Plural (\textit{Kaffees}, \textit{Kameras}) \rot{niemals mit s-Fuge}
      \Halbzeile
    \item \alert{derivierte} Substantive (meist Abstrakta) (\textit{-heit}, \textit{-keit}, \textit{-tum}) \alert{prototypisch s-Fuge}
      \begin{itemize}[<+->]
        \item sehr viele Feminina, Fuge nicht paradigmatisch (= keine Flexionsform)
      \end{itemize}
      \Halbzeile
    \item starke\slash gemischte Maskulina | manchmal -(\textit{e})\textit{s}
      \begin{itemize}[<+->]
        \item Genitiv? Welche Funktion sollte ein Genitiv im Kompositum haben?
        \item Lassen sich die Komposita mit s-Fuge mit Genitiv umformulieren?
        \item \textit{Freundeskreis → \rot{*Kreis des Freundes}}
        \item \textit{Geschlechtsverkehr → \rot{*Verkehr des Geschlechts}}
        \item \textit{Berufstätigkeit → \rot{*Tätigkeit des Berufs}}
        \item \textit{Auslandsaufenthalt → \rot{*Aufenthalt des Auslands}}
      \end{itemize}
    \Halbzeile
  \item die s-Fugen an \alert{Feminina} sowieso nicht als Genitiv möglich
      \begin{itemize}
        \item \textit{Gelegenheitsdieb} → \rot{*\textit{Dieb der Gelegenheits}}
      \end{itemize}
  \end{itemize}
\end{frame}

\section{Zur nächsten Woche | Überblick}

\begin{frame}
  {Morphologie und Lexikon des Deutschen | Plan}
  \rot{Alle} angegebenen Kapitel\slash Abschnitte aus \rot{\citet{Schaefer2018b}} sind \rot{Klausurstoff}!\\
  \Halbzeile
  \begin{enumerate}
    \item Grammatik und Grammatik im Lehramt (Kapitel 1 und 3)
    \item Morphologie und Grundbegriffe (Kapitel 2, Kapitel 7 und Abschnitte 11.1--11.2)
    \item Wortklassen als Grundlage der Grammatik (Kapitel 6)
    \item Wortbildung | Komposition (Abschnitt 8.1)
    \item \rot{Wortbildung | Derivation und Konversion (Abschnitte 8.2--8.3)}
    \item Flexion | Nomina außer Adjektiven (Abschnitte 9.1--9.3)
    \item Flexion | Adjektive und Verben (Abschnitt 9.4 und Kapitel 10)
    \item Valenz (Abschnitte 2.3, 14.1 und 14.3)
    \item Verbtypen als Valenztypen (Abschnitte 14.4--14.5, 14.7--14.9) 
    \item Kernwortschatz und Fremdwort (vorwiegend Folien)
  \end{enumerate}
  \Halbzeile
  \centering 
  \url{https://langsci-press.org/catalog/book/224}
\end{frame}



  \let\subsection\section\let\section\woopsi
  
  \section{Wortbildung | Derivation und Konversion}
  \let\woopsi\section\let\section\subsection\let\subsection\subsubsection
  \section{Überblick}

\begin{frame}
  {Andere Wortbildungsmuster}
  \onslide<+->
  \begin{itemize}[<+->]
    \item \alert{Konversion} | Stamm\Sub{1} → Stamm\Sub{2} \\ 
      \textit{laufen} → (\textit{der}) \textit{Lauf}
      \Zeile
    \item \alert{Derivation} | Stamm\Sub{1} + Affix → Stamm\Sub{2}\\
      \textit{schön} → (\textit{die}) \textit{Schönheit}
      \Halbzeile
    \item Typische Anwendungsbereiche für \alert{Präfigierung} und \alert{Suffigierung} im Deutschen
  \end{itemize}
\end{frame}

\section{Konversion}

\begin{frame}
  {Beispiele für Konversion}
  \pause
  Konversion | \alert{Stamm\Sub{1} oder Wortform → Stamm\Sub{2}}
  \Halbzeile
  \pause
  \begin{exe}
    \ex[ ]{einkauf-en → Einkauf}
    \pause
    \ex[ ]{einkauf-en → Einkaufen}
    \pause
    \ex[ ]{ernst → Ernst}
    \pause
    \ex[ ]{schwarz → Schwarz}
    \pause
    \ex[ ]{gestrichen → gestrichen}
    \pause
    \ex[!]{schwarz → schwärzen}
    \pause
    \ex[!]{schieß-en → Schuss}
    \pause
    \ex[?]{stech-en → Stich}
  \end{exe}
\end{frame}

\begin{frame}
  {Stammkonversion}
  \pause
  \begin{itemize}[<+->]
    \item \rot{Stamm} → Stamm \alert{(mit Wortklassenwechsel)}
      \Halbzeile
    \item produktiv vor allem 
      \Halbzeile
      \begin{itemize}[<+->]
        \item \alert{Verbstammnominalisierung} | \textit{\alert{einkauf-en}} → \textit{der \alert{Einkauf}}\\
          \grau{Flexion wie ein normales maskulines Substantiv}
          \Halbzeile

        \item \alert{(Farb-)Adjektivnominalisierung} | \textit{das Kleid ist \alert{rot}} → \textit{das \alert{Rot} des Kleids}\\
          \grau{Flexion wie ein normales neutrales Substantiv}
          \Halbzeile

        \item \alert{metasprachliche Nominalisierung} | \textit{saturiert, \alert{aber} unzufrieden} → \textit{das ständige \alert{Aber}}\\
          \grau{Flexion wir ein normales neutrales Substantiv}

      \end{itemize}
  \end{itemize}
\end{frame}

\begin{frame}
  {Wortformenkonversion}
  \pause
  \begin{itemize}[<+->]
    \item \rot{flektierte Wortform} → Stamm \alert{oder} Wortform \alert{(mit Wortklassenwechsel)}
      \Halbzeile
    \item produktiv vor allem
      \Halbzeile
      \begin{itemize}[<+->]
        \item \alert{Infinitivnominalisierung} | \textit{Ich gehe \alert{einkaufen}.} → \textit{Das \alert{Einkaufen} macht Spaß.}\\
          \grau{Flexion wie ein normales neutrales Substantiv}
          \Halbzeile

        \item \alert{Adjektivnominalisierung} | \textit{Zwei \alert{doppelte} Brötchen bitte.} → \textit{Zwei \alert{Doppelte} bitte.}\\
          \grau{Flexion wie ein Adjektiv | daher Konversion Wortform → Wortform}
          \Halbzeile

        \item \alert{Adjektiadverbialisierung} | \textit{Das Auto ist \alert{schnell}.} → \textit{Das Auto fährt \alert{schnell}.}\\
          \grau{keine Flexion außer Komparativ}
      \end{itemize}
  \end{itemize}
\end{frame}

\section{Derivation}

\begin{frame}
  {Beispiele für Derivation}
  \pause
  Derivation | \alert{Stamm\Sub{1} + Affix → Stamm\Sub{2}}
  \Halbzeile
  \pause
  \begin{exe}
    \ex
    \begin{xlist}
      \ex Scherz → scherz\alert{:haft}
      \pause
      \ex brenn-en → brenn\alert{:bar}
      \pause
      \ex grün → grün\alert{:lich}
    \end{xlist}
    \pause
    \Halbzeile
    \ex
    \begin{xlist}
      \ex doof → Doof\alert{:heit}
      \pause
      \ex Fahrer → Fahrer\alert{:in}
      \pause
      \ex Kunde → Kund\alert{:schaft}
      \pause
      \ex Hund → Hünd\alert{:chen}
    \end{xlist}
    \pause
    \Halbzeile
    \ex
    \begin{xlist}
      \ex Schlange → schläng\alert{:el}-n
      \pause
      \ex Ruck → ruck\alert{:el}-n
    \end{xlist}
  \end{exe}
\end{frame}

\begin{frame}
  {Mit und ohne Wortklassenwechsel}
  \pause
  \begin{itemize}[<+->]
    \item \alert{mit} Wortklassenwechsel | Wortart ändert sich (\textit{Hand} → \textit{händ:isch})
    \item \alert{ohne} Wortklassenwechsel | Wortart bleibt gleich (\textit{rot} → röt:lich)
      \Zeile
    \item ohne Wortklassenwechsel | geänderte statische Merkmale?
      \begin{itemize}[<+->]
        \item in jedem Fall \alert{Bedeutung}
        \item prototypisch \textit{Dank → Un:dank}, \textit{bedeutend → un:bedeutend}
      \end{itemize}
  \end{itemize}
\end{frame}

\begin{frame}
  {Etwas schwierigere Fälle}
  \pause
  \begin{exe}
    \ex
    \begin{xlist}
      \ex{bebeispielen, bestuhlen, bevölkern}
      \ex{entvölkern, entgräten, entwanzen}
      \ex{verholzen, vernageln, verwanzen, verzinnen}
    \end{xlist}
    \pause
    \ex
    \begin{xlist}
      \ex{ergrauen, ermüden, erneuern}
      \ex{befreien, beengen, begrünen}
    \end{xlist}
  \end{exe}
  \pause
  \Halbzeile
  \begin{itemize}[<+->]
    \item entweder \alert{Stammkonversion + Präfigierung}
      \begin{itemize}[<+->]
        \item \textit{grau} (Adjektiv)
        \item[→] \textit{grau-en} (Stammkonversion zum Verb)
        \item[→] \textit{er:grau-en} (Präfigierung ohne Wortklassenwechsel)
      \end{itemize}
    \item oder \alert{wortartenverändernde Präfixe}
      \begin{itemize}[<+->]
        \item \textit{grau} (Adjektiv)
        \item[→] \textit{er:grau-en} (Präfigierung mit Wortklassenwechsel zum Verb)
      \end{itemize}
  \end{itemize}
\end{frame}

\begin{frame}
  {In welchem Bereich wird vor allem suffigiert?}
  \pause
  \begin{center}
    \scalebox{0.5}{
      \begin{tabular}{llll}
        \toprule
        \textbf{Ausgangsklasse} & \textbf{Substantiv-Affix} & \textbf{Adjektiv-Affix} & \textbf{Verb-Affix} \\
       \midrule
       \multirow{8}{*}{\textbf{Substantiv}} & :chen & :haft & \\
       & \textit{Äst:chen} & \textit{schreck:haft} & \\
       \cmidrule{2-4}
       
       & :in & :ig & \\
       & \textit{Arbeiter:in} & \textit{fisch:ig} & \\
       \cmidrule{2-4}
       
       & :ler & :isch & \\
       & \textit{Volkskund:ler} & \textit{händ:isch} & \\
       \cmidrule{2-4}
       
       & :schaft & :lich & \\
       & \textit{Wissen:schaft} & \textit{häus:lich} & \\
       
       \midrule
       \multirow{6}{*}{\textbf{Adjektiv}} & :heit & :lich & \\
       & \textit{Schön:heit} & \textit{röt:lich} & \\
       \cmidrule{2-4}
       
       & :keit && \\
       & \textit{Heiter:keit} & & \\
       \cmidrule{2-4}
       
       & :igkeit && \\
       & \textit{Neu:igkeit} & & \\
       
       \midrule
       \multirow{6}{*}{\textbf{Verb}} & :er & :bar & :el \\
       & \textit{Arbeit:er} & \textit{bieg:bar} & \textit{kreis:el-n} \\
       \cmidrule{2-4}
       
       & :erei && \\
       & \textit{Arbeit:erei} & & \\
       \cmidrule{2-4}
       
       & :ung && \\
       & \textit{Les:ung} & & \\
       
       \bottomrule
      \end{tabular}
    }\\
    \Zeile
    \pause
    \alert{\large \ldots\ zum Nomen, vor allem zum Substantiv.}\\
  \end{center}
\end{frame}

\begin{frame}
  {In welchem Bereich wird prototypisch präfigiert?}
  \onslide<+->
  \onslide<+->
  \alert{Verbpräfixe} | Verb → Verb\\
  \Halbzeile
  \begin{itemize}[<+->]
    \item kauf-en → ver:kauf-en
    \item hol-en → über:hol-en
    \item stell-en → unter:stell-en
  \end{itemize}
  \Zeile
  \onslide<+->
  \alert{Verpartikeln} | Verb → Verb\\
  \Halbzeile
  \begin{itemize}[<+->]
    \item leg-en → um=leg-en
    \item geh-en → entlang=geh-en
    \item trenn-en → ab=trenn-en
  \end{itemize}
\end{frame}

\begin{frame}
  {Unterschiede zwischen Verbpräfixen und Verbpartikeln}
  \onslide<+->
  \begin{itemize}[<+->]
    \item Trennbarkeit
      \begin{itemize}[<+->]
        \item \ldots\ weil wir es \alert{verkaufen} | Wir \alert{verkaufen} es.
        \item \ldots\ weil wir es \rot{abtrennen} | Wir \rot{trennen} es \rot{ab}.
      \end{itemize}
      \Zeile
    \item Partizipbildung
      \begin{itemize}[<+->]
        \item \alert{ver:kauf}-en → \alert{ver:kauf}-t
        \item \rot{ab=trenn}-en → \rot{ab=ge-trenn}-t
      \end{itemize}
  \end{itemize}
  \onslide<+->

  \Zeile
  \centering 
  Wir kommen auf die Formen später nochmal kurz zurück.
\end{frame}

\begin{frame}
  {Notationskonvention im Buch}
  \pause
  \begin{itemize}[<+->]
    \item \alert{Flexion (und Fuge)} mit Bindestrich: \textit{Tisch-es}, \textit{Fäng-e}
    \item \alert{Komposition} mit Punkt | \textit{Tasche-n.tuch}
    \item \alert{Derivation} mit Doppelpunkt | \textit{Läuf:er}, \textit{ver:blühen}
    \item \alert{Verbpartikeln} mit Gleichheitszeichen | \textit{ab=trenn-en}, \textit{auf=schieb-en}
    \Halbzeile
  \item Markierung für angeblich umlautauslösende Affixe aus EGBD3 \rot{entfällt}
      \begin{itemize}[<+->]
        \item \grau{\char`~ bei Flexion (Plural \textit{\char`~er}, \textit{Männ-er})}
        \item \grau{\~: bei Derivation (wie bei \textit{\~:lich}, töd:lich)}
      \end{itemize}
    \Halbzeile
  \item spezifisch EGBD, keine allgemeine Konvention
  \item \rot{Die Notation muss für die Klausur sicher beherrscht werden!}
  \end{itemize}
\end{frame}


\ifdefined\TITLE
  \section{Zur nächsten Woche | Überblick}

  \begin{frame}
    {Morphologie und Lexikon des Deutschen | Plan}
    \rot{Alle} angegebenen Kapitel\slash Abschnitte aus \rot{\citet{Schaefer2018b}} sind \rot{Klausurstoff}!\\
    \Halbzeile
    \begin{enumerate}
      \item Grammatik und Grammatik im Lehramt (Kapitel 1 und 3)
      \item Morphologie und Grundbegriffe (Kapitel 2, Kapitel 7 und Abschnitte 11.1--11.2)
      \item Wortklassen als Grundlage der Grammatik (Kapitel 6)
      \item Wortbildung | Komposition (Abschnitt 8.1)
      \item Wortbildung | Derivation und Konversion (Abschnitte 8.2--8.3)
      \item \rot{Flexion | Nomina außer Adjektiven (Abschnitte 9.1--9.3)}
      \item Flexion | Adjektive und Verben (Abschnitt 9.4 und Kapitel 10)
      \item Valenz (Abschnitte 2.3, 14.1 und 14.3)
      \item Verbtypen als Valenztypen (Abschnitte 14.4--14.5, 14.7--14.9) 
      \item Kernwortschatz und Fremdwort (vorwiegend Folien)
    \end{enumerate}
    \Halbzeile
    \centering 
    \url{https://langsci-press.org/catalog/book/224}
  \end{frame}
\fi

  \let\subsection\section\let\section\woopsi
  
  \section{Flexion | Nomina außer Adjektiven}
  \let\woopsi\section\let\section\subsection\let\subsection\subsubsection
  \section{Überblick}

\begin{frame}
  {Flexion | Nomina}
  \onslide<+->
  \begin{itemize}[<+->]
    \item Funktion in der Nominalflexion
    \item Flexion(sklassen) der Substantive
    \item Flexion der Pronomina und Artikel
  \end{itemize}
\end{frame}

\begin{frame}
  {Flexion im Lehramtsstudium}
  \pause
  \begin{itemize}[<+->]
    \item \alert{Wir beherrschen doch alle Formen!}
      \Halbzeile
    \item Funktion der Flexionskategorien
      \begin{itemize}
        \item semantisch\slash pragmatisch
        \item \alert{systemintern} als Hilfe zu \alert{Rekonstruktion der Satzstruktur}
      \end{itemize}
      \Halbzeile
    \item Flexion im Deutschen ein ideales und gut durchschaubares Beispiel\\
      für die klassische \alert{reduktionistische} Methode der Linguistik\\
      (= Analyse der Sprache als \alert{System})
      \Halbzeile
    \item \alert{Können} vs.\ \rot{Erklären}
    \item Reaktion auf Erwerbsschwierigkeiten (L1)
    \item inkl.\ Schwierigkeiten wegen nicht-deutscher Erstsprache (L2)
  \end{itemize}
\end{frame}


\section{Funktion}

\begin{frame}
  {Was heißt Funktion?}
  \pause
  Rückgriff auf Kapitel 3:
  \pause
  \Halbzeile
  \begin{itemize}[<+->]
    \item \alert{externe} Funktion | kommunikativ, pragmatisch, textuell, kulturell, \dots
    \item \alert{interne} Funktion | innerhalb der Grammatik Relationen kennzeichnend,
      Rekonstruktion der Struktur ermöglichend, Schnittstelle zur Semantik | \rot{Kompositionalität}
    \item nicht immer trennbar
      \Halbzeile
    \item Paradebeispiel für interne Funktion | \alert{Kasussystem}
  \end{itemize}
\end{frame}

\begin{frame}
  {Numerus}
  \pause
  \begin{exe}
    \ex
    \begin{xlist}
      \ex[ ]{Die Trainerin beobachtet [einen guten Wettkampf].}
      \pause
      \ex[*]{Die Trainerin beobachtet [einen guten \rot{Wettkämpfe}].}
    \end{xlist}
    \pause
    \ex
    \begin{xlist}
      \ex[ ]{Die Trainerin beobachtet [einige gute Wettkämpfe].}
      \pause
      \ex[*]{Die Trainerin beobachtet [einige gute \rot{Wettkampf}].}
    \end{xlist}
  \end{exe}
  \pause
  \Halbzeile
  \begin{itemize}[<+->]
    \item \alert{Anzahl von Objekten ("`Gegenständen"')} | konzeptuell beim Subst motiviert
    \item notwendigerweise volatiles Merkmal beim Subst
    \item Pluraliatantum wie \textit{Ferien} oder Singulariatantum wie \textit{Gesundheit}
  \end{itemize}
\end{frame}

\begin{frame}
  {Kasus}
  \pause
  Was ist Kasus? Haben die Kasus an sich eine Bedeutung?
  \Halbzeile
  \pause
  \begin{exe}
    \ex
    \begin{xlist}
      \ex{Wir sehen \rot{den Rasen}.}
      \pause
      \ex{Wir begehen \rot{den Rasen}.}
      \pause
      \ex{Wir säen \rot{den Rasen}.}
      \pause
      \ex{Wir fürchten \rot{uns}.}
    \end{xlist}
    \pause
    \ex
    \begin{xlist}
      \ex \rot{Nächsten März} fahre ich zum Bergwandern in die Tatra.
      \ex Es waren \rot{den ganzen Tag} Menschen zum Gipfel unterwegs.
    \end{xlist}
    \pause
    \ex
    \begin{xlist}
      \ex{Sarah backt \rot{ihrer Freundin} einen Marmorkuchen.}
      \pause
      \ex{Wir kaufen \rot{dir} ein Kilo Rohrzucker.}
      \pause
      \ex{Die Mannschaft spielt \rot{mir} zu drucklos.}
      \pause
      \ex{Der Marmorkuchen schmeckt \rot{den Freundinnen} gut.}
    \end{xlist}
  \end{exe}
\end{frame}


\begin{frame}
  {Kasus | Eigenschaften}
  \pause
  \centering

  \Large
  Kasus stellt \alert{Relationen zwischen\\
  den kasustragenden Nomina und anderen Wörtern}\\
  (Verben, Präpositionen, anderen Nomina) her.\\
\end{frame}

\begin{frame}
  {Person | Deixis}
  \pause
  Was ist die grammatische Person?

  \Halbzeile
  \pause
  \begin{exe}
    \ex
    \begin{xlist}
      \ex{\alert{Ich} unterstütze den FCR Duisburg.}
      \pause
      \ex{\alert{Ihr} unterstützt den FCR Duisburg.}
      \pause
      \ex{\alert{Sie/Diese/Jene/Eine/Man\ldots} unterstützt den FCR Duisburg.}
      \pause
      \ex{\alert{Sie/Diese/Jene/Einige/\ldots} unterstützen den FCR Duisburg.}
    \end{xlist}
  \end{exe}
  \pause
  \Halbzeile
  \begin{itemize}[<+->]
    \item prototypisch beim \alert{Pronomen} funktional motiviert
    \item Substantive | statisch dritte Person
      \Halbzeile
    \item hier | \rot{deiktische Pronomina}
      \begin{itemize}[<+->]
        \item in einer Situation verweisend
        \item nur relativ zu einer Situation interpretierbar
      \end{itemize} 
  \end{itemize}
\end{frame}

\begin{frame}
  {Person | Anaphorik}
  \pause
  \begin{exe}
    \ex \alert{Sarah$_{\textnormal{1}}$} backt \rot{[ihrer Freundin]$_{\textnormal{2}}$} \gruen{[einen Kuchen]$_{\textnormal{3}}$}.\\
      \alert{Sie$_{\textnormal{1}}$} verwendet nur fair gehandelten unraffinierten Rohrzucker.
    \pause
      \ex \alert{Sarah$_{\textnormal{1}}$} backt \rot{[ihrer Freundin]$_{\textnormal{2}}$} \gruen{[einen Kuchen]$_{\textnormal{3}}$}.\\
      \gruen{Er$_{\textnormal{3}}$} besteht nur aus fair gehandelten Zutaten.
    \pause
      \ex \alert{Sarah$_{\textnormal{1}}$} backt \rot{[ihrer Freundin]$_{\textnormal{2}}$} \gruen{[einen Kuchen]$_{\textnormal{3}}$}.\\
      \rot{Sie$_{\textnormal{2}}$} soll \gruen{ihn$_{\textnormal{3}}$} zum Geburtstag geschenkt bekommen.
  \end{exe}
  \Halbzeile
  \pause
  \begin{itemize}[<+->]
    \item anaphorische Pronomina
    \item Rückverweis im Text, Satz, Diskurs
    \item gleiche Indizes zeigen Bedeutungsidentität (Korreferenz)
    \item \rot{die Indizes setzen wir, um eine bestimmte Interpretation zu markieren.}\\
      Diese Interpretation kann möglich oder unmöglich sein.
  \end{itemize}
\end{frame}

\begin{frame}
  {Genus, Geschlecht, Gender?}
  \pause
  \begin{exe}
    \ex \label{ex:genus039}
    \begin{xlist}
      \ex \alert{Die Petunie} ist \orongsch{eine Blume}.
      \ex \rot{Der Enzian} ist \orongsch{eine Blume}.
      \ex \gruen{Das Veilchen} ist \orongsch{eine Blume}.
    \end{xlist}
  \end{exe}
  \pause
  \Halbzeile
  \begin{itemize}[<+->]
    \item reine Subklassenbildung beim Substantiv
    \item nicht in Geschlecht oder Gender motiviert
    \item teilweise Korrespondenz von maskulin und männlich\\
      sowie feminin und weiblich bei Menschen bzw.\ Lebewesen
    \item \rot{aber}
      \begin{itemize}[<+->]
        \item der Mensch
        \item die Person
        \item das (menschliche) Wesen
        \item das Individuum
      \end{itemize}
  \end{itemize}
\end{frame}

\section{Nominalflexion}

\subsection{Substantive}

\begin{frame}
  {Substantive | Kasus und Numerus}
  Das traditionelle Chaos der Flexionstypen mit Kasus-Numerus-Formen\ldots\\
  \Zeile
  \pause
  \Zeile
  \resizebox{\textwidth}{!}{
    \begin{tabular}{llp{0mm}lp{2mm}llp{1mm}lp{2mm}llp{2mm}l}
      \toprule
      \multicolumn{2}{c}{} && \multicolumn{1}{l}{\textbf{Maskulinum}} && \multicolumn{4}{l}{\textbf{Maskulinum und Neutrum}} && \multicolumn{2}{l}{\textbf{Femininum}} && \multicolumn{1}{l}{\textbf{s-Flexion}} \\
      \multicolumn{2}{c}{} && \multicolumn{1}{l}{\textbf{schwach (S1)}} && \multicolumn{2}{l}{\textbf{stark (S2)}} && \multicolumn{1}{l}{\textbf{gemischt (S3)}} && \multicolumn{2}{l}{\textbf{(S4)}} && \multicolumn{1}{l}{\textbf{(S5)}} \\
      \midrule
      \multirow{4}{*}{\textbf{Sg}} & \textbf{Nom} && Mensch && Stuhl & Haus && Staat && Frau & \multicolumn{1}{l}{Sau} && Auto \\
      & \textbf{Akk} && Mensch-en && Stuhl & Haus && Staat && Frau & \multicolumn{1}{l}{Sau} && Auto \\
      & \textbf{Dat} && Mensch-en && Stuhl & Haus && Staat && Frau & \multicolumn{1}{l}{Sau} && Auto \\
      & \textbf{Gen} && Mensch-en && Stuhl-es & Haus-es && Staat-(e)s && Frau & \multicolumn{1}{l}{Sau} && Auto-s \\
      \midrule
      \multirow{4}{*}{\textbf{Pl}} & \textbf{Nom} && Mensch-en && Stühl-e & Häus-er && Staat-en && Frau-en & \multicolumn{1}{l}{Säu-e} && Auto-s \\
      & \textbf{Akk} && Mensch-en && Stühl-e & Häus-er && Staat-en && Frau-en & \multicolumn{1}{l}{Säu-e} && Auto-s \\
      & \textbf{Dat} && Mensch-en && Stühl-en & Häus-ern && Staat-en && Frau-en & \multicolumn{1}{l}{Säu-en} && Auto-s \\
      & \textbf{Gen} && Mensch-en && Stühl-e & Häus-er && Staat-en && Frau-en & \multicolumn{1}{l}{Säu-e} && Auto-s \\
      \bottomrule
    \end{tabular}
  }
\end{frame}

\begin{frame}
  {Das traditionelle Chaos als "`System"'}
  \pause
  Das geht irgendwie nach Genus und Pluralbildung, aber nicht nur\ldots\\
  \pause
  \Zeile
  \begin{center}
    \resizebox{0.8\textwidth}{!}{
    \begin{tikzpicture}[every text node part/.style={align=center}]
      \node (MaskN)    at (2,4)   {\textbf{Maskulin}};
      \node (NeutN)    at (4,4)   {\textbf{Neutral}};
      \node (FemN)     at (8.5,4) {\textbf{Feminin}};

      \node (schwachN) at (0,2)   {\textbf{schwach}\\\textit{Mensch-en}};
      \node (starkN)   at (2,2)   {\textbf{stark}};
      \node (gemistN)  at (4,2)   {\textbf{gemischt}\\\textit{Staat-en}};
      \node (sFlexN)   at (6.5,2) {\textbf{s-Flexion}\\\textit{Auto-s}\\\textit{Papaya-s}};
      \node (s4N)      at (8.5,2) {\textbf{"`S4"'}};

      \node (EPlu)     at (0.5,0) {Plural\\\textit{\char`~e}\\\textit{Stühl-e}};
      \node (ePlu)     at (2,0)   {Plural\\\textit{-e}\\\textit{Gurt-e}};
      \node (erPlu)    at (3.5,0) {Plural\\\textit{\char`~er}\\Lämm-er};

      \node (enPlu)    at (7.5,0) {Plural\\\textit{-en}\\\textit{Frau-en}};
      \node (EnPlu)    at (9.5,0) {Plural\\\textit{\char`~e}\\\textit{Säu-e}};

      \draw (MaskN.south)  -- (schwachN.north);
      \draw (MaskN.south)  -- (starkN.north);
      \draw (MaskN.south)  -- (gemistN.north);
      \draw (MaskN.south)  -- (sFlexN.north);

      \draw (NeutN.south)  -- (starkN.north);
      \draw (NeutN.south)  -- (gemistN.north);
      \draw (NeutN.south)  -- (sFlexN.north);

      \draw (FemN.south)   -- (s4N.north);
      \draw (FemN.south)   -- (sFlexN.north);

      \draw [dashed] (starkN.south) -- (EPlu.north);
      \draw [dashed] (starkN.south) -- (ePlu.north);
      \draw [dashed] (starkN.south) -- (erPlu.north);

      \draw [dashed] (s4N.south)    -- (enPlu.north);
      \draw [dashed] (s4N.south)    -- (EnPlu.north);
    \end{tikzpicture}
  }      
  \end{center}
\end{frame}

\begin{frame}
  {Aber das war noch nicht alles | mit und ohne Schwa}
  \pause
  Es gibt außerdem noch Varianten der Affixe \rot{ohne Schwa}:\\
  \Zeile
  \pause
  \begin{center}
    \resizebox{\textwidth}{!}{
      \begin{tabular}{llp{1mm}llp{1mm}llp{1mm}ll}
        \toprule
        \multicolumn{2}{l}{\textbf{schwach}} && \multicolumn{2}{l}{\textbf{gemischt}} && \multicolumn{2}{l}{\textbf{Fem S4a}} && \multicolumn{2}{l}{\textbf{Fem S4b}}\\
        \textbf{voll} & \textbf{reduziert} && \textbf{voll} & \textbf{reduziert} && \textbf{voll} & \textbf{reduziert} && \textbf{voll} & \textbf{reduziert} \\
        \midrule
        Mensch\alert{-en} & Löwe\rot{-n} && Staat\alert{-en} & Ende\rot{-n} && Frau\alert{-en} & Nudel\rot{-n} && Säu\alert{-e} & Mütter\rot{-$\emptyset$} \\
        \bottomrule
      \end{tabular}
    }
  \end{center}
\end{frame}


\begin{frame}
  {Der Ansatz in EGBD}
  \large \alert{Sauber trennen zwischen Numerus- und Kasusmarkierung!}\\
  \pause
  \Halbzeile
  \normalsize
  Erstens | Der Plural ist nahezu immer \orongsch{stärker markiert} als\\
  oder mindestens \gruen{gleich stark markiert} wie der Singular.\\
  → Pluralbildung ist die \alert{dominante Flexionseigenschaft}.
  \pause
  \Zeile
  \begin{center}
    \scalebox{0.8}{
      \begin{tabular}{llll}
        \toprule
        \textbf{Klasse} & \textbf{Kasus} & \textbf{Sg} & \textbf{Pl} \\
        \midrule
        S1 & Nom & (der) Mensch & (die) Mensch\orongsch{-en} \\
        S2a & Gen & (des) Stuhl\rot{-es} & (der) Stühl\rot{-e} \\
        S2b & Akk & (den) Gurt & (die) Gurt\orongsch{-e} \\
        S2c & Dat & (dem) Haus & (den) Häus\orongsch{-ern} \\
        S3 & Akk & (den) Staat & (die) Staat\orongsch{-en} \\
        S4a & Nom & (die) Frau & (die) Frau\orongsch{-en} \\
        S4b & Nom & (die) Sau & (die) Säu\orongsch{-e} \\
        \midrule
        S1 & Akk & (den) Mensch\gruen{-en} & (die) Mensch\gruen{-en} \\
        S5 & Gen & (des) Auto\gruen{-s} & (der) Auto\gruen{-s} \\
        \bottomrule
      \end{tabular}
    }
  \end{center}
\end{frame}


\begin{frame}
  {Pluralbildungen}
  \pause
  Isolierung der Plural-Affixe.\\
  \Zeile
  \pause
  \begin{center}
    \resizebox{\textwidth}{!}{
      \begin{tabular}{llp{0mm}lp{2mm}llp{1mm}lp{2mm}llp{2mm}l}
        \toprule
        \multicolumn{2}{c}{} && \multicolumn{1}{l}{\textbf{Maskulinum}} && \multicolumn{4}{l}{\textbf{Maskulinum und Neutrum}} && \multicolumn{2}{l}{\textbf{Femininum}} && \multicolumn{1}{l}{\textbf{s-Flexion}} \\
        \multicolumn{2}{c}{} && \multicolumn{1}{l}{\textbf{schwach (S1)}} && \multicolumn{2}{l}{\textbf{stark (S2)}} && \multicolumn{1}{l}{\textbf{gemischt (S3)}} && \multicolumn{2}{l}{\textbf{(S4)}} && \multicolumn{1}{l}{\textbf{(S5)}} \\
        \midrule
        \multirow{4}{*}{\textbf{Sg}} & \textbf{Nom} && Mensch && Stuhl & Haus && Staat && Frau & \multicolumn{1}{l}{Sau} && Auto \\
        & \textbf{Akk} && Mensch\rot<5->{-en} && Stuhl & Haus && Staat && Frau & \multicolumn{1}{l}{Sau} && Auto \\
        & \textbf{Dat} && Mensch\rot<5->{-en} && Stuhl(-e) & Haus(-e) && Staat(-e) && Frau & \multicolumn{1}{l}{Sau} && Auto \\
        & \textbf{Gen} && Mensch\rot<5->{-en} && Stuhl-(e)s & Haus-(e)s && Staat-(e)s && Frau & \multicolumn{1}{l}{Sau} && Auto\orongsch<6->{-s} \\
        \midrule
        \multirow{4}{*}{\textbf{Pl}} & \textbf{Nom} && Mensch\rot<5->{\alert<4>{-en}} && Stühl\alert<4->{-e} & Häus\alert<4->{-er}   && Staat\alert<4->{-en} && Frau\alert<4->{-en} & \multicolumn{1}{l}{Säu\alert<4->{-e}}   && Auto\alert<4->{-s} \\
        & \textbf{Akk} && Mensch\rot<5->{\alert<4>{-en}} && Stühl\alert<4->{-e}                             & Häus\alert<4->{-er}   && Staat\alert<4->{-en} && Frau\alert<4->{-en} & \multicolumn{1}{l}{Säu\alert<4->{-e}}   && Auto\alert<4->{-s} \\
        & \textbf{Dat} && Mensch\rot<5->{\alert<4>{-en}} && Stühl\alert<4->{-e}-n                           & Häus\alert<4->{-er}-n && Staat\alert<4->{-en} && Frau\alert<4->{-en} & \multicolumn{1}{l}{Säu\alert<4->{-e}-n} && Auto\alert<4->{-s} \\
        & \textbf{Gen} && Mensch\rot<5->{\alert<4>{-en}} && Stühl\alert<4->{-e}                             & Häus\alert<4->{-er}   && Staat\alert<4->{-en} && Frau\alert<4->{-en} & \multicolumn{1}{l}{Säu\alert<4->{-e}}   && Auto\alert<4->{-s} \\
        \bottomrule
      \end{tabular}
    }
  \end{center}
  \pause
  \pause
  \pause
  \pause
  \begin{itemize}[<+->]
    \item schwache Maskulina | \alert{Sonderklasse mit niedriger Typfrequenz}
    \item Genitiv Singular bei s-Flexion | \rot{nicht} rausnehmen (s.~unten)
    \item was an Affixen übrig bleibt | \alert{Kasus}
  \end{itemize}
\end{frame}


\begin{frame}
  {Kasusmarkierungen}
  \pause
  Was bleibt denn übrig für Kasus?
  \Zeile
  \pause
  \begin{center}
    \resizebox{\textwidth}{!}{
      \begin{tabular}{llp{0mm}llp{1mm}lp{2mm}llp{2mm}l}
        \toprule
        \multicolumn{2}{c}{} && \multicolumn{4}{l}{\textbf{Maskulinum und Neutrum}} && \multicolumn{2}{l}{\textbf{Femininum}} && \multicolumn{1}{l}{\textbf{s-Flexion}} \\
        \multicolumn{2}{c}{} && \multicolumn{2}{l}{\textbf{stark (S2)}} && \multicolumn{1}{l}{\textbf{gemischt (S3)}} && \multicolumn{2}{l}{\textbf{(S4)}} && \multicolumn{1}{l}{\textbf{(S5)}} \\
        \midrule
        \multirow{4}{*}{\textbf{Sg}} & \textbf{Nom} && Stuhl & Haus && Staat && Frau & \multicolumn{1}{l}{Sau} && Auto \\
        & \textbf{Akk} && Stuhl & Haus && Staat && Frau & \multicolumn{1}{l}{Sau} && Auto \\
        & \textbf{Dat} && Stuhl & Haus && Staat && Frau & \multicolumn{1}{l}{Sau} && Auto \\
        & \textbf{Gen} && Stuhl\gruen{-es} & Haus\gruen{-(e)s} && Staat\gruen{-(e)s} && Frau\onslide<5->{\orongsch{*-s}} & \multicolumn{1}{l}{Sau\onslide<5->{\orongsch{*-s}}} && Auto\gruen{-s} \\
        \midrule
        \multirow{4}{*}{\textbf{Pl}} & \textbf{Nom} && Stühl-e & Häus-er && Staat-en && Frau-en & \multicolumn{1}{l}{Säu-e} && Auto-s \\
        & \textbf{Akk} && Stühl-e & Häus-er && Staat-en && Frau-en & \multicolumn{1}{l}{Säu-e} && Auto-s \\
        & \textbf{Dat} && Stühl-e\alert{-n} & Häus-er\alert{-n} && Staat-en\onslide<4->{\rot{*-n}} && Frau-en\onslide<4->{\rot{*-n}} & \multicolumn{1}{l}{Säu-e\alert{-n}} && Auto-s\onslide<4->{\rot{*-n}} \\
        & \textbf{Gen} && Stühl-e & Häus-er && Staat-en && Frau-en & \multicolumn{1}{l}{Säu-e} && Auto-s \\
        \bottomrule
      \end{tabular}
    }
  \end{center}
\end{frame}

\begin{frame}
  {Regularitäten der Substantivflexion}
  \pause
  \begin{itemize}[<+->]
%    \item Die schwachen Maskulina sind die einzige "`Sonderklasse"'.
    \item \alert{Die Pluralklasse determiniert das Flexionsverhalten.}
    \item \alert{Und das Genus determiniert teilweise Pluralklasse.}
      \begin{itemize}[<+->]
        \item \alert{Mask prototypisch \textit{\char`~e} oder \textit{-e}}
        \item \alert{Fem prototypisch \textit{-en}}
%        \item Kleinstklasse | Mask und Neut \textit{-er}
        \item Subst endet mit Vollkvokal (\textit{Kanu-s}) oder Kurzwort (\textit{LKWs}) | s-Plural
      \end{itemize}
    \Halbzeile
  \item \alert{Maskulin Genitiv Singular | \textit{-(e)s}} \rot{außer phonotaktisch unmöglich}
    \item \alert{alle Genera Dativ Plural | \textit{-(e)n}} \rot{außer phonotaktisch unmöglich}
    \item Genitiv-Regularität (Mask/Neut) auch bei s-Substantiven
      \begin{itemize}[<+->]
        \item \textit{des Kanu-s}
        \item \rot{\textit{*der Papaya-s}} (Sg)
      \end{itemize}
  \Halbzeile
    \item keine Sequenzen von Schwa-Silben | \textit{die Tüte-n} statt \rot{\textit{*Tüte-en}}
    \item \ldots oder \textit{die Bolzen} statt \rot{\textit{*Bolzen-e}} oder \rot{\textit{*Bolzen-en}}
    \item keine /nn/-Sequenzen | \textit{die Bolzen} statt \rot{\textit{Bolzen-n}}
  \end{itemize}
\end{frame}

\begin{frame}
  {Grafische Darstellung des Klassensystems}
  \pause
  \begin{center}
    \resizebox{0.7\textwidth}{!}{
      \begin{forest}
        [Substantive, calign=last, l sep+=2em
          [\textit{en}-Maskulina]
          [normale Flexion{,}\\differenziert\\nur nach\\Pluralbildung, l sep+=2em
            [\textit{\char`~er}\\nur Maskulina\\und Neutra\\(Kleinstklasse)]
            [\textit{\char`~e}\slash\textit{-e}\\Protoyp\\der \textbf{Maskulina}\\und \textbf{Neutra}]
            [\textit{-en}\\Prototyp\\der \textbf{Feminina}]
            [\textit{-s}\\lexikalisch oder\\phonotaktisch\\motiviert]
          ]
        ]
      \end{forest}
    }
  \end{center}
\end{frame}

\subsection{Pronomina und Artikel}

\begin{frame}
  {Pronomina in Pronominalfunktion}
  \pause
  \begin{exe}
    \ex
    \begin{xlist}
      \ex{\orongsch{[Der Autor des Textes]} schreibt \gruen{[Sätze, die niemand zuvor geschrieben hat]}.}
      \ex{\orongsch{[Dieser]} schreibt \gruen{[etwas]}.}
    \end{xlist}
  \end{exe}
  \pause
  \Zeile
  In dieser Funktion stehen Pronomina \alert{anstelle einer vollen Nominalphrase}.
\end{frame}

\begin{frame}
  {Personalpronomina}
  \onslide<+->
  \onslide<+->
  Uninteressant unsystematisch, wenn auch primäre Träger der Personmarkierung\ldots\\
  \Zeile
  \Zeile
  \onslide<+->
  \centering 
  \scalebox{0.8}{\begin{tabular}[h]{lllllll}
    \toprule
    \textbf{Numerus}             & \textbf{Kasus} & \multicolumn{5}{c}{\textbf{Person\slash Genus}} \\\cline{3-7}
                                 &                &  \textbf{1} & \textbf{2} & \multicolumn{3}{c}{\textbf{3}} \\\cline{5-7}
                                 &                &        &        & \textbf{Mask} & \textbf{Neut} & \textbf{Fem} \\
    \midrule
    \multirow{4}{*}{\textbf{Sg}} & \textbf{Nominativ}      & ich    & du     & er     & \multirow{2}{*}{es}     & \multirow{2}{*}{sie}    \\
                                 & \textbf{Akkusativ}      & mich   & dich   & ihn    &      &     \\
                                 & \textbf{Dativ}          & mir    & dir    & \multicolumn{2}{c}{ihm}    & ihr    \\
                                 & \textbf{Genitiv}        & meiner & deiner & \multicolumn{2}{c}{seiner} & ihrer  \\
    \midrule
    \multirow{4}{*}{\textbf{Pl}} & \textbf{Nominativ}      & wir    & ihr    & &  \multirow{2}{*}{sie}          &        \\
                                 & \textbf{Akkusativ}      & \multirow{2}{*}{uns} & \multirow{2}{*}{euch}   & &    &        \\
                                 & \textbf{Dativ}          &     &    & & ihnen        &        \\
                                 & \textbf{Genitiv}        & unser  & euer   & & ihrer        &        \\
    \bottomrule
  \end{tabular}}\\

  \Zeile
  \onslide<+->
  \alert{Die Formen müssen Sie natürlich jederzeit sicher bestimmen können!}
\end{frame}


\begin{frame}
  {Pronomina in Artikelfunktion}
  \pause
  \begin{exe}
    \ex \label{ex:gemeinsamkeitenundunterschiede074}
    \begin{xlist}
      \ex{[\alert{Dieser} frische Marmorkuchen] schmeckt lecker.}
      \ex{[\alert{Jeder} leckere Marmorkuchen] ist mir recht.}
    \end{xlist}
  \end{exe}
  \pause
  \Zeile
  \begin{itemize}[<+->]
    \item In dieser Funktion stehen Pronomina \alert{vor einem Substantiv,\\
      mit dem sie kongruieren}.
      \Halbzeile
    \item \alert{Artikelwörter} (auch Determinative) | alle Wörter in dieser Position
      \Halbzeile
    \item im weiteren hier nur regelmäßig flektierende ("`normale"') Pronomina,\\
      keine Exoten wie \textit{ich}, \textit{du}, \textit{man}, \textit{etwas} usw.
  \end{itemize}
\end{frame}


\begin{frame}
  {Warum ist das so schwer? I}
  \pause
    \begin{center}
      \resizebox{1\textwidth}{!}{
        \begin{tabular}[h]{lp{3em}cllp{5em}cl}
          \toprule
          \textbf{Kasus (Singular)} &&&  \gruen{\textbf{Artikel}} &       &&& \alert{\textbf{Pronomen}} \\
          \midrule
          \textbf{Nominativ}        && \onslide<3->{\rot{\HandCuffRight}} & \textbf{\gruen{ein}}   & Mantel  && \onslide<3->{\rot{\HandCuffRight}} & \textbf{\alert{ein-er}} \\
          \textbf{Akkusativ}        &&&  \gruen{ein-en} & Mantel  &&& \alert{ein-en} \\
          \textbf{Dativ}            &&&  \gruen{ein-em} & Mantel  &&& \alert{ein-em} \\
          \textbf{Genitiv}          &&&  \gruen{ein-es} & Mantels &&& \alert{ein-es} \\
          \bottomrule
        \end{tabular}
    }
    \end{center}
    \pause
    \pause
    \Halbzeile
    Also gibt es \gruen{einen Artikel \textit{ein}} und \alert{ein Pronomen \textit{ein}}.
\end{frame}


\begin{frame}
  {Warum ist das so schwer? II}
  \pause
    \begin{center}
      \resizebox{1\textwidth}{!}{
        \begin{tabular}[h]{lp{2em}cllp{4em}cl}
          \toprule
          \textbf{Kasus (Plural)}   &&&  \gruen{\textbf{Artikel}} &        &&& \alert{\textbf{Pronomen}} \\
          \midrule
          \textbf{Nominativ}        &&& \gruen{die}               & Rottweiler  &&& \alert{die} \\
            \textbf{Akkusativ}      &&& \gruen{die}               & Rottweiler  &&&  \alert{die} \\
            \textbf{Dativ}            && \onslide<3->{\rot{\HandCuffRight}} & \textbf{\gruen{den}}         & Rottweilern && \onslide<3->{\rot{\HandCuffRight}} & \textbf{\alert{denen}} \\
            \textbf{Genitiv}          && \onslide<3->{\rot{\HandCuffRight}} & \textbf{\gruen{der}}         & Rottweiler  &&  \onslide<3->{\rot{\HandCuffRight}} & \textbf{\alert{derer}} \\
          \bottomrule
        \end{tabular}
      }
    \end{center}
    \pause
    \pause
    \Halbzeile
    Also gibt es \gruen{einen Artikel \textit{d-}} und \alert{ein Pronomen \textit{d-}}.\\
    \Halbzeile
    \grau{\textit{d-} ist der Stamm für \textit{der}, \textit{die}, \textit{das}.}
\end{frame}


\begin{frame}
  {Warum ist das so schwer? III}
  \pause
    \begin{center}
      \resizebox{0.9\textwidth}{!}{
        \begin{tabular}[h]{llp{1em}llp{2em}l}
          \toprule
          &\textbf{Kasus}       &&  \alert{\textbf{Pronomen}} &         && \alert{\textbf{Pronomen}} \\
          &&& \multicolumn{2}{l}{\textbf{in Artikelfunktion}} && \textbf{in Pronominalfunktion} \\
          \midrule 
          \textbf{Sg} & \textbf{Nominativ}   && dies\alert{-er}  & Rottweiler         && dies\alert{-er} \\
          &\textbf{Akkusativ}   && dies\alert{-en}  & Rottweiler         && dies\alert{-en} \\
          &\textbf{Dativ}       && dies\alert{-em}  & Rottweiler         && dies\alert{-em} \\
          &\textbf{Genitiv}     && dies\alert{-es}  & Rottweilers        && dies\alert{-es} \\
          \midrule 
          \textbf{Pl}&\textbf{Nominativ}   && dies\alert{-e}   & Rottweiler         && dies\alert{-e} \\
          &\textbf{Akkusativ}   && dies\alert{-e}   & Rottweiler         && dies\alert{-e} \\
          &\textbf{Dativ}       && dies\alert{-en}  & Rottweilern        && dies\alert{-en} \\
          &\textbf{Genitiv}     && dies\alert{-er}  & Rottweiler         && dies\alert{-er} \\
          \bottomrule
        \end{tabular}
      }
    \end{center}
    \Halbzeile
    \pause
    Also gibt es nur \alert{ein Pronomen \textit{dies}}, das \alert{in beiden Funktionen} auftritt.\\
    \pause
    \Halbzeile
   Es gibt \rot{keinen Artikel \textit{dies}!}
\end{frame}


\begin{frame}
  {Warum ist das so schwer? IV}
  \pause
  \Zeile
  \begin{block}{Artikel und Pronomen}
    Wenn die Formen eines Stamms in Artikelfunktion und Pronominalfunktion nicht durchgehend gleich sind, handelt es sich um \alert{\textbf{zwei verschiedene lexikalische Wörter mit gleichlautendem Stamm: einen Artikel und ein Pronomen}}.\\
    \Halbzeile
    Ansonsten handelt es sich bei jedem Wort, das in Artikel- und Pronominalfunktion auftreten kann, um \alert{\textbf{ein lexikalisches Wort, nämlich ein reines Pronomen, das in Artikelfunktion und Pronominalfunktion auftreten kann}}.\\
    \Halbzeile
    Es gibt folglich \rot{\textbf{keine Artikel in Pronominalfunktion}}.
  \end{block}
\end{frame}

\begin{frame}
  {Warum ist das so schwer? V}
  \pause
  \begin{block}{Artikel und Pronomina mit gleichlautendem Stamm I}
    Treten die Stämme \textit{ein}, \textit{kein}, \textit{mein}, \textit{dein}, \textit{sein}, \textit{ihr}, \textit{euer}, \textit{unser} oder \textit{d-} in Artikelfunktion auf, \gruen{\textbf{sind sie Artikel}}.
  \end{block}
  \pause
  \begin{block}{Artikel und Pronomina mit gleichlautendem Stamm II}
    Treten die Stämme \textit{ein}, \textit{kein}, \textit{mein}, \textit{dein}, \textit{sein}, \textit{ihr}, \textit{euer}, \textit{unser} oder \textit{d-} in Pronominalfunktion auf, \alert{\textbf{sind sie Pronomina}}.
  \end{block}
  \Zeile
  \pause
  \begin{block}{Reine Pronomina (\textbf{kein} gleichlautender Artikel)}
    Alle anderen pronominalen Stämme wie \textit{dies}, \textit{jen}, \textit{welch} sind \alert{\textbf{immer ein Pronomen}} und treten in Artikel- oder Pronominalfunktion auf.
  \end{block}
\end{frame}


\begin{frame}
  {Das (ganz) normale Pronomen}
  \pause
  \begin{center}
    \begin{tabular}{lllll}
      \toprule
      \multicolumn{1}{c}{} & \textbf{Mask} & \textbf{Neut} & \textbf{Fem} & \textbf{Pl} \\
      \midrule
      \textbf{Nom} & dies-er & dies-es & dies-e & dies-e \\
      \textbf{Akk} & dies-en & dies-es & dies-e & dies-e \\
      \textbf{Dat} & dies-em & dies-em & dies-er & dies-en \\
      \textbf{Gen} & dies-es & dies-es & dies-er & dies-er \\
      \bottomrule
    \end{tabular}
  \end{center}
\end{frame}


\begin{frame}
  {Synkretismen}
  \pause
  Wo ist das Vier-Kasus-System?
  \pause
  \Zeile
  \begin{center}
    \begin{tabular}{|l|c|c|c|c|}
      \cline{2-5}
      \multicolumn{1}{c|}{} & \alert<4->{\textbf{Mask}} & \textbf{Neut} & \textbf{Fem} & \textbf{Pl} \\
      \hline
      \textbf{Nom} & \alert<4->{-er} & \multirow{2}{*}{-es} & \multicolumn{2}{c|}{\multirow{2}{*}{-e}} \\ \cline{1-2}
      \textbf{Akk} & \alert<4->{-en} && \multicolumn{2}{c|}{} \\ \hline
      \textbf{Dat} & \multicolumn{2}{c|}{\alert<4->{-em}} && -en \\ \cline{1-3} \cline{5-5}
      \textbf{Gen} & \multicolumn{2}{c|}{\alert<4->{-es}} & \multicolumn{2}{c|}{-er} \\
      \hline
    \end{tabular}
  \end{center}
\end{frame}

\begin{frame}
  {Abweichungen bei den Definita}
  \pause
  Stamm-Affix-Trennprobleme beim Definitartikel:\\
  \begin{center}
    \begin{tabular}{lllll}
      \toprule
      \multicolumn{1}{c}{} & \textbf{Mask} & \textbf{Neut} & \textbf{Fem} & \textbf{Pl} \\
      \midrule
      \textbf{Nom} & d-er & \Dimblue d-as & \Dimblue d-ie & \Dimblue d-ie \\
      \textbf{Akk} & d-en & \Dimblue d-as & \Dimblue d-ie & \Dimblue d-ie \\
      \textbf{Dat} & d-em & d-em & d-er & d-en \\
      \textbf{Gen} & d-es & d-es & d-er & d-er \\
      \bottomrule
    \end{tabular}
  \end{center}
  \pause 

  Zusätzliche Affixdopplung beim Definitpronomen:\\
  \begin{center}
    \begin{tabular}{lllll}
      \toprule
      \multicolumn{1}{c}{} & \textbf{Mask} & \textbf{Neut} & \textbf{Fem} & \textbf{Pl} \\
      \hline
      \textbf{Nom} & d-er & \Dimblue d-as & \Dimblue d-ie & \Dimblue d-ie \\
      \textbf{Akk} & d-en & \Dimblue d-as & \Dimblue d-ie & \Dimblue d-ie \\
      \textbf{Dat} & d-em & d-em & d-er & d-en-en \Dimgreen \\
      \textbf{Gen} & \Dimgreen d-ess-en & \Dimgreen d-ess-en & \Dimgreen d-er-er & \Dimgreen d-er-er \\
      \bottomrule
    \end{tabular}
  \end{center}
\end{frame}


\begin{frame}
  {Abweichung beim Indefinitartikel}
  \pause
  Das Indefinitpronomen flektiert als normales Pronomen.\\
  \begin{center}
    \begin{tabular}{lllll}
      \lsptoprule
      \multicolumn{1}{c}{} & \textbf{Mask} & \textbf{Neut} & \textbf{Fem} & \textbf{Pl} \\
      \midrule
      \textbf{Nom} & kein-er & kein-es & kein-e & kein-e \\
      \textbf{Akk} & kein-en & kein-es & kein-e & kein-e \\
      \textbf{Dat} & kein-em & kein-em & kein-er & kein-en \\
      \textbf{Gen} & kein-es & kein-es & kein-er & kein-er \\
      \lspbottomrule
    \end{tabular}
  \end{center}
  \pause
  \Zeile
  Aber der Indefinitartikel hat Affixlücken:\\
  \begin{center}
    \begin{tabular}{lllll}
      \lsptoprule
      \multicolumn{1}{c}{} & \textbf{Mask} & \textbf{Neut} & \textbf{Fem} & \textbf{Pl} \\
      \midrule
      \textbf{Nom} & \Dim kein & \Dim kein & kein-e & kein-e \\
      \textbf{Akk} & kein-en & \Dim kein & kein-e & kein-e \\
      \textbf{Dat} & kein-em & kein-em & kein-er & kein-en \\
      \textbf{Gen} & kein-es & kein-es & kein-er & kein-er \\
      \lspbottomrule
    \end{tabular}
  \end{center}
\end{frame}



\begin{frame}
  {Nochmal zurück zu Artikel vs.\ Pronomen}
  \pause
  Die auf den letzten Folien gezeigten Abweichungen von der normalen Pronominalflexion sind die systematische Aufarbeitung des eingangs gemachten Unterschieds zwischen Pronomina und Artikeln.\\
  \pause
  \Zeile
  \begin{center}
    \resizebox{0.8\textwidth}{!}{
      \begin{forest}
        [systematisch flektierende\\Pronomina und Artikel
          [Indefinit- und\\Possessivartikel\\(\textit{kein}{,} \textit{mein} usw.)]
          [normale Pronomina\\und Definita
            [normale Pronomina\\(\textit{jener}{,} \textit{meiner} usw.)]
            [Definita
              [Definitartikel\\(\textit{der}{,} \textit{des} usw.)]
              [Definitpronomina\\(\textit{der}{,} \textit{dessen} usw.)]
            ]
          ]
        ]
      \end{forest}
    }
  \end{center}
  \pause
  \Halbzeile
  Übrigens, wir definieren hier gerade weitere Wortklassen.
\end{frame}

\section{Zur nächsten Woche | Überblick}

\begin{frame}
  {Morphologie und Lexikon des Deutschen | Plan}
  \rot{Alle} angegebenen Kapitel\slash Abschnitte aus \rot{\citet{Schaefer2018b}} sind \rot{Klausurstoff}!\\
  \Halbzeile
  \begin{enumerate}
    \item Grammatik und Grammatik im Lehramt (Kapitel 1 und 3)
    \item Morphologie und Grundbegriffe (Kapitel 2, Kapitel 7 und Abschnitte 11.1--11.2)
    \item Wortklassen als Grundlage der Grammatik (Kapitel 6)
    \item Wortbildung | Komposition (Abschnitt 8.1)
    \item Wortbildung | Derivation und Konversion (Abschnitte 8.2--8.3)
    \item Flexion | Nomina außer Adjektiven (Abschnitte 9.1--9.3)
    \item \rot{Flexion | Adjektive und Verben (Abschnitt 9.4 und Kapitel 10)}
    \item Valenz (Abschnitte 2.3, 14.1 und 14.3)
    \item Verbtypen als Valenztypen (Abschnitte 14.4--14.5, 14.7--14.9) 
    \item Kernwortschatz und Fremdwort (vorwiegend Folien)
  \end{enumerate}
  \Halbzeile
  \centering 
  \url{https://langsci-press.org/catalog/book/224}
\end{frame}



  \let\subsection\section\let\section\woopsi
  
  \section{Flexion | Adjektive und Verben}
  \let\woopsi\section\let\section\subsection\let\subsection\subsubsection
  \input{includes/07.+Flexion+--+Adjektive+und+Verben.tex}
  \let\subsection\section\let\section\woopsi
  
  \section{Valenz}
  \let\woopsi\section\let\section\subsection\let\subsection\subsubsection
  \section{Überblick}

\begin{frame}
  {Funktionale Wortschatzgliederung bei Verben}
  \onslide<+->
  \begin{itemize}[<+->]
    \item bisher | \alert{morphologisch motivierte} Gliederung des Lexikons
    \item \zB\ Pluralklassen bei Substantiven
      \Zeile
    \item weitere Gliederung | \alert{morphosyntaktisch-funktional}
    \item inbesondere \alert{Verbklassen}
      \Halbzeile
      \begin{itemize}[<+->]
        \item \alert{passivierbare} Verben
        \item \alert{Valenzklassen} (transitiv, intransitiv etc.)
        \item Verben mit Präpositionalobjekten
        \item \ldots\ nur ein Ausschnitt der möglichen Klassen
      \end{itemize}
  \end{itemize}
\end{frame}

\section{Valenz}

\begin{frame}
  {Ergänzungen und Angaben}
  \onslide<+->
  \onslide<+->
  \begin{exe}
    \ex\label{ex:valenz034}
    \begin{xlist}
      \ex{Gabriele malt \alert{[ein Bild]}.}
      \onslide<+->
      \ex{Gabriele malt \orongsch{[gerne]}.}
      \onslide<+->
      \ex{Gabriele malt \orongsch{[den ganzen Tag]}.}
      \onslide<+->
      \ex{Gabriele malt \orongsch{[ihrem Mann]} \rot{[zu figürlich]}.}
    \end{xlist}
  \end{exe}
  \Halbzeile
  \begin{itemize}[<+->]
    \item \alert{[ein Bild]} mit besonderer Relation zum Verb | \alert{Objekt\slash Ergänzung}
    \item keine solche Relation bei den anderen | \orongsch{Adverbial\slash Angaben}
    \item "`Weglassbarkeit"' (Optionalität) nicht entscheidend
  \end{itemize}
\end{frame}

\begin{frame}
  {Lizenzierung}
  \pause
  \begin{exe}
    \ex 
    \begin{xlist}
      \ex[ ]{Gabriele isst \orongsch{[den ganzen Tag]} Walnüsse.}
    \pause
      \ex[ ]{Gabriele läuft \orongsch{[den ganzen Tag]}.}
      \pause
      \ex[ ]{Gabriele backt ihrer Schwester \orongsch{[den ganzen Tag]} Stollen.}
      \pause
      \ex[ ]{Gabriele litt \orongsch{[den ganzen Tag]} unter Sonnenbrand.}
    \end{xlist}
    \pause\Halbzeile
    \ex 
    \begin{xlist}
      \ex[*]{Gabriele isst \alert{[ein Bild]} Walnüsse.}
      \pause
      \ex[*]{Gabriele läuft \alert{[ein Bild]}.}
      \pause
      \ex[*]{Gabriele backt ihrer Schwester \alert{[ein Bild]} Stollen.}
      \pause
      \ex[*]{Gabriele litt \alert{[ein Bild]} unter Sonnenbrand. }
      \pause
    \end{xlist}
  \end{exe}
  \pause\Halbzeile
  \begin{itemize}[<+->]
    \item \orongsch{Angaben} sind verb-unspezifisch lizenziert
    \item \alert{Ergänzungen} sind verb(klassen)spezifisch lizenziert
    \item \gruen{Valenz = Liste der Ergänzungen eines lexikalischen Worts}
  \end{itemize}
\end{frame}


\begin{frame}
  {Weitere Eigenschaften von Ergänzungen und Angaben}
  \pause
  \alert{Iterierbarkeit} (= Wiederholbarkeit) von Angaben, nicht Ergänzungen\\
  \pause
  \Halbzeile
  \begin{exe}
    \ex[ ]{Wir müssen den Wagen \orongsch{[jetzt]} \orongsch{[mit aller Kraft]} \orongsch{[vorsichtig]} anschieben.}
    \pause
    \ex[ ]{Wir essen \orongsch{[schnell]} \orongsch{[mit Appetit]} \orongsch{[an einem Tisch]}\\
      \orongsch{[mit der Gabel]} \alert{[einen Salat]}.}
    \pause
    \ex[*]{Wir essen \orongsch{[schnell]} \rot{[ein Tofugericht]} \orongsch{[mit Appetit]} \orongsch{[an einem Tisch]}\\
      \orongsch{[mit der Gabel]} \alert{[einen Salat]}.}
  \end{exe}
\end{frame}


\begin{frame}
  {Ergänzungen | Schnittstelle von Syntax und Semantik}
  \onslide<+->
  \onslide<+->
  Verbsemantik | Welche \alert{Rolle} spielen die von den Satzgliedern bezeichneten Dinge\\
  in der vom Verb beschriebenen Situation?\\
  \Zeile
  \onslide<+->
  Semantik von \alert{Ergänzungen} | \alert{abhängig} vom Verb\\
  \onslide<+->
  \Viertelzeile
  Semantik von \gruen{Angaben} | \gruen{unabhängig} vom Verb\\
  \Halbzeile
  \pause
  \begin{exe}
    \ex\label{ex:valenz071}
    \begin{xlist}
      \ex{\label{ex:valenz072}Ich lösche \alert{[den Ordner]} \gruen{[während der Hausdurchsuchung]}.}
      \pause
      \ex{\label{ex:valenz073}Ich mähe \alert{[den Rasen]} \gruen{[während der Ferien]}.}
      \pause
      \ex{\label{ex:valenz074}Ich fürchte \alert{[den Sturm]} \gruen{[während des Sommers]}.}
    \end{xlist}
  \end{exe}
\end{frame}

\begin{frame}
  {Valenz}
  \onslide<+->
  \onslide<+->
  \begin{block}{Angaben}
    \alert{Angaben} sind grammatisch immer lizenziert und bringen\\
    ihre eigene semantische Rolle mit.\\
    \Halbzeile
    \grau{Sie können aber semantisch\slash pragmatisch inkompatibel sein.}
  \end{block}
  \Zeile
  \onslide<+->
  \begin{block}{Ergänzungen}
    \gruen{Ergänzungen} werden spezifisch vom Verb lizenziert und in ihrer semantischen Rolle\\
    vom Verb festgelegt. Jede dieser Rollen kann nur einmal vergeben werden.
  \end{block}
\end{frame}


\section{Rollen}

\begin{frame}
  {Was sind "`Rollen"'}
  \pause
  \begin{exe}
    \ex
    \begin{xlist}
      \ex{\alert{Michelle} kauft einen Rottweiler.}
      \pause
      \ex{\alert{Der Rottweiler} schläft.}
      \pause
      \ex{\alert{Der Rottweiler} erfreut Marina.}
    \end{xlist}
  \end{exe}
  \pause
  \Halbzeile
  \begin{itemize}[<+->]
    \item semantische Generalisierung über \alert{Käuferin}, \alert{Schläfer}, \alert{Erfreuer}?
    \item \rot{"`Das Subjekt drückt aus, wer oder was im Satz handelt."' --- Unsinn!}
    \item Nur die \alert{Käuferin} handelt!
      \Halbzeile
    \item Verben als Kodierung eines \alert{Situationstyps} 
    \item Situationstypen mit charakteristischen \alert{Mitspielern}
    \item Handelnde, Betroffene, Veränderte, Emotionen Erfahrende, \ldots
    \item "`Mitspieler"' im weiteren Sinn, auch Gegenstände, Zeitpunkte usw.
      \Halbzeile
    \item Gleichsetzung von Rollen mit Kasus \rot{absoluter Unsinn}
  \end{itemize}
\end{frame}

\begin{frame}
  {Agens und Experiencer}
  \pause
  \begin{exe}
    \ex
    \begin{xlist}
      \ex{\alert{Michelle} kauft \orongsch{einen Rottweiler}.}
      \ex{\orongsch{Der Rottweiler} schläft.}
      \ex{\orongsch{Der Rottweiler} erfreut \rot{Marina}.}
    \end{xlist}
  \end{exe}
  \pause
  \Halbzeile
  \begin{itemize}[<+->]
    \item Rollen in den Beispielen
      \begin{itemize}[<+->]
        \item \alert{Michelle} → Handelnde = \alert{Agens}
        \item \rot{Marina} → psychischen Zustand Erfahrende: \rot{Experiencer}
        \item \orongsch{Rottweiler} → andere Rollen, hier nicht weiter analysiert (Rx)
      \end{itemize}
  \end{itemize}
\end{frame}

\begin{frame}
  {Rollenzuweisung\ldots\ und Ergänzungen und Angaben}
  \pause
  \begin{itemize}[<+->]
    \item für einen Situationstyp charakteristische Rollen?
      \Viertelzeile
    \item (fast) \alert{immer} \zB
      \begin{itemize}[<+->]
        \item \alert{Zeitpunkt}
        \item \alert{Ort}
        \item \alert{Dauer}
      \end{itemize}
      \Viertelzeile
    \item \rot{nicht immer} \zB
      \begin{itemize}[<+->]
        \item \rot{Handelnde} (\textit{schlafen}, \textit{fallen}, \textit{gefallen}, \ldots)
        \item \rot{psychischen Zustand Erfahrende} (\textit{laufen}, \textit{reparieren}, \textit{häkeln}, \ldots)
        \item \rot{physisch Veränderte} (\textit{betrachten}, \textit{belassen}, \textit{verkaufen}, \ldots)
      \end{itemize}
      \Viertelzeile
    \item Auch wenn Kaufen, Fallen usw.\ Emotionen auslöst:\\
      \alert{Das jeweilige Verb (\textit{kaufen}, \textit{fallen} usw.) sagt darüber nichts aus!}
      \Viertelzeile
    \item \rot{Ergänzung}: gekoppelt an \alert{verbspezifische} Rolle 
    \item \alert{Angabe}: gekoppelt an \alert{verbunspezifische} Rolle
%    \item Ergänzung = subklassenspezifische Lizenzierung
  \end{itemize}
\end{frame}

\begin{frame}
  {Das Prinzip der Rollenzuweisung}
  \onslide<+-> 
  \begin{itemize}[<+->]
    \item situationsspezifische Rollen: \alert{nur einmal vergebbar}\\
    = Prinzip der Rollenzuweisung
      \Halbzeile
    \item semantische Motivation für:
      \begin{itemize}[<+->]
        \item Angaben sind iterierbar,
        \item Ergänzungen nicht.
      \end{itemize}
      \Halbzeile
    \item und \alert{Koordinationen}?
  \end{itemize}
  \onslide<+->
  \begin{exe}
    \ex \alert{Marina und Michelle} kaufen bei \rot{einer seriösen Züchterin\\
    und ihrer Freundin} einen \orongsch{Dobermann und einen Rottweiler}.
  \end{exe}
  \begin{itemize}[<+->]
    \item koordinierte NPs = \alert{ein} syntaktisches Argument
    \item eine Rolle pro syntaktischem Argument
    \item semantisch dann Summenindividuen oder Ähnliches
  \end{itemize}
\end{frame}

\section{Passive}

\begin{frame}
  {Valenzänderungen | Vorbemerkung}
  \onslide<+->
  \onslide<+->
  \alert{Wir beschreiben Passivbildung als Valenzänderung\ldots}\\
  \Halbzeile
  \begin{itemize}[<+->]
    \item im Prinzip eine Art von \alert{Wortbildung}
    \item Valenz von \textit{kaufen} \{\alert{Nominativ-NP\Sub{1}}, \orongsch{Akkusativ-NP\Sub{2}}\}\\
      → Valenz des Passivs von \textit{kaufen} \{\orongsch{Nominativ-NP\Sub{2}}\}
      \Halbzeile
    \item andere Wortbildungsprozesse mit Valenzänderungen
      \begin{itemize}[<+->]
        \item Valenzanreicherung beim Applikativ \textit{be:}
        \item \textit{geh-en} → \textit{be:geh-en}
        \item Valenzänderung \{Nominativ-NP\Sub{1}\} → \{Nominativ-NP\Sub{1}, Akkusativ-NP\Sub{2}\}
        \item \textit{Ich gehe auf der Straße.} → \textit{Ich begehe die Straße.}
      \end{itemize}
  \end{itemize}
\end{frame}

\begin{frame}
  {\textit{werden}-Passiv oder Vorgangspassiv}
  \pause
  "`Nur transitive Verben können passiviert werden."'\pause\rot{--- Nein!}
  \pause
    \begin{exe}
    \addtolength\itemsep{-0.25\baselineskip}
      \ex\label{ex:werdenpassivundverbtypen110}
      \begin{xlist}\addtolength\itemsep{-0.5\baselineskip}
          \ex[ ]{\label{ex:werdenpassivundverbtypen111} \alert{Johan} wäscht \orongsch{den Wagen}.}
          \ex[ ]{\label{ex:werdenpassivundverbtypen112} \orongsch{Der Wagen} wird \alert{(von Johan)} gewaschen.}
      \end{xlist}
      \pause
      \ex\label{ex:werdenpassivundverbtypen113}
      \begin{xlist}\addtolength\itemsep{-0.5\baselineskip}
          \ex[ ]{\label{ex:werdenpassivundverbtypen114} \alert{Alma} schenkt \gruen{dem Schlossherrn} \orongsch{den Roman}.}
          \ex[ ]{\label{ex:werdenpassivundverbtypen115} \orongsch{Der Roman} wird \gruen{dem Schlossherrn} \alert{(von Alma)} geschenkt.}
      \end{xlist}
      \pause
      \ex\label{ex:werdenpassivundverbtypen116}
      \begin{xlist}\addtolength\itemsep{-0.5\baselineskip}
          \ex[ ]{\label{ex:werdenpassivundverbtypen117} \alert{Johan} bringt \orongsch{den Brief} zur Post.}
          \ex[ ]{\label{ex:werdenpassivundverbtypen118} \orongsch{Der Brief} wird \alert{(von Johan)} zur Post gebracht.}
      \end{xlist}
      \pause
      \ex\label{ex:werdenpassivundverbtypen119}
      \begin{xlist}\addtolength\itemsep{-0.5\baselineskip}
          \ex[ ]{\label{ex:werdenpassivundverbtypen120} \alert{Der Maler} dankt \gruen{den Fremden}.}
          \ex[ ]{\label{ex:werdenpassivundverbtypen121} \gruen{Den Fremden} wird \alert{(vom Maler)} gedankt.}
      \end{xlist}
      \pause
      \ex\label{ex:werdenpassivundverbtypen122}
      \begin{xlist}\addtolength\itemsep{-0.5\baselineskip}
          \ex[ ]{\label{ex:werdenpassivundverbtypen123} \alert{Johan} arbeitet hier immer montags.}
          \ex[ ]{\label{ex:werdenpassivundverbtypen124} Montags wird hier \alert{(von Johan)} immer gearbeitet.}
      \end{xlist}
      \pause
      \ex\label{ex:werdenpassivundverbtypen125}
      \begin{xlist}\addtolength\itemsep{-0.5\baselineskip}
          \ex[ ]{\label{ex:werdenpassivundverbtypen126} \alert{Der Ball} platzt bei zu hohem Druck.}
          \ex[*]{\label{ex:werdenpassivundverbtypen127} Bei zu hohem Druck wird \rot{(vom Ball)} geplatzt.}
      \end{xlist}
      \pause
      \ex\label{ex:werdenpassivundverbtypen128}
      \begin{xlist}\addtolength\itemsep{-0.5\baselineskip}
          \ex[ ]{\label{ex:werdenpassivundverbtypen129} \alert{Der Rottweiler} fällt \gruen{Michelle} auf.}
          \ex[*]{\label{ex:werdenpassivundverbtypen130} \alert{Michelle} wird \rot{(von dem Rottweiler)} aufgefallen.}
      \end{xlist}
    \end{exe}
\end{frame}

\begin{frame}
  {Was passiert beim Vorgangspassiv?}
  \pause
  \begin{itemize}[<+->]
    \item Auxiliar: \textit{werden}, Verbform: Partizip
    \item für Passivierbarkeit relevant: \alert{die Nominativ-Ergänzung!}
      \Halbzeile
    \item \alert{Passivierung als Valenzänderung}:
      \begin{itemize}[<+->]
        \item Nominativ-Ergänzung → optionale \textit{von}-PP-Angabe
        \item eventuelle Akkusativ-Ergänzung → obligatorische Nominativ-Ergänzung
        \item kein Akkusativ: kein "`Subjekt"' = keine Nom-Erg (\textit{es} ist positional)
        \item \grau{Dativ-Ergänzung → Dativ-Ergänzung (usw.)}
        \item \grau{Angaben: keine Änderung}
      \end{itemize}
    \Halbzeile
  \item \alert{nicht passivierbare Verben}?
    \begin{itemize}[<+->]
      \item {ohne }\rot{agentivische}\alert{ Nominativ-Ergänzung}
      \item Achtung! Gilt nur mit prototypischem Charakter\ldots
      \item Siehe Vertiefung 14.2 auf S.~439!
    \end{itemize}
  \end{itemize}
\end{frame}

\begin{frame}
  {Feinere Klassifikation von Verben}
  \pause
  \begin{itemize}[<+->]
    \item Neuklassifikation vor dem Hintergrund des Vorgangspassivs
    \item Wenn so eine Klassifikation einen Wert haben soll:\\
      \alert{Berücksichtigung der semantischen Rollen unabdinglich!}
    \item Bedingung für Vorgangs-Passiv: \alert{Nom\_Ag}
  \end{itemize} 
  \pause
  \Zeile
  \centering
  \scalebox{0.9}{\begin{tabular}{lllll}
    \toprule
    \textbf{Valenz} & \textbf{Passiv} & \textbf{Name} & \textbf{Beispiel} \\
    \midrule
    \alert{Nom\_Ag} & ja & Unergative & \textit{arbeiten} \\
    Nom & nein & Unakkusative & \textit{platzen} \\
    \alert{Nom\_Ag}, Akk & ja & Transitive & \textit{waschen} \\
    \alert{Nom\_Ag}, Dat & ja & unergative Dativverben & \textit{danken} \\
    Nom, Dat & nein & unakkusative Dativverben & \textit{auf"|fallen} \\
    \alert{Nom\_Ag}, Dat, Akk & ja & Ditransitive & \textit{geben} \\
    \bottomrule
  \end{tabular}}\\
  \raggedright
  \Zeile
  \pause
  Immer noch nichts als eine reine Bequemlichkeitsterminologie,\\
  um bestimmte (durchaus wichtige) Valenzmuster hervorzuheben.
\end{frame}

\section{Verben mit Präpositionalobjekten}

\begin{frame}
  {Präpositionalobjekte}
  \pause
  PP-Angabe vs.\ PP-Ergänzung: oft schwierig zu entscheiden.\\
  \Viertelzeile
  \pause
  \begin{exe}
    \ex\label{ex:ppergaenzungenundppangaben189}
    \begin{xlist}
      \ex{\label{ex:ppergaenzungenundppangaben190} Viele Menschen leiden \alert{unter Vorurteilen}.}
      \pause
      \ex{\label{ex:ppergaenzungenundppangaben191} Viele Menschen schwitzen \orongsch{unter Sonnenschirmen}.}
    \end{xlist}
  \end{exe}
  \Viertelzeile
  \pause
  \begin{itemize}[<+->]
    \item \alert{Ergänzungen}:
      \begin{itemize}[<+->]
        \item Semantik der PP nur verbgebunden interpretierbar
        \item = semantische Rolle der PP vom Verb zugewiesen
      \end{itemize}
    \item \orongsch{Angaben}:
      \begin{itemize}[<+->]
        \item Semantik der PP selbständig erschließbar (lokal unter)
        \item = "`semantische Rolle"' der PP von der Präposition zugewiesen
      \end{itemize}
      \Viertelzeile
    \item \alert{Sehen Sie, wie schnell man in der (Grund-)Schulgrammatik\\
      in gefährliche linguistische Fahrwasser gerät?}
    \item \rot{Wenn Sie dieses Wissen nicht haben, unterrichten Sie sehr leicht\\
      komplett Falsches, zumal wenn es im Lehrbuch falsch steht.}
  \end{itemize}
\end{frame}


\begin{frame}
  {Der umstrittene PP-Angaben-Test}
  \pause
  Die PP mit \textit{"`Dies geschieht PP."'} aus dem Satz auskoppeln.\\
  \Halbzeile
  \pause
  \begin{exe}
    \ex\label{ex:ppergaenzungenundppangaben192}
    \begin{xlist}
      \ex[*]{\label{ex:ppergaenzungenundppangaben193} Viele Menschen leiden.
      \rot{Dies geschieht unter Vorurteilen.}}
        \pause
      \ex[ ]{\label{ex:ppergaenzungenundppangaben194} Viele Menschen schwitzen.
      \alert{Dies geschieht unter Sonnenschirmen.}}
        \pause
      \ex[*]{\label{ex:ppergaenzungenundppangaben195} Mausi schickt einen Brief.
      \rot{Dies geschieht an ihre Mutter.}}
        \pause
      \ex[*]{\label{ex:ppergaenzungenundppangaben196} Mausi befindet sich.
      \rot{Dies geschieht in Hamburg.}}
        \pause
      \ex[?]{\label{ex:ppergaenzungenundppangaben197} Mausi liegt.
      \orongsch{Dies geschieht auf dem Bett.}}
    \end{xlist}
  \end{exe}
  \Halbzeile
  \pause
  \begin{itemize}[<+->]
    \item der beste Test, den es gibt
    \item trotz Problemen
    \item \rot{Verlangen Sie von Schülern keine Entscheidungen,\\
    die Sie selber nicht operationalisieren können!}
  \end{itemize}
\end{frame}

\section{Zur nächsten Woche | Überblick}

\begin{frame}
  {Morphologie und Lexikon des Deutschen | Plan}
  \rot{Alle} angegebenen Kapitel\slash Abschnitte aus \rot{\citet{Schaefer2018b}} sind \rot{Klausurstoff}!\\
  \Halbzeile
  \begin{enumerate}
    \item Grammatik und Grammatik im Lehramt (Kapitel 1 und 3)
    \item Morphologie und Grundbegriffe (Kapitel 2, Kapitel 7 und Abschnitte 11.1--11.2)
    \item Wortklassen als Grundlage der Grammatik (Kapitel 6)
    \item Wortbildung | Komposition (Abschnitt 8.1)
    \item Wortbildung | Derivation und Konversion (Abschnitte 8.2--8.3)
    \item Flexion | Nomina außer Adjektiven (Abschnitte 9.1--9.3)
    \item Flexion | Adjektive und Verben (Abschnitt 9.4 und Kapitel 10)
    \item Valenz (Abschnitte 2.3, 14.1 und 14.3)
    \item \rot{Verbtypen als Valenztypen (Abschnitte 14.4--14.5, 14.7--14.9)}
    \item Kernwortschatz und Fremdwort (vorwiegend Folien)
  \end{enumerate}
  \Halbzeile
  \centering 
  \url{https://langsci-press.org/catalog/book/224}
\end{frame}




  \let\subsection\section\let\section\woopsi
  
  \section{Verbtypen als Valenztypen}
  \let\woopsi\section\let\section\subsection\let\subsection\subsubsection
  \section{Überblick}

\begin{frame}
  {Weitere Unterteilung des Verbwortschatzes}
  \onslide<+->
  \begin{itemize}[<+->]
    \item \alert{Doppelakkusative} und Objektstatus
      \Zeile
    \item \alert{Dative} als Ergänzungen (Objekte)
    \item Dativpassiv als Test
      \Zeile
    \item \alert{Statusrektion} | Modalverben, Halbmodalverben, Hilfsverben
  \end{itemize}
\end{frame}

\section{Objekte und Valenz}

\begin{frame}
  {Terminologische Zuordnung}
  \onslide<+->
  \begin{itemize}[<+->]
    \item \alert{Subjekt} | mit Verb kongruierende Nominativ-Ergänzung
    \item \alert{direktes Objekt} | Akkusativ-Ergänzung eines Verbs
    \item \alert{indirektes Objekt} | Dativ-Ergänzung eines Verbs
    \item \alert{Präpositionalobjekt} | Präpositionsgruppe mit Ergänzungsstatus
      \Zeile
    \item \rot{Nichts davon} ist zwangsläufig immer vorhanden!
      \begin{itemize}[<+->]
        \item \textit{Mir graut.} | \rot{kein Subjekt}
        \item \textit{Der Ballon platzt.} | \rot{kein Objekt}
      \end{itemize}
      \Zeile
    \item \alert{adverbiale Bestimmung} | Angabe zum Verb(?)
  \end{itemize}
\end{frame}

\begin{frame}
  {Direkte Objekte und Doppelakkusative}
  \onslide<+->
  \onslide<+->
  Was ist ein direktes Objekt\slash Akkusativobjekt?
  \begin{itemize}[<+->]
    \item \alert{Akkusativ-Ergänzungen zum Verb}
    \item \alert{oder Nebensätze an deren Stelle}
  \end{itemize}
  \onslide<+->
  \Halbzeile
  Und Doppelakkusative?\\
  \onslide<+->
  \begin{exe}
    \ex\label{ex:akkusativeunddirekteobjekte158}
    \begin{xlist}
      \ex[ ]{\label{ex:akkusativeunddirekteobjekte159} Ich lehre \alert{ihn} \orongsch{das Schwimmen}.}
      \onslide<+->
      \ex[*]{\label{ex:akkusativeunddirekteobjekte160} \orongsch{Das Schwimmen} wird \alert{ihn} gelehrt.}
      \onslide<+->
      \ex[*]{\label{ex:akkusativeunddirekteobjekte161} \alert{Er} wird \orongsch{das Schwimmen} gelehrt.}
      \onslide<+->
      \ex[ ]{\label{ex:akkusativeunddirekteobjekte161} Hier wird \orongsch{das Schwimmen} gelehrt.}
    \end{xlist}
  \end{exe}
  \begin{itemize}[<+->]
    \item beide Akkusative im Passiv nicht nominativfähig
    \item \grau{Korrektur zum Buch: Doppelakkusative bilden unpersönliche Passive.}
  \end{itemize} 
\end{frame}

\section{Dative}

\begin{frame}
  {\textit{bekommen}-Passiv oder Rezipientenpassiv}
  \pause
  Es gibt nicht "`das Passiv im Deutschen"'.\\
  \Halbzeile
  \pause
  \begin{exe}
    \ex\label{ex:bekommenpassiv138}
    \begin{xlist}
      \ex[ ]{\small\label{ex:bekommenpassiv139} \gruen{Mein Kollege} bekommt \orongsch{den Wagen} \alert{(von Johan)} gewaschen.}
      \pause
      \ex[ ]{\small\label{ex:bekommenpassiv140} \gruen{Der Schlossherr} bekommt \orongsch{den Roman} \alert{(von Alma)} geschenkt.}
      \pause
      \ex[ ]{\small\label{ex:bekommenpassiv141} \gruen{Mein Kollege} bekommt \orongsch{den Brief} \alert{(von Johan)} zur Post gebracht.}
      \pause
      \ex[ ]{\small\label{ex:bekommenpassiv142} \gruen{Die Fremden} bekommen \alert{(von dem Maler)} gedankt.}
      \pause
      \ex[?]{\small\label{ex:bekommenpassiv143} \gruen{Mein Kollege} bekommt hier immer montags \alert{(von Johan)} gearbeitet.}
      \pause
      \ex[*]{\small\label{ex:bekommenpassiv144} \gruen{Mein Kollege} bekommt bei zu hohem Druck \rot{(von dem Ball)} geplatzt.}
      \pause
      \ex[*]{\small\label{ex:bekommenpassiv145} \gruen{Michelle} bekommt \rot{(von dem Rottweiler)} aufgefallen.}
    \end{xlist}
  \end{exe}
  \pause\Halbzeile
  \alert{Das ist eine Passivbildung, die genauso den Nom\_Ag betrifft\\
  wie das Vorgangspassiv.}
\end{frame}

\begin{frame}
  {Was passiert beim Rezipientenpassiv?}
  \pause
  Alles, was sich verglichen mit Vorgangspassiv nicht unterscheidet, grau.\\
  \Halbzeile
  \pause
  \begin{itemize}[<+->]
    \item Auxiliar: \textit{bekommen} (evtl.\ \textit{kriegen}), \grau{Verbform: Partizip}
    \item \grau{für Passivierbarkeit relevant: die Nominativ-Ergänzung!}
      \Halbzeile
    \item \grau{Passivierung = Valenzänderung}:
      \begin{itemize}[<+->]
        \item \grau{Nominativ-Ergänzung → optionale \textit{von}-PP-Angabe}
        \item eventuelle Akkusativ-Ergänzung: → Akkusativ-Ergänzung
        \item \alert{Dativ-Ergänzung → Nominativ-Ergänzung}
        \item \rot{kein Dativ: kein Rezipientenpassiv}
        \item \grau{Angaben: keine Änderung}
      \end{itemize}
    \Halbzeile
  \item \grau{nicht passivierbare Verben?}
    \begin{itemize}[<+->]
      \item \grau{ohne agentivische Nominativ-Ergänzung}
      \item \grau{Achtung! Gilt nur mit prototypischem Charakter\ldots}
      \item \grau{Siehe Vertiefung 14.2 auf S.~439!}
    \end{itemize}
  \end{itemize}
\end{frame}

\begin{frame}
  {Rezipientenpassiv bei unergativen Verben}
  \onslide<+->
  \onslide<+->
  Warum war dieser Satz zweifelhaft?\\
  \onslide<+->
  \begin{exe}
    \ex[?]{\small \gruen{Mein Kollege} bekommt hier immer montags \alert{(von Johan)} gearbeitet.}
  \end{exe}
  \onslide<+->
  \Halbzeile
  Ist der zugehörige Aktivsatz besser?\\
  \onslide<+->
  \begin{exe}
    \ex[?]{\small Montags arbeitet \alert{Johan} \gruen{meinem Kollegen} hier immer.}
  \end{exe}
  \begin{itemize}[<+->]
    \item Nein.
    \item \alert{keine Frage des Rezipientenpassivs}
    \item bei diesen Verben: eher \textit{für}-PP
  \end{itemize}
\end{frame}


\begin{frame}
  {Indirekte Objekte}
  \pause
  Welche Dative sind Ergänzungen (= Teil der Valenz)?\\
  \pause
  \Halbzeile
  \begin{exe}
    \ex\label{ex:dativeundindirekteobjekte166}
    \begin{xlist}
      \ex[ ]{\label{ex:dativeundindirekteobjekte167} \alert{Alma} gibt \gruen{ihm} heute ein Buch.}
      \pause
      \ex[ ]{\label{ex:dativeundindirekteobjekte168} \alert{Alma} fährt \gruen{mir} heute aber wieder schnell.}
      \pause
      \ex[ ]{\label{ex:dativeundindirekteobjekte169} \alert{Alma} mäht \gruen{mir} heute den Rasen.}
      \pause
      \ex[ ]{\label{ex:dativeundindirekteobjekte170} \alert{Alma} klopft \gruen{mir} heute auf die Schulter.}
    \end{xlist}
  \end{exe}
  \Halbzeile
  \pause
  Recht einfache Entscheidung, da wir Passiv\\
  als \alert{Valenzänderung} beschreiben:\\
  \pause
  \begin{exe}
    \ex\label{ex:dativeundindirekteobjekte171}
    \begin{xlist}
      \ex[ ]{\label{ex:dativeundindirekteobjekte172} \gruen{Er} bekommt \alert{von Alma} heute ein Buch gegeben.}
      \ex[*]{\label{ex:dativeundindirekteobjekte173} \rot{Ich} bekomme \alert{von Alma} heute aber wieder schnell gefahren.}
      \ex[ ]{\label{ex:dativeundindirekteobjekte174} \gruen{Ich} bekomme \alert{von Alma} heute den Rasen gemäht.}
      \ex[ ]{\label{ex:dativeundindirekteobjekte175} \gruen{Ich} bekomme \alert{von Alma} heute auf die Schulter geklopft.}
    \end{xlist}
  \end{exe}
\end{frame}

\begin{frame}
  {Die vier wichtigen verbabhängigen Dative}
  \pause
  \begin{exe}
    \ex\label{ex:dativeundindirekteobjekte166x}
    \begin{xlist}
      \ex{\label{ex:dativeundindirekteobjekte167x} Alma gibt \gruen{ihm} heute ein Buch.}
      \pause
      \ex{\label{ex:dativeundindirekteobjekte168x} Alma fährt \orongsch{mir} heute aber wieder schnell.}
      \pause
      \ex{\label{ex:dativeundindirekteobjekte169x} Alma mäht \alert{mir} heute den Rasen.}
      \pause
      \ex{\label{ex:dativeundindirekteobjekte170x} Alma klopft \alert{mir} heute auf die Schulter.}
    \end{xlist}
  \end{exe}
  \Halbzeile
  \pause
  \begin{itemize}[<+->]
    \item (\ref{ex:dativeundindirekteobjekte167x}) = \gruen{Ergänzung} bei ditransitivem Verb
    \item (\ref{ex:dativeundindirekteobjekte168x}) = \orongsch{Bewertungsdativ} (Angabe, im Vorfeld\slash direkt nach finitem Verb)
    \item (\ref{ex:dativeundindirekteobjekte169x}) = \alert{Nutznießerdativ} (\alert{Ergänzung per Valenzerweiterung})
    \item (\ref{ex:dativeundindirekteobjekte170x}) = \alert{Pertinenzdativ} (\alert{Ergänzung per Valenzerweiterung})
      \Halbzeile
    \item Bewertungsdativ, Nutznießerdativ und Pertinenzdativ\\
      nennt man auch \textit{freie Dative}.
  \end{itemize}
\end{frame}

\begin{frame}
  {Valenzveränderungen im Beispiel}
  \pause
  1.~Wir beginnen mit einem Verb mit \alert{Nom\_Ag} und einem \orongsch{Akk}:\\
  \pause
  \Halbzeile
  \begin{exe}
    \ex \alert{Alma} mäht \orongsch{den Rasen}.
  \end{exe}
  \Zeile
  \pause
  2.~Der \gruen{Nutznießerdativ} wird als Valenzerweiterung hinzugefügt:\\
  \pause
  \Halbzeile
  \begin{exe}
    \ex \alert{Alma} mäht \gruen{meinem Kollegen} \orongsch{den Rasen}.
  \end{exe}
  \Zeile
  \pause
  3.~Das Rezipientenpassiv (Valenzänderung) kann jetzt gebildet werden:
  \pause
  \Halbzeile
  \begin{exe}
    \ex \gruen{Mein Kollege} bekommt \alert{(von Alma)} \orongsch{den Rasen} gemäht.
  \end{exe}
\end{frame}

\section{Statusrektion}

\begin{frame}
  {Statusrektion | Verben regieren Verben}
  \onslide<+->
  \begin{itemize}[<+->]
    \item bisher | \alert{nominale} und \alert{präpositionale} Objekte
    \item andere Verben | \alert{Statusrektion}, valenzgebundene infinite Verben
      \Zeile
    \item die drei Status des infiniten Verbs
      \begin{itemize}[<+->]
        \item \tuerkis{1.~Status} | reiner Infinitiv (\textit{kaufen})
        \item \gruen{2.~Status} | Infinitiv mit \textit{zu} (\textit{zu kaufen})
        \item \rot{3.~Status} | Partizip
      \end{itemize}
      \Zeile
    \item Die folgende Zusammenfassung ist nicht exhaustiv!
  \end{itemize}
\end{frame}

\begin{frame}
  {Valenzgebundener 3. Status}
  \onslide<+->
  \onslide<+->
  \begin{exe}
    \ex Nadezhda \alert{hat} meine Hantel \rot{signiert}.
    \ex Nadezhda \alert{ist} zur Siegerehrung \rot{gegangen}.
    \Halbzeile
    \onslide<+->
    \ex Nadezhda \orongsch{wurde} mit meiner Hantel \rot{fotografiert}.
  \end{exe}
  \Zeile
  \begin{itemize}[<+->]
    \item \alert{Perfekt-Hilfsverben (\textit{haben}\slash \textit{sein})} regieren \rot{3.~Status}.
    \item Das \orongsch{Passiv-Hilfsverb (\textit{werden})} regiert ebenfalls \rot{3.~Status}.
  \end{itemize}
\end{frame}


\begin{frame}
  {Valenzgebundener 2.~Status}
  \onslide<+->
  \onslide<+->
  \begin{exe}
    \ex Der Hufschmied \alert{beschließt} die Pferde \gruen{zu behufen}.
    \ex Der Hufschmied \alert{wünscht} die Pferde \gruen{zu behufen}.
    \Halbzeile
    \onslide<+->
    \ex Der Hufschmied \orongsch{scheint} die Pferde \gruen{zu behufen}.
  \end{exe}
  \begin{itemize}[<+->]
    \item Sog.\ \alert{Kontrollverben (\textit{beschließen}\slash \textit{wünschen} usw.)} regieren \gruen{2.~Status}.
    \item Sog.\ \orongsch{Halbmodalverben (\textit{scheinen})} regieren ebenfalls \gruen{2.~Status}.
  \end{itemize}
\end{frame}

\begin{frame}
  {Valenzgebundener 1.~Status}
  \onslide<+->
  \onslide<+->
  \begin{exe}
    \ex Der Hufschmied \alert{wird} die Pferde \tuerkis{behufen}.
    \Halbzeile
    \onslide<+->
    \ex Der Hufschmied \orongsch{möchte} die Pferde \tuerkis{behufen}.
    \ex Der Hufschmied \orongsch{kann} die Pferde \tuerkis{behufen}.
  \end{exe}
  \Zeile
  \begin{itemize}[<+->]
    \item Das \alert{Futur-Hilfsverb (\textit{werden})} regiert \tuerkis{1.~Status}.
    \item \orongsch{Modalverben (\textit{dürfen}, \textit{können}, \textit{mögen}, \textit{müssen}, \textit{sollen}, \textit{wollen})}\\
      regieren ebenfalls \tuerkis{1.~Status}.
  \end{itemize}
\end{frame}

\section{Verbklassen}

\begin{frame}
  {Gliederung des verbalen Lexikons I}
  \onslide<+->
  \onslide<+->
  Nominale\slash präpositionale Valenz:\\
  \begin{itemize}[<+->]
    \item \alert{Nominativ-Ergänzung} (Subjekt) oder nicht
    \item \alert{agentivischer Nominativ} oder nicht-agentivisches 
    \item erste \alert{Akkusativergänzung} (Objekt) oder nicht
    \item zweite Akkusativergänzung (Objekt)
    \item \alert{Dativergänzung} (Objekt) oder nicht
    \item \alert{Präpositionalergänzung} (Objekt) oder nicht
  \end{itemize}
\end{frame}

\begin{frame}
  {Gliederung des verbalen Lexikons II}
  \onslide<+->
  \onslide<+->
  Verben auf der Valenzliste\slash Statusrektion:\\
  \begin{itemize}[<+->]
    \item \tuerkis{1.~Status} (Hilfsverben, Modalverben)
    \item \gruen{2.~Status} (Kontrollverben, Halbmodalverben)
    \item \rot{3.~Status} (Hilfsverben)
  \end{itemize}
\end{frame}

\section{Zur nächsten Woche | Überblick}

\begin{frame}
  {Morphologie und Lexikon des Deutschen | Plan}
  \rot{Alle} angegebenen Kapitel\slash Abschnitte aus \rot{\citet{Schaefer2018b}} sind \rot{Klausurstoff}!\\
  \Halbzeile
  \begin{enumerate}
    \item Grammatik und Grammatik im Lehramt (Kapitel 1 und 3)
    \item Morphologie und Grundbegriffe (Kapitel 2, Kapitel 7 und Abschnitte 11.1--11.2)
    \item Wortklassen als Grundlage der Grammatik (Kapitel 6)
    \item Wortbildung | Komposition (Abschnitt 8.1)
    \item Wortbildung | Derivation und Konversion (Abschnitte 8.2 und 8.3)
    \item Flexion | Nomina außer Adjektiven (Abschnitte 9.1--9.3)
    \item Flexion | Adjektive und Verben (Abschnitt 9.4 und Kapitel 10)
    \item Valenz (Abschnitte 2.3, 14.1 und 14.3)
    \item Verbtypen als Valenztypen (Abschnitte 14.4, 14.5, 14.7--14.9) 
    \item \rot{Kernwortschatz und Fremdwort (vorwiegend Folien)}
  \end{enumerate}
  \Halbzeile
  \centering 
  \url{https://langsci-press.org/catalog/book/224}
\end{frame}


  \let\subsection\section\let\section\woopsi
  
  \section{Kernwortschatz und Fremdwort}
  \let\woopsi\section\let\section\subsection\let\subsection\subsubsection
  \section{Überblick}

\begin{frame}
  {Fremdwort und\slash oder Erbwort}
  \onslide<+->
  \begin{itemize}[<+->]
    \item Entlehnung aus anderen Sprachen
    \item Fremdheit ungleich Entlehnung
    \item Definition Kernwortschatz
      \Zeile
    \item \citet{Eisenberg2018}, \citet{Schaefer2018b}\\
      \grau{Die meisten Beispiele hier entnommen aus \citet{Eisenberg2018}.}
      \Zeile
    \item Das Wichtigste für mich ist, dass Sie hier etwas\\
      über den \alert{Kernwortschatz} lernen -- im Kontrast zu den Fremdwörtern.
  \end{itemize}
\end{frame}

\section{Fremdwort}

\begin{frame}
  {Was kommt uns \textit{fremd} vor?}
  \onslide<+->
  \onslide<+->
  \begin{exe}
    \ex Herzmuskelentzündung, Säurebindungsmittel, Nebennierenschwäche
    \onslide<+->
    \Halbzeile
    \ex Hypolyseninsuffizienz, Thyroxintherapie, Osteoporoseminimierung
    \onslide<+->
    \Halbzeile
    \ex Herzrhythmusstörung, Plasmaeiweißbindung, Schilddrüsenunterfunktion
  \end{exe}
  \onslide<+->
  \Zeile
  \alert{Entlehnung} | Das Wort ist im überblickbaren historischen Rahmen\\
  nicht schon immer im Wortschatz, sondern wurde\\
  aus einer Gebersprache übernommen.\\
  \onslide<+->
  \Zeile
  Spielt das wirklich eine Rolle für den Eindruck von \alert{Fremdheit}?
\end{frame}

\begin{frame}
  {Lehn-\slash und Fremdwörter | Welche Wortklassen?}
  \onslide<+->
  \onslide<+->
  Welche Wortklassen…\\
  \Halbzeile
  \begin{itemize}[<+->]
    \item \ldots sind überhaupt \alert{aufnahmefähig}?
    \item \ldots sind mächtig genug für Prototyp und Abweichung?
    \item \ldots haben starke formale Prototypen?
  \end{itemize}
  \onslide<+->
  \Zeile
  \alert{Substantive} > \alert{Adjektive} > \alert{Verben} > \grau{Adverben} > Rest
\end{frame}

\begin{frame}
  {Vorab | Simplicia}
  \onslide<+->
  \onslide<+->
  Das \alert{einfache} Wort…\\
  \Halbzeile
  \begin{itemize}[<+->]
    \item keine erkennbare Ableitung (\alert{Haus}, \rot{häuslich})
    \item keine Komposition (\alert{Tür}, \rot{Türschloss})
    \item bei Verben | ohne Präfix? (\alert{laufen}, \orongsch{verlaufen})
      \Halbzeile
    \item \alert{Wir betrachten hier erstmal nur Simplizia.}
  \end{itemize}
      \Zeile
      \onslide<+->
  \rot{Achtung!} Terminologie!\\
  \Halbzeile
  \begin{itemize}[<+->]
    \item \alert{Simplex} (Singular)
    \item \alert{Simplicia} oder \alert{Simplizia} (Plural)
      \Halbzeile
    \item niemals \rot{*Simplicium} (Singular)
  \end{itemize}
\end{frame}

\section{Kernwortschatz}

\begin{frame}
  {Kernwortschatz | Substantive}
  \onslide<+->
  \onslide<+->
  \begin{exe}
    \ex Baum, Mensch, Strich, Hand, Frist, Buch, Kind
    \Halbzeile
    \onslide<+->
    \ex \alert{Maskulin} | Hase, Falke, Anker, Krater, Hobel, Igel, Graben, Faden
    \onslide<+->
    \ex \alert{Feminin} | Farbe, Hose, Elster, Kelter, Amsel, Sichel
    \onslide<+->
    \ex \alert{Neutral} | Auge, Erbe, Leder, Wasser, Kabel, Rudel, Becken, Wappen
  \end{exe}
  \onslide<+->
  \Zeile
  \begin{itemize}[<+->]
    \item im Singular einsilbig oder
    \item zweisilbige Trochäen, zweite Silbe enthält \alert{Schwa} (<e> bzw.\ [ə])
      \Halbzeile
    \item im Plural immer zweisilbig
  \end{itemize}
\end{frame}

\begin{frame}
  {Kernwortschatz | Adjektive}
  \onslide<+->
  \onslide<+->
  \begin{exe}
    \ex blau, heiß, klein, lang, nackt, schön, stolz, wild
    \onslide<+->
    \ex lose, müde, heiter, mager, edel, nobel, eben, offen
  \end{exe}
  \onslide<+->
  \Zeile
  Eigenschaften?\\
  \onslide<+->
  \Halbzeile
  Und in anderen Formen?
\end{frame}

\begin{frame}
  {Kernwortschatz | Verben}
  \onslide<+->
  \onslide<+->
  \begin{exe}
  \ex baden, denken, leben, schieben, stehen, tragen, wohnen
  \ex rudern, hadern, zetern, bügeln, jubeln, segeln
  \ex atmen, ordnen, öffnen, regnen, zeichnen
  \end{exe}
  \onslide<+->
  \Zeile
  Eigenschaften?\\
  \onslide<+->
  \Halbzeile
  Und in anderen Formen?
\end{frame}

\begin{frame}
  {Kernwortschatz | Lehnwörter, nicht fremd}
  \onslide<+->
  \onslide<+->
  \begin{exe}
    \ex \alert{Englisch} | Akte, Boss, Film, grillen, Lift, Rocker, sponsern, starten, streiken, Stress, tippen, Toner, Tunnel
    \onslide<+->
    \ex \alert{Französisch} | Bluse, Dame, Lärm, Möbel, Mode, nett, nobel, Onkel, Plüsch, Puder, Robe, Soße, Suppe, Tante, Tasse, Torte, Weste
    \onslide<+->
    \ex \alert{Italienisch} | Bank, Barke, Bratsche, Fuge, Kasse, Kurs, Kuppel, Lanze, Liste, Mole, Null, Oper, Paste, Posten, Putte, Reis, Rest
    \onslide<+->
    \ex \alert{Griechisch} | Arzt, Ball, Engel, Fieber, Leier, Ketzer, Kirche, Lesbe, Meter, Pfarrer, Pflaster, Sarg, taufen, Teufel, Tisch, Zone
    \onslide<+->
    \ex \alert{Lateinisch} | Eimer, Esel, Fenster, Kerker, krass, Kreuz, Küche, Mauer, Meile, Mühle, Schule, Straße, Wanne, Wein, Ziegel
    \onslide<+->
    \ex \alert{Hebräisch\slash Jiddisch} | Bammel, dufte, Jubel, Kaff, kotzen, koscher, Nepp, petzen, Ramsch, Zoff
  \end{exe}
\end{frame}

\begin{frame}
  {Fremdwort}
  \onslide<+->
  \onslide<+->
  \centering 
  \alert{Fremdwort} | Fremdwörter sind \alert{nicht im Kern des Systems}.\\
  Sie weichen von den (proto)typischen phonologischen, morphologischen\\
  oder graphematischen Mustern ab, denen die \alert{meisten Wörter} folgen.\\
  \onslide<+->
  \Zeile
  Fremdwörter sind oft intuitiv als \alert{fremd} erkennbar.\\
  \onslide<+->
  \Zeile
  Es gibt \alert{fremde Erbwörter} und \alert{nicht-fremde Lehnwörter}.
\end{frame}

\section{Gradueller Kern}

\begin{frame}
  {Genauer hingeschaut | \textit{Ramsch} usw.}
  \onslide<+->
  \onslide<+->
  Die folgenden Wörter sind nicht im ganz engen Kernwortschatz. Warum?\\
  \Zeile
  \begin{itemize}[<+->]
    \item Bratsche
    \item Bronze
    \item Arzt
    \item Fenster
    \item Ramsch
  \end{itemize}
  \onslide<+->
  \Zeile
  Es kommen jeweils \alert{extrem seltene Konsonantenverbindungen} vor.\\
  \onslide<+->
  Vergleiche \alert{\textit{Mensch}}.
\end{frame}

\begin{frame}
  {Nahe Fremd-\slash Lehnwörter | \textit{quasseln}, \textit{Bagger} usw.}
  \onslide<+->
  \onslide<+->
  Die folgenden Wörter sind Kernwortschatz nach der einfachen Definition.\\
  Wieso sind sie trotzdem ungewöhnlich bzw. vom Kern entfernt?\\
  \Zeile
  \onslide<+->
  \begin{exe}
    \ex Ebbe, Krabbe, kribbeln, Robbe, sabbern, schrubben
    \ex Buddel, Kladde, paddeln, Pudding, Widder
    \ex Bagger, Dogge, Egge, Flagge, Roggen
    \Halbzeile
    \ex quasseln (kontrastiere \textit{prasseln})
  \end{exe}
  \onslide<+->
  \Zeile
  \alert{Stimmhafte Obstruenten am Silbengelenk} sollte es nicht geben.\\
  Siehe Graphematik | Warum \textit{quasseln} besonders schwierig ist.
\end{frame}

\begin{frame}
  {Kern und Peripherie | Abstufungen}
  \onslide<+->
  \onslide<+->
  Was ist an diesen Wörtern etwas fremder als am innersten Kern?\\
  \Halbzeile
  \begin{exe}
    \ex Arbeit, Bischof, Echo, Efeu, Gulasch, Heimat, Oma, Pfirsich, Uhu
    \onslide<+->
    \ex Forelle, Holunder, Hornisse, Kaliber, Kamille, Marone, Maschine
    \onslide<+->
    \ex Ameise, Abenteuer, Akelei, Kehricht, Kleinod, Kobold, Nachtigall\\
    \onslide<+->
    \ex Azur, Bovist, Delfin, Granit, Kanal, Hermelin, Humor, Taifun, Topas
  \end{exe}
  \onslide<+->
  \Zeile
  \alert{Vollvokale} in Nebensilben, \alert{mehr als zwei Silben}, \alert{Pseudokomposita}, \alert{Endsilbenbetonung}.\\
  \Halbzeile
  \onslide<+->
  Welche von diesen Wörtern sind entlehnt?
\end{frame}

\begin{frame}
  {Sind Lehn-\slash Fremdwörter kein Deutsch?}
  \onslide<+->
  \onslide<+->
  Eine Anekdote aus meinem Japanologie-Studium (1998 Bochum):\\
  \Viertelzeile
  \textit{\rot{"`Diphthong ist ein griechisches Wort!} Es wird nach dem Präfix Di- getrennt!"'}\\
  \onslide<+->
  \Viertelzeile
  → \rot{Unsinn!} \grau{Auch wenn die Trennung nach \textit{Di-} bildungssprachlich zu empfehlen ist.}\\
  \onslide<+->
  \Zeile
  Sprechen wir \ldots
  \begin{itemize}[<+->]
    \item \ldots\ Japanisch beim \alert{Sushi}?
    \item \ldots\ Italienisch beim \alert{Cappuccino}?
    \item \ldots\ Französisch beim \alert{Soufflet}?
    \item \ldots\ Englisch beim \alert{Burger}?
  \end{itemize}
  \Halbzeile
  \onslide<+->
  Natürlich nicht. Die Wörter wurden \alert{ins Deutsche entlehnt und sind Deutsch}.\\
  \Viertelzeile
  Auch \alert{Kern und Peripherie} sind nicht mehr oder weniger Deutsch.
\end{frame}

\section{Fremde Wortbildung}

\begin{frame}
  {Lehnwortbildung und Stämme}
  \onslide<+->
  \onslide<+->
  Besonders bei Lehnwortbildungen | Der \alert{Stamm} ist oft selber \alert{nicht wortfähig}.\\
  \Zeile
  \onslide<+->
  \alert{Provider} ist ein deutsches Wort. \onslide<+-> Aber \rot{*provide(n)} ist es nicht.\\
  \Viertelzeile
  \onslide<+->
  Ähnlich ist es bei \alert{Clearing} und \rot{*clear(en)}.\\
  \onslide<+->
  \Zeile
  Inwiefern solche Bildungen als Wortbildungen wahrgenommen werden,\\
  ist schwer und ggf.\ nur im Einzelfall zu entscheiden.
\end{frame}

\begin{frame}
  {Anglizistische Wortbildung | \orongsch{-er}}
  \onslide<+->
  \onslide<+->
  \begin{exe}
    \ex \alert{Kernwörter} | Denker, Fälscher, Leser, Schläger, Turner
    \onslide<+->
    \Zeile
    \ex \alert{Anglizismen} | Beater, Camper, Carrier, Catcher, Dealer, Globetrotter, Hacker, Hitchhiker, Jazzer, Jobber, Jogger, Keeper, Killer, Manager, Producer, Promoter, Provider, Pusher, Surfer, Swinger, User, Walker
  \end{exe}
  \onslide<+->
  \Zeile
  \begin{itemize}[<+->]
    \item Sind die Bildungen \alert{fremd} im Sinn des Nicht-Kerns?
    \item Beziehen Sie sich für Einzelwörter auch auf einzelne der vorkommenden Laute.
  \end{itemize}
\end{frame}

\begin{frame}
  {Anglizistische Wortbildung | \orongsch{-ing}}
  \onslide<+->
  \onslide<+->
  \begin{exe}
    \ex Boarding, Clearing, Coaching, Dumping, Jogging, Mailing, Recycling,
Scratching, Skimming, Shopping, Surfing
    \onslide<+->
    \Halbzeile
    \ex Bodybuilding, Canyoning, Dribbling, Forechecking, Nordic Walking, Slacklining, Tackling, Trekking
  \end{exe}
  \onslide<+->
  \Zeile
  \begin{itemize}[<+->]
    \item Was unterscheidet die erste von der zweiten Gruppe?
    \item Welche Stämme sind \alert{wortfähig}?
    \item Bei wortfähigen Stämmen | Können Sie sich vorstellen,\\
      dass \alert{zuerst das abgeleitete Wort entlehnt wurde} und\\
      der Stamm nachträglich abgetrennt wurde?
  \end{itemize}
\end{frame}

\begin{frame}
  {Einige gallizistische Wortbildungsmuster I}
  \onslide<+->
  \onslide<+->
  \begin{exe}
    \ex \alert{Adjektive auf esk}
    \begin{xlist}
      \ex \small arabesk, balladesk, burlesk, clownesk, gigantesk, karnevalesk, karrikaturesk, pittoresk, romanesk
      \ex \small chaplinesk, dantesk, donjuanesk, godardesk, goyaesk, hoffmannesk, kafkaesk, zappaesk
    \end{xlist}
    \onslide<+->
    \ex \alert{Adjektive auf ös}
    \begin{xlist}
      \ex \small bravourös, desaströs, fibrös, medikamentös, monströs, nervös, pompös, porös, ruinös, schikanös, skandalös, venös, virös
      \ex \small graziös, infektiös, minutiös, sentenziös, tendenziös
      \ex \small bituminös, libidinös, mirakulös, muskulös, nebulös, tuberkulös, voluminös
      \ex \small leprös, kariös, dubiös, ingeniös, kapriziös, luxuriös, melodiös, mysteriös
    \end{xlist}
  \end{exe}
  \onslide<+->
  \Viertelzeile
  Siehe auch Adjektive auf \alert{är}.
\end{frame}

% \begin{frame}
%   {Einige gallizistische Wortbildungsmuster II}
%   \begin{exe}
%   \ex {Adjektive auf är}
%   \begin{xlist}
%     \ex doktrinär, familiär, legendär, reaktionär, sekundär, singulär, stationär, visionär
%     \ex intermediär, konträr, radiär, sekulär
%     \ex muskulär, regulär, zirkulär, zellulär
%     \ex arbiträr, binär, ordinär, pekuniär, sanitär, solitär, subsidiär, temporär
%   \end{xlist}
%     \ex
%   \end{exe}
% \end{frame}

\begin{frame}
  {Einige gallizistische Wortbildungsmuster II}
  \onslide<+->
  \onslide<+->
  \begin{exe}
    \ex \alert{Substantive auf age}
    \begin{xlist}
      \ex \small Blamage, Karambolage, Massage, Montage, Passage, Reportage,\\
      Sabotage, Spionage
      \ex \small Bandage, Collage, Dränage, Etage, Garage, Passage, Plantage,\\
      Reportage, Trikotage
    \end{xlist}
    \onslide<+->
    \ex \alert{Substantive auf eur}
    \begin{xlist}
      \ex \small Akteur, Bankrotteur, Charmeur, Kontrolleur, Parfümeur, Rechercheur
      \ex \small Arrangeur, Chauffeur, Deserteur, Flaneur, Friseur, Hasardeur, Hypnotiseur, Jongleur, Kommandeur, Masseur, Monteur, Saboteur, Souffleur
      \ex \small Installateur, Konstrukteur, Operateur, Provokateur, Redakteur, Restaurateur, Spediteur
    \end{xlist}
  \end{exe}
  \onslide<+->
  \Viertelzeile
  Siehe auch Nomina auf \alert{ee}
\end{frame}

\section{Zur nächsten Woche | Überblick}

\begin{frame}
  {Morphologie und Lexikon des Deutschen | Plan}
  \rot{Alle} angegebenen Kapitel\slash Abschnitte aus \rot{\citet{Schaefer2018b}} sind \rot{Klausurstoff}!\\
  \Halbzeile
  \begin{enumerate}
    \item \rot{Grammatik und Grammatik im Lehramt (Kapitel 1 und 3)}
    \item \rot{Morphologie und Grundbegriffe (Kapitel 2, Kapitel 7 und Abschnitte 11.1--11.2)}
    \item \rot{Wortklassen als Grundlage der Grammatik (Kapitel 6)}
    \item \rot{Wortbildung | Komposition (Abschnitt 8.1)}
    \item \rot{Wortbildung | Derivation und Konversion (Abschnitte 8.2 und 8.3)}
    \item \rot{Flexion | Nomina außer Adjektiven (Abschnitte 9.1--9.3)}
    \item \rot{Flexion | Adjektive und Verben (Abschnitt 9.4 und Kapitel 10)}
    \item \rot{Valenz (Abschnitte 2.3, 14.1 und 14.3)}
    \item \rot{Verbtypen als Valenztypen (Abschnitte 14.4, 14.5, 14.7--14.9)}
    \item \rot{Kernwortschatz und Fremdwort (vorwiegend Folien)}
  \end{enumerate}
  \Halbzeile
  \centering 
  \url{https://langsci-press.org/catalog/book/224}
\end{frame}




  \let\subsection\section\let\section\woopsi

  \section{Vor der Klausur}

  \begin{frame}
    {Morphologie und Lexikon des Deutschen | Plan}
    \rot{Alle} angegebenen Kapitel\slash Abschnitte aus \rot{\citet{Schaefer2018b}} sind \rot{Klausurstoff}!\\
    \Halbzeile
    \begin{enumerate}
      \item Grammatik und Grammatik im Lehramt \rot{(Kapitel 1 und 3)}
      \item Morphologie und Grundbegriffe \rot{(Kapitel 2, Kapitel 7 und Abschnitte 11.1--11.2)}
      \item Wortklassen als Grundlage der Grammatik \rot{(Kapitel 6)}
      \item Wortbildung | Komposition \rot{(Abschnitt 8.1)}
      \item Wortbildung | Derivation und Konversion \rot{(Abschnitte 8.2--8.3)}
      \item Flexion | Nomina außer Adjektiven \rot{(Abschnitte 9.1--9.3)}
      \item Flexion | Adjektive und Verben \rot{(Abschnitt 9.4 und Kapitel 10)}
      \item Valenz \rot{(Abschnitte 2.3, 14.1 und 14.3)}
      \item Verbtypen als Valenztypen \rot{(Abschnitte 14.4--14.5, 14.7--14.9)}
      \item Kernwortschatz und Fremdwort \rot{(vorwiegend Folien)}
    \end{enumerate}
    \Halbzeile
    \centering 
    \url{https://langsci-press.org/catalog/book/224}
  \end{frame}

\fi

\makeatletter
\setcounter{lastpagemainpart}{\the\c@framenumber}
\makeatother

\appendix

\begin{frame}[allowframebreaks]
  {Literatur}
  \renewcommand*{\bibfont}{\footnotesize}
  \setbeamertemplate{bibliography item}{}
  \printbibliography
\end{frame}

\begin{frame}
  {Autor}
  \begin{block}{Kontakt}
    Prof.\ Dr.\ Roland Schäfer\\
    Institut für Germanistische Sprachwissenschaft\\
    Friedrich-Schiller-Universität Jena\\
    Fürstengraben 30\\
    07743 Jena\\[\baselineskip]
    \url{https://rolandschaefer.net}\\
    \texttt{roland.schaefer@uni-jena.de}
  \end{block}
\end{frame}

\begin{frame}
  {Lizenz}
  \begin{block}{Creative Commons BY-SA-3.0-DE}
    Dieses Werk ist unter einer Creative Commons Lizenz vom Typ \textit{Namensnennung - Weitergabe unter gleichen Bedingungen 3.0 Deutschland} zugänglich.
    Um eine Kopie dieser Lizenz einzusehen, konsultieren Sie \url{http://creativecommons.org/licenses/by-sa/3.0/de/} oder wenden Sie sich brieflich an Creative Commons, Postfach 1866, Mountain View, California, 94042, USA.
  \end{block}
\end{frame}

\mode<beamer>{\setcounter{framenumber}{\thelastpagemainpart}}

\end{document}
