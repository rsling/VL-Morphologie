% Communication with make =============================================

\def\GRAPHPATH{localgraphics}

\ifdefined\HANDOUT
  \documentclass[handout,aspectratio=1610,dvipsnames]{beamer}
  \def\GRAPHPATH{graphics}
\else
  \documentclass[aspectratio=1610,dvipsnames]{beamer}
\fi

\ifdefined\TITLE
\else
  \def\TITLE{}
\fi

\usepackage[ngerman]{babel}
\usepackage{ifthen}
\usepackage{color}
\usepackage{colortbl}
\usepackage{textcomp}
\usepackage{multirow}
\usepackage{nicefrac}
\usepackage{multicol}
\usepackage{langsci-gb4e}
\usepackage{verbatim}
\usepackage{cancel}
\usepackage{graphicx}
\usepackage{hyperref}
\usepackage{verbatim}
\usepackage{boxedminipage}
\usepackage{adjustbox}
\usepackage{rotating}
\usepackage{booktabs}
\usepackage{bbding}
\usepackage{pifont}
\usepackage{multicol}
\usepackage{stmaryrd}
\usepackage{FiraSans}
\usepackage{soul}
\usepackage{tikz}

\usetikzlibrary{calc,decorations.pathmorphing,tikzmark,positioning,chains,trees,graphs,shapes,shadows,arrows}

\renewcommand\tikzmark[2]{%
  \tikz[remember picture,baseline=(chain-1.base),start chain] \node[on chain,inner sep=2pt,outer sep=0] (#1){#2};%
}

\newcommand\link[2]{%
  \begin{tikzpicture}[remember picture, overlay, >=stealth, shift={(0,0)}]
    \draw[-] (#1) --++(0,-12pt) -| (#2);
   \end{tikzpicture}%
}


\usepackage{tikz-qtree}
\usepackage[linguistics]{forest}
\usepackage[maxbibnames=99,
  maxcitenames=2,
  uniquelist=false,
  backend=biber,
  doi=false,
  url=false,
  isbn=false,
  bibstyle=biblatex-sp-unified,
  citestyle=sp-authoryear-comp]{biblatex}

% Biblatex ============================================================

\addbibresource{rs.bib}

% Colors ==============================================================

\definecolor{grau}{rgb}{0.5,0.5,0.5}
\definecolor{lg}{rgb}{0.8,0.8,0.8}
\definecolor{trueblue}{rgb}{0.3,0.3,1}
\definecolor{ltb}{rgb}{0.8,0.8,1}
\definecolor{lgr}{rgb}{0.5,1,0.5}
\definecolor{orongsch}{RGB}{255,165,0}
\definecolor{gruen}{rgb}{0,0.4,0}
\definecolor{rot}{rgb}{0.7,0.2,0.0}
\definecolor{tuerkis}{RGB}{63,136,143}
\definecolor{braun}{RGB}{108,71,65}
\definecolor{blaw}{rgb}{0,0,0.9}
\newcommand{\gruen}[1]{\textcolor{gruen}{#1}}
\newcommand{\blaw}[1]{\textcolor{blaw}{#1}}
\newcommand{\rot}[1]{\textcolor{rot}{#1}}
\newcommand{\blau}[1]{\textcolor{trueblue}{#1}}
\newcommand{\orongsch}[1]{\textcolor{orongsch}{#1}}
\newcommand{\grau}[1]{\textcolor{grau}{#1}}
\newcommand{\whyte}[1]{\textcolor{white}{#1}}
\newcommand{\tuerkis}[1]{\textcolor{tuerkis}{#1}}
\newcommand{\braun}[1]{\textcolor{braun}{#1}}

% Newcommands =========================================================

\newcommand{\Dim}{\cellcolor{lg}}
\newcommand{\Dimblue}{\cellcolor{ltb}}
\newcommand{\Dimgreen}{\cellcolor{lgr}}
\newcommand{\Sub}[1]{\ensuremath{_{\text{#1}}}}
\newcommand{\Up}[1]{\ensuremath{^{\text{#1}}}}
\newcommand{\UpSub}[2]{\ensuremath{^{\text{#1}}_{\text{#2}}}}
\newcommand{\Spur}[1]{t\Sub{#1}}
\newcommand{\Ti}{\Spur{1}}
\newcommand{\Tii}{\Spur{2}}
\newcommand{\Tiii}{\Spur{3}}
\newcommand{\Tiv}{\Spur{4}}
\newcommand{\Ck}{\CheckmarkBold}
\newcommand{\Fl}{\XSolidBrush}
\newcommand{\xxx}{\hspaceThis{[}}
\newcommand{\zB}{z.\,B.\ }
\newcommand{\down}[1]{\ensuremath{\mathrm{#1}}}
\newcommand{\Zeile}{\vspace{\baselineskip}}
\newcommand{\Halbzeile}{\vspace{0.5\baselineskip}}
\newcommand{\Viertelzeile}{\vspace{0.25\baselineskip}}
\newcommand{\KTArr}[1]{\ding{226}~\textit{#1}~\ding{226}}
\newcommand{\Ast}{*}
\newcommand{\SL}{\ensuremath{\llbracket}}
\newcommand{\SR}{\ensuremath{\rrbracket}}
\def\lspbottomrule{\bottomrule}
\def\lsptoprule{\toprule}
\newcommand{\Sw}[1]{\begin{sideways}#1\end{sideways}}
\newcommand{\Lab}{\ensuremath{\langle}}
\newcommand{\Rab}{\ensuremath{\rangle}}
\newcommand{\AbUmlautBreaker}{}
\ifdefined\HANDOUT
  \renewcommand{\AbUmlautBreaker}{\ /}
\fi
\newcommand{\LocStrutGrph}{\hspace{0.1\textwidth}}
\newcommand{\Nono}{---}

% Beamer ==============================================================

\usetheme[hideothersubsections]{PaloAlto}

\renewcommand<>{\rot}[1]{%
  \alt#2{\beameroriginal{\rot}{#1}}{#1}%
}
\renewcommand<>{\blau}[1]{%
  \alt#2{\beameroriginal{\blau}{#1}}{#1}%
}
\renewcommand<>{\orongsch}[1]{%
  \alt#2{\beameroriginal{\orongsch}{#1}}{#1}%
}
\renewcommand<>{\gruen}[1]{%
  \alt#2{\beameroriginal{\gruen}{#1}}{#1}%
}

\setbeamercolor{alerted text}{fg=trueblue}

\addtobeamertemplate{navigation symbols}{}{%
    \usebeamerfont{footline}%
    \usebeamercolor[fg]{footline}%
    \hspace{1em}%
    \insertframenumber/\inserttotalframenumber
}

\newcounter{lastpagemainpart}

\resetcounteronoverlays{exx}

\AtBeginSection[]{
  \begingroup
  \setbeamertemplate{navigation symbols}{}
  \begin{frame}[noframenumbering]
  \vfill
  \centering
  \begin{beamercolorbox}[sep=8pt,center,shadow=true,rounded=true]{title}
    \usebeamerfont{title}\insertsectionhead\par%
  \end{beamercolorbox}
  \vfill
  \end{frame}
  \endgroup
}

\setbeamertemplate{navigation symbols}{\insertframenumber/\inserttotalframenumber\hspace{5em}}

% Tikz ================================================================

\usetikzlibrary{positioning,arrows,cd}
\tikzset{>=latex}

% Forest

\forestset{
  Ephr/.style={draw, ellipse, thick, inner sep=2pt},
  Eobl/.style={draw, rounded corners, inner sep=5pt},
  Eopt/.style={draw, rounded corners, densely dashed, inner sep=5pt},
  Erec/.style={draw, rounded corners, double, inner sep=5pt},
  Eoptrec/.style={draw, rounded corners, densely dashed, double, inner sep=5pt},
  Ehd/.style={rounded corners, fill=gray, inner sep=5pt,
    delay={content=\whyte{##1}}
  },
  Emult/.style={for children={no edge}, for tree={l sep=0pt}},
  phrasenschema/.style={for tree={l sep=2em, s sep=2em}},
  decide/.style={draw, chamfered rectangle, inner sep=2pt},
  finall/.style={rounded corners, fill=gray, text=white},
  intrme/.style={draw, rounded corners},
  yes/.style={edge label={node[near end, above, sloped, font=\scriptsize]{Ja}}},
  no/.style={edge label={node[near end, above, sloped, font=\scriptsize]{Nein}}},
  sake/.style={tier=preterminal},
  ake/.style={
    tier=preterminal
    },
}

\tikzset{
    invisible/.style={opacity=0,text opacity=0},
    visible on/.style={alt=#1{}{invisible}},
    alt/.code args={<#1>#2#3}{%
      \alt<#1>{\pgfkeysalso{#2}}{\pgfkeysalso{#3}}
    },
}

\forestset{
  visible on/.style={
    for tree={
      /tikz/visible on={#1},
      edge+={/tikz/visible on={#1}}}}}

\useforestlibrary{edges}

\forestset{
  narroof/.style={roof, inner xsep=-0.25em, rounded corners},
  forky/.style={forked edge, fork sep-=7.5pt},
  bluetree/.style={for tree={trueblue}, for children={edge=trueblue}},
  orongschtree/.style={for tree={orongsch}, for children={edge=orongsch}},
  rottree/.style={for tree={rot}, for children={edge=rot}},
  gruentree/.style={for tree={gruen}, for children={edge=gruen}},
  tuerkistree/.style={for tree={tuerkis}, for children={edge=tuerkis}},
  brauntree/.style={for tree={braun}, for children={edge=braun}},
}


% Drawing sonority diagrams =========================================== 

\makeatletter

\long\def\ifnodedefined#1#2#3{%
  \@ifundefined{pgf@sh@ns@#1}{#3}{#2}}

\newcommand\aeundefinenode[1]{%
  \expandafter\ifx\csname pgf@sh@ns@#1\endcsname\relax
  \else
    \typeout{Undefining node "#1"}%
    \global\expandafter\let\csname pgf@sh@ns@#1\endcsname\relax
  \fi
}

\newcommand\aeundefinethesenodes[1]{%
  \foreach \myn  in {#1}
    {%
      \ifnodedefined{\myn}{%
      \expandafter\aeundefinenode\expandafter{\myn}%
    }{}
    }%
}

\newcommand\aeundefinenumericnodes{%
  \foreach \myn in {1,2,...,50}
    {%
      \ifnodedefined{\myn}{%
      \expandafter\aeundefinenode\expandafter{\myn}%
    }{}
    }%
}
\makeatother

\newcommand{\plo}{0}
\newcommand{\fri}{0.5}
\newcommand{\nas}{1}
\newcommand{\liq}{1.5}
\newcommand{\vok}{2}

% Save text.
\newcommand{\lastsaved}{}
\newcommand{\textsave}[1]{\gdef\lastsaved{#1}#1}

\newcommand{\SonDiag}[2][0]{%
  \begin{tikzpicture}
    \textsave{.}
    \tikzset{
      normalseg/.style={fill=white},
      extrasyll/.style={circle, draw, fill=white},
      sylljoint/.style={diamond, draw, fill=white}
    }
    \node at (0,\plo) {P};
    \node at (0,\fri) {F};
    \node at (0,\nas) {N};
    \node at (0,\liq) {L};
    \node at (0,\vok) {V};

    % Draw the helper lines if required.
    \ifthenelse{\equal{#1}{0}}{}{%
      \foreach \y in {\plo, \fri, \nas, \liq,\vok} {%
	\draw [dotted, |-|] (0.25, \y) -- (#1.75, \y);
      }
    }

    \foreach [count=\x from 1, remember=\x as \lastx] \p / \y / \g in #2 {
      \ifthenelse{\equal{\y}{-1}}{\textsave{.}}{%

	% Draw the node, either plain, as Silbenbgelenk, or as extrasyllabic.
        \ifthenelse{\equal{\g}{1}}{%
	  \node (\x) [sylljoint] at (\x, \y) {\p};
	}{%
	  \ifthenelse{\equal{\g}{2}}{%
	    \node (\x) [extrasyll] at (\x, \y) {\p};
	  }{%
	    \node (\x) [normalseg] at (\x, \y) {\p};
	  }
	}

	% Draw the connection unless the previous node was not or was empty.
	\ifthenelse{\NOT\equal{\lastsaved}{.}}{%
	  \draw [->] (\lastx) to (\x);
	}{}
	\textsave{1}
      }
    }
    \aeundefinenumericnodes
  \end{tikzpicture}
}


% Meta ================================================================

\title[Morphologie, Lexikon]{Einführung in die Morphologie und Lexikologie\\\TITLE}
\author{Roland Schäfer}
\institute{Institut für Germanistische Sprachwissenschaft\\Friedrich-Schiller-Universität Jena}
\date{Diese Version ist vom \today.\\\Zeile%
  \scriptsize \grau{stets aktuelle Fassungen: \url{https://github.com/rsling/SE-Einfuehrung-in-die-Morphologie-und-Lexikologie}}}

\begin{document}

\begingroup
\setbeamertemplate{navigation symbols}{}
\begin{frame}[noframenumbering]
 \titlepage
\end{frame}
\endgroup

\ifdefined\LECTURE
  \include{includes/\LECTURE}
\else

  \makeatletter
  \setbeamertemplate{section in sidebar}{\vspace{0.25\baselineskip}\vbox{%
      \beamer@sidebarformat{3pt}{section in sidebar}{\insertsectionhead}\vspace{-0.25\baselineskip}}}
  \setbeamertemplate{section in sidebar shaded}{\vspace{0.25\baselineskip}\vbox{%
      \beamer@sidebarformat{3pt}{section in sidebar shaded}{\insertsectionhead}\vspace{-0.25\baselineskip}}}
  \setbeamertemplate{subsection in sidebar}{\hspace{1em}\vbox{%
    \beamer@sidebarformat{3pt}{subsection in sidebar}{\insertsubsectionhead}\vspace{-0.5\baselineskip}}}
  \setbeamertemplate{subsection in sidebar shaded}{\hspace{1em}\vbox{%
      \beamer@sidebarformat{3pt}{subsection in sidebar shaded}{\insertsubsectionhead}\vspace{-0.5\baselineskip}}}
  \makeatother

%   \begin{frame}
%     {Stand der Überarbeitung}
%     \begin{center}
%       \Large Dieser Foliensatz ist erst bis\\
%       einschließlich \alert{Vorlesung 10}\\
%       für das Wintersemester 2019\slash 2020 überarbeitet.
%     \end{center}
%   \end{frame}

  \section[Sprache]{Sprache \& Sprache und Lehramt}
  \let\woopsi\section\let\section\subsection\let\subsection\subsubsection
  \section{Satzglieder}\label{sec:satzglieder}

Bestimmen Sie für die unterstrichenen Teile in den folgenden Sätzen, ob sie Satzglieder sind, indem Sie den \textbf{Vorfeld-Test} anwenden und alle Satzglieder im vorgesehenen Feld ankreuzen.
Zur Erinnerung: Beim Vorfeld-Test versuchen Sie, den Satz so umzustellen, dass das potentielle Satzglied vor dem finiten Verb zu stehen kommt.
Finite Verben sind diejenigen Verben, die nach Tempus (Präsens\slash Präteritum), Numerus (Singular\slash Plural) und Person (1\slash 2\slash 3) flektiert sind.

\Halbzeile

\Sol{Die angekreuzten Fälle sind die, für die der Test erfolgreich verläuft.
Es handelt sich dabei nicht immer um Satzglieder gemäß Schulgrammatik, da der Begriff schulgrammatisch schlicht nicht hinreichend genau definiert und operationalisiert ist.}

\begin{enumerate}
  \item \Solalt{\XBox}{\Square}\ul{Der Winter} ist vorbei.\Sol{
      \begin{enumerate}
        \item Der Winter ist vorbei.\\
          \blau{Steht bereits im Vorfeld.}
      \end{enumerate}
    }
  \item Otje schickt \Solalt{\Square}{\Square}~\ul{seinen Kindern aus} dem Urlaub \Solalt{\XBox}{\Square}~\ul{eine Karte}.\Sol{
      \begin{enumerate}
        \item * Seinen Kindern aus schickt Otje dem Urlaub eine Karte.
        \item Eine Karte schickt Otje seinen Kindern aus dem Urlaub.
      \end{enumerate}
    }
  \item Wir kaufen öfter \Solalt{\Square}{\Square}\ul{Produkte}, die regional gefertigt wurden.\Sol{
      \begin{enumerate}
        \item * Produkte kaufen wir öfter, die regional gefertigt wurden.
      \end{enumerate}
    }
  \item K.\,R.\ Popper ist \Solalt{\XBox}{\Square}\ul{der Philosoph, auf dessen Werken alle falsifikationistischen Wissenschaftstheorien basieren}.\Sol{
      \begin{enumerate}
        \item Der Philosoph, auf dessen Werken alle falsifikationistischen Wissenschaftstheorien basieren, ist K.\,R.\ Popper.
      \end{enumerate}
    }
  \item Zu dieser Jahreszeit gibt es keine \Solalt{\Square}{\Square}\ul{regionalen} Erdbeeren \Solalt{\XBox}{\Square}\ul{in Deutschland}.\Sol{
      \begin{enumerate}
        \item * Regionalen gibt es zu dieser Jahreszeit keine Erdbeeren in Deutschland.
        \item In Deutschland gibt es zu dieser Jahreszeit keine regionalen Erdbeeren.
      \end{enumerate}
    }
  \item Ich glaube \Solalt{\XBox}{\Square}\ul{überhaupt nicht}, \Solalt{\XBox}{\Square}\ul{dass ein solcher Unsinn überhaupt ernstgenommen wird}.\Sol{
      \begin{enumerate}
        \item Überhaupt nicht glaube ich, dass ein solcher Unsinn überhaupt ernstgenommen wird.
        \item Dass ein solcher Unsinn überhaupt ernstgenommen wird, glaube ich überhaupt nicht.
      \end{enumerate}
    }
  \item Alle Wissenschaftler möchten gerne \Solalt{\XBox}{\Square}\ul{einen großen Erfolg für sich verbuchen} können.\Sol{
      \begin{enumerate}
        \item Einen großen Erfolg für sich verbuchen möchten alle Wissenschaftler gerne können.\\
          \blau{Schulgrammatisch handelt es sich nicht um ein Satzglied, aber es ist trotzdem vorfeldfähig.}
      \end{enumerate}
    }
  \item Man darf \Solalt{\Square}{\Square}\ul{seinen Hund} \Solalt{\Square}{\Square}\ul{beim Einkaufen} nicht \Solalt{\Square}{\Square}\ul{im Auto} zurücklassen.\Sol{
      \begin{enumerate}
        \item Seinen Hund darf man beim Einkaufen nicht im Auto zurücklassen.
        \item Beim Einkaufen darf man seinen Hund nicht im Auto zurücklassen.
        \item Im Auto darf man seinen Hund beim Einkaufen nicht zurücklassen.
      \end{enumerate}
    }
  \item \Solalt{\Square}{\Square}\ul{Heute} hat es keinen Zweck, \Solalt{\Square}{\Square}\ul{rudern zu gehen}.\Sol{
      \begin{enumerate}
        \item Heute hat es keinen Zweck, rudern zu gehen.\\
          \blau{Steht bereits im Vorfeld.}
        \item * Rudern zu gehen, hat es heute keinen Zweck.\\
          \blau{Der Infinitiv soll eigentlich ein Satzglied sein. Der Test funktioniert hier aber nicht, weil das sogenannte Korrelat des \textit{zu}-Infinitivs -- nämlich \textit{es} -- nicht nach dem Infinitiv stehen darf. Man muss \textit{es} weglassen, damit es funktioniert.}
      \end{enumerate}
    }
  \item Der abgewählte Präsident goss bei einer Wahlkampfveranstaltung \Solalt{\XBox}{\Square}\ul{Öl ins Feuer}.\Sol{
      \begin{enumerate}
        \item Öl ins Feuer goss der abgewählte Präsident bei einer Wahlkampfveranstaltung.\\
          \blau{Auch dieses scheinbar doppelt besetzte Vorfeld sollte es gemäß der Schulgrammatik nicht geben. Jedenfalls soll \textit{Öl ins Feuer} kein Satzglied sein.}
      \end{enumerate}
    }
  \item \Solalt{\XBox}{\Square}\ul{Der Hund unter dem Tisch} will endlich \Solalt{\XBox}{\Square}\ul{sein Fressen} haben.\Sol{
      \begin{enumerate}
        \item Der Hund unter dem Tisch will endlich sein Fressen haben.\\
          \blau{Steht bereits im Vorfeld.}
        \item Sein Fressen will der Hund unter dem Tisch endlich haben.
      \end{enumerate}
    }
  \item \Solalt{\Square}{\Square}\ul{Dass es heute} regnet, ist \Solalt{\XBox}{\Square}\ul{so gut wie sicher}.\Sol{
      \begin{enumerate}
        \item * Dass es heute ist regnet so gut wie sicher.
        \item So gut wie sicher ist, dass es heute regnet.
      \end{enumerate}
    }
  \item \Solalt{\XBox}{\Square}\ul{Eine Entscheidung für den Frieden} ist \Solalt{\Square}{\Square}\ul{nicht} \Solalt{\XBox}{\Square}\ul{generell} \Solalt{\XBox}{\Square}\ul{unvereinbar} mit \Solalt{\Square}{\Square}\ul{einer Entscheidung für militärische Aufrüstung}.\Sol{
      \begin{enumerate}
        \item Eine Entscheidung für den Frieden ist nicht generell unvereinbar mit einer Entscheidung für militärische Aufrüstung.\\
          \blau{Steht bereits im Vorfeld.}
        \item Nicht ist eine Entscheidung für den Frieden generell unvereinbar mit einer Entscheidung für militärische Aufrüstung.
        \item Generell ist eine Entscheidung für den Frieden nicht unvereinbar mit einer Entscheidung für militärische Aufrüstung.
        \item Unvereinbar ist eine Entscheidung für den Frieden nicht generell mit einer Entscheidung für militärische Aufrüstung.
        \item * Einer Entscheidung für militärische Aufrüstung ist eine Entscheidung für den Frieden nicht generell unvereinbar mit.
      \end{enumerate}
    }
\end{enumerate}

\section{Nominal- und Präpositionalphrasen}

Welche der unterstrichenen Teile aus Aufgabe~\ref{sec:satzglieder} sind NPs und PPs?
Gibt es andere nicht unterstrichene NPs und PPs in den Sätzen?

\Sol{%
  \begin{enumerate}
    \item
      \begin{enumerate}
        \item NP: \textit{der Winter}
      \end{enumerate}
    \item 
      \begin{enumerate}
        \item NP: \textit{Otje}
        \item NP: \textit{seinen Kindern}
        \item PP: \textit{aus dem Urlaub}
        \item NP: \textit{dem Urlaub}
        \item NP: \textit{eine Karte}
      \end{enumerate}
    \item 
      \begin{enumerate}
        \item NP: wir
        \item NP: Produkte, die regional gefertigt wurden
      \end{enumerate}
    \item 
      \begin{enumerate}
        \item NP: K.\,R.\ Popper
        \item NP: der Philosoph, auf dessen Werken alle falsifikationistischen Wissenschaftstheorien basieren
        \item PP: auf dessen Werken
        \item NP: dessen Werken
        \item NP: dessen
        \item NP: alle falsifikationistischen Wissenschaftstheorien
      \end{enumerate}
    \item
      \begin{enumerate}
        \item PP: zu dieser Jahreszeit
        \item NP: dieser Jahreszeit
        \item NP: keine regionalen Erdbeeren
        \item PP: in Deutschland
        \item NP: Deutschland
      \end{enumerate}
    \item 
      \begin{enumerate}
        \item NP: ich
        \item NP: ein solcher Unsinn
      \end{enumerate}
    \item 
      \begin{enumerate}
        \item NP: alle Wissenschaftler
        \item NP: einen großen Erfolg
        \item PP: für sich
        \item NP: sich
      \end{enumerate}
    \item 
      \begin{enumerate}
        \item NP: man
        \item NP: seinen Hund
        \item PP: beim Einkaufen
        \item NP: Einkaufen
        \item PP: im Auto
        \item NP: Auto
      \end{enumerate}
    \item 
      \begin{enumerate}
        \item NP: keinen Zweck
      \end{enumerate}
    \item 
      \begin{enumerate}
        \item NP: der abgewählte Präsident
        \item PP: bei einer Wahlkampfveranstaltung
        \item NP: einer Wahlkampfveranstaltung
        \item NP: Öl
        \item PP: ins Feuer
        \item NP: Feuer
      \end{enumerate}
    \item 
      \begin{enumerate}
        \item NP: der Hund unter dem Tisch
        \item PP: unter dem Tisch
        \item NP: dem Tisch
        \item NP: sein Fressen
      \end{enumerate}
    \item (keine)
    \item 
      \begin{enumerate}
        \item NP: eine Entscheidung für den Frieden
        \item PP: für den Frieden
        \item NP: den Frieden
        \item PP: mit einer Entscheidung für militärische Aufrüstung
        \item NP: einer Entscheidung für militärische Aufrüstung
        \item PP: für militärische Aufrüstung
        \item NP: militärische Aufrüstung
      \end{enumerate}
  \end{enumerate}
}


  \let\subsection\section\let\section\woopsi
  \section{Phonetik}
  \let\woopsi\section\let\section\subsection\let\subsection\subsubsection
  \section{Stämme und Affixe}\label{sec:form}

Finden Sie in den folgenden Wörtern Stämme und Affixe.
Analysieren Sie die Wörter, soweit Sie können.
Wenn also mehrere Stämme oder Affixe im Wort vorkommen, zerlegen Sie es so detailliert wie möglich.
Wir benutzen hier noch nicht die Trennzeichen, die später in der Vorlesung \slash\ im Buch eingeführt werden.
Trennen Sie einfach alle Morphe mit | ab und \ul{unterstreichen} Sie alle Stämme.

Zur Erinnerung:

\begin{itemize}\Lf
  \item Stämme sind die Morphe mit lexikalischer Markierungsfunktion, also vor allem Bedeutung.
  \item Affixe (Präfixe und Suffixe) sind Morphe ohne lexikalische Markierungsfunktion, sie haben also keine Bedeutung.
    Außerdem können Affixe prinzipiell nicht alleine stehen, sind also nicht wortfähig.
\end{itemize}

\begin{longtable}[h]{cp{0.2\textwidth}p{0.75\textwidth}}
  &&\\
  (0) & \grau{schmeißen}    & \ul{schmeiß} | en\\
  &&\\
  (0) & \grau{liebäugeln}   & \ul{lieb} | \ul{äug} | el | n\\
  &&\\
  (0) & \grau{Verwerfungen} & Ver | \ul{werf} | ung | en \\
  && \\
  (1) & Überholung          & \Sol{Über|\ul{hol}|ung} \\\Solalt{}{\cline{3-3}}
  && \\
  (2) & schreit             & \Sol{\ul{schrei}|t} \\\Solalt{}{\cline{3-3}}
  && \\
  (3) & bläulicheres        & \Sol{\ul{bläu}|lich|er|es} \\\Solalt{}{\cline{3-3}}
  && \\
  (4) & unterschwellige     & \Sol{unter|\ul{schwell}|ig|e} \\\Solalt{}{\cline{3-3}}
  && \\
  (5) & begrünen            & \Sol{be|\ul{grün}|en} \\\Solalt{}{\cline{3-3}}
  && \\
  (6) & denke               & \Sol{\ul{denk}|e} \\\Solalt{}{\cline{3-3}}
  && \\
  (7) & Zeitmessung         & \Sol{\ul{Zeit}|\ul{mess}|ung} \\\Solalt{}{\cline{3-3}}
  && \\
  (8) & Grauslichkeiten     & \Sol{\ul{Graus}|lich|keit|en} \\\Solalt{}{\cline{3-3}}
  && \\
  (9) & Tüchern             & \Sol{\ul{Tüch}|er|n} \\\Solalt{}{\cline{3-3}}
  && \\
  (10) & Rumänen            & \Sol{\ul{Rumän}|e|n} \\\Solalt{}{\cline{3-3}}
\end{longtable}


\section{Wortbildung und Flexion}\label{sec:funktion}

Entscheiden Sie für die \ul{unterstrichenen} Wörter im folgendem Text, inwiefern in ihnen Wortbildung durch Affixe oder Flexion durch Affixe (oder beides) zu beobachten sind.
Trennen Sie dazu die Wörter in Stämme und Affixe auf wie in Aufgabe~\ref{sec:form}.
Wenn es Flexion ist, versuchen Sie zu beschreiben, welche \textbf{Markierungsfunktion} die Affixe haben.
Wenn es Wortbildung ist, versuchen Sie zu beschreiben, welche Merkmale sich durch die Anfügung des Affixes ändern.

Zur Erinnerung:

\begin{itemize}
  \item Bei der Flexion ändern sich Werte \textbf{volatiler} Merkmale, aber das lexikalische Wort bleibt dasselbe.
    Typische Flexionsmerkmale sind:
    \begin{itemize}\Lf
      \item Tempus, Modus, Person und Numerus bei den Verben
      \item Kasus und Numerus bei den Substantiven
      \item Kasus, Genus und Numerus bei den Artikeln und Pronomina
      \item Kasus, Genus, Numerus und die sogenannte Stärke bei den Adjektiven
    \end{itemize}
  \item Bei der Wortbildung ändern sich Werte \textbf{statischer} Merkmale, die sich sonst nicht ändern (\zB die Bedeutung oder die Wortklasse), oder es werden Merkmale gelöscht\slash hinzugefügt.
    Dies kann mit der Anfügung von Affixen einhergehen (\textit{beobachten → \textit{Beobachtung}}, manchmal aber auch ganz ohne Affixe (\textit{wandern} als Verb → \textit{Wandern} als Substantiv).
    Den ersten Fall nennen wir später Derivation, den zweiten Konversion.
    Hier geht es nur um Derivation.
\end{itemize}

\subsection{Text}

\textbf{Wikipedia | Weltraum}\\
\footnotesize Quelle: \url{https://de.wikipedia.org/wiki/Weltraum}\\

\begin{quote}

  Der Weltraum \grau{\ul{bezeichnet}} den Raum zwischen Himmelskörpern. Die \grau{\ul{Atmosphären}} von \ul{festen} und gasförmigen Himmelskörpern (wie Sternen und Planeten) haben keine feste Grenze nach oben, \ul{sondern} werden mit \ul{zunehmendem} Abstand zum Himmelskörper allmählich immer \ul{dünner}. Ab einer \ul{bestimmten} \ul{Höhe} \ul{spricht} man vom Beginn des Weltraums.

  Im Weltraum herrscht ein Hochvakuum mit niedriger Teilchendichte. Er ist aber kein leerer Raum, sondern enthält Gase, kosmischen Staub und Elementarteilchen (Neutrinos, kosmische \ul{Strahlung}, Partikel), außerdem elektrische und magnetische Felder, Gravitationsfelder und elektromagnetische Wellen (Photonen). Das fast vollständige Vakuum im Weltraum macht ihn außerordentlich durchsichtig und erlaubt die Beobachtung extrem \ul{entfernter} Objekte, etwa anderer Galaxien. Jedoch können Nebel aus \ul{interstellarer} Materie die Sicht auf dahinterliegende Objekte auch stark behindern.

  Der Begriff des Weltraums ist nicht gleichzusetzen mit dem Weltall, welches eine \ul{eingedeutschte} Bezeichnung für das Universum insgesamt ist und somit alles, also auch die Sterne und Planeten selbst, mit einschließt. Dennoch wird das deutsche Wort Weltall oder All \ul{umgangssprachlich} (eigentlich inkorrekt) mit der Bedeutung Weltraum verwendet.

  Die Erforschung des Weltraums \ul{wird} Weltraumforschung \ul{genannt}. Reisen oder Transporte in oder durch den Weltraum werden als Raumfahrt bezeichnet.
\end{quote}

\subsection{Lösungsbeispiele}

Hinweis: Die Aufgabenstellung ist freier als typische Klausuraufgaben, und sie ist zum gegenwärtigen Zeitpunkt noch eine Transferaufgabe.
Dies gibt Ihnen die Möglichkeit, nachzudenken und selber ein Gespür für Morphologie zu entwickeln.
Es geht \textbf{nicht} darum, eine perfekte Lösung abzuliefern.
Wir kommen auf alle Details, die Ihnen momentan noch fehlen, in späteren Sitzungen zurück.

\Halbzeile

\begin{itemize}
  \item \textbf{bezeichnet} 
    \begin{itemize}
      \item \textit{be} ist ein Wortbildungsaffix zum Stamm \textit{zeichn}(\textit{e}), das die Bedeutung verändert (\textit{jemand zeichnet etwas} → \textit{ein Wort o.\,Ä.\ bezeichnet etwas} oder \textit{jemand bezeichnet etwas mit einem Wort o.\,Ä.})
      \item Das Flexionssuffix (\textit{e})\textit{t} legt Person und Numerus auf P3 und Singular fest, eventuell auch auf Tempus und Modus auf Präsens und Indikativ.
    \end{itemize}
  \item \textbf{Atmosphären}
    \begin{itemize}
      \item Das Flexionssuffix \textit{n} am Stamm \textit{Atmosphäre} legt das Numerus-Merkmal auf Plural fest.
    \end{itemize}
\end{itemize}

\Sol{Diese Musterlösung entspricht ihrem Wissensstand nach Lektion sieben des Seminars.
Was Sie jetzt nicht verstehen, müssten Sie sich dann später nochmals ansehen.}

\Sol{\begin{enumerate}
  \item be:zeichne-t | Verbpräfix, Präsensstamm, PN1 3. Sg.
  \item Atmosphäre-n | Stamm, Plural
  \item fest-en | Stamm, Kasus-Numerus-Flexion (pronominal)
  \item sondern | nicht flektierbares Wort (Konjunktion)
  \item zu=nehm:end-en | Verbpartikel, Stamm, Adjektivderivation (KEIN Partizip!), Kasus-Numerus-Flexion (pronominal)
  \item dünn-er (ab EGBD4 dünn:er) | Stamm, Komparativ
  \item be:stimm-t-en (ab EGBD4 be:stimm:t-en) | Verbpräfix, Stamm, Partizip, Kasus-Numerus-Flexion (adjektivisch)
  \item Höhe | Stamm
  \item sprich-t | Präsensstamm (Alternanzstufe), PN1 3. Sg.
  \item Strahl:ung | Stamm, Substantiv-Derivationssuffix
  \item ent:fern-t-er (ab EGBD4 ent:fern:t-er) | Verbpräfix, Stamm, Partizip, Kasus-Numerus-Entscheiden (pronominal)
  \item interstellar-er | Stamm, Kasus-Numerus-Endung (pronominal)
  \item ein=ge-deutsch-t-e (ab EGBD4 ein=ge:deutsch:t-e) | Verbpartikel, Partizip, Stamm, Partizip, Kasus-Numerus-Flexion (adjektivisch)
  \item um=gang-s.sprach:lich | Verbpartikel, Stamm, Fugenelement, Stamm, Adjektiv-Derivationssuffix
  \item wird | keine Segmentierung möglich
  \item ge-nann-t (ab EGBD4 ge:nann:t) | Partizip, Stamm, Partizip
\end{enumerate}
}

\section{Transfer\slash Vertiefung}

Die Funktionsbestimmung für Flexionsformen aus Aufgabe~\ref{sec:funktion} ist für die Beispiele in Aufgabe~\ref{sec:form} nicht immer ohne Weiteres möglich.
Dies liegt daran, dass die Formen ohne Satzkontext gegeben werden.
Erklären Sie diese Behauptung und illustrieren Sie an konkreten Formen aus Aufgabe~\ref{sec:form} das Problem.

\Sol{\begin{enumerate}
  \item \textit{bläulicheres} kann ein Nominativ oder Akkusativ sein
  \item \textit{unterschwellige} kann ein Nominativ oder Akkusativ Plural in einer NP ohne flektierendes Artikelwort sein (\textit{unterschwellige Anschuldigungen}). Es könnte außerdem ein Nominativ oder Akkusativ Singular Femininum mit oder ohne flektierendes Artikelwort sein.
    \begin{enumerate}
      \item (Die) unterschwellige Aggression macht mich nervös.
      \item Ich spüre (die) unterschwellige Aggression.
    \end{enumerate}
  \item \textit{begrünen} kann ein Infinitiv oder eine Präsensform der ersten oder dritten Person im Plural sein, hypothetisch auch im Konjunktiv Präsens.
  \item \textit{denke} kann eine Präsensform der 1. Person oder eine Imperativform sein.
  \item Bei \textit{Grauslichkeiten} sind unterschiedliche Kasus denkbar.
  \item Bei \textit{Rumänen} kann es sich um jede Kasus-Numerus-Form außer dem Nominativ Singular handeln.
  \end{enumerate}
}

  
\fi

\makeatletter
\setcounter{lastpagemainpart}{\the\c@framenumber}
\makeatother

\appendix

\begin{frame}[allowframebreaks]
  {Literatur}
  \renewcommand*{\bibfont}{\footnotesize}
  \setbeamertemplate{bibliography item}{}
  \printbibliography
\end{frame}

\begin{frame}
  {Autor}
  \begin{block}{Kontakt}
    Prof.\ Dr.\ Roland Schäfer\\
    Institut für Germanistische Sprachwissenschaft\\
    Friedrich-Schiller-Universität Jena\\
    Fürstengraben 30\\
    07743 Jena\\[\baselineskip]
    \url{https://rolandschaefer.net}\\
    \texttt{roland.schaefer@uni-jena.de}
  \end{block}
\end{frame}

\begin{frame}
  {Lizenz}
  \begin{block}{Creative Commons BY-SA-3.0-DE}
    Dieses Werk ist unter einer Creative Commons Lizenz vom Typ \textit{Namensnennung - Weitergabe unter gleichen Bedingungen 3.0 Deutschland} zugänglich.
    Um eine Kopie dieser Lizenz einzusehen, konsultieren Sie \url{http://creativecommons.org/licenses/by-sa/3.0/de/} oder wenden Sie sich brieflich an Creative Commons, Postfach 1866, Mountain View, California, 94042, USA.
  \end{block}
\end{frame}

\mode<beamer>{\setcounter{framenumber}{\thelastpagemainpart}}

\end{document}
