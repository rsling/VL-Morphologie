\documentclass[12pt,a4paper,twoside]{article}

\usepackage[margin=2cm]{geometry}

\usepackage[ngerman]{babel}

\usepackage{setspace}
\usepackage{booktabs}
\usepackage{array,graphics}
\usepackage{color}
\usepackage{soul}
\usepackage[linecolor=gray,backgroundcolor=yellow!50,textsize=tiny]{todonotes}
\usepackage[linguistics]{forest}
\usepackage{multirow}
\usepackage{pifont}
\usepackage{wasysym}
\usepackage{langsci-gb4e}
\usepackage{soul}
\usepackage{enumitem}
\usepackage{marginnote}
\usepackage{ulem}

\usepackage[maxbibnames=99,
  maxcitenames=2,
  uniquelist=false,
  backend=biber,
  doi=false,
  url=false,
  isbn=false,
  bibstyle=biblatex-sp-unified,
  citestyle=sp-authoryear-comp]{biblatex}

\definecolor{rot}{rgb}{0.7,0.2,0.0}
\newcommand{\rot}[1]{\textcolor{rot}{#1}}
\definecolor{blau}{rgb}{0.1,0.2,0.7}
\newcommand{\blau}[1]{\textcolor{blau}{#1}}
\definecolor{gruen}{rgb}{0.0,0.7,0.2}
\newcommand{\gruen}[1]{\textcolor{gruen}{#1}}
\definecolor{grau}{rgb}{0.6,0.6,0.6}
\newcommand{\grau}[1]{\textcolor{grau}{#1}}
\definecolor{orongsch}{RGB}{255,165,0}
\newcommand{\orongsch}[1]{\textcolor{orongsch}{#1}}
\definecolor{tuerkis}{RGB}{63,136,143}
\definecolor{braun}{RGB}{108,71,65}
\newcommand{\tuerkis}[1]{\textcolor{tuerkis}{#1}}
\newcommand{\braun}[1]{\textcolor{braun}{#1}}

\newcommand*\Rot{\rotatebox{75}}
\newcommand*\RotRec{\rotatebox{90}}

\newcommand{\zB}{z.\,B.\ }
\newcommand{\ZB}{Z.\,B.\ }
\newcommand{\Sub}[1]{\ensuremath{_{\text{#1}}}}
\newcommand{\Up}[1]{\ensuremath{^{\text{#1}}}}
\newcommand{\UpSub}[2]{\ensuremath{^{\text{#1}}_{\text{#2}}}}
\newcommand{\Doppelzeile}{\vspace{2\baselineskip}}
\newcommand{\Zeile}{\vspace{\baselineskip}}
\newcommand{\Halbzeile}{\vspace{0.5\baselineskip}}
\newcommand{\Viertelzeile}{\vspace{0.25\baselineskip}}

\newcommand{\whyte}[1]{\textcolor{white}{#1}}

\newcommand{\Spur}[1]{t\Sub{#1}}
\newcommand{\Ti}{\Spur{1}}
\newcommand{\Tii}{\Spur{2}}
\newcommand{\Tiii}{\Spur{3}}
\newcommand{\Tiv}{\Spur{4}}
\newcommand*{\mybox}[1]{\framebox{#1}}
\newcommand\ol[1]{{\setul{-0.9em}{}\ul{#1}}}

\newenvironment{nohyphens}{%
  \par
  \hyphenpenalty=10000
  \exhyphenpenalty=10000
  \sloppy
}{\par}

\newcommand{\Lf}{
  \setlength{\itemsep}{1pt}
  \setlength{\parskip}{0pt}
  \setlength{\parsep}{0pt}
}

\forestset{
  Ephr/.style={draw, ellipse, thick, inner sep=2pt},
  Eobl/.style={draw, rounded corners, inner sep=5pt},
  Eopt/.style={draw, rounded corners, densely dashed, inner sep=5pt},
  Erec/.style={draw, rounded corners, double, inner sep=5pt},
  Eoptrec/.style={draw, rounded corners, densely dashed, double, inner sep=5pt},
  Ehd/.style={rounded corners, fill=gray, inner sep=5pt,
    delay={content=\whyte{##1}}
  },
  Emult/.style={for children={no edge}, for tree={l sep=0pt}},
  phrasenschema/.style={for tree={l sep=2em, s sep=2em}},
  sake/.style={tier=preterminal},
  ake/.style={
    tier=preterminal
    },
}

\forestset{
  decide/.style={draw, chamfered rectangle, inner sep=2pt},
  finall/.style={rounded corners, fill=gray, text=white},
  intrme/.style={draw, rounded corners},
  yes/.style={edge label={node[near end, above, sloped, font=\scriptsize]{Ja}}},
  no/.style={edge label={node[near end, above, sloped, font=\scriptsize]{Nein}}}
}

\usepackage{tikz}
\usetikzlibrary{arrows,positioning} 


\author{Prof.\ Dr.\ Roland Schäfer | Germanistische Linguistik FSU Jena}
\title{Morphologie | 06 | Flexion der Substantive, Pronomina, Artikel}
\date{Version Sommer 2023 (\today)}


\usepackage{fontspec}
\defaultfontfeatures{Ligatures=TeX,Numbers=OldStyle, Scale=MatchLowercase}
\setmainfont{Linux Libertine O}
\setsansfont{Linux Biolinum O}

\setlength{\parindent}{0pt}

% \usepackage[headings]{fancyhdr}
% \fancyhead[E,O]{}
% \fancyfoot[E,O]{}
% \renewcommand{\headrulewidth}{0pt}
% \pagestyle{fancy}
% \setlength{\headsep}{50pt}
% \setlength{\textheight}{\textheight-25pt}


\newenvironment{spread}
{%
  \newdimen\origiwspc%
  \newdimen\origiwstr%
  \origiwspc=\fontdimen2\font%
  \origiwstr=\fontdimen3\font%
  \fontdimen2\font=1em%
  \doublespacing%
}{%
  \fontdimen2\font=\origiwspc%
  \fontdimen3\font=\origiwstr%
}



\begin{document}

\maketitle

\section{Traditionelle Flexionsklassen der Substantive}


\section{Pluralklasse und prototypisches Genus der Substantive}


\section{Anaphern}

Koindizieren Sie die Anaphern (und ggf.\ ihren Antezedenzien) in den folgenden Sätzen so, dass die beschriebene Situation korrekt von den Sätzen wiedergegeben wird.

\begin{enumerate}
  \item\doublespacing%
    \begin{spread}
      \textbf{Situation}: Eine Person kauft für eine andere ein Geschenk.\\
      \textbf{Sätze}: (i) Sie betritt das KaDeWe und überlegt, was ihr gefallen könnte.
      (ii) Sie findet zunächst nichts passendes für sie.
      (iii) Sie hat ihr ausdrücklich gesagt, dass sie gar kein Geschenk zu besorgen braucht.
      (iv) Auf jeden Fall will sie ihr kein Klischeegeschenk mitbringen.
      (v) Im Obergeschoss entdeckt sie dann zufällig den Beaujolais, den sie damals nach ihrem MA-Abschluss getrunken haben, und nimmt zwei Flaschen mit.\end{spread}
  \item\doublespacing
    \begin{spread}
      \textbf{Situation}: Max schickt Julius einen Brief über den Vorstand mit dem firmeneigenen Briefboten.\\
      \textbf{Sätze}: 
  \end{spread}
\end{enumerate}


\section{Pronomina und Artikel unterscheiden}

Entscheiden Sie für die unterstrichenen Wörter, ob sie Artikel (A) oder Pronomina in Artikelfunktion (PA) oder Pronomina in Pronominalfunktion (PP) sind.

\begin{center}
  \begin{tabular}[h]{cll}
    \toprule
    & \textbf{Wort im Satzkontext} & \textbf{Klassifkation} \\
    \midrule
    (1) & \textit{Es hat sich \uline{kein} Junge ins Wasser getraut.} & \Square~A\ \ \Square~PA\ \ \Square~PP \\
    (2) & \textit{Da ist der Kollege, \uline{dessen} Kinder immer nerven.} & \Square~A\ \ \Square~PA\ \ \Square~PP \\
    (3) & \textit{Mit \uline{diesem} Milieu will ich nichts zu tun haben.} & \Square~A\ \ \Square~PA\ \ \Square~PP \\
    (4) & \textit{\uline{Einer} wollte auf jeden Fall schwimmen.} & \Square~A\ \ \Square~PA\ \ \Square~PP \\
    (5) & \textit{Ich fahre ungern mit \uline{deinem} Auto.} & \Square~A\ \ \Square~PA\ \ \Square~PP \\
    (6) & \textit{Die Kinder \uline{des} Kollegen waren heute ruhig.} & \Square~A\ \ \Square~PA\ \ \Square~PP \\
    (7) & \textit{\uline{Unseres} hatte leider gestern eine Reifenpanne.} & \Square~A\ \ \Square~PA\ \ \Square~PP \\
    (8) & \textit{\uline{Die} ist gemein!} & \Square~A\ \ \Square~PA\ \ \Square~PP \\
    (9) & \textit{Wir erinnerten uns \uline{seiner}, als er hereinkam.} & \Square~A\ \ \Square~PA\ \ \Square~PP \\
    (10) & \textit{Ich traf gestern \uline{die} Schwester meines Kollegen.} & \Square~A\ \ \Square~PA\ \ \Square~PP \\
    \bottomrule
  \end{tabular}
\end{center}

\end{document}
