\section{Wörter in Kern und Peripherie}\label{sec:woerter}

\ul{Unterstreichen} Sie im folgenden Text \textit{Relativitätstheorie} zehn Substantive, Verben oder Adjektive, die Simplizia sind, und die zum morphophonologischen Kern des Deutschen gehören.
\mybox{Umrahmen} Sie außerdem zehn Wörter aus aus denselben Wortklassen, die nicht zu diesem Kern gehören und damit fremd sind.
Die fremden Wörter dürfen auch Komposita sein und müssen dann mindestens ein fremdes Glied enthalten.
Direkte fremdsprachliche Zitatformen (wie \textit{Theory of Everything}) ignorieren Sie bitte.

\section{Fremdwörter analysieren}\label{sec:analyse}

Nennen Sie die Gründe dafür, dass die zehn Fremdwörter aus Aufgabe~\ref{sec:woerter} nicht zum Kern gehören.
Wenn das Wort ein Kompositum ist, geben Sie im dafür vorgesehenen Feld den Stamm desjenigen Gliedes an, auf das Sie sich beziehen.
Wenn mehrere Gründe zutreffen, nennen Sie alle.
Schreiben Sie das Wort in seiner Nennform auf, nicht unbedingt in der Flexionsform, die im Text vorkommt.
Es kommen folgende Gründe infrage:

\begin{itemize}\Lf
  \item drei- oder mehrsilbiger Stamm (S)
  \item Stammbetonung ist nicht trochäisch (T)
  \item Vollvokale in unbetonten Silben (V)
  \item nicht-zweisilbiger Plural beim Substantiv (P)
  \item gleiche Form im Akkusativ Singular und Plural (A)
  \item Lehnwort aus einer anderen Sprache (L)
\end{itemize}

In (0) sehen Sie ein Beispiel, das nicht aus dem Text stammt.

\begin{center}
  \begin{tabular}[h]{cp{0.28\textwidth}p{0.1em}p{0.2\textwidth}p{0.1em}l}
    \toprule
    & \textbf{Wort} && \textbf{ggf. fremdes Glied} && \textbf{Gründe für Fremdwortstatus} \\
    \midrule
    &&&&& \\
    (0) & \textit{Bovisternte} && \textit{Bovist} && S~\Square\ \ \ T~\XBox\ \ \ V~\XBox\ \ \ P~\XBox\ \ \ A~\Square\ \ \ L~\Square \\
    &&&&& \\
    (1) & \Sol{Struktur} &&&& S~\Solalt{\Square}{\Square}\ \ \ T~\Solalt{\XBox}{\Square}\ \ \ V~\Solalt{\XBox}{\Square}\ \ \ P~\Solalt{\XBox}{\Square}\ \ \ A~\Solalt{\Square}{\Square}\ \ \ L~\Solalt{\Square}{\Square}\\\cline{2-2}\cline{4-4}
    &&&&& \\
    (2) & \Sol{Gravitation} &&&& S~\Solalt{\XBox}{\Square}\ \ \ T~\Solalt{\XBox}{\Square}\ \ \ V~\Solalt{\XBox}{\Square}\ \ \ P~\Solalt{\XBox}{\Square}\ \ \ A~\Solalt{\Square}{\Square}\ \ \ L~\Solalt{\Square}{\Square}\\\cline{2-2}\cline{4-4}
    &&&&& \\
    (3) & \Sol{Theorie} &&&& S~\Solalt{\XBox}{\Square}\ \ \ T~\Solalt{\XBox}{\Square}\ \ \ V~\Solalt{\XBox}{\Square}\ \ \ P~\Solalt{\XBox}{\Square}\ \ \ A~\Solalt{\Square}{\Square}\ \ \ L~\Solalt{\Square}{\Square}\\\cline{2-2}\cline{4-4}
    &&&&& \\
    (4) & \Sol{speziell} &&&& S~\Solalt{\Square}{\Square}\ \ \ T~\Solalt{\XBox}{\Square}\ \ \ V~\Solalt{\XBox}{\Square}\ \ \ P~\Solalt{\Square}{\Square}\ \ \ A~\Solalt{\Square}{\Square}\ \ \ L~\Solalt{\Square}{\Square}\\\cline{2-2}\cline{4-4}
    &&&&& \\
    (5) & \Sol{relativ} &&&& S~\Solalt{\XBox}{\Square}\ \ \ T~\Solalt{\XBox}{\Square}\ \ \ V~\Solalt{\XBox}{\Square}\ \ \ P~\Solalt{\Square}{\Square}\ \ \ A~\Solalt{\Square}{\Square}\ \ \ L~\Solalt{\Square}{\Square}\\\cline{2-2}\cline{4-4}
    &&&&& \\
    (6) & \Sol{Phänomen} &&&& S~\Solalt{\XBox}{\Square}\ \ \ T~\Solalt{\XBox}{\Square}\ \ \ V~\Solalt{\XBox}{\Square}\ \ \ P~\Solalt{\XBox}{\Square}\ \ \ A~\Solalt{\Square}{\Square}\ \ \ L~\Solalt{\Square}{\Square}\\\cline{2-2}\cline{4-4}
    &&&&& \\
    (7) & \Sol{revolutionieren} &&&& S~\Solalt{\XBox}{\Square}\ \ \ T~\Solalt{\XBox}{\Square}\ \ \ V~\Solalt{\XBox}{\Square}\ \ \ P~\Solalt{\Square}{\Square}\ \ \ A~\Solalt{\Square}{\Square}\ \ \ L~\Solalt{\Square}{\Square}\\\cline{2-2}\cline{4-4}
    &&&&& \\
    (8) & \Sol{präzis} &&&& S~\Solalt{\Square}{\Square}\ \ \ T~\Solalt{\XBox}{\Square}\ \ \ V~\Solalt{\XBox}{\Square}\ \ \ P~\Solalt{\Square}{\Square}\ \ \ A~\Solalt{\Square}{\Square}\ \ \ L~\Solalt{\Square}{\Square}\\\cline{2-2}\cline{4-4}
    &&&&& \\
    (9) & \Sol{Axiom} &&&& S~\Solalt{\XBox}{\Square}\ \ \ T~\Solalt{\XBox}{\Square}\ \ \ V~\Solalt{\XBox}{\Square}\ \ \ P~\Solalt{\XBox}{\Square}\ \ \ A~\Solalt{\Square}{\Square}\ \ \ L~\Solalt{\Square}{\Square}\\\cline{2-2}\cline{4-4}
    &&&&& \\
    (10) & \Sol{Elektrodynamik} && \Sol{Dynamik} && S~\Solalt{\XBox}{\Square}\ \ \ T~\Solalt{\XBox}{\Square}\ \ \ V~\Solalt{\XBox}{\Square}\ \ \ P~\Solalt{\XBox}{\Square}\ \ \ A~\Solalt{\Square}{\Square}\ \ \ L~\Solalt{\Square}{\Square}\\\cline{2-2}\cline{4-4}
  \end{tabular}
\end{center}

\section{Bonusaufgabe}

Suchen Sie im unten stehenden Text über die Relativitätstheorie Wörter, die Sie als Lehnwörter erkennen, die aber dennoch zum morphophonologischen Kern gehören (ggf.\ auch hier einzelne Glieder von Komposita).

\Halbzeile

\Sol{
  \textbf{Beispiele}
  \begin{enumerate}
    \item \textit{Grenze} aus altpoln. \textit{granica}\slash \textit{grańca}
    \item \textit{Titel} aus lat. \textit{titulus}
    \item \textit{Formel} aus lat. \textit{formula}
    \item \textit{Äther} aus gr. \textit{αἰθήρ}
  \end{enumerate}}

\section*{Text \textit{Relativitätstheorie}}

Quelle: Wikipedia (\url{https://de.wikipedia.org/wiki/Relativitätstheorie})

\begin{nohyphens}

\subsection*{Einleitung}

Die Relativitätstheorie befasst sich mit der \Solalt{\mybox{Struktur}}{Struktur} von \Solalt{\ul{Raum}}{Raum} und \Solalt{\ul{Zeit}}{Zeit} sowie mit dem Wesen der \Solalt{\mybox{Gravitation.}}{Gravitation.} Sie besteht aus zwei maßgeblich von Albert Einstein entwickelten physikalischen \Solalt{\mybox{Theorien:}}{Theorien:} der 1905 veröffentlichten \Solalt{\mybox{speziellen}}{speziellen} Relativitätstheorie und der 1916 abgeschlossenen allgemeinen Relativitätstheorie. Die spezielle Relativitätstheorie beschreibt das Verhalten von Raum und Zeit aus der Sicht von Beobachtern, die sich \Solalt{\mybox{relativ}}{relativ} zueinander bewegen, und die damit verbundenen \Solalt{\mybox{Phänomene.}}{Phänomene.} Darauf aufbauend führt die allgemeine Relativitätstheorie die Gravitation auf eine Krümmung von Raum und Zeit zurück, die unter anderem durch die beteiligten Massen verursacht wird.

Der in der physikalischen Fachsprache häufige Ausdruck \textit{relativistisch} bedeutet üblicherweise, dass eine Geschwindigkeit nicht vernachlässigbar \Solalt{\ul{klein}}{klein} gegenüber der Lichtgeschwindigkeit ist; die \Solalt{\ul{Grenze}}{Grenze} wird oft bei 10 Prozent gezogen. Bei relativistischen Geschwindigkeiten gewinnen die von der speziellen Relativitätstheorie beschriebenen Effekte zunehmende Bedeutung, die Abweichungen von der klassischen Mechanik können dann nicht mehr vernachlässigt werden.

In diesem Artikel werden die grundlegenden Strukturen und Phänomene lediglich zusammenfassend aufgeführt. Für Erläuterungen und Details siehe die Artikel spezielle Relativitätstheorie und allgemeine Relativitätstheorie sowie die Verweise im Text.


\subsection*{Bedeutung}

Die Relativitätstheorie hat das Verständnis von Raum und Zeit \Solalt{\mybox{revolutioniert}}{revolutioniert} und Zusammenhänge aufgedeckt, die sich der anschaulichen Vorstellung entziehen. Diese lassen sich jedoch mathematisch \Solalt{\mybox{präzise}}{präzise} in Formeln fassen und durch Experimente bestätigen. Die Relativitätstheorie enthält die newtonsche Physik als Grenzfall. Sie erfüllt damit das Korrespondenzprinzip.

Das Standardmodell der Teilchenphysik beruht auf der Vereinigung der speziellen Relativitätstheorie mit der Quantentheorie zu einer relativistischen Quantenfeldtheorie.

Die allgemeine Relativitätstheorie ist neben der Quantenphysik eine der beiden \Solalt{\ul{Säulen}}{Säulen} des Theoriengebäudes Physik. Es wird allgemein angenommen, dass eine Vereinigung dieser beiden Säulen zu einer Theory of Everything (Theorie von allem) im Prinzip möglich ist. Trotz großer Anstrengungen ist solch eine Vereinigung jedoch noch nicht vollständig gelungen. Sie zählt zu den großen Herausforderungen der physikalischen Grundlagenforschung.

\subsection{Das Relativitätsprinzip}

Die beiden folgenden Feststellungen lassen sich als \Solalt{\mybox{Axiome}}{Axiome} der Relativitätstheorie interpretieren, aus denen alles Weitere hergeleitet werden kann:

\begin{itemize}\Lf
  \item Messen verschiedene Beobachter die Geschwindigkeit eines Lichtstrahls relativ zu ihrem Standort, so \Solalt{\ul{kommen}}{kommen} sie unabhängig von ihrem \Solalt{\ul{eigenen}}{eigenen} Bewegungszustand zum selben Ergebnis. Dies ist das sogenannte Prinzip von der Konstanz der Lichtgeschwindigkeit.
  \item Die physikalischen Gesetze haben für alle Beobachter, die sich relativ zueinander mit konstanter Geschwindigkeit bewegen, also keiner Beschleunigung unterliegen, dieselbe Gestalt. Diesen Umstand nennt man Relativitätsprinzip.
\end{itemize}

Das Relativitätsprinzip an sich ist wenig spektakulär, denn es gilt auch für die newtonsche Mechanik. Aus ihm \Solalt{\ul{folgt}}{folgt} unmittelbar, dass es keine Möglichkeit gibt, eine absolute Geschwindigkeit eines Beobachters im Raum zu ermitteln und damit ein absolut ruhendes Bezugssystem zu definieren. Ein solches Ruhesystem müsste sich in irgendeiner Form von allen anderen unterscheiden – es würde damit aber im Widerspruch zum Relativitätsprinzip stehen, wonach die Gesetze der Physik in allen Bezugssystemen dieselbe Gestalt haben. Nun beruhte vor der Entwicklung der Relativitätstheorie die Elektrodynamik auf der Annahme des Äthers als Träger elektromagnetischer Wellen. Würde ein solcher Äther als \Solalt{\ul{starres}}{starres} Gebilde den Raum füllen, dann würde er ein Bezugssystem definieren, in dem im Widerspruch zum Relativitätsprinzip die physikalischen Gesetze eine besonders einfache Form hätten und welches überdies das einzige System wäre, in dem die Lichtgeschwindigkeit konstant ist. Jedoch scheiterten alle Versuche, die Existenz des Äthers nachzuweisen, wie beispielsweise das berühmte Michelson-Morley-Experiment von 1887.

Durch die Aufgabe der konventionellen Vorstellungen von Raum und Zeit und die Verwerfung der Ätherhypothese gelang es Einstein, den scheinbaren Widerspruch zwischen dem Relativitätsprinzip und der aus der \Solalt{\mybox{Elektrodynamik}}{Elektrodynamik} folgenden Konstanz der Lichtgeschwindigkeit aufzulösen. Nicht zufällig waren es Experimente und Überlegungen zur Elektrodynamik, die zur Entdeckung der Relativitätstheorie führten. So lautete der unscheinbare \Solalt{\ul{Titel}}{Titel} der einsteinschen Publikation von 1905, die die spezielle Relativitätstheorie begründete, \textit{Zur Elektrodynamik bewegter Körper}.

\end{nohyphens}

