% \usepackage{setspace}
% \usepackage{booktabs}
% \usepackage{array,graphics}
% \usepackage{color}
% \usepackage{soul}
% \usepackage[linecolor=gray,backgroundcolor=yellow!50,textsize=tiny]{todonotes}
% \usepackage[linguistics]{forest}
% \usepackage{multirow}
% \usepackage{pifont}
% \usepackage{wasysym}
% \usepackage{langsci-gb4e}
% \usepackage{soul}
% \usepackage{enumitem}
% \usepackage{marginnote}

\section{Wortklassen}\label{sec:bestimmen}

Entscheiden Sie für die im folgenden Text unterstrichenen Wörter, zu welche Wortklasse sie gehören.
Kreuzen Sie in der Tabelle dazu alle Filter an, die das Wort durchlassen (bei denen die Antwort also "`Ja"' ist). 

\begin{nohyphens}\begin{quote}
  \textbf{Wikipedia | Wasserspringen}\\
  https://de.wikipedia.org/wiki/Wasserspringen, modifiziert\\[0.5\baselineskip]
  \textbf{Definition}
  (1)\ul{Das} (2)\ul{Wasserspringen} ist eine Wassersportart, (3)\ul{bei} der (4)\ul{es} (5)\ul{darum} (6)\ul{geht}, aus (7)\ul{unterschiedlichen} Höhen (8)\ul{und} mit verschiedenen Techniken (9)\ul{möglichst} (10)\ul{elegant} (11)\ul{ins} Wasser (12)\ul{zu} springen.
  Dieser Wettkampfsport ist mit mehreren Disziplinen (13)\ul{seit} (14)\ul{1904} (15)\ul{Bestandteil} der Olympischen Spiele.
  Es wird zwischen Kunstspringen (1-m- und 3-m-Brett), Turmspringen (5-m-, 7,5-m- und 10-m-Turm) und Synchronspringen (3-m-Brett und 10-m-Turm) unterschieden.\\[0.5\baselineskip]
  \textbf{Ausführung der Sprünge}
  Bei den Frauen (16)\ul{besteht} ein Wettbewerb aus fünf, bei den Männern aus (17)\ul{sechs} Sprungdurchgängen.
  In den Einzelwettbewerben gibt es (18)\ul{ausschließlich} Kürsprünge, die Springer (19)\ul{können} Sprungtyp, Schwierigkeitsgrad und Reihenfolge ihrer Sprünge (20)\ul{frei} wählen.
  Im Kunstspringen ist es jedoch Pflicht, (21)\ul{dass} aus jeder der fünf Sprunggruppen ein Sprung gezeigt wird.
  Die Männer können (22)\ul{somit} aus einer frei wählbaren Sprunggruppe zwei Sprünge zeigen, (23)\ul{die} allerdings nicht (24)\ul{identisch} (25)\ul{sein} dürfen.
  Im Turmspringen gibt es sechs Sprunggruppen, (26)\ul{aber} während die Frauen (27)\ul{nur} aus fünf frei wählbaren Gruppen einen Sprung zeigen müssen, zeigen die Männer aus (28)\ul{jeder} der sechs Gruppen einen Sprung.
  In den Synchronwettbewerben bestehen die ersten beiden Durchgänge aus Pflichtsprüngen, (29)\ul{damit} bei einer geringen Höchstschwierigkeit (30)\ul{so} die exakte Synchronität bei (31)\ul{mindestens} zwei Sprüngen im Vordergrund steht.
  (32)\ul{Auch} (33)\ul{hier} müssen Sprünge aus unterschiedlichen Sprunggruppen gezeigt werden.\\[0.5\baselineskip]
  \textbf{Kontroversen} [\ldots] An den Missständen ist auch eine über Jahrzehnte geduldete Kultur des Verschweigens (34)\ul{schuld}.
\end{quote}\end{nohyphens}

\begin{center}
  \begin{tabular}[h]{clp{0.5em}cp{0.5em}cccp{0.5em}ccccccp{0.5em}l}
    \toprule
    &&&&& \multicolumn{3}{l}{\textbf{Flektierbar}} && \multicolumn{6}{l}{\textbf{Nicht-flektierbar}} && \\\cline{6-8}\cline{10-15}
    &&&&&&&&&&&&&&&& \\
    & \textbf{Wort} && \RotRec{\textbf{Numerus?}} && \RotRec{\textbf{Finit?}} & \RotRec{\textbf{Genusfest?}} & \RotRec{\textbf{Stärkeflexion?}} && \RotRec{\textbf{Valenz?}} & \RotRec{\textbf{Nebensatzeinleiter?}} & \RotRec{\textbf{Vorfeldbesetzer?}} & \RotRec{\textbf{Mit Kopulaverb?}} & \RotRec{\textbf{Satzersetzer?}} & \RotRec{\textbf{Konstituentenverbinder?}} && \textbf{Wortklasse} \\
    \midrule
    (1) & \textit{das} && \Square && \Square & \Square &\Square && \Square & \Square & \Square & \Square & \Square & \Square && \\
    (2) & \textit{Wasserspringen} && \Square && \Square & \Square &\Square && \Square & \Square & \Square & \Square & \Square & \Square && \\
    (3) & \textit{bei} && \Square && \Square & \Square &\Square && \Square & \Square & \Square & \Square & \Square & \Square && \\
    (4) & \textit{es} && \Square && \Square & \Square &\Square && \Square & \Square & \Square & \Square & \Square & \Square && \\
    (5) & \textit{darum} && \Square && \Square & \Square &\Square && \Square & \Square & \Square & \Square & \Square & \Square && \\
    (6) & \textit{geht} && \Square && \Square & \Square &\Square && \Square & \Square & \Square & \Square & \Square & \Square && \\
    (7) & \textit{unterschiedlichen} && \Square && \Square & \Square &\Square && \Square & \Square & \Square & \Square & \Square & \Square && \\
    (8) & \textit{und} && \Square && \Square & \Square &\Square && \Square & \Square & \Square & \Square & \Square & \Square && \\
    (9) & \textit{möglichst} && \Square && \Square & \Square &\Square && \Square & \Square & \Square & \Square & \Square & \Square && \\
    (10) & \textit{elegant} && \Square && \Square & \Square &\Square && \Square & \Square & \Square & \Square & \Square & \Square && \\
    (11) & \textit{ins} && \Square && \Square & \Square &\Square && \Square & \Square & \Square & \Square & \Square & \Square && \\
    (12) & \textit{zu} && \Square && \Square & \Square &\Square && \Square & \Square & \Square & \Square & \Square & \Square && \\
    (13) & \textit{seit} && \Square && \Square & \Square &\Square && \Square & \Square & \Square & \Square & \Square & \Square && \\
    (14) & \textit{1904} && \Square && \Square & \Square &\Square && \Square & \Square & \Square & \Square & \Square & \Square && \\
    (15) & \textit{Bestandteil} && \Square && \Square & \Square &\Square && \Square & \Square & \Square & \Square & \Square & \Square && \\
    (16) & \textit{besteht} && \Square && \Square & \Square &\Square && \Square & \Square & \Square & \Square & \Square & \Square && \\
    (17) & \textit{sechs} && \Square && \Square & \Square &\Square && \Square & \Square & \Square & \Square & \Square & \Square && \\
    (18) & \textit{ausschließlich} && \Square && \Square & \Square &\Square && \Square & \Square & \Square & \Square & \Square & \Square && \\
    (19) & \textit{können} && \Square && \Square & \Square &\Square && \Square & \Square & \Square & \Square & \Square & \Square && \\
    (20) & \textit{frei} && \Square && \Square & \Square &\Square && \Square & \Square & \Square & \Square & \Square & \Square && \\
    (21) & \textit{dass} && \Square && \Square & \Square &\Square && \Square & \Square & \Square & \Square & \Square & \Square && \\
    (22) & \textit{somit} && \Square && \Square & \Square &\Square && \Square & \Square & \Square & \Square & \Square & \Square && \\
    (23) & \textit{die} && \Square && \Square & \Square &\Square && \Square & \Square & \Square & \Square & \Square & \Square && \\
    (24) & \textit{identisch} && \Square && \Square & \Square &\Square && \Square & \Square & \Square & \Square & \Square & \Square && \\
    (25) & \textit{sein} && \Square && \Square & \Square &\Square && \Square & \Square & \Square & \Square & \Square & \Square && \\
    (26) & \textit{aber} && \Square && \Square & \Square &\Square && \Square & \Square & \Square & \Square & \Square & \Square && \\
    (27) & \textit{nur} && \Square && \Square & \Square &\Square && \Square & \Square & \Square & \Square & \Square & \Square && \\
    (28) & \textit{jeder} && \Square && \Square & \Square &\Square && \Square & \Square & \Square & \Square & \Square & \Square && \\
    (29) & \textit{damit} && \Square && \Square & \Square &\Square && \Square & \Square & \Square & \Square & \Square & \Square && \\
    (30) & \textit{so} && \Square && \Square & \Square &\Square && \Square & \Square & \Square & \Square & \Square & \Square && \\
    (31) & \textit{mindestens} && \Square && \Square & \Square &\Square && \Square & \Square & \Square & \Square & \Square & \Square && \\
    (32) & \textit{auch} && \Square && \Square & \Square &\Square && \Square & \Square & \Square & \Square & \Square & \Square && \\
    (33) & \textit{hier} && \Square && \Square & \Square &\Square && \Square & \Square & \Square & \Square & \Square & \Square && \\
    (34) & \textit{schuld} && \Square && \Square & \Square &\Square && \Square & \Square & \Square & \Square & \Square & \Square && \\
    \bottomrule
  \end{tabular}
\end{center}

\newpage

\section{Filterhierarchie}

Was ist der in der Hierarchie \textbf{höchste} (also in der Anwendung früheste) Filter, aufgrund dessen die folgenden Wortklassenzuweisungen für die unterstrichenen Filter nicht korrekt sein kann.
Benennen Sie die Filter mit den Bezeichnungen aus der Tabelle in Aufgabe~\ref{sec:bestimmen}.

\begin{center}
  \scalebox{0.85}{\begin{tabular}[h]{cllp{0.4\textwidth}}
    \toprule
    & \textbf{Wort im Satzkontext} & \textbf{Wortklasse} & \textbf{Filter} \\
    &                              & \textbf{(falsch)}   &                 \\
    \midrule
    &&& \\
    (1) & \textit{Die zweite Sitzung wurde \ul{leider} gestört.}         & Komplementierer & \\ \cline{4-4}
    &&& \\
    (2) & \textit{Die \ul{Trockenheit} ist heuer nicht so schlimm.}      & Satzäquivalent & \\ \cline{4-4}
    &&& \\
    (3) & \textit{Marie kann Paul mit dem Generator \ul{nicht} helfen.}  & Substantiv & \\ \cline{4-4}
    &&& \\
    (4) & \textit{Da kann \ul{man} nichts machen.}                       & Verb & \\ \cline{4-4}
    &&& \\
    (5) & \textit{Ich rauche keine Zigaretten, \ul{sondern} Zigarren.}   & Konjunktion & \\ \cline{4-4}
    &&& \\
    (6) & \textit{Der FCR war so \ul{pleite} wie sonst kein Verein.}     & Adjektiv & \\ \cline{4-4}
    &&& \\
    (7) & \textit{Wir fahren selten \ul{mit} dem Auto zur Arbeit.}       & Adkopula & \\ \cline{4-4}
    &&& \\
    (8) & \textit{Es regnet, \ul{obwohl} es sehr warm ist.}              & Adverb & \\ \cline{4-4}
    &&& \\
    (9) & \textit{Gestern \ul{habe} der Regen nicht mehr aufgehört.}     & Präposition & \\ \cline{4-4}
    &&& \\
    (10)& \textit{Einen \ul{bestimmten} Espresso trinken wir gern.}      & Substantiv & \\ \cline{4-4}
  \end{tabular}}
\end{center}

