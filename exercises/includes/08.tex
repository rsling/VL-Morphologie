\section{Agentivische Verben}

Bilden Sie das Vorgangspassiv der folgenden Sätze (ohne Präpositionalphrase mit \textit{von}), ganz gleich ob das Ergebnis grammatisch ist oder nicht.
Fügen Sie kein Material hinzu und lassen Sie kein Material weg (außer dem Subjekt des Aktivs).
Wenn der Passivsatz ungrammatisch ist, vergessen Sie bitte nicht den Asterisk!
Entscheiden Sie anhand der Grammatikalität des Passivsatzes, ob die unterstrichenen Verben einen agentivischen Nominativ auf ihrer Valenzliste haben.

\Zeile

\begin{center}
  \begin{longtable}[h]{cp{0.8\textwidth}c}
    \toprule
    & \textbf{Zu passivierender Satz} & \textbf{Agentiv} \\
    \midrule
    &&\\
    (1) & \textit{Gestern ist die Petunie auf dem Balkon \ul{vertrocknet}.} & \\
    &&\\
    & \Sol{* Gestern ist auf dem Balkon vertrocknet worden.} & \\\cline{2-2}
    &&\\
    && \Solalt{\Square}{\Square} \\\cline{2-2}
    &&\\
    (2) & \textit{Ich werde den Wagen in die Werkstatt \ul{gefahren} haben.} & \\
    &&\\
    & \Sol{Der Wagen wird in die Werkstatt gefahren worden sein.} &\\\cline{2-2}
    &&\\
    && \Solalt{\XBox}{\Square} \\\cline{2-2}
    &&\\
    (3) & \textit{Die Kollegin hat Matthias \ul{geraten} zu kündigen.} & \\
    &&\\
    & \Sol{Matthias ist geraten worden zu kündigen.} &\\\cline{2-2}
    &&\\
    && \Solalt{\XBox}{\Square} \\\cline{2-2}
    &&\\
    (4) & \textit{Wir \ul{brachten} den Neuen die Regeln \ul{bei}.} & \\
    &&\\
    & \Sol{Die Regeln wurden den Neuen beigebracht.} &\\\cline{2-2}
    &&\\
    && \Solalt{\XBox}{\Square} \\\cline{2-2}
    &&\\
    (5) & \textit{Doro \ul{gießt} die Petunie.} & \\
    &&\\
    & \Sol{Die Petunie wird gegossen.} &\\\cline{2-2}
    &&\\
    && \Solalt{\XBox}{\Square} \\\cline{2-2}
    &&\\
    (6) & \textit{Die Petunie \ul{gefällt} mir überhaupt nicht.} & \\
    &&\\
    & \Sol{* Mir wird überhaupt nicht gefallen.} &\\\cline{2-2}
    &&\\
    && \Solalt{\Square}{\Square} \\\cline{2-2}
  \end{longtable}
\end{center}

\section{Verbtypen}

Entscheiden Sie für die unterstrichenen Verben in der folgenden Tabelle, zu welchem Valenztyp sie gehören.
Es kommen folgende Valenztypen infrage:

\begin{enumerate}\Lf
  \item unergatives Verb (UE)
  \item unakkusativ (UA)
  \item transitiv (TR)
  \item unergatives Dativverb (UED) % +antworten glauben +raten befehlen +danken vergeben drohen 
  \item unakkusatives Dativverb (UAD) % +gefallen passieren schmecken passen einfallen nützen +gelingen ähneln
  \item ditransitives Verb (DTR) % geben verkaufen +zeigen +beibringen +anbieten glauben
\end{enumerate}

\begin{center}
  \begin{longtable}[h]{clp{0.2\textwidth}}
    \toprule
    & \textbf{Verb im Satzkontext} & \textbf{Valenztyp} \\
    \midrule
    &&\\
    (1) & \textit{Doro \ul{dankt} Swem nicht.} &  \Sol{UED} \\\cline{3-3}
    &&\\
    (2) & \textit{Ich habe leider die Wanderjahre \ul{gelesen}.} &  \Sol{TR} \\\cline{3-3}
    &&\\
    (3) & \textit{Gestern wurde fleißig \ul{trainiert}.} &  \Sol{UE\slash TR} \\\cline{3-3}
    &&\\
    (4) & \textit{Man \ul{bot} der gnädigen Frau Konfekt \ul{an}.} &  \Sol{DTR} \\\cline{3-3}
    &&\\
    (5) & \textit{Im Kreisverkehr \ul{kracht} es öfters.} &  \Sol{UA} \\\cline{3-3}
    &&\\
    (6) & \textit{Mir ist in letzter Zeit mal wieder rein gar nichts \ul{gelungen}.} &  \Sol{UAD} \\\cline{3-3}
    &&\\
    (7) & \textit{\ul{Schieb} das weg!} &  \Sol{TR} \\\cline{3-3}
    &&\\
    (8) & \textit{Nächste Woche \ul{zeigen} wir euch den Weg.} &  \Sol{DTR} \\\cline{3-3}
    &&\\
    (9) & \textit{Warum \ul{antworten} Sie dem Gericht nicht?} &  \Sol{UED} \\\cline{3-3}
    &&\\
    (10) & \textit{Das Bild \ul{ruckelte} kurz und fiel dann aus.} &  \Sol{UA} \\\cline{3-3}
    &&\\
    (11) & \textit{Ich würde niemals Bier \ul{trinken}.} &  \Sol{TR} \\\cline{3-3}
    &&\\
    (12) & \textit{\ul{Vergib}, damit dir vergeben wird.} &  \Sol{UED} \\\cline{3-3}
  \end{longtable}
\end{center}

\section{Präpositionalobjekte}

% abstimmen über, achten auf, aufklären über, bitten um, denken an, gehen um, glauben an, halten für, 
% impfen gegen, nachdenken über, reagieren auf, sorgen für, warten auf

Wenden Sie den Auskopplungstest auf die unterstrichene Konstituente an und entscheiden Sie, ob sie ein Präpositionalobjekt (PO) ist.

\begin{center}
  \begin{longtable}[h]{cp{0.8\textwidth}c}
    \toprule
    & \textbf{Auszukoppelnde Konstituente im Satzkontext} & \textbf{PO} \\
    \midrule
    &&\\
    (1) & \textit{Das Volk hat \ul{über die Verfassung} abgestimmt.} & \\
    &&\\
    & \Sol{*Das Volk hat abgestimmt. Dies geschah über die Verfassung.} &\\\cline{2-2}
    &&\\
    && \Solalt{\XBox}{\Square} \\\cline{2-2}
    &&\\
    (2) & \textit{Wir stehen \ul{über dem Dorf} auf der Aussichtsplattform.} & \\
    &&\\
    & \Sol{? Wir stehen auf der Aussichtsplattform. Dies geschieht über dem Dorf.} &\\\cline{2-2}
    &&\\
    && \Solalt{\Square}{\Square} \\\cline{2-2}
    &&\\
    (3) & \textit{Wir essen über dem Dorf \ul{auf der Aussichtsplattform}.} & \\
    &&\\
    & \Sol{Wir essen auf der Aussichtsplattform. Dies geschieht über dem Dorf.} &\\\cline{2-2}
    &&\\
    && \Solalt{\Square}{\Square} \\\cline{2-2}
    &&\\
    (4) & \textit{Matthias besteht \ul{auf einer Abfindung}.} & \\
    &&\\
    & \Sol{*Matthias besteht. Dies geschieht auf einer Abfindung.} &\\\cline{2-2}
    &&\\
    && \Solalt{\XBox}{\Square} \\\cline{2-2}
    &&\\
    (5) & \textit{Doro sorgt \ul{für ihre Katze}.} & \\
    &&\\
    & \Sol{*Doro sorgt. Dies geschieht für ihre Katze.} &\\\cline{2-2}
    &&\\
    && \Solalt{\XBox}{\Square} \\\cline{2-2}
    &&\\
    (6) & \textit{Matthias fährt den Umzugswagen \ul{für mich}.} & \\
    &&\\
    & \Sol{Matthias fährt den Umzugswagen. Dies geschieht er für mich.} &\\\cline{2-2}
    &&\\
    && \Solalt{\Square}{\Square} \\\cline{2-2}
  \end{longtable}
\end{center}

