\section{Verbformen strategisch bilden}

Als Germanisten sollten Sie zu jedem lexikalischen Verb der deutschen Sprache jede mögliche Flexionsform sicher bilden können.
Stellen Sie sich vor, Sie hätten Englisch auf Lehramt studiert und wüssten zum Beispiel nicht, wie das \textit{present perfect} von \textit{to bring} gebildet wird.
Das wäre undenkbar, und gleichermaßen undenkbar sollte es sein, dass Ihnen so etwas im Deutschen unterläuft, zumal es sich bei Ihnen mehrheitlich um Erstsprecher des Deutschen handelt.

Wir unterscheiden gemäß \textit{Einführung in die grammatische Beschreibung des Deutschen} (EGBD) und gemäß der zugehörigen Videos im Deutschen die folgenden für die Flexion und analytische Formenbildung relevanten Merkmale mit ihren Werten:

\begin{enumerate}\Lf
  \item \textbf{Person} | 1, 2, 3
  \item \textbf{Numerus} | Singular, Plural
  \item \textbf{finites Tempus} | Präsens, Präteritum, Futur
  \item \textbf{nicht-finites Quasitempus} | Perfekt, Nicht-Perfekt
  \item \textbf{Modus} | Indikativ, Konjunktiv
  \item \textbf{Diathese} | Aktiv, Passiv
\end{enumerate}

Falls Ihnen die Trennung von Tempus und Perfekt nichts sagt, lesen Sie bitte nach.
Generell gilt, dass diese Erläuterungen und Übungen nur für Personen geeignet sind, die zunächst gründlich die anderen Materialien durchgearbeitet haben.
Rechnerisch haben wir es mit 144 Formen (oder 288, wenn Varianten mit Modalverb einbezogen werden) zu tun, auch wenn manche davon vielleicht nicht in nennenswerter Häufigkeit verwendet werden.
Der Versuch, in dieser Frage irgendetwas mechanisch auswendig zu lernen, ist also offensichtlich weniger als gar nicht zielführend.
Sie müssen verstehen, wie die Formen gebildet werden.

Beim Passiv betrachten wir nur das \textit{werden}-Passiv, da die anderen Passivkonstruktionen bezüglich der Flexion parallel, aber mit anderen Hilfsverben (Auxiliaren) gebildet werden.
Sie müssen außerdem die Formenbildung auch dann beherrschen, wenn Modalverben beteiligt sind (\textit{laufen müssen}, \textit{singen dürfen} usw.).

Ist eine Nennform (wie \textit{laufen}) und eine Merkmalspezifikation (wie 1.~Person Singular Präteritum Perfekt Konjunktiv Passiv) gegeben, können und sollten Sie \textbf{strategisch} die Form bilden.
Mit Intuition ist Ihnen vermutlich in dieser sprachlich ernsten Angelegenheit ab einem gewissen Komplexitätsgrad nicht geholfen.
Im Folgenden finden Sie eine Empfehlung, wie die Formen zu bilden sind.
Als allgemeine Strategie wird empfohlen, dem Hauptverb zunächst die gegebenenfalls erforderlichen Hilfsverben für Passiv, Perfekt und Futur hinzuzufügen, um dann das höchste Verb der Kette finit (nach Tempus, Modus, Person und Numerus) zu flektieren.

\subsection{Hierarchie der Hilfsverben}

Hier wird viel von der hierarchischen Ordnung der an analytischen Formen beteiligten Verben gesprochen.
Damit ist die Rektionsfolge (Statusrektion) gemeint, die Sie aus EGBD kennen.
Die Statusrektion der nicht-lexikalischen Verben, die wir hier betrachten, ist wie folgt:

\begin{itemize}\Lf
  \item Die Perfekt-Auxiliare \textit{sein} und \textit{haben} regieren den \textbf{3.~Status} (Partizip).
  \item Das Futur-Auxiliar \textit{werden} regiert den \textbf{1.~Status} (Infinitiv ohne \textit{zu}).
  \item Das Passiv-Auxiliar \textit{werden} regiert den \textbf{3.~Status} (Partizip).
  \item Modalverben wie \textit{wollen} regieren den \textbf{1.~Status} (Infinitiv ohne \textit{zu}).
\end{itemize}

In der normalen Abfolge stehen Verben von rechts nach links in einer Rektionsfolge, und die Hierarchie (rechts das hierarchisch höchste Verb, links das hierarchisch niedrigste Verb) ergibt sich daraus, dass das jeweils rechte Verb das links von ihm stehende Verb regiert.
Wenn das Zeichen $<$ als \textit{wird regiert von} gelesen wird, ergibt sich beispielsweise die folgende Darstellung der Hierarchie des lexikalischen Verbs \textit{kaufen} als Hauptverb (Träger der lexikalischen Semantik) und dreier Hilfsverben in einer analytischen Form:

\begin{center}
  \textit{gekauft} $<$ \textit{worden} $<$ \textit{sein} $<$ \textit{werden}
\end{center}

Das Futur-Auxiliar \textit{werden} regiert den ersten Status des Perfekt-Auxiliars \textit{sein}, das wiederum den dritten Status des Passiv-Auxiliars \textit{worden} regiert, welches seinerseits den dritten Status des lexikalischen Verbs \textit{gekauft} regiert.

\subsection{Hilfsverben hinzufügen}
\label{sec:hilfsverben}

Zur Bildung sowohl des Passivs als auch des Perfekts sind Hilfsverben notwendig.
Falls ein Passiv gebildet werden soll, fügen Sie zuerst \textit{werden} im Infinitiv hinzu und setzen Sie das lexikalische Verb ins Partizip.
Die Passivbildung steht als lexikalische Bildung (s.\ EGBD) dem Verb am nächsten und muss daher zuerst durchgeführt werden.
Die jeweils im relevanten Schritt hinzutretenden beziehungsweise veränderten Elemente sind in den Tabellen farbig hervorgehoben.

\begin{center}
  \begin{tabular}{lllr}
    \toprule
    \textbf{Lex.~Verb} & \textbf{Diathese} & \textbf{Form} & \\
    \midrule
    \textit{kaufen} & Aktiv & \textit{kaufen} & \grau{1.1} \\
    \textit{kaufen} & Passiv & \textit{\blau{ge}kauf\blau{t} \blau{werden}} & \grau{1.2}\\
    \bottomrule
  \end{tabular}
\end{center}

Als nächstes fügen Sie -- falls gefordert -- das Perfekt-Hilfsverb im Infinitiv hinzu und setzen Sie das bisher hierarchisch höchste Verb (d.\,h.\ das lexikalische Verb oder -- falls vorhanden -- \textit{werden}) ins Partizip.
Bei Hilfsverben brauchen Sie \textit{sein}, bei Modalverben \textit{haben}, ansonsten entscheiden Sie abhängig vom lexikalischen Verb, ob \textit{haben} oder \textit{sein} das angemessene Perfekt-Auxiliar ist.

\begin{center}
  \begin{tabular}{llllr}
    \toprule
    \textbf{Lex.~Verb} & \textbf{Diathese} & \textbf{Perfekt} & \textbf{Form} & \\
    \midrule
    \textit{kaufen} & Aktiv & Nicht-Perfekt & \textit{kaufen} & \grau{2.1 ← 1.1} \\
    \textit{kaufen} & Passiv & Nicht-Perfekt & \textit{gekauft werden} & \grau{2.2 ← 1.2} \\
    \midrule
    \textit{kaufen} & Aktiv & Perfekt & \textit{\blau{ge}kauf\blau{t haben}} & \grau{2.3 ← 1.1} \\
    \textit{kaufen} & Passiv & Perfekt & \textit{gekauft \blau{worden sein}} & \grau{2.4 ← 1.2} \\
    \bottomrule
  \end{tabular}
\end{center}

\newpage

Bei den finiten Tempora wird schließlich das Futur mit \textit{werden}, welches den ersten Status regiert, gebildet, und dieses Hilfsverb ist stets das hierarchisch höchste.
Sie sollten es als letztes hinzufügen.
Einige der Formen mögen vielleicht ungewöhnlich oder ungewohnt klingen, was aber nur ihrer Seltenheit geschuldet ist.
Bei den eingeklammerten Formen handelt es sich um solche, die vermutlich im Infinitiv nicht vorkommen, die also stets in eine finite Form gesetzt werden müssen (s.\,u.).

\begin{center}
  \begin{tabular}{lllllr}
    \toprule
    \textbf{Lex.~Verb} & \textbf{Diathese} & \textbf{Perfekt} & \textbf{Tempus} & \textbf{Form} & \\
    \midrule
    \textit{kaufen} & Aktiv & Nicht-Perfekt & --- & \textit{kaufen} & \grau{3.1 ← 2.1}  \\
    \textit{kaufen} & Passiv & Nicht-Perfekt & --- & \textit{gekauft werden} & \grau{3.2 ← 2.2}  \\
    \textit{kaufen} & Aktiv & Perfekt & --- & \textit{gekauft haben} & \grau{3.3 ← 2.3} \\
    \textit{kaufen} & Passiv & Perfekt & --- & \textit{gekauft worden sein} & \grau{3.4 ← 2.4}  \\
    \midrule
    \textit{kaufen} & Aktiv & Nicht-Perfekt & Futur & (\textit{kaufen \blau{werden}}) & \grau{3.5 ← 2.1}  \\
    \textit{kaufen} & Passiv & Nicht-Perfekt & Futur & (\textit{gekauft werden \blau{werden}}) & \grau{3.6 ← 2.2}  \\
    \textit{kaufen} & Aktiv & Perfekt & Futur & (\textit{gekauft haben \blau{werden}}) & \grau{3.7 ← 2.3}  \\
    \textit{kaufen} & Passiv & Perfekt & Futur & (\textit{gekauft worden sein \blau{werden}}) & \grau{3.8 ← 2.4}  \\
    \bottomrule
  \end{tabular}
\end{center}

\subsection{Finite Flexion hinzufügen}

Die finite Flexion nach Tempus, Modus und Person-Numerus (PN) muss nun dem hierarchisch höchsten Verb synthetisch (i.\,d.\,R.\ durch Antritt verschiedener Endungen) hinzugefügt werden.
Dies geschieht bei starken und schwachen Verben sowie Hilfsverben auf jeweils unterschiedliche Weise.

Das Futur-Hilfsverb \textit{werden} etwa muss finit flektieren und nimmt dazu formal seine Präsensformen an.
Wenn das Präteritum gebildet werden soll, wird bei den Vollverben der Präteritalstamm verwendet: Schwachen Verben wird \textit{-te} angehängt, starke Verben haben einen Präteritalstamm mit Ablaut (z.\,B.\ \textit{geb-} → \textit{gab-}).
Für den Konjunktiv wird ein \textit{-e} hinzugefügt, wenn der Präsens- oder Präteritalstamm nicht schon auf eines ausgeht.
Es treten weiterhin die PN1- oder PN2-Endungen für Person und Numerus an.
Das Futur-Hilfsverb \textit{werden} flektiert in Teilen unregelmäßig.
Für die damit hinreichend beschriebenen Formen von \textit{kaufen} ergibt sich jeweils mit der 2.~Person Singular als Beispiel für eine PN-Form die unten stehende Tabelle.%
\footnote{Die Formenpaare wie \textit{werdest\slash würdest} illustrieren ein faszinierendes Phänomen.
Wie Sie aus EGBD wissen, haben wir es im gegenwärtigen Deutsch mit der Situation zu tun, dass der Konjunktiv Präsens und der Konjunktiv Präteritum ihre Tempusbedeutung verloren haben, weswegen die Schulterminologie schließlich auch funktional orientiert von Konjunktiv 1 und Konjunktiv 2 spricht.
Das stets finite Futur bildet seine Formen im Indikativ formal ausschließlich mit dem Präsens des Hilfsverbs, da ein Präteritum semantisch widersprüchlich zur Futurbedeutung des Hilfsverbs wäre.
Im Konjunktiv eröffnet sich aber wegen des Verlusts der Tempusbedeutung die Möglichkeit, Formen des Futurs im Präsens und Präteritum zu bilden.
Formen wie \textit{werde} stellen dann formal ein Futur-Präsens und Formen wie \textit{würde} formal ein Futur-Präteritum dar.
Das grammatische System kämpft hier mit sprachgeschichtlichem Ballast und schafft zugleich neue funktionale Nischen, die langfristig zu völlig neuen grammatischen Wegen ausgebaut werden.}

\begin{center}
  \resizebox{\textwidth}{!}{\begin{tabular}{lllllllr}
    \toprule
    \textbf{Lex.~Verb} & \textbf{Dia} & \textbf{Perf} & \textbf{Temp} & \textbf{Mod} & \textbf{Per-Num} & \textbf{Form} & \\
    \midrule
    \textit{kaufen} & Aktiv  & Nicht-Perfekt & Präs  & Ind  & 2 Sg & \textit{kauf\blau{st}}                                      & \grau{4.1 ← 3.1}  \\
    \textit{kaufen} & Passiv & Nicht-Perfekt & Präs  & Ind  & 2 Sg & \textit{gekauft \blau{wirst}}                               & \grau{4.2 ← 3.2}  \\
    \textit{kaufen} & Aktiv  & Perfekt       & Präs  & Ind  & 2 Sg & \textit{gekauft \blau{hast}}                                & \grau{4.3 ← 3.3} \\
    \textit{kaufen} & Passiv & Perfekt       & Präs  & Ind  & 2 Sg & \textit{gekauft worden \blau{bist}}                         & \grau{4.4 ← 3.4}  \\
    \textit{kaufen} & Aktiv  & Nicht-Perfekt & Präs  & Konj & 2 Sg & \textit{kauf\blau{est}}                                     & \grau{4.5 ← 3.1}  \\
    \textit{kaufen} & Passiv & Nicht-Perfekt & Präs  & Konj & 2 Sg & \textit{gekauft werd\blau{est}}                             & \grau{4.6 ← 3.2}  \\
    \textit{kaufen} & Aktiv  & Perfekt       & Präs  & Konj & 2 Sg & \textit{gekauft hab\blau{est}}                              & \grau{4.7 ← 3.3} \\
    \textit{kaufen} & Passiv & Perfekt       & Präs  & Konj & 2 Sg & \textit{gekauft worden sei\blau{st}}                        & \grau{4.8 ← 3.4}  \\
    \midrule
    \textit{kaufen} & Aktiv  & Nicht-Perfekt & Prät  & Ind  & 2 Sg & \textit{kauf\blau{test}}                                    & \grau{4.9 ← 3.1}  \\
    \textit{kaufen} & Passiv & Nicht-Perfekt & Prät  & Ind  & 2 Sg & \textit{gekauft \blau{wurdest}}                             & \grau{4.10 ← 3.2}  \\
    \textit{kaufen} & Aktiv  & Perfekt       & Prät  & Ind  & 2 Sg & \textit{gekauft \blau{hattest}}                             & \grau{4.11 ← 3.3} \\
    \textit{kaufen} & Passiv & Perfekt       & Prät  & Ind  & 2 Sg & \textit{gekauft worden \blau{warst}}                        & \grau{4.12 ← 3.4}  \\
    \textit{kaufen} & Aktiv  & Nicht-Perfekt & Prät  & Konj & 2 Sg & \textit{kauf\blau{test}}                                    & \grau{4.13 ← 3.1}  \\
    \textit{kaufen} & Passiv & Nicht-Perfekt & Prät  & Konj & 2 Sg & \textit{gekauft \blau{würdest}}                             & \grau{4.14 ← 3.2}  \\
    \textit{kaufen} & Aktiv  & Perfekt       & Prät  & Konj & 2 Sg & \textit{gekauft \blau{hättest}}                             & \grau{4.15 ← 3.3} \\
    \textit{kaufen} & Passiv & Perfekt       & Prät  & Konj & 2 Sg & \textit{gekauft worden \blau{wärst}}                        & \grau{4.16 ← 3.4}  \\
    \midrule
    \textit{kaufen} & Aktiv  & Nicht-Perfekt & Futur & Ind  & 2 Sg & \textit{kaufen \blau{wirst}}                               & \grau{4.17 ← 3.5}  \\
    \textit{kaufen} & Passiv & Nicht-Perfekt & Futur & Ind  & 2 Sg & \textit{gekauft werden \blau{wirst}}                       & \grau{4.18 ← 3.6}  \\
    \textit{kaufen} & Aktiv  & Perfekt       & Futur & Ind  & 2 Sg & \textit{gekauft haben \blau{wirst}}                        & \grau{4.19 ← 3.7}  \\
    \textit{kaufen} & Passiv & Perfekt       & Futur & Ind  & 2 Sg & \textit{gekauft worden sein \blau{wirst}}                  & \grau{4.20 ← 3.8}  \\
    \textit{kaufen} & Aktiv  & Nicht-Perfekt & Futur & Konj & 2 Sg & \textit{kaufen \blau{werdest\slash würdest}}               & \grau{4.21 ← 3.5}  \\
    \textit{kaufen} & Passiv & Nicht-Perfekt & Futur & Konj & 2 Sg & \textit{gekauft werden \blau{werdest\slash würdest}}       & \grau{4.22 ← 3.6}  \\
    \textit{kaufen} & Aktiv  & Perfekt       & Futur & Konj & 2 Sg & \textit{gekauft haben \blau{werdest\slash würdest}}        & \grau{4.23 ← 3.7}  \\
    \textit{kaufen} & Passiv & Perfekt       & Futur & Konj & 2 Sg & \textit{gekauft worden sein \blau{werdest\slash würdest}}  & \grau{4.24 ← 3.8}  \\
    \bottomrule
  \end{tabular}}
\end{center}

In der nachstehenden Tabelle finden Sie ein Beispiel für dasselbe mit einem starken Verb, das zudem \textit{sein} als Perfekt-Hilfsverb verlangt.
Die Passivformen scheinen eventuell zunächst nicht akzeptabel zu sein, aber in der Bedeutung wie im Beispiel \textit{einen Marathon laufen} sind sie einwandfrei.
Wegen dieser Formen wurde hier auf die 3.~Person Singular als Beispielform ausgewichen.%
\footnote{Strenggenommen mischen wir hier damit zwei verschiedene Verben \textit{laufen}:
Das eine ist transitiv wie in \textit{einen Marathon laufen} und bildet das Perfekt daher mit \textit{haben}.
Das andere ist intransitiv und bildet als Bewegungsverb das Perfekt selbstverständlich mit \textit{sein}.
Für Kenner sollte das allerdings eher eine Bereicherung als einen Nachteil darstellen.}

\begin{center}
  \resizebox{\textwidth}{!}{\begin{tabular}{llllllll}
    \toprule
    \textbf{Lex.~Verb} & \textbf{Dia} & \textbf{Perf} & \textbf{Temp} & \textbf{Mod} & \textbf{Per-Num} & \textbf{Form} & \\
    \midrule
    \textit{laufen} & Aktiv  & Nicht-Perfekt & Präs  & Ind  & 3 Sg & \textit{läuf\blau{t}}                                       & \grau{5.1} \\
    \textit{laufen} & Passiv & Nicht-Perfekt & Präs  & Ind  & 3 Sg & \textit{gelaufen \blau{wird}}                               & \grau{5.2} \\
    \textit{laufen} & Aktiv  & Perfekt       & Präs  & Ind  & 3 Sg & \textit{gelaufen \blau{sei}}                                & \grau{5.3} \\
    \textit{laufen} & Passiv & Perfekt       & Präs  & Ind  & 3 Sg & \textit{gelaufen worden \blau{ist}}                         & \grau{5.4} \\
    \textit{laufen} & Aktiv  & Nicht-Perfekt & Präs  & Konj & 3 Sg & \textit{lauf\blau{est}}                                     & \grau{5.5} \\
    \textit{laufen} & Passiv & Nicht-Perfekt & Präs  & Konj & 3 Sg & \textit{gelaufen werd\blau{e}}                              & \grau{5.6} \\
    \textit{laufen} & Aktiv  & Perfekt       & Präs  & Konj & 3 Sg & \textit{gelaufen \blau{sei}}                                & \grau{5.7} \\
    \textit{laufen} & Passiv & Perfekt       & Präs  & Konj & 3 Sg & \textit{gelaufen worden \blau{sei}}                         & \grau{5.8} \\
    \midrule
    \textit{laufen} & Aktiv  & Nicht-Perfekt & Prät  & Ind  & 3 Sg & \textit{\blau{lief}}                                        & \grau{5.9} \\
    \textit{laufen} & Passiv & Nicht-Perfekt & Prät  & Ind  & 3 Sg & \textit{gelaufen \blau{wurde}}                              & \grau{5.10} \\
    \textit{laufen} & Aktiv  & Perfekt       & Prät  & Ind  & 3 Sg & \textit{gelaufen \blau{war}}                                & \grau{5.11} \\
    \textit{laufen} & Passiv & Perfekt       & Prät  & Ind  & 3 Sg & \textit{gelaufen worden \blau{war}}                         & \grau{5.12} \\
    \textit{laufen} & Aktiv  & Nicht-Perfekt & Prät  & Konj & 3 Sg & \textit{\blau{liefe}}                                       & \grau{5.13} \\
    \textit{laufen} & Passiv & Nicht-Perfekt & Prät  & Konj & 3 Sg & \textit{gelaufen \blau{würde}}                              & \grau{5.14} \\
    \textit{laufen} & Aktiv  & Perfekt       & Prät  & Konj & 3 Sg & \textit{gelaufen \blau{wäre}}                               & \grau{5.15} \\
    \textit{laufen} & Passiv & Perfekt       & Prät  & Konj & 3 Sg & \textit{gelaufen worden \blau{wäre}}                        & \grau{5.16} \\
    \midrule
    \textit{laufen} & Aktiv  & Nicht-Perfekt & Futur & Ind  & 3 Sg & \textit{laufen \blau{wird}}                                & \grau{5.17}  \\
    \textit{laufen} & Passiv & Nicht-Perfekt & Futur & Ind  & 3 Sg & \textit{gelaufen werden \blau{wird}}                       & \grau{5.18}  \\
    \textit{laufen} & Aktiv  & Perfekt       & Futur & Ind  & 3 Sg & \textit{gelaufen sein \blau{wird}}                         & \grau{5.19}  \\
    \textit{laufen} & Passiv & Perfekt       & Futur & Ind  & 3 Sg & \textit{gelaufen worden sein \blau{wird}}                  & \grau{5.20}  \\
    \textit{laufen} & Aktiv  & Nicht-Perfekt & Futur & Konj & 3 Sg & \textit{laufen \blau{werde\slash würde}}                   & \grau{5.21}  \\
    \textit{laufen} & Passiv & Nicht-Perfekt & Futur & Konj & 3 Sg & \textit{gelaufen werden \blau{werde\slash würde}}          & \grau{5.22}  \\
    \textit{laufen} & Aktiv  & Perfekt       & Futur & Konj & 3 Sg & \textit{gelaufen sein \blau{werde\slash würde}}            & \grau{5.23}  \\
    \textit{laufen} & Passiv & Perfekt       & Futur & Konj & 3 Sg & \textit{gelaufen worden sein \blau{werde\slash würde}}     & \grau{5.24}  \\
    \bottomrule
  \end{tabular}}
\end{center}

Formen mit Perfekt- oder Passiv-Hilfsverb als hierarchisch höchstes Verb müssen mit den entsprechenden Formen der Auxiliare gebildet werden.
Die unten stehenden Tabellen listen diese Formen auf.

Das Verb \textit{haben} bildet seine Formen nahezu vorhersagbar, wenn man einen Präsensstamm \textit{hab-} und einen Präteritalstamm \textit{hat-} annimmt.
Das Verb \textit{sein} ist hingegen unregelmäßig beziehungsweise suppletiv, da es diachron aus drei Stämmen ursprünglich verschiedener Verben entstanden ist.
Infolgedessen ergeben sich Formen wie \textit{gekauft worden wäre} als 1.~Person Singular Präteritum Perfekt Konjunktiv Passiv, weil \textit{gekauft worden sein} der analytische Teil der Form ist (Perfekt Passiv Infinitiv) und \textit{wäre} die 1.~Person Singular Präteritum Konjunktiv von \textit{sein} ist.

\begin{center}
  \begin{tabular}{lllll}
    \toprule
    & \textbf{Präs Ind} & \textbf{Prät Ind} & \textbf{Präs Konj} & \textbf{Prät Konj} \\
    \midrule
    \textbf{1 Sg} & habe  & hatte   & habe   & hätte \\
    \textbf{2 Sg} & hast  & hattest & habest & hättest \\
    \textbf{3 Sg} & hat   & hatte   & habe   & hätte \\
    \textbf{1 Pl} & haben & hatten  & haben  & hätten \\
    \textbf{2 Pl} & habt  & hattet  & habet  & hättet \\
    \textbf{3 Pl} & haben & hatten  & haben  & hätten \\
    \bottomrule
  \end{tabular}
\end{center}

\begin{center}
  \begin{tabular}{lllll}
    \toprule
    & \textbf{Präs Ind} & \textbf{Prät Ind} & \textbf{Präs Konj} & \textbf{Prät Konj} \\
    \midrule
    \textbf{1 Sg} & bin   & war     & sei    & wäre   \\
    \textbf{2 Sg} & bist  & warst   & seist  & wärest \\
    \textbf{3 Sg} & ist   & war     & sei    & wäre   \\
    \textbf{1 Pl} & sind  & waren   & seien  & wären  \\
    \textbf{2 Pl} & seid  & wart    & seid   & wäret  \\
    \textbf{3 Pl} & sind  & waren   & seien  & wären \\
    \bottomrule
  \end{tabular}
\end{center}

\subsection{Beteiligung von Modalverben}

Unter Beteiligung eines Modalverbs ergeben sich lange und in einigen Fällen selbst für Erstsprecher fragwürdige Ketten von Verben.
Zu beachten ist, dass Modalverben im Perfekt (wenn sie also vom Hilfsverb \textit{haben} abhängen) normgerecht \textbf{nicht} im Partizip stehen, sondern im sogenannten Ersatzinfinitiv.
Statt \textit{dass er laufen gemusst hat} heißt es also \textit{dass er laufen müssen hat} oder mit der sogenannten Oberfeldumstellung, die nur mit Ersatzinfinitiv vorkommt, \textit{dass er hat laufen müssen}.
Hier werden grundsätzlich die Varianten mit Ersatzinfinitiv, aber ohne Oberfeldumstellung angegeben.

Die hierarchische Position des Modalverbs liegt \textbf{über} der des Passiv-Hilfsverbs und \textbf{unter} der des Perfekt-Hilfsverbs.
Von den Sätzen in (\ref{ex:modalpassiv}) ist daher nur (\ref{ex:modalpassiva}) grammatisch, und nur solche Strukturen finden sich in großen Textkorpora.

\begin{exe}
  \ex\label{ex:modalpassiv}
  \begin{xlist}
    \ex[ ]{Es hat gekauft werden dürfen.\label{ex:modalpassiva}}
    \ex[*]{Es ist kaufen gedurft worden.\label{ex:modalpassivb}}
  \end{xlist}
\end{exe}

In der unten stehenden Tabelle werden nur die Infinitive angegeben, da die finiten Formen auf Basis dieser leicht gebildet werden können:

\begin{center}
  \begin{tabular}{lllll}
    \toprule
    \textbf{Lex.~Verb} & \textbf{Diathese} & \textbf{Perfekt} & \textbf{Form} \\
    \midrule
    \textit{kaufen wollen} & Aktiv & Perfekt & \textit{kaufen wollen \blau{haben}} \\
    \textit{kaufen wollen} & Passiv & Perfekt & \textit{gekauft \blau{werden} wollen \blau{haben}} \\
    \midrule
    \textit{kaufen wollen} & Aktiv & Nicht-Perfekt & \textit{kaufen wollen} \\
    \textit{kaufen wollen} & Passiv & Nicht-Perfekt & \textit{gekauft \blau{werden} wollen} \\
    \bottomrule
  \end{tabular}
\end{center}

Um die Formen nun finit zu flektieren, muss nur das jeweils letzte und hierarchisch höchste Verb nach Tempus, Modus und Person-Numerus verändert werden.
Im Futur kommt außerdem das Hilfsverb \textit{werden} hinzu.
Es folgen einige Beispiele, jeweils in Verb-Zweit-Stellung, bei der das finite Verb nach vorne gestellt wird.
In dieser Satzgliedstellung klingen die Beispiele in der Regel akzeptabler.
Das verschobene (bewegte) finite Verb ist dabei hervorgehoben:

\begin{exe}
  \ex[ ]{Er \blau{habe} kaufen wollen.\\\grau{\small Aktiv Perfekt Konjunktiv Präsens 3. Person Singular}\label{ex:komplexa}}
  \ex[ ]{Er \blau{hätte} gekauft werden wollen.\\\grau{\small Passiv Perfekt Konjunktiv Präsens 3. Person Singular}\label{ex:komplexb}}
  \ex[ ]{Er \blau{wird} kaufen wollen haben.\\\grau{\small Aktiv Perfekt Indikativ Futur 3. Person Singular}\label{ex:komplexc}}
  \ex[?]{Er \blau{werde} gekauft werden wollen haben.\\\grau{\small Passiv Perfekt Konjunktiv Futur 3. Person Singular}\label{ex:komplexd}}
\end{exe}

Insbesondere Satz (\ref{ex:komplexd}) ist grenzwertig.
Dies liegt allerdings eher an seiner kaum nachvollziehbaren Bedeutung als an seiner Morphosyntax.
Eine mögliche Paraphrase des Satzes wäre:
\textit{In der Zukunft wird ein Zustand abgeschlossen sein, in dem er es sich gewünscht hat, gekauft zu werden.}
Durch den Konjunktiv wird dieser Inhalt zusätzlich als Redewiedergabe markiert.
Sie müssen sich letztlich selbst überlegt haben werden, ob Sie sich eine solche Situation vorstellen können und ob Sie dafür eine Verbform benötigen.
Die Form an sich müssen Sie in jedem Fall selbständig bilden können.
Es wird erzählt, nach der Klausur werde diese Form gebildet werden können müssen haben.

Als Pendants ohne Ersatzinfinitiv ergeben sich (\ref{ex:komplexzweia})--(\ref{ex:komplexzweid}), die für manche Sprecher -- je nach dialektaler Herkunft -- möglicherweise akzeptabler sind.

\begin{exe}
  \ex[ ]{Er \blau{habe} kaufen gewollt.\label{ex:komplexzweia}}
  \ex[ ]{Er \blau{hätte} gekauft werden gewollt.\label{ex:komplexzweib}}
  \ex[ ]{Er \blau{wird} kaufen gewollt haben.\label{ex:komplexzweic}}
  \ex[?]{Er \blau{werde} gekauft werden gewollt haben.\label{ex:komplexzweid}}
\end{exe}

\subsection{Zusammenfassung des Ablaufs der Formenbildung}

Wenn Sie die Form eines Verbs mit vorgegebenen Eigenschaften bilden sollen, sollten Sie sich streng an den folgenden Ablauf in genau der angegebenen Reihenfolge halten und sich daher so etwas einprägen wie: \textit{\textbf{Voll -- Passiv -- Modal -- Futur -- und ganz hinten die Flexion!}}
Alles, was Sie bei diesem Ablauf hinzufügen (Auxiliare oder Flexionsendungen), fügen Sie stets \textbf{rechts} an.
Es wird ein Beispiel in wachsender Komplexität gegeben.

\begin{enumerate}
  \item Beginnen mit dem \textbf{lexikalischen Verb} (auch \textbf{Vollverb} genannt) im Infinitiv.
    \begin{enumerate}
      \item \textit{graben}
    \end{enumerate} 
  \item Falls ein \textbf{Passiv} gebildet werden soll: Das Auxiliar \textit{werden} hinzufügen und lexikalisches Verb ins Partizip setzen.
    \begin{enumerate}
      \item \textit{gegraben werden}
    \end{enumerate}
  \item Falls ein \textbf{Modalverb} beteiligt ist: Dieses Modalverb im Infinitiv hinzufügen, das Verb davor bleibt im Infinitiv.
    \begin{enumerate}
      \item \textit{graben dürfen}
      \item \textit{gegraben werden dürfen}
    \end{enumerate} 
  \item Falls ein \textbf{Perfekt} gebildet werden soll: Das passende Perfekt-Hilfsverb (\textit{haben} oder \textit{sein}) hinzufügen und das Verb davor ins Partizip setzen -- mit Ausnahme der Modalverben und ein paar anderer Verben, die in den Infinitiv gesetzt werden müssen.
    \begin{enumerate}
      \item \textit{gegraben haben}
      \item \textit{gegraben worden sein}
      \item \textit{graben dürfen haben}
      \item \textit{gegraben werden dürfen haben}
    \end{enumerate} 
  \item Falls ein \textbf{Futur} gebildet werden soll: Das Auxiliar \textit{werden} hinzufügen und das Verb davor im Infinitiv belassen.
    \begin{enumerate}
      \item \textit{graben werden}
      \item \textit{gegraben werden werden}
      \item \textit{graben dürfen werden}
      \item \textit{gegraben werden dürfen werden}
      \item \textit{gegraben haben werden}
      \item \textit{gegraben worden sein werden}
      \item \textit{graben dürfen haben werden}
      \item \textit{gegraben werden dürfen haben werden} (?)
    \end{enumerate}
  \item Die \textbf{Modus-Tempus-Person-Numerus-Flexion} des \textbf{letzten Verbs} in der Kette an die geforderten Formen anpassen. Beim Futur nehmen Sie dabei die formalen Präsensformen des Futur-Hilfsverbs \textit{werden}. Für den Rest gelten die bekannten Regeln. Die Beispiele sind hier alle im Konjunktiv Präteritum der dritten Person.
    \begin{enumerate}
      \item \textit{grübe}
      \item \textit{gegraben würde}
      \item \textit{graben dürfte}
      \item \textit{gegraben werden dürfte}
      \item \textit{gegraben hätte}
      \item \textit{gegraben worden wäre}
      \item \textit{graben dürfen hätte}
      \item \textit{gegraben werden dürfen hätte}
    \end{enumerate} 
  \item \textbf{Schlussprüfung}: Fragen Sie sich, ob die gebildete Form überhaupt Deutsch ist, oder ob Ihnen ein Fehler unterlaufen ist. Erst zu diesem Zeitpunkt können Sie (für sich selbst zur Prüfung) die letzte Form aus der Kette nach links stellen und \textit{er} oder ein anderes in Person und Numerus passendes Pronomen davorstellen. Für die vorherigen Beispiele ergibt das die untenstehenden grammatischen Sätze. (Achtung: Die Aufgabenstellung in der Klausur fordert in der Regel die zuletztgenannte Reihenfolge, nicht die untenstehende.)
    \begin{enumerate}
      \item \textit{Er grübe.}
      \item \textit{Er würde gegraben.}
      \item \textit{Er dürfte graben.}
      \item \textit{Er dürfte gegraben werden.}
      \item \textit{Er hätte gegraben.}
      \item \textit{Er wäre gegraben worden.}
      \item \textit{Er hätte graben dürfen.}
      \item \textit{Er hätte gegraben werden dürfen.}
    \end{enumerate}
\end{enumerate}

\newpage

Zum Abschluss: Wenn Sie zu früh anfangen, finite Formen zu bilden und diese gar nach links zu verschieben, werden Sie ziemlich sicher durcheinanderkommen.
Schauen Sie sich die folgende Ableitung nach dem obenstehenden Schema an:

\begin{enumerate}
  \item \textit{graben}\\
    \grau{Vollverb}
  \item \textit{gegraben werden}\\
    \grau{Vollverb + Passivauxiliar}
  \item \textit{gegraben werden dürfen}\\
    \grau{Vollverb + Passivauxiliar + Modalverb}
  \item \textit{gegraben werden dürfen haben}\\
    \grau{Vollverb + Passivauxiliar + Modalverb + Perfektauxiliar}
  \item \textit{gegraben werden dürfen hätte}\\
    \grau{Vollverb + Passivauxiliar + Modalverb + Perfektauxiliar 3.~Person Konjunktiv Präteritum}
\end{enumerate}

Wenn Sie diese Ableitung durchführen und dabei von Anfang an finite Formen in der 3.~Person Konjunktiv Präteritum in der "`Schulreihenfolge"' (= finites Verb immer zuerst) bilden, sieht die Reihe folgendermaßen aus:

\begin{enumerate}
  \item \textit{grübe}
  \item \textit{würde gegraben}
  \item \textit{dürfte gegraben werden}
  \item \textit{hätte gegraben werden dürfen}
\end{enumerate}

Sie können die Formen freilich so bilden, wie Sie es gerne mögen, aber das Springen der Flexionsmerkmale von einem Verb zum anderen und der ständige Wechsel der Verbfolge führt bei dieser Technik vorhersagbar und nachweislich zu unnötigen Fehlern.

\subsection{Korrespondenzen zur Schulterminologie}

Es gelten folgende Korrespondenzen zwischen Schulterminologie (zuerst genannt) und einer grammatisch informierten Beschreibung, die -- wie Sie oben gesehen haben -- unabdinglich für ein vertieftes Verständnis deutscher Verbformen ist.

\begin{itemize}\Lf
  \item Perfekt → Präsens Perfekt
  \item Plusquamperfekt → Präteritum Perfekt
  \item Futur 1 → Futur
  \item Futur 2 → Futur Perfekt
  \item Konjunktiv 1 → Konjunktiv Präsens
  \item Konjunktiv 2 → Konjunktiv Präteritum
\end{itemize}

\newpage

\section{Übung zur Bildung verbaler Flexionsformen}

Bilden Sie nach dem eingeführten Schema die angegebenen Formen der genannten lexikalischen Verben.
Wählen Sie stets die Wortstellung, wie sie im Nebensatz auftritt, in der also das finite Verb am Ende steht und sich die Rektionshierarchie in der Reihenfolge der Verben spiegelt.\\

Bilden Sie ausdrücklich \ul{\textbf{keine Ersatzformen}}, also \ul{\textbf{nicht}} den Konjunktiv Präteritum bei Formengleichheit (des Konjunktiv Präsens mit dem Indikativ) und ebenfalls \ul{\textbf{nicht}} die \textit{würde}-Paraphrase für Formen des Konjunktiv Präteritums bei Formengleichheit (mit dem Indikativ).\\

Beim Konjunktiv des Futurs können Sie sich entscheiden zwischen einem formalen Präsensfutur (das dem Konjunktiv I entspricht, also \textit{werde}, \textit{werdest}, \textit{werde} usw.) und einem Präteritumsfutur (das dem Konjunktiv II entspricht, also \textit{würde}, \textit{würdest}, \textit{würde} usw.).
Der Unterschied kann in unserer Analyse nicht klar ausformuliert werden.

\Zeile

\aufgabeginn
\centering
\begingroup\footnotesize
\begin{longtable}{clllllllp{0.3\textwidth}}
  \toprule
  & \textbf{Lex.~Verb} & \textbf{Dia} & \textbf{Perf} & \textbf{Temp} & \textbf{Mod} & \textbf{Per} & \textbf{Num} & \textbf{Form} \\
  \midrule
  &&&&&&&& \\
  \aufg & \textit{geben              } & Pass   & Perf     & Prät   & Ind    & 1      & Sg     & \Sol{gegeben worden war} \\\cline{9-9}
  &&&&&&&& \\
  \aufg & \textit{springen           } & Akt    & Nicht-Perf   & Prät   & Konj   & 2      & Pl     & \Sol{spränget} \\\cline{9-9}
  &&&&&&&& \\
  \aufg & \textit{boxen              } & Akt    & Nicht-Perf   & Prät   & Konj   & 3      & Sg     & \Sol{boxte} \\\cline{9-9}
  &&&&&&&& \\
  \aufg & \textit{intensivieren      } & Pass   & Perf     & Präs   & Ind    & 1      & Pl     & \Sol{intensiviert worden sind} \\\cline{9-9}
  &&&&&&&& \\
  \aufg & \textit{versägen           } & Akt    & Perf     & Präs   & Konj   & 2      & Sg     & \Sol{versägt habest} \\\cline{9-9}
  &&&&&&&& \\
  \aufg & \textit{lachen müssen      } & Pass   & Nicht-Perf   & Präs   & Konj   & 3      & Pl     & \Sol{\rot{Verb nicht passivierbar!}} \\\cline{9-9}
  &&&&&&&& \\
  \aufg & \textit{unterstellen       } & Pass   & Nicht-Perf   & Fut    & Ind    & 1      & Sg     & \Sol{untergestellt werden werde} \\\cline{9-9}
  &&&&&&&& \\
  \aufg & \textit{besorgen           } & Akt    & Perf     & Fut    & Ind    & 2      & Pl     & \Sol{besorgt haben werdet} \\\cline{9-9}
  &&&&&&&& \\
  \aufg & \textit{gutheißen          } & Akt    & Perf     & Fut    & Ind    & 3      & Sg     & \Sol{gutgeheißen haben wird} \\\cline{9-9}
  &&&&&&&& \\
  \aufg & \textit{zeichnen dürfen    } & Pass   & Nicht-Perf   & Präs   & Konj   & 1      & Pl     & \Sol{gezeichnet werden dürfen} \\\cline{9-9}
  &&&&&&&& \\
  \aufg & \textit{singen             } & Pass   & Nicht-Perf   & Prät   & Konj   & 2      & Sg     & \Sol{gesungen würdest} \\\cline{9-9}
  &&&&&&&& \\
  \aufg & \textit{sprechen           } & Akt    & Perf     & Fut    & Ind    & 3      & Pl     & \Sol{gesprochen haben werden} \\\cline{9-9}
  &&&&&&&& \\
  \aufg & \textit{testen             } & Akt    & Perf     & Fut    & Ind    & 1      & Sg     & \Sol{getestet haben werde} \\\cline{9-9}
  &&&&&&&& \\
  \aufg & \textit{bedenken           } & Pass   & Nicht-Perf   & Präs   & Konj   & 2      & Pl     & \Sol{bedacht werdet} \\\cline{9-9}
  &&&&&&&& \\
  \aufg & \textit{lesen können müssen} & Akt    & Nicht-Perf   & Prät   & Ind    & 3      & Sg     & \Sol{lesen können musste} \\\cline{9-9}
  &&&&&&&& \\
  \aufg & \textit{nehmen wollen      } & Pass   & Perf     & Prät   & Konj   & 1      & Pl     & \Sol{genommen werden wollen hätten} \\\cline{9-9}
  &&&&&&&& \\
  \aufg & \textit{sehen mögen        } & Akt    & Perf     & Präs   & Konj   & 2      & Sg     & \Sol{sehen mögen habest} \\\cline{9-9}
  &&&&&&&& \\
  \aufg & \textit{brechen            } & Pass   & Nicht-Perf   & Präs   & Ind    & 3      & Pl     & \Sol{gebrochen werden} \\\cline{9-9}
  &&&&&&&& \\
  \aufg & \textit{diktieren können   } & Pass   & Nicht-Perf   & Fut    & Ind    & 1      & Sg     & \Sol{diktiert werden können werde} \\\cline{9-9}
  &&&&&&&& \\
  \aufg & \textit{husten             } & Akt    &  Perf    & Fut    & Konj   & 2      & Pl     & \Sol{gehustet haben werdet} \\\cline{9-9}
  &&&&&&&& \\
  \aufg & \textit{rumpeln            } & Akt    &  Perf    & Prät   & Konj   & 3      & Sg     & \Sol{gerumpelt hätte} \\\cline{9-9}
  &&&&&&&& \\
  \aufg & \textit{beschlagen         } & Pass   &  Perf    & Präs   & Ind    & 1      & Pl     & \Sol{beschlagen worden sind} \\\cline{9-9}
  &&&&&&&& \\
  \aufg & \textit{verschlechtern     } & Akt    &  Perf    & Prät   & Konj   & 2      & Sg     & \Sol{verschlechtert hättest} \\\cline{9-9}
  &&&&&&&& \\
  \aufg & \textit{aufsammeln         } & Pass   &  Perf    & Prät   & Ind    & 3      & Pl     & \Sol{aufgesammelt worden waren} \\\cline{9-9}
  &&&&&&&& \\
  \aufg & \textit{kriechen           } & Akt    &  Nicht-Perf  & Präs   & Konj   & 1      & Sg     & \Sol{krieche} \\\cline{9-9}
\end{longtable}
\endgroup
