\section{Traditionelle Flexionsklassen der Substantive}

Bilden Sie den Nominativ Plural der Substantive in der nachstehenden Tabelle und bestimmen Sie das Genus (M, N, F) sowie die traditionelle Flexionsklasse.
Die traditionellen Flexionsklassen sind:

\begin{enumerate}\Lf
  \item schwache Maskulina (Schw) % Hase, Graf 
  \item starke Maskulina bzw.\ Neutra (St) % Zahn, Brot
  \item starke im Plural endungslose Maskulina bzw.\ Neutra (St$-$) % Boden
  \item gemischte Maskulina bzw.\ Neutra (Gem) % Schmerz, Auge
  \item Maskulina bzw.\ Neutra auf \textit{-er} im Plural (Er)\footnotemark[1] % Strauch, Holz
  \item \textit{e}-Feminina (Fe)\footnotemark[1] % Hand
  \item im Plural endungslose Feminina (F$-$)\footnotemark[1] % Tochter
  \item Feminina auf \textit{-en} bzw.\ \textit{-n} im Plural (Fn)\footnotemark[2] % Achse, Schicht
  \item \textit{s}-Klasse (S) % Kamera, Risiko
\end{enumerate}

\footnotetext[1]{Diese Substantive werden manchmal auch als stark bezeichnet.}
\footnotetext[2]{Diese Substantive werden manchmal auch als schwach bezeichnet.}

Weiterhin nehmen Sie die Unterklassifikation nach Schwa-Haltigkeit (orthographisches <e> in der Endsilbe) für Schw, Gem und Fn. Gemeint ist damit, ob das Pluralsuffix \textit{-en} oder \textit{-n} lautet:

\begin{itemize}\Lf
  \item $+$e | mit <e>\slash Schwa
  \item $-$e | ohne <e>\slash Schwa
\end{itemize}

Schließlich klassifizieren Sie danach, ob der Stammvokal im Zuge der Pluralbildung umgelautet wird oder nicht:

\begin{itemize}\Lf
  \item $+$U | mit Umlaut
  \item $-$U | ohne Umlaut
\end{itemize}

Für das erste Wort wird die Lösung beispielhaft gegeben.

\newpage

\begin{center}
  \begin{longtable}[h]{rlp{0.25\textwidth}cp{0.075\textwidth}cp{0.075\textwidth}cp{0.075\textwidth}cp{0.075\textwidth}}
    \toprule
    & \textbf{Substantiv} & \textbf{Pluralform} && \textbf{Genus} && \textbf{Klasse} && \textbf{<e>} && \textbf{Umlaut} \\
    \midrule{}
    \endhead
    (0) & \textit{Ohr} & \grau{\textit{Ohren}} && \grau{N} && \grau{\textit{Gem}} && \grau{$+$e} && \grau{$-$U} \\
    &&&&&&&& \\
    (1) & \textit{Schmerz} & \Sol{Schmerzen} && \Sol{M} && \Sol{Gem} && \Sol{$+$e} && \Sol{$-$U} \\\cline{3-3}\cline{5-5}\cline{7-7}\cline{9-9}\cline{11-11}
    &&&&&&&& \\
    (2) & \textit{Brot} & \Sol{Brote} && \Sol{N} && \Sol{St} && \Sol{} && \Sol{$-$U} \\\cline{3-3}\cline{5-5}\cline{7-7}\cline{9-9}\cline{11-11}
    &&&&&&&& \\
    (3) & \textit{Tochter} & \Sol{Töchter} && \Sol{F} && \Sol{F$-$} && \Sol{} && \Sol{$+$U} \\\cline{3-3}\cline{5-5}\cline{7-7}\cline{9-9}\cline{11-11}
    &&&&&&&& \\
    (4) & \textit{Boden} & \Sol{Böden} && \Sol{M} && \Sol{St$-$} && \Sol{} && \Sol{$+$U} \\\cline{3-3}\cline{5-5}\cline{7-7}\cline{9-9}\cline{11-11}
    &&&&&&&& \\
    (5) & \textit{Zahn} & \Sol{Zähne} && \Sol{M} && \Sol{St} && \Sol{} && \Sol{$+$U} \\\cline{3-3}\cline{5-5}\cline{7-7}\cline{9-9}\cline{11-11}
    &&&&&&&& \\
    (6) & \textit{Auge} & \Sol{Augen} && \Sol{N} && \Sol{Gem} && \Sol{$-$e} && \Sol{$-$U} \\\cline{3-3}\cline{5-5}\cline{7-7}\cline{9-9}\cline{11-11}
    &&&&&&&& \\
    (7) & \textit{Holz} & \Sol{Hölzer} && \Sol{N} && \Sol{Er} && \Sol{} && \Sol{$+$U} \\\cline{3-3}\cline{5-5}\cline{7-7}\cline{9-9}\cline{11-11}
    &&&&&&&& \\
    (8) & \textit{Strauch} & \Sol{Sträucher} && \Sol{M} && \Sol{Er} && \Sol{} && \Sol{$+$U} \\\cline{3-3}\cline{5-5}\cline{7-7}\cline{9-9}\cline{11-11}
    &&&&&&&& \\
    (9) & \textit{Kamera} & \Sol{Kameras} && \Sol{F} && \Sol{S} && \Sol{} && \Sol{$-$U} \\\cline{3-3}\cline{5-5}\cline{7-7}\cline{9-9}\cline{11-11}
    &&&&&&&& \\
    (10) & \textit{Hand} & \Sol{Hände} && \Sol{F} && \Sol{Fe} && \Sol{} && \Sol{$+$U} \\\cline{3-3}\cline{5-5}\cline{7-7}\cline{9-9}\cline{11-11}
    &&&&&&&& \\
    (11) & \textit{Achse} & \Sol{Achsen} && \Sol{F} && \Sol{Fn} && \Sol{$-$e} && \Sol{$-$U} \\\cline{3-3}\cline{5-5}\cline{7-7}\cline{9-9}\cline{11-11}
    &&&&&&&& \\
    (12) & \textit{Risiko} & \Sol{Risiken} && \Sol{N} && \Sol{(Gem)} && \Sol{($+$e)} && \Sol{$-$U} \\\cline{3-3}\cline{5-5}\cline{7-7}\cline{9-9}\cline{11-11}
    &&&&&&&& \\
    (13) & \textit{Graf} & \Sol{Grafen} && \Sol{M} && \Sol{Schw} && \Sol{$+$e} && \Sol{$-$U} \\\cline{3-3}\cline{5-5}\cline{7-7}\cline{9-9}\cline{11-11}
    &&&&&&&& \\
    (14) & \textit{Hase} & \Sol{Hasen} && \Sol{M} && \Sol{Schw} && \Sol{$-$e} && \Sol{$-$U} \\\cline{3-3}\cline{5-5}\cline{7-7}\cline{9-9}\cline{11-11}
    &&&&&&&& \\
    (15) & \textit{Schicht} & \Sol{Schichten} && \Sol{F} && \Sol{Fn} && \Sol{$+$e} && \Sol{$-$U} \\\cline{3-3}\cline{5-5}\cline{7-7}\cline{9-9}\cline{11-11}
  \end{longtable}
\end{center}

\Sol{\textbf{Hinweis:} Die Bildung \textit{Risiken} ist ungewöhnlich, weil das \textit{-os} des Singularstamms getilgt wird. Die Klassifikation als S-Substantiv hinkt daher ein bisschen.}

\section{Pluralklasse und prototypisches Genus der Substantive}

Welches Genus müssen die unterstrichenen Kunstwörter haben, wenn sie den wichtigsten Generalisierungen der Pluralbildung und deren Genusspezifik folgen?

\begin{center}
  \begin{tabular}[h]{rll}
    \toprule
    & \textbf{Wort im Satzkontext} & \textbf{Erwartbares Genus} \\
    \midrule
    (1) & \textit{Die \uline{Paugen} sind verschwunden.} & \Square~Mask\slash Neut\ \ \ \XBox~Fem \\
    (2) & \textit{Wir haben im 3.~Jahrhundert gegen \uline{Dimalchonten} gekämpft.} & \XBox~Mask\slash Neut\ \ \ \Square~Fem \\
    (3) & \textit{Er hat gleich mehrere \uline{Pümmer} entsorgt.} & \XBox~Mask\slash Neut\ \ \ \Square~Fem \\
    (4) & \textit{\uline{Klütsche} darf man hier tragen, Pantoffeln aber nicht.} & \XBox~Mask\slash Neut\ \ \ \Square~Fem \\
    \bottomrule
  \end{tabular}
\end{center}

\section{Anaphern}

Koindizieren Sie die unterstrichenen Anaphern bzw.\ Kataphern und Antezedenzien in den folgenden beiden Texten so, dass die beschriebenen Situationen korrekt von den Texten wiedergegeben werden.
Doppeldeutige Anaphern markieren sie durch alle infragekommenden Indizes, getrennt durch Schrägstriche, also \textit{ihn}\Sub{\textit{2}\slash\textit{3}} oder ähnlich.

\begin{enumerate}
  \item\doublespacing%
%    \begin{spread}
      \textbf{Situation}: Eine Person1 kauft für eine andere2 ein Geschenk.
      
      \textbf{Text}: \uline{Sie}~\Sol{\Sub{1}} betritt das KaDeWe und überlegt, was \uline{ihr}~\Sol{\Sub{2}} gefallen könnte .
      \uline{Sie}~\Sol{\Sub{1}} findet zunächst nichts passendes für \uline{sie}~\Sol{\Sub{2}} .
      \uline{Sie}~\Sol{\Sub{2}} hat \uline{ihr}~\Sol{\Sub{1}} ausdrücklich gesagt, dass \uline{sie}~\Sol{\Sub{1}} gar kein Geschenk zu besorgen braucht .
      Auf jeden Fall will \uline{sie}~\Sol{\Sub{1}} \uline{ihr}~\Sol{\Sub{2}} kein Klischeegeschenk mitbringen .
      Im Obergeschoss entdeckt \uline{sie}~\Sol{\Sub{1}} dann zufällig den Beaujolais, den \uline{sie}~\Sol{\Sub{3 (1 und 2)}} damals nach ihrem MA-Abschluss getrunken haben, und nimmt zwei Flaschen mit.
%    \end{spread}
    \Zeile
  \item\doublespacing
%    \begin{spread}
      \textbf{Situation}: Max1 schickt Julius2 per firmeneigenem Briefboten3 einen konspirativen Brief4 über den Firmenvorstand5.

      \textbf{Text}: \uline{Max}~\Sol{\Sub{1}} weiß, dass \uline{er}~\Sol{\Sub{1}} in \uline{dem Brief}~\Sol{\Sub{4}} an \uline{Julius}~\Sol{\Sub{2}} über \uline{den Vorstand}~\Sol{\Sub{5}} keine vertraulichen Details über \uline{seine}~\Sol{\Sub{5}} Beschlussfindung preisgeben darf.
      Trotzdem will \uline{er}~\Sol{\Sub{1}} \uline{ihn}~\Sol{\Sub{2}} dringend über \uline{den Vorstand}~\Sol{\Sub{5}} und \uline{seine}~\Sol{\Sub{1 oder 5}} Ansichten in Kenntnis setzen.
      \uline{Er}~\Sol{\Sub{1}} war \uline{sich}~\Sol{\Sub{1}} letzte Woche auch nicht sicher, ob \uline{der Bote}~\Sol{\Sub{3}} nicht von \uline{ihm}~\Sol{\Sub{2 oder 5}} beauftragt worden ist, alle Briefe zu öffnen und \uline{ihm}~\Sol{\Sub{2 oder 5}} weiterzuleiten.
      \uline{Er}~\Sol{\Sub{3}} wird in fünf Minuten kommen, um \uline{ihn}~\Sol{\Sub{4}} abzuholen und zuzustellen.
      Also schreibt \uline{er}~\Sol{\Sub{1}} schnell die wichtigsten Informationen in Andeutungen hinein, klebt \uline{ihn}~\Sol{\Sub{4}} zu und hofft, dass \uline{er}~\Sol{\Sub{3}} keinen Verdacht schöpft und \uline{ihn}~\Sol{\Sub{4}} ausliefert.
%  \end{spread}
\end{enumerate}

\newpage

\section{Pronomina und Artikel unterscheiden}

Entscheiden Sie für die unterstrichenen Wörter, ob sie Artikel (Art) oder Pronomina in Artikelfunktion (ProAF) oder Pronomina in Pronominalfunktion (ProPF) darstellen.

\begin{center}
  \begin{tabular}[h]{rll}
    \toprule
    & \textbf{Wort im Satzkontext} & \textbf{Klassifikation} \\
    \midrule
    (1) & \textit{Es hat sich \uline{kein} Junge ins Wasser getraut.} & \Solalt{\XBox}{\Square}~Art\ \ \Solalt{\Square}{\Square}~ProAF\ \ \Solalt{\Square}{\Square}~ProPF \\
    (2) & \textit{Da ist der Kollege, \uline{dessen} Kinder immer nerven.} & \Solalt{\Square}{\Square}~Art\ \ \Solalt{\Square}{\Square}~ProAF\ \ \Solalt{\XBox}{\Square}~ProPF \\
    (3) & \textit{Mit \uline{diesem} Milieu will ich nichts zu tun haben.} & \Solalt{\Square}{\Square}~Art\ \ \Solalt{\XBox}{\Square}~ProAF\ \ \Solalt{\Square}{\Square}~ProPF \\
    (4) & \textit{\uline{Einer} wollte auf jeden Fall schwimmen.} & \Solalt{\Square}{\Square}~Art\ \ \Solalt{\Square}{\Square}~ProAF\ \ \Solalt{\XBox}{\Square}~ProPF \\
    (5) & \textit{Ich fahre ungern mit \uline{deinem} Auto.} & \Solalt{\XBox}{\Square}~Art\ \ \Solalt{\Square}{\Square}~ProAF\ \ \Solalt{\Square}{\Square}~ProPF \\
    (6) & \textit{Die Kinder \uline{des} Kollegen waren heute ruhig.} & \Solalt{\XBox}{\Square}~Art\ \ \Solalt{\Square}{\Square}~ProAF\ \ \Solalt{\Square}{\Square}~ProPF \\
    (7) & \textit{\uline{Unseres} hatte leider gestern eine Reifenpanne.} & \Solalt{\Square}{\Square}~Art\ \ \Solalt{\Square}{\Square}~ProAF\ \ \Solalt{\XBox}{\Square}~ProPF \\
    (8) & \textit{\uline{Die} ist gemein!} & \Solalt{\Square}{\Square}~Art\ \ \Solalt{\Square}{\Square}~ProAF\ \ \Solalt{\XBox}{\Square}~ProPF \\
    (9) & \textit{Wir erinnerten uns \uline{seiner}, als er hereinkam.} & \Solalt{\Square}{\Square}~Art\ \ \Solalt{\Square}{\Square}~ProAF\ \ \Solalt{\XBox}{\Square}~ProPF \\
    (10) & \textit{Ich traf gestern \uline{die} Schwester meines Kollegen.} & \Solalt{\XBox}{\Square}~Art\ \ \Solalt{\Square}{\Square}~ProAF\ \ \Solalt{\Square}{\Square}~ProPF \\
    \bottomrule
  \end{tabular}
\end{center}

\section{Flexion der Pronomina und Artikel}

Entscheiden Sie, ob die folgenden Aussagen korrekt sind.
Mit Pronomina sind hier nur die regelmäßig flektierenden gemeint, um die es in der Vorlesung und in EGBD3 hauptsächlich geht.
Dementsprechend bleiben Personalpronomina und sonstige Exoten hier unbeachtet.

\begin{center}
  \renewcommand{\arraystretch}{1.5}
  \begin{tabular}[h]{rp{0.6\textwidth}l}
    \toprule
    & \textbf{Aussage} & \textbf{Bewertung} \\ 
    \midrule
    (1) & Artikel flektieren genau wie Pronomina. & \Solalt{\Square}{\Square}~trifft zu\ \ \ \Solalt{\XBox}{\Square}~trifft nicht zu \\
    (2) & Definitpronomina haben im Gegensatz zu Definitartikeln einige zweisilbige Formen. & \Solalt{\XBox}{\Square}~trifft zu\ \ \ \Solalt{\Square}{\Square}~trifft nicht zu \\
    (3) & Artikel in Pronominalfunktion treten immer ohne nachfolgendes Substantiv auf. & \Solalt{\Square}{\Square}~trifft zu\ \ \ \Solalt{\XBox}{\Square}~trifft nicht zu \\
    (4) & Im Gegensatz zum Indefinitpronomen fehlt beim Indefinitartikel im Akkusativ Singular Neutrum das Suffix. & \Solalt{\XBox}{\Square}~trifft zu\ \ \ \Solalt{\Square}{\Square}~trifft nicht zu \\
    (5) & Beim Definitartikel ist die Trennung von Stamm und Endung teilweise problematisch. & \Solalt{\XBox}{\Square}~trifft zu\ \ \ \Solalt{\Square}{\Square}~trifft nicht zu \\
    (6) & Die Form \textit{der} kann kein feminines Pronomen sein. & \Solalt{\Square}{\Square}~trifft zu\ \ \ \Solalt{\XBox}{\Square}~trifft nicht zu \\
    (7) & Die Form \textit{dessen} kann ein Artikel sein. & \Solalt{\Square}{\Square}~trifft zu\ \ \ \Solalt{\XBox}{\Square}~trifft nicht zu \\
    (8) & Die Pronomina flektieren im Femininum Singular genauso wie im Plural. & \Solalt{\Square}{\Square}~trifft zu\ \ \ \Solalt{\XBox}{\Square}~trifft nicht zu \\
    (9) & Possessiva flektieren wie Indefinita. & \Solalt{\XBox}{\Square}~trifft zu\ \ \ \Solalt{\Square}{\Square}~trifft nicht zu \\
    (10) & In Flexionsendungen der Nomina (eventuell mit Ausnahme -- je nach Analyse -- der Definitartikel) kommt als Vokal ausschließlich Schwa (orthographisch <e>) vor. & \Solalt{\XBox}{\Square}~trifft zu\ \ \ \Solalt{\Square}{\Square}~trifft nicht zu \\
  \end{tabular}
\end{center}


