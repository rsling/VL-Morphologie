\section{Satzglieder}\label{sec:satzglieder}

Bestimmen Sie für die unterstrichenen Teile in den folgenden Sätzen, ob sie Satzglieder sind, indem Sie den \textbf{Vorfeld-Test} anwenden und alle Satzglieder im vorgesehenen Feld ankreuzen.
Zur Erinnerung: Beim Vorfeld-Test versuchen Sie, den Satz so umzustellen, dass das potentielle Satzglied vor dem finiten Verb zu stehen kommt.
Finite Verben sind diejenigen Verben, die nach Tempus (Präsens\slash Präteritum), Numerus (Singular\slash Plural) und Person (1\slash 2\slash 3) flektiert sind.

\Halbzeile

\Sol{Die angekreuzten Fälle sind die, für die der Test erfolgreich verläuft.
Es handelt sich dabei nicht immer um Satzglieder gemäß Schulgrammatik, da der Begriff schulgrammatisch schlicht nicht hinreichend genau definiert und operationalisiert ist.}

\begin{enumerate}
  \item \Solalt{\XBox}{\Square}\ul{Der Winter} ist vorbei.\Sol{
      \begin{enumerate}
        \item Der Winter ist vorbei.\\
          \blau{Steht bereits im Vorfeld.}
      \end{enumerate}
    }
  \item Otje schickt \Solalt{\Square}{\Square}~\ul{seinen Kindern aus} dem Urlaub \Solalt{\XBox}{\Square}~\ul{eine Karte}.\Sol{
      \begin{enumerate}
        \item * Seinen Kindern aus schickt Otje dem Urlaub eine Karte.
        \item Eine Karte schickt Otje seinen Kindern aus dem Urlaub.
      \end{enumerate}
    }
  \item Wir kaufen öfter \Solalt{\Square}{\Square}\ul{Produkte}, die regional gefertigt wurden.\Sol{
      \begin{enumerate}
        \item * Produkte kaufen wir öfter, die regional gefertigt wurden.
      \end{enumerate}
    }
  \item K.\,R.\ Popper ist \Solalt{\XBox}{\Square}\ul{der Philosoph, auf dessen Werken alle falsifikationistischen Wissenschaftstheorien basieren}.\Sol{
      \begin{enumerate}
        \item Der Philosoph, auf dessen Werken alle falsifikationistischen Wissenschaftstheorien basieren, ist K.\,R.\ Popper.
      \end{enumerate}
    }
  \item Zu dieser Jahreszeit gibt es keine \Solalt{\Square}{\Square}\ul{regionalen} Erdbeeren \Solalt{\XBox}{\Square}\ul{in Deutschland}.\Sol{
      \begin{enumerate}
        \item * Regionalen gibt es zu dieser Jahreszeit keine Erdbeeren in Deutschland.
        \item In Deutschland gibt es zu dieser Jahreszeit keine regionalen Erdbeeren.
      \end{enumerate}
    }
  \item Ich glaube \Solalt{\XBox}{\Square}\ul{überhaupt nicht}, \Solalt{\XBox}{\Square}\ul{dass ein solcher Unsinn überhaupt ernstgenommen wird}.\Sol{
      \begin{enumerate}
        \item Überhaupt nicht glaube ich, dass ein solcher Unsinn überhaupt ernstgenommen wird.
        \item Dass ein solcher Unsinn überhaupt ernstgenommen wird, glaube ich überhaupt nicht.
      \end{enumerate}
    }
  \item Alle Wissenschaftler möchten gerne \Solalt{\XBox}{\Square}\ul{einen großen Erfolg für sich verbuchen} können.\Sol{
      \begin{enumerate}
        \item Einen großen Erfolg für sich verbuchen möchten alle Wissenschaftler gerne können.\\
          \blau{Schulgrammatisch handelt es sich nicht um ein Satzglied, aber es ist trotzdem vorfeldfähig.}
      \end{enumerate}
    }
  \item Man darf \Solalt{\Square}{\Square}\ul{seinen Hund} \Solalt{\Square}{\Square}\ul{beim Einkaufen} nicht \Solalt{\Square}{\Square}\ul{im Auto} zurücklassen.\Sol{
      \begin{enumerate}
        \item Seinen Hund darf man beim Einkaufen nicht im Auto zurücklassen.
        \item Beim Einkaufen darf man seinen Hund nicht im Auto zurücklassen.
        \item Im Auto darf man seinen Hund beim Einkaufen nicht zurücklassen.
      \end{enumerate}
    }
  \item \Solalt{\Square}{\Square}\ul{Heute} hat es keinen Zweck, \Solalt{\Square}{\Square}\ul{rudern zu gehen}.\Sol{
      \begin{enumerate}
        \item Heute hat es keinen Zweck, rudern zu gehen.\\
          \blau{Steht bereits im Vorfeld.}
        \item * Rudern zu gehen, hat es heute keinen Zweck.\\
          \blau{Der Infinitiv soll eigentlich ein Satzglied sein. Der Test funktioniert hier aber nicht, weil das sogenannte Korrelat des \textit{zu}-Infinitivs -- nämlich \textit{es} -- nicht nach dem Infinitiv stehen darf. Man muss \textit{es} weglassen, damit es funktioniert.}
      \end{enumerate}
    }
  \item Der abgewählte Präsident goss bei einer Wahlkampfveranstaltung \Solalt{\XBox}{\Square}\ul{Öl ins Feuer}.\Sol{
      \begin{enumerate}
        \item Öl ins Feuer goss der abgewählte Präsident bei einer Wahlkampfveranstaltung.\\
          \blau{Auch dieses scheinbar doppelt besetzte Vorfeld sollte es gemäß der Schulgrammatik nicht geben. Jedenfalls soll \textit{Öl ins Feuer} kein Satzglied sein.}
      \end{enumerate}
    }
  \item \Solalt{\XBox}{\Square}\ul{Der Hund unter dem Tisch} will endlich \Solalt{\XBox}{\Square}\ul{sein Fressen} haben.\Sol{
      \begin{enumerate}
        \item Der Hund unter dem Tisch will endlich sein Fressen haben.\\
          \blau{Steht bereits im Vorfeld.}
        \item Sein Fressen will der Hund unter dem Tisch endlich haben.
      \end{enumerate}
    }
  \item \Solalt{\Square}{\Square}\ul{Dass es heute} regnet, ist \Solalt{\XBox}{\Square}\ul{so gut wie sicher}.\Sol{
      \begin{enumerate}
        \item * Dass es heute ist regnet so gut wie sicher.
        \item So gut wie sicher ist, dass es heute regnet.
      \end{enumerate}
    }
  \item \Solalt{\XBox}{\Square}\ul{Eine Entscheidung für den Frieden} ist \Solalt{\Square}{\Square}\ul{nicht} \Solalt{\XBox}{\Square}\ul{generell} \Solalt{\XBox}{\Square}\ul{unvereinbar} mit \Solalt{\Square}{\Square}\ul{einer Entscheidung für militärische Aufrüstung}.\Sol{
      \begin{enumerate}
        \item Eine Entscheidung für den Frieden ist nicht generell unvereinbar mit einer Entscheidung für militärische Aufrüstung.\\
          \blau{Steht bereits im Vorfeld.}
        \item Nicht ist eine Entscheidung für den Frieden generell unvereinbar mit einer Entscheidung für militärische Aufrüstung.
        \item Generell ist eine Entscheidung für den Frieden nicht unvereinbar mit einer Entscheidung für militärische Aufrüstung.
        \item Unvereinbar ist eine Entscheidung für den Frieden nicht generell mit einer Entscheidung für militärische Aufrüstung.
        \item * Einer Entscheidung für militärische Aufrüstung ist eine Entscheidung für den Frieden nicht generell unvereinbar mit.
      \end{enumerate}
    }
\end{enumerate}

\section{Nominal- und Präpositionalphrasen}

Welche der unterstrichenen Teile aus Aufgabe~\ref{sec:satzglieder} sind NPs und PPs?
Gibt es andere nicht unterstrichene NPs und PPs in den Sätzen?

\Sol{%
  \begin{enumerate}
    \item
      \begin{enumerate}
        \item NP: \textit{der Winter}
      \end{enumerate}
    \item 
      \begin{enumerate}
        \item NP: \textit{Otje}
        \item NP: \textit{seinen Kindern}
        \item PP: \textit{aus dem Urlaub}
        \item NP: \textit{dem Urlaub}
        \item NP: \textit{eine Karte}
      \end{enumerate}
    \item 
      \begin{enumerate}
        \item NP: wir
        \item NP: Produkte, die regional gefertigt wurden
      \end{enumerate}
    \item 
      \begin{enumerate}
        \item NP: K.\,R.\ Popper
        \item NP: der Philosoph, auf dessen Werken alle falsifikationistischen Wissenschaftstheorien basieren
        \item PP: auf dessen Werken
        \item NP: dessen Werken
        \item NP: dessen
        \item NP: alle falsifikationistischen Wissenschaftstheorien
      \end{enumerate}
    \item
      \begin{enumerate}
        \item PP: zu dieser Jahreszeit
        \item NP: dieser Jahreszeit
        \item NP: keine regionalen Erdbeeren
        \item PP: in Deutschland
        \item NP: Deutschland
      \end{enumerate}
    \item 
      \begin{enumerate}
        \item NP: ich
        \item NP: ein solcher Unsinn
      \end{enumerate}
    \item 
      \begin{enumerate}
        \item NP: alle Wissenschaftler
        \item NP: einen großen Erfolg
        \item PP: für sich
        \item NP: sich
      \end{enumerate}
    \item 
      \begin{enumerate}
        \item NP: man
        \item NP: seinen Hund
        \item PP: beim Einkaufen
        \item NP: Einkaufen
        \item PP: im Auto
        \item NP: Auto
      \end{enumerate}
    \item 
      \begin{enumerate}
        \item NP: keinen Zweck
      \end{enumerate}
    \item 
      \begin{enumerate}
        \item NP: der abgewählte Präsident
        \item PP: bei einer Wahlkampfveranstaltung
        \item NP: einer Wahlkampfveranstaltung
        \item NP: Öl
        \item PP: ins Feuer
        \item NP: Feuer
      \end{enumerate}
    \item 
      \begin{enumerate}
        \item NP: der Hund unter dem Tisch
        \item PP: unter dem Tisch
        \item NP: dem Tisch
        \item NP: sein Fressen
      \end{enumerate}
    \item (keine)
    \item 
      \begin{enumerate}
        \item NP: eine Entscheidung für den Frieden
        \item PP: für den Frieden
        \item NP: den Frieden
        \item PP: mit einer Entscheidung für militärische Aufrüstung
        \item NP: einer Entscheidung für militärische Aufrüstung
        \item PP: für militärische Aufrüstung
        \item NP: militärische Aufrüstung
      \end{enumerate}
  \end{enumerate}
}
