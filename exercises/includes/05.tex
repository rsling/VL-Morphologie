\section{Derivation identifizieren und analysieren}

(a) Unterstreichen Sie im folgenden Textausschnitt \textit{Ausnahmebehandlung} alle Wörter, in denen Derivation stattgefunden hat.
Unterstreichen Sie auch solche, in denen die Wortbildung vermutlich nicht mehr produktiv oder transparent ist, sondern nur noch das lexikalisierte Ergebnis eines historischen Prozesses.

 \begin{quote}\onehalfspacing
   \textbf{Ausnahmebehandlung (Ausschnitt)}\\
   {\footnotesize\url{https://de.wikipedia.org/wiki/Ausnahmebehandlung}}\\

   Eine Ausnahme oder Ausnahmesituation (englisch exception oder Trap) bezeichnet in der Computertechnik ein Verfahren, Informationen über bestimmte Programmzustände – meistens Fehlerzustände – an andere Programmebenen zur Weiterbehandlung weiterzureichen.
Kann in einem Programm beispielsweise einer Speicheranforderung nicht stattgegeben werden, wird eine Speicheranforderungsausnahme ausgelöst. Ein Computerprogramm kann zur Behandlung dieses Problems dafür definierte Algorithmen abarbeiten, die den Fehler beheben oder anzeigen.
Exceptions haben in weiten Teilen die Behandlung von Fehlern mittels Fehlercodes oder Sprunganweisungen abgelöst und stellen im technischen Sinne einen zweiten, optionalen Rückgabewert einer Methode / Funktion dar.
 \end{quote}

\Zeile

(b) Analysieren Sie zehn der unterstrichenen Wörter vollständig gemäß der Konvention aus EGBD3.

\begin{itemize}\Lf
  \item Flexion und Fugenelemente abtrennen mit --
  \item Komposition abtrennen mit .
  \item Derivation abtrennen mit :
  \item Verbpartikeln abtrennen mit =
  \item \textbf{Die Anzeige von umlautauslösenden Affixen mit Tilden entfällt.}
\end{itemize}

\newpage

\begin{center}
  \begin{tabular}[h]{cp{0.8\textwidth}}
    &\\
    (1) & \\\cline{2-2}
    &\\
    (2) & \\\cline{2-2}
    &\\
    (3) & \\\cline{2-2}
    &\\
    (4) & \\\cline{2-2}
    &\\
    (5) & \\\cline{2-2}
    &\\
    (6) & \\\cline{2-2}
    &\\
    (7) & \\\cline{2-2}
    &\\
    (8) & \\\cline{2-2}
    &\\
    (9) & \\\cline{2-2}
    &\\
    (10) & \\\cline{2-2}
  \end{tabular}
\end{center}


\section{Konversion}


Handelt es sich bei den in der folgenden Tabelle unterstrichenen Wörtern um Wortformenkonversion (WFK), Stammkonversion (SK) oder gar nicht um Konversion (nichts ankreuzen)?
Was ist die Ausgangswortklasse und was die Zielwortklasse (V, Subst, Adj, Art, Präp, Komp, Adk, Adv, Satzä, Konj, Rest)?

\begin{center}
  \begin{tabular}[h]{cllp{0.125\textwidth}p{0.01\textwidth}p{0.125\textwidth}}
    \toprule
    & \textbf{Wort im Satzkontext} & \textbf{Klassifikation} & \textbf{Ausgangswk.} && \textbf{Zielwk.} \\
    \midrule
    &&&&& \\
    (1) & \textit{Wir \uline{ackern} ohne Pause.} & \Square~WFK\ \ \Square~SK &&& \\\cline{4-4}\cline{6-6}
    &&&&& \\
    (2) & \textit{Das ewige \uline{Gegeneinander} nervt.} & \Square~WFK\ \ \Square~SK &&& \\\cline{4-4}\cline{6-6}
    &&&&& \\
    (3) & \textit{Er ist mir \uline{feind}.} & \Square~WFK\ \ \Square~SK &&& \\\cline{4-4}\cline{6-6}
    &&&&& \\
    (4) & \textit{Mir macht das \uline{Lesen} ohne Brille Mühe.} & \Square~WFK\ \ \Square~SK &&& \\\cline{4-4}\cline{6-6}
    &&&&& \\
    (5) & \textit{Sie sollten den Salat weniger stark \uline{süßen}.} & \Square~WFK\ \ \Square~SK &&& \\\cline{4-4}\cline{6-6}
    &&&&& \\
    (6) & \textit{Der \uline{Putz} war von guter Qualität.} & \Square~WFK\ \ \Square~SK &&& \\\cline{4-4}\cline{6-6}
    &&&&& \\
    (7) & \textit{Gesunder \uline{Schlaf} fördert die Fitness.} & \Square~WFK\ \ \Square~SK &&& \\\cline{4-4}\cline{6-6}
  \end{tabular}
\end{center}

\newpage

\section{Verbpartikeln und Verbpräfixe}

(a) Entscheiden Sie für die folgenden Verben, ob sie mit einer Verbpartikel oder einem Verbpräfix abgeleitet wurden.
Markieren Sie die Art der Ableitung, indem Sie Präfixe mit : und Partikeln mit = abtrennen.
Bilden Sie einen kurzen Satz, an dem über die Satzgliedstellung oder die Morphologie des Verbs eindeutig erkennbar ist, um welche Art Ableitung (Präfix\slash Partikel) es sich handelt.

(b) Denken Sie anhand der hier gegebenen Beispiele und anderer einschlägiger Verben darüber nach, ob es tendenzielle Unterschiede in der Form von Präfixen und Partikeln gibt, und ob sie semantisch andere Effekte haben.
(Diese Teilaufgabe kann prinzipiell keine Musterlösung haben.
Sie ergibt nur dann einen Sinn, wenn Sie tatsächlich versuchen, über das Phänomen nachzudenken.)

\begin{center}
  \begin{tabular}[h]{clp{0.25\textwidth}p{0.01\textwidth}p{0.45\textwidth}}
    \toprule
    & Verb & Analyse && Satz \\
    \midrule
    &&&& \\
    (1) & \textit{entgegenstellen} &&& \\\cline{3-3}\cline{5-5}
        &&&& \\
        &&&& \\\cline{5-5}
    &&&& \\
    (2) & \textit{entlaufen} &&& \\\cline{3-3}\cline{5-5}
        &&&& \\
        &&&& \\\cline{5-5}
    &&&& \\
    (3) & \textit{versenken} &&& \\\cline{3-3}\cline{5-5}
        &&&& \\
        &&&& \\\cline{5-5}
    &&&& \\
    (4) & \textit{anheften} &&& \\\cline{3-3}\cline{5-5}
        &&&& \\
        &&&& \\\cline{5-5}
    &&&& \\
    (5) & \textit{nebenordnen} &&& \\\cline{3-3}\cline{5-5}
        &&&& \\
        &&&& \\\cline{5-5}
    &&&& \\
    (6) & \textit{beschleunigen} &&& \\\cline{3-3}\cline{5-5}
        &&&& \\
        &&&& \\\cline{5-5}
    &&&& \\
    (7) & \textit{zuwiderhandeln} &&& \\\cline{3-3}\cline{5-5}
        &&&& \\
        &&&& \\\cline{5-5}
  \end{tabular}
\end{center}

