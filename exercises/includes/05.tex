\section{Derivation identifizieren und analysieren}

(a) Unterstreichen Sie im folgenden Textausschnitt \textit{Ausnahmebehandlung} alle Wörter, in denen Derivation stattgefunden hat.
Unterstreichen Sie auch solche, in denen die Wortbildung vermutlich nicht mehr produktiv oder transparent ist, sondern nur noch das lexikalisierte Ergebnis eines historischen Prozesses.

 \begin{quote}\onehalfspacing
   \textbf{Ausnahmebehandlung (Ausschnitt)}\\
   {\footnotesize\url{https://de.wikipedia.org/wiki/Ausnahmebehandlung}}\\

   Eine \Solalt{\ul{Ausnahme}}{Ausnahme} oder \Solalt{\ul{Ausnahmesituation}}{Ausnahmesituation} (\Solalt{\ul{englisch}}{englisch} \textit{exception} oder \textit{Trap}) \Solalt{\ul{bezeichnet}}{bezeichnet} in der Computertechnik ein \Solalt{\ul{Verfahren}}{Verfahren}, Informationen über \Solalt{\ul{bestimmte}}{bestimmte} \Solalt{\ul{Programmzustände}}{Programmzustände} – \Solalt{\ul{meistens}}{meistens} \Solalt{\ul{Fehlerzustände}}{Fehlerzustände} – an andere \Solalt{\ul{Programmebenen}}{Programmebenen} zur \Solalt{\ul{Weiterbehandlung}}{Weiterbehandlung} \Solalt{\ul{weiterzureichen}}{weiterzureichen}.
Kann in einem Programm \Solalt{\ul{beispielsweise}}{beispielsweise} einer \Solalt{\ul{Speicheranforderung}}{Speicheranforderung} nicht \Solalt{\ul{stattgegeben}}{stattgegeben} werden, wird eine \Solalt{\ul{Speicheranforderungsausnahme}}{Speicheranforderungsausnahme} \Solalt{\ul{ausgelöst}}{ausgelöst}. Ein Computerprogramm kann zur \Solalt{\ul{Behandlung}}{Behandlung} dieses Problems \Solalt{\ul{dafür}}{dafür} definierte Algorithmen \Solalt{\ul{abarbeiten}}{abarbeiten}, die den Fehler \Solalt{\ul{beheben}}{beheben} oder \Solalt{\ul{anzeigen}}{anzeigen}.
Exceptions haben in weiten Teilen die \Solalt{\ul{Behandlung}}{Behandlung} von \Solalt{\ul{Fehlern}}{Fehlern} \Solalt{\ul{mittels}}{mittels} \Solalt{\ul{Fehlercodes}}{Fehlercodes} oder \Solalt{\ul{Sprunganweisungen}}{Sprunganweisungen} \Solalt{\ul{abgelöst}}{abgelöst} und \Solalt{\ul{stellen}}{stellen} im \Solalt{\ul{technischen}}{technischen} Sinne einen \Solalt{\ul{zweiten}}{zweiten}, \Solalt{\ul{optionalen}}{optionalen} \Solalt{\ul{Rückgabewert}}{Rückgabewert} einer Methode\slash Funktion \Solalt{\ul{dar}}{dar}.
 \end{quote}

 \Sol{\textbf{Hinweis:} Derivationen, die vermutlich aus dem Englischen oder anderen Gebersprachen übernommen wurden oder die anderweitig Pseudoderivationen darstellen (wie \textit{Computer}, \textit{Exception}, \textit{Technik}) wurden nicht markiert.}

\Zeile

(b) Analysieren Sie zehn der unterstrichenen Wörter vollständig gemäß der Konvention aus EGBD3.

\begin{itemize}\Lf
  \item Flexion und Fugenelemente abtrennen mit --
  \item Komposition abtrennen mit .
  \item Derivation abtrennen mit :
  \item Verbpartikeln abtrennen mit =
  \item \textbf{Die Anzeige von umlautauslösenden Affixen mit Tilden wie in EGBD3 entfällt.}
\end{itemize}

\newpage

\aufgabeginn
\begin{center}
  \begin{tabular}[h]{cp{0.8\textwidth}}
    &\\
    \aufg & \Sol{Aus=nahm:e} \\\cline{2-2}
    &\\
    \aufg & \Sol{engl:isch (aus $^{\dagger}$\textit{engel:isch} zur Volksbezeichnung \textit{Engeln} bzw. \textit{Angeln})} \\\cline{2-2}
    &\\
    \aufg & \Sol{be:zeichn(e)-t} \\\cline{2-2}
    &\\
    \aufg & \Sol{Ver:fahr-en} \\\cline{2-2}
    &\\
    \aufg & \Sol{be:stimm:t-e} \\\cline{2-2}
    &\\
    \aufg & \Sol{Programm.zu(=)ständ-e} \\\cline{2-2}
    &\\
    \aufg & \Sol{meist:ens (wie \textit{übrigens usw.})} \\\cline{2-2}
    &\\
    \aufg & \Sol{Programm.eben:e} \\\cline{2-2}
    &\\
    \aufg & \Sol{Weiter=be:handl:ung} \\\cline{2-2}
    &\\
    \aufg & \Sol{bei:spiel-s.weise} \\\cline{2-2}
  \end{tabular}
\end{center}

\Halbzeile

\Sol{\textbf{Anmerkung:} Bei \textit{bezeichn(e)-t} muss man davon ausgehen, dass der Stamm \textit{bezeichn} ist. Die PN1-Endung ist auf jeden Fall \textit{-t}. Da \textit{*bezeichn-t} phonotaktisch nicht wohlgeformt wäre, wird das zusätzliche \textit{e} [ə] eingefügt.}

\Halbzeile

\section{Konversion}


Handelt es sich bei den in der folgenden Tabelle unterstrichenen Wörtern um Wortformenkonversion (WFK), Stammkonversion (SK) oder gar nicht um Konversion (nichts ankreuzen)?
Was ist die Ausgangswortklasse und was die Zielwortklasse (V, Subst, Adj, Art, Präp, Komp, Adk, Adv, Satzä, Konj, Rest)?

\begin{center}
  \begin{longtable}[h]{cllp{0.125\textwidth}p{0.01\textwidth}p{0.125\textwidth}}
    \toprule
    & \textbf{Wort im Satzkontext} & \textbf{Klassifikation} & \textbf{Ausgangswk.} && \textbf{Zielwk.} \\
    \midrule
    &&&&& \\
    (1) & \textit{Wir \uline{ackern} ohne Pause.} & \Solalt{\Square}{\Square}~WFK\ \ \Solalt{\XBox}{\Square}~SK & \Sol{Subst} && \Sol{V} \\\cline{4-4}\cline{6-6}
    &&&&& \\
    (2) & \textit{Das ewige \uline{Gegeneinander} nervt.} & \Solalt{\Square}{\Square}~WFK\ \ \Solalt{\XBox}{\Square}~SK & \Sol{Adv} && \Sol{Subst} \\\cline{4-4}\cline{6-6}
    &&&&& \\
    (3) & \textit{Er ist mir \uline{feind}.} & \Solalt{\Square}{\Square}~WFK\ \ \Solalt{\XBox}{\Square}~SK & \Sol{Subst} && \Sol{Adk} \\\cline{4-4}\cline{6-6}
    &&&&& \\
    (4) & \textit{Mir macht das \uline{Lesen} ohne Brille Mühe.} & \Solalt{\XBox}{\Square}~WFK\ \ \Solalt{\Square}{\Square}~SK & \Sol{V} && \Sol{Subst} \\\cline{4-4}\cline{6-6}
    &&&&& \\
    (5) & \textit{Sie sollten den Salat weniger stark \uline{süßen}.} & \Solalt{\Square}{\Square}~WFK\ \ \Solalt{\XBox}{\Square}~SK & \Sol{Adj} && \Sol{V} \\\cline{4-4}\cline{6-6}
    &&&&& \\
    (6) & \textit{Der \uline{Putz} war von guter Qualität.} & \Solalt{\Square}{\Square}~WFK\ \ \Solalt{\XBox}{\Square}~SK & \Sol{V} && \Sol{Subst} \\\cline{4-4}\cline{6-6}
    &&&&& \\
    (7) & \textit{Gesunder \uline{Schlaf} fördert die Fitness.} & \Solalt{\Square}{\Square}~WFK\ \ \Solalt{\XBox}{\Square}~SK & \Sol{V} && \Sol{Subst} \\\cline{4-4}\cline{6-6}
    &&&&& \\
    (8) & \textit{Der soeben erst \uline{gekaufte} Wagen ist defekt.} & \Solalt{\XBox}{\Square}~WFK\ \ \Solalt{\XBox}{\Square}~SK & \Sol{V} && \Sol{Adj} \\\cline{4-4}\cline{6-6}
  \end{longtable}
\end{center}

\Sol{\textbf{Anmerkung:} Für \textit{gekaufte} existieren zwei Analysen, und zwischen EGBD3 und EGBD4 gibt es einen Wechsel von der einen zur anderen. In EGBD3 wird die Partizipbildung als Flexion aufgefasst. Damit wäre das Partizip \textit{gekauft} eine Wortform, und es würde sich bei \textit{gekauft-e} um eine Wortformenkonversion plus Flexion handeln. In EGBD4 wird die Partizipbildung als Wortbildung bzw.\ Stammbildung beschrieben, und dann handelt es sich bei \textit{gekaufte} um Stammkonversion. Beide Interpretationen existieren in der germanistischen Literatur.}

\section{Verbpartikeln und Verbpräfixe}

(a) Entscheiden Sie für die folgenden Verben, ob sie mit einer Verbpartikel oder einem Verbpräfix abgeleitet wurden.
Markieren Sie die Art der Ableitung, indem Sie Präfixe mit : und Partikeln mit = abtrennen.
Bilden Sie einen kurzen Satz, an dem über die Satzgliedstellung oder die Morphologie des Verbs eindeutig erkennbar ist, um welche Art Ableitung (Präfix\slash Partikel) es sich handelt.

(b) Denken Sie anhand der hier gegebenen Beispiele und anderer einschlägiger Verben darüber nach, ob es tendenzielle Unterschiede in der Form von Präfixen und Partikeln gibt, und ob sie semantisch andere Effekte haben.
(Diese Teilaufgabe kann prinzipiell keine Musterlösung haben.
Sie ergibt nur dann einen Sinn, wenn Sie tatsächlich versuchen, über das Phänomen nachzudenken.)

\begin{center}
  \begin{longtable}[h]{clp{0.25\textwidth}p{0.01\textwidth}p{0.45\textwidth}}
    \toprule
    & Verb & Analyse && Satz \\
    \midrule
    &&&& \\
    (1) & \textit{entgegenstellen} & \Sol{ent(:)gegen=stellen} && \Sol{Die Polizei stellte sich ihm entgegen.} \\\cline{3-3}\cline{5-5}
        &&&& \\
        &&&& \\\cline{5-5}
    &&&& \\
    (2) & \textit{entlaufen} & \Sol{ent:laufen} && \Sol{Der Gefangene ist entlaufen.} \\\cline{3-3}\cline{5-5}
        &&&& \\
        &&&& \\\cline{5-5}
    &&&& \\
    (3) & \textit{versenken} & \Sol{ver:senken} && \Sol{Sie hatten das Schiff versenkt.} \\\cline{3-3}\cline{5-5}
        &&&& \\
        &&&& \\\cline{5-5}
    &&&& \\
    (4) & \textit{anheften} & \Sol{an=heften} && \Sol{Er heftete die Notiz ans Buch an.} \\\cline{3-3}\cline{5-5}
        &&&& \\
        &&&& \\\cline{5-5}
    &&&& \\
    (5) & \textit{nebenordnen} & \Sol{neben=ordnen} && \Sol{Das Beispiel besteht aus zwei} \\\cline{3-3}\cline{5-5}
        &&&& \\
        &&&& \Sol{nebengeordneten Hauptsätzen.} \\\cline{5-5}
    &&&& \\
    (6) & \textit{beschleunigen} & \Sol{be:schleunigen} && \Sol{Er beschleunigte vor der Kurve.} \\\cline{3-3}\cline{5-5}
        &&&& \\
        &&&& \\\cline{5-5}
    &&&& \\
    (7) & \textit{zuwiderhandeln} & \Sol{zuwider=handeln} && \Sol{Sie handelte unserem Rat zuwider.} \\\cline{3-3}\cline{5-5}
        &&&& \\
        &&&& \\\cline{5-5}
  \end{longtable}
\end{center}

