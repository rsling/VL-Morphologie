\section{Stämme und Affixe}\label{sec:form}

Finden Sie in den folgenden Wörtern Stämme und Affixe.
Analysieren Sie die Wörter, soweit Sie können.
Wenn also mehrere Stämme oder Affixe im Wort vorkommen, zerlegen Sie es so detailliert wie möglich.
Wir benutzen hier noch nicht die Trennzeichen, die später in der Vorlesung \slash\ im Buch eingeführt werden.
Trennen Sie einfach alle Morphe mit | ab und \ul{unterstreichen} Sie alle Stämme.

Zur Erinnerung:

\begin{itemize}\Lf
  \item Stämme sind die Morphe mit lexikalischer Markierungsfunktion, also vor allem Bedeutung.
  \item Affixe (Präfixe und Suffixe) sind Morphe ohne lexikalische Markierungsfunktion, sie haben also keine Bedeutung.
    Außerdem können Affixe prinzipiell nicht alleine stehen, sind also nicht wortfähig.
\end{itemize}

\begin{longtable}[h]{cp{0.2\textwidth}p{0.75\textwidth}}
  &&\\
  (0) & \grau{schmeißen}    & \ul{schmeiß} | en\\
  &&\\
  (0) & \grau{liebäugeln}   & \ul{lieb} | \ul{äug} | el | n\\
  &&\\
  (0) & \grau{Verwerfungen} & Ver | \ul{werf} | ung | en \\
  && \\
  (1) & Überholung          & \Sol{Über|\ul{hol}|ung} \\\Solalt{}{\cline{3-3}}
  && \\
  (2) & schreit             & \Sol{\ul{schrei}|t} \\\Solalt{}{\cline{3-3}}
  && \\
  (3) & bläulicheres        & \Sol{\ul{bläu}|lich|er|es} \\\Solalt{}{\cline{3-3}}
  && \\
  (4) & unterschwellige     & \Sol{unter|\ul{schwell}|ig|e} \\\Solalt{}{\cline{3-3}}
  && \\
  (5) & begrünen            & \Sol{be|\ul{grün}|en} \\\Solalt{}{\cline{3-3}}
  && \\
  (6) & denke               & \Sol{\ul{denk}|e} \\\Solalt{}{\cline{3-3}}
  && \\
  (7) & Zeitmessung         & \Sol{\ul{Zeit}|\ul{mess}|ung} \\\Solalt{}{\cline{3-3}}
  && \\
  (8) & Grauslichkeiten     & \Sol{\ul{Graus}|lich|keit|en} \\\Solalt{}{\cline{3-3}}
  && \\
  (9) & Tüchern             & \Sol{\ul{Tüch}|er|n} \\\Solalt{}{\cline{3-3}}
  && \\
  (10) & Rumänen            & \Sol{\ul{Rumän}|e|n} \\\Solalt{}{\cline{3-3}}
\end{longtable}


\section{Wortbildung und Flexion}\label{sec:funktion}

Entscheiden Sie für die \ul{unterstrichenen} Wörter im folgendem Text, inwiefern in ihnen Wortbildung durch Affixe oder Flexion durch Affixe (oder beides) zu beobachten sind.
Trennen Sie dazu die Wörter in Stämme und Affixe auf wie in Aufgabe~\ref{sec:form}.
Wenn es Flexion ist, versuchen Sie zu beschreiben, welche \textbf{Markierungsfunktion} die Affixe haben.
Wenn es Wortbildung ist, versuchen Sie zu beschreiben, welche Merkmale sich durch die Anfügung des Affixes ändern.

Zur Erinnerung:

\begin{itemize}
  \item Bei der Flexion ändern sich Werte \textbf{volatiler} Merkmale, aber das lexikalische Wort bleibt dasselbe.
    Typische Flexionsmerkmale sind:
    \begin{itemize}\Lf
      \item Tempus, Modus, Person und Numerus bei den Verben
      \item Kasus und Numerus bei den Substantiven
      \item Kasus, Genus und Numerus bei den Artikeln und Pronomina
      \item Kasus, Genus, Numerus und die sogenannte Stärke bei den Adjektiven
    \end{itemize}
  \item Bei der Wortbildung ändern sich Werte \textbf{statischer} Merkmale, die sich sonst nicht ändern (\zB die Bedeutung oder die Wortklasse), oder es werden Merkmale gelöscht\slash hinzugefügt.
    Dies kann mit der Anfügung von Affixen einhergehen (\textit{beobachten → \textit{Beobachtung}}, manchmal aber auch ganz ohne Affixe (\textit{wandern} als Verb → \textit{Wandern} als Substantiv).
    Den ersten Fall nennen wir später Derivation, den zweiten Konversion.
    Hier geht es nur um Derivation.
\end{itemize}

\subsection{Text}

\textbf{Wikipedia | Weltraum}\\
\footnotesize Quelle: \url{https://de.wikipedia.org/wiki/Weltraum}\\

\begin{quote}

  Der Weltraum \grau{\ul{bezeichnet}} den Raum zwischen Himmelskörpern. Die \grau{\ul{Atmosphären}} von \ul{festen} und gasförmigen Himmelskörpern (wie Sternen und Planeten) haben keine feste Grenze nach oben, \ul{sondern} werden mit \ul{zunehmendem} Abstand zum Himmelskörper allmählich immer \ul{dünner}. Ab einer \ul{bestimmten} \ul{Höhe} \ul{spricht} man vom Beginn des Weltraums.

  Im Weltraum herrscht ein Hochvakuum mit niedriger Teilchendichte. Er ist aber kein leerer Raum, sondern enthält Gase, kosmischen Staub und Elementarteilchen (Neutrinos, kosmische \ul{Strahlung}, Partikel), außerdem elektrische und magnetische Felder, Gravitationsfelder und elektromagnetische Wellen (Photonen). Das fast vollständige Vakuum im Weltraum macht ihn außerordentlich durchsichtig und erlaubt die Beobachtung extrem \ul{entfernter} Objekte, etwa anderer Galaxien. Jedoch können Nebel aus \ul{interstellarer} Materie die Sicht auf dahinterliegende Objekte auch stark behindern.

  Der Begriff des Weltraums ist nicht gleichzusetzen mit dem Weltall, welches eine \ul{eingedeutschte} Bezeichnung für das Universum insgesamt ist und somit alles, also auch die Sterne und Planeten selbst, mit einschließt. Dennoch wird das deutsche Wort Weltall oder All \ul{umgangssprachlich} (eigentlich inkorrekt) mit der Bedeutung Weltraum verwendet.

  Die Erforschung des Weltraums \ul{wird} Weltraumforschung \ul{genannt}. Reisen oder Transporte in oder durch den Weltraum werden als Raumfahrt bezeichnet.
\end{quote}

\subsection{Lösungsbeispiele}

Hinweis: Die Aufgabenstellung ist freier als typische Klausuraufgaben, und sie ist zum gegenwärtigen Zeitpunkt noch eine Transferaufgabe.
Dies gibt Ihnen die Möglichkeit, nachzudenken und selber ein Gespür für Morphologie zu entwickeln.
Es geht \textbf{nicht} darum, eine perfekte Lösung abzuliefern.
Wir kommen auf alle Details, die Ihnen momentan noch fehlen, in späteren Sitzungen zurück.

\Halbzeile

\begin{itemize}
  \item \textbf{bezeichnet} 
    \begin{itemize}
      \item \textit{be} ist ein Wortbildungsaffix zum Stamm \textit{zeichn}(\textit{e}), das die Bedeutung verändert (\textit{jemand zeichnet etwas} → \textit{ein Wort o.\,Ä.\ bezeichnet etwas} oder \textit{jemand bezeichnet etwas mit einem Wort o.\,Ä.})
      \item Das Flexionssuffix (\textit{e})\textit{t} legt Person und Numerus auf P3 und Singular fest, eventuell auch auf Tempus und Modus auf Präsens und Indikativ.
    \end{itemize}
  \item \textbf{Atmosphären}
    \begin{itemize}
      \item Das Flexionssuffix \textit{n} am Stamm \textit{Atmosphäre} legt das Numerus-Merkmal auf Plural fest.
    \end{itemize}
\end{itemize}

\Sol{Diese Musterlösung entspricht ihrem Wissensstand nach Lektion sieben des Seminars.
Was Sie jetzt nicht verstehen, müssten Sie sich dann später nochmals ansehen.}

\Sol{\begin{enumerate}
  \item be:zeichne-t | Verbpräfix, Präsensstamm, PN1 3. Sg.
  \item Atmosphäre-n | Stamm, Plural
  \item fest-en | Stamm, Kasus-Numerus-Flexion (pronominal)
  \item sondern | nicht flektierbares Wort (Konjunktion)
  \item zu=nehm:end-en | Verbpartikel, Stamm, Adjektivderivation (KEIN Partizip!), Kasus-Numerus-Flexion (pronominal)
  \item dünn-er (ab EGBD4 dünn:er) | Stamm, Komparativ
  \item be:stimm-t-en (ab EGBD4 be:stimm:t-en) | Verbpräfix, Stamm, Partizip, Kasus-Numerus-Flexion (adjektivisch)
  \item Höhe | Stamm
  \item sprich-t | Präsensstamm (Alternanzstufe), PN1 3. Sg.
  \item Strahl:ung | Stamm, Substantiv-Derivationssuffix
  \item ent:fern-t-er (ab EGBD4 ent:fern:t-er) | Verbpräfix, Stamm, Partizip, Kasus-Numerus-Entscheiden (pronominal)
  \item interstellar-er | Stamm, Kasus-Numerus-Endung (pronominal)
  \item ein=ge-deutsch-t-e (ab EGBD4 ein=ge:deutsch:t-e) | Verbpartikel, Partizip, Stamm, Partizip, Kasus-Numerus-Flexion (adjektivisch)
  \item um=gang-s.sprach:lich | Verbpartikel, Stamm, Fugenelement, Stamm, Adjektiv-Derivationssuffix
  \item wird | keine Segmentierung möglich
  \item ge-nann-t (ab EGBD4 ge:nann:t) | Partizip, Stamm, Partizip
\end{enumerate}
}

\section{Transfer\slash Vertiefung}

Die Funktionsbestimmung für Flexionsformen aus Aufgabe~\ref{sec:funktion} ist für die Beispiele in Aufgabe~\ref{sec:form} nicht immer ohne Weiteres möglich.
Dies liegt daran, dass die Formen ohne Satzkontext gegeben werden.
Erklären Sie diese Behauptung und illustrieren Sie an konkreten Formen aus Aufgabe~\ref{sec:form} das Problem.

\Sol{\begin{enumerate}
  \item \textit{bläulicheres} kann ein Nominativ oder Akkusativ sein
  \item \textit{unterschwellige} kann ein Nominativ oder Akkusativ Plural in einer NP ohne flektierendes Artikelwort sein (\textit{unterschwellige Anschuldigungen}). Es könnte außerdem ein Nominativ oder Akkusativ Singular Femininum mit oder ohne flektierendes Artikelwort sein.
    \begin{enumerate}
      \item (Die) unterschwellige Aggression macht mich nervös.
      \item Ich spüre (die) unterschwellige Aggression.
    \end{enumerate}
  \item \textit{begrünen} kann ein Infinitiv oder eine Präsensform der ersten oder dritten Person im Plural sein, hypothetisch auch im Konjunktiv Präsens.
  \item \textit{denke} kann eine Präsensform der 1. Person oder eine Imperativform sein.
  \item Bei \textit{Grauslichkeiten} sind unterschiedliche Kasus denkbar.
  \item Bei \textit{Rumänen} kann es sich um jede Kasus-Numerus-Form außer dem Nominativ Singular handeln.
  \end{enumerate}
}
