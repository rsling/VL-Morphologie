\section{Dative und Valenzerweiterung}\label{sec:dative}

Fügen Sie den folgenden Beispielsätzen die Dative der jeweils genannten Arten hinzu und prüfen Sie, ob das Ergebnis grammatisch ist.
Formulieren Sie den ergänzten Satz, indem Sie eine entsprechenden Nominalphrasen hinzufügen.
Achtung: Außer den Dativ-NPs dürfen Sie nichts hinzufügen!

\begin{center}
  \begin{longtable}[h]{cp{0.8\textwidth}}
    (1)  & \textit{Ein Oldtimter \ul{verbraucht} viel.} \textbf{+ Bewertungsdativ}\\
    &\\
    & \Sol{*Ein Oldtimer verbraucht \ul{mir} viel.} \\\cline{2-2}
    &\\
    &\\\cline{2-2}
    &\\
    (2)  & \textit{Sarah \ul{backt} einen Christstollen.} \textbf{+ Nutznießerdativ}\\
    &\\
    & \Sol{Sarah backt \ul{ihm} einen Christstollen.} \\\cline{2-2}
    &\\
    &\\\cline{2-2}
    &\\
    (3)  & \textit{Dem Kollegen passieren zu viele Fehler.} \textbf{+ Bewertungsdativ}\\
    &\\
    & \Sol{Dem Kollegen passieren \ul{mir} zu viele Fehler.} \\\cline{2-2}
    &\\
    &\\\cline{2-2}
    &\\
    (4)  & \textit{Sarah \ul{backt} zu viele Kekse.} \textbf{+ Nutznießerdativ + Bewertungsdativ}\\
    &\\
    & \Sol{Sarah backt \ul{mir} \ul{den Kindern} zu viele Kekse.} \\\cline{2-2}
    &\\
    &\\\cline{2-2}
    &\\
    (5)  & \textit{Matthias \ul{schenkt} Doro ein Mixtape.} \textbf{+ Nutznießerdativ}\\
    &\\
    & \Sol{*Matthias schenkt Doro \ul{seinem Vater} ein Mixtape.} \\\cline{2-2}
    &\\
    & \Sol{\blau{Der lexikalische Dativ hat bereits die Nutznießer-Bedeutung!}} \\\cline{2-2}
    &\\
    (6)  & \textit{Das alte MacBook \ul{läuft} inzwischen zu langsam.} \textbf{+ Bewertungsdativ}\\
    &\\
    & \Sol{Das alte MacBook läuft \ul{Sarahs Freundin} inzwischen zu langsam.} \\\cline{2-2}
    &\\
    &\\\cline{2-2}
  \end{longtable}
\end{center}

\section{Dativpassiv und Valenz}\label{sec:dativpassiv}

Bilden Sie zu den von Ihnen ergänzten Sätzen aus Aufgabe~\ref{sec:dative} das Dativpassiv, sofern dies möglich ist.
Lassen Sie dabei immer die \textit{von}-PP weg.
Entscheiden Sie, was für ein Dativ von der Passivierung betroffen ist: (1) ein lexikalischer Dativ, (2) ein per Valenzanreicherung hinzugefügter Dativ, oder (3) eine Dativangabe.
Falls ein \textit{bekommen}-Passiv nicht bildbar ist, geben Sie an, warum.

\begin{center}
  \begin{longtable}[h]{cp{0.9\textwidth}}
    &\\
    (1) & \Sol{*Ich bekomme viel verbraucht.} \\\cline{2-2}
    &\\
    & \Sol{\blau{Aktivsatz ungrammatisch, weil Vergleichselement fehlt (\textit{zu viel} o.\,ä.).}}\\\cline{2-2}
    &\\
    & \Sol{\blau{Bewertungsdative (Angaben) sind auch nicht passivierbar.}}\\\cline{2-2}
    &\\
    (2) & \Sol{Er bekommt einen Christstollen gebacken.} \\\cline{2-2}
    &\\
    & \Sol{\blau{Nutznießerdative sind passivierbar.}} \\\cline{2-2}
    &\\
    (3) & \Sol{*Ich bekomme dem Kollegen passiert.} \\\cline{2-2}
    &\\
    & \Sol{\blau{Bewertungsdative (Angaben) sind nicht passivierbar.}} \\\cline{2-2}
    &\\
    & \Sol{*Der Kollege bekommt mir passiert.} \\\cline{2-2}
    &\\
    & \Sol{\blau{\textit{Passieren} ist kein agentivisches Verb und kann nicht passiviert werden.}} \\\cline{2-2}
    &\\
    (4) & \Sol{*Ich bekomme den Kindern zu viele Kekse gebacken.} \\\cline{2-2}
    &\\
    & \Sol{\blau{Bewertungsdative (Angaben) sind konsequent nicht passivfähig.}} \\\cline{2-2}
    &\\
    & \Sol{Die Kinder bekommen mir zu viele Kekse gebacken.} \\\cline{2-2}
    &\\
    & \Sol{\blau{Das Passiv zum Nutznießerdativ ist bildbar.}}\\\cline{2-2}
    &\\
    (5) & \Sol{*Sein Vater bekommt Doro ein Mixtape geschenkt.} \\\cline{2-2}
    &\\
    & \Sol{\blau{Ungrammatische Aktivsätze können nicht passiviert werden.}} \\\cline{2-2}
    &\\
    (6) & \Sol{Sarahs Freundin bekommt inzwischen zu langsam gelaufen.} \\\cline{2-2}
    &\\
    & \Sol{\blau{Bewertungsdative (Angaben) sind konsequent nicht passivfähig.}} \\\cline{2-2}
  \end{longtable}
\end{center}

\section{Statusrektion}

Bestimmen Sie in den folgenden Sätzen die Status der infiniten Verben, indem Sie jeweils einen Pfeil einzeichnen, der vom statusregierenden Verb auf as statusregierte Verb zeigt.
Beschriften Sie diesen Pfeil mit der Nummer (1, 2 oder 3) des regierten Status.


\Zeile

\begin{spread}
  \begin{tabular}[h]{cl}
    (1) & \textit{Adrianna \mkword{wird} wieder Salat \mkword{bestellen} \mkword{wollen}.} \Sol{\setlength{\arrowheight}{3.5ex}\mvarrow[above=1]{wird}{wollen}\mvarrow*[below=1]{wollen}{bestellen}} \\
    (2) & \textit{Man kolportiert, dass der Kollege \mkword{plagiiert} \mkword{zu haben} \mkword{scheint}.} \Sol{\setlength{\arrowheight}{3.5ex}\mvarrow[above=2]{scheint}{zu haben}\mvarrow*[below=3]{zu haben}{plagiiert}} \\
    (3) & \textit{Ich finde es schade, dass man den Bär \mkword{hat} \mkword{einfangen} \mkword{lassen}.} \Sol{\setlength{\arrowheight}{3.5ex}\mvarrow[above=1]{hat}{lassen}\mvarrow*[below=1]{lassen}{einfangen}} \\
    (4) & \textit{Der Ruin \mkword{wurde} von der Firma \mkword{versucht}, unbedingt \mkword{zu vermeiden}.} \Sol{\setlength{\arrowheight}{3.5ex}\mvarrow[above=3]{wurde}{versucht}\mvarrow*[below=2]{versucht}{zu vermeiden}} \\
    (5) & \textit{Der Rottweiler \mkword{war} sehr gut \mkword{erzogen} \mkword{worden}.} \Sol{\setlength{\arrowheight}{3.5ex}\mvarrow[above=3]{war}{worden}\mvarrow*[below=3]{worden}{erzogen}} \\
    (6) & \textit{Michelle \mkword{hat} uns \mkword{versprochen}, auf den Kater \mkword{aufzupassen}.} \Sol{\setlength{\arrowheight}{3.5ex}\mvarrow[above=3]{hat}{versprochen}\mvarrow*[below=2]{versprochen}{aufzupassen}} \\
  \end{tabular}
\end{spread}

