\section{Dative und Valenzerweiterung}\label{sec:dative}

Fügen Sie den folgenden Beispielsätzen die Dative der jeweils genannten Arten hinzu, sofern dies möglich ist.
Formulieren Sie den ergänzten Satz, indem Sie eine entsprechenden Nominalphrasen hinzufügen.
Achtung: Außer den Dativ-NPs dürfen Sie nichts hinzufügen!

\begin{center}
  \begin{longtable}[h]{cp{0.8\textwidth}}
    (1)  & \textit{Ein Oldtimter \ul{verbraucht} viel.} \textbf{+ Bewertungsdativ}\\
    &\\
    &\\\cline{2-2}
    &\\
    &\\\cline{2-2}
    &\\
    (2)  & \textit{Sarah \ul{backt} einen Christstollen.} \textbf{+ Nutznießerdativ}\\
    &\\
    &\\\cline{2-2}
    &\\
    &\\\cline{2-2}
    &\\
    (3)  & \textit{Dem Kollegen passieren zu viele Fehler.} \textbf{+ Bewertungsdativ}\\
    &\\
    &\\\cline{2-2}
    &\\
    &\\\cline{2-2}
    &\\
    (4)  & \textit{Sarah \ul{backt} zu viele Kekse.} \textbf{+ Nutznießerdativ + Bewertungsdativ}\\
    &\\
    &\\\cline{2-2}
    &\\
    &\\\cline{2-2}
    &\\
    (5)  & \textit{Matthias \ul{schenkt} Doro ein Mixtape.} \textbf{+ Nutznießerdativ}\\
    &\\
    &\\\cline{2-2}
    &\\
    &\\\cline{2-2}
    &\\
    (6)  & \textit{Das alte MacBook \ul{läuft} inzwischen zu langsam.} \textbf{+ Bewertungsdativ}\\
    &\\
    &\\\cline{2-2}
    &\\
    &\\\cline{2-2}
  \end{longtable}
\end{center}

\section{Dativpassiv und Valenz}\label{sec:dativpassiv}

Bilden Sie zu den von Ihnen ergänzten Sätzen aus Aufgabe~\ref{sec:dative} das Dativpassiv, sofern dies möglich ist.
Entscheiden Sie, was für ein Dativ von der Passivierung betroffen ist: (1) ein lexikalischer Dativ, (2) ein per Valenzanreicherung hinzugefügter Dativ, oder (3) eine Dativangabe.

\begin{center}
  \begin{longtable}[h]{cp{0.8\textwidth}}
    &\\
    (1) & \\\cline{2-2}
    &\\
    &\\\cline{2-2}
    &\\
    (2) & \\\cline{2-2}
    &\\
    &\\\cline{2-2}
    &\\
    (3) & \\\cline{2-2}
    &\\
    &\\\cline{2-2}
    &\\
    (4) & \\\cline{2-2}
    &\\
    &\\\cline{2-2}
    &\\
    (5) & \\\cline{2-2}
    &\\
    &\\\cline{2-2}
    &\\
    (6) & \\\cline{2-2}
    &\\
    &\\\cline{2-2}
  \end{longtable}
\end{center}

\newpage

\section{Statusrektion}

Bestimmen Sie in den folgenden Sätzen die Status der infiniten Verben, indem Sie (1) die ersten Status \ul{unterstreichen}, (2) die zweiten Status \mybox{umrahmen} und (3) die dritten Status \sout{durchstreichen}.
Zeichnen Sie jeweils einen Pfeil ein, der vom statusregierenden Verb auf as statusregierte Verb zeigt.

\begin{spread}
  \begin{tabular}[h]{cl}
    (1) & \textit{Adrianna wird wieder Salat bestellen wollen.}\\
    &\\
    (2) & \textit{Man kolportiert, dass der Kollege plagiiert zu haben scheint.}\\
    &\\
    (3) & \textit{Ich finde es schade, dass man den Bär hat einfangen lassen.}\\
    &\\
    (4) & \textit{Der Ruin wurde von der Firma versucht, unbedingt zu vermeiden.}\\
    &\\
    (5) & \textit{Der Rottweiler war sehr gut erzogen worden.}\\
    &\\
    (6) & \textit{Michelle hat uns versprochen, auf den Kater aufzupassen.}\\
    &\\
  \end{tabular}
\end{spread}

