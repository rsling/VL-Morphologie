\section{Adjektivflexion}

Entscheiden Sie, ob die unterstrichenen Adjektivformen in den Sätzen in der nachstehenden Tabelle adjektivisch (adj) oder pronominal (pron) flektiert sind.
Geben Sie jeweils den Grund dafür an, dass das zu klassifizierende Adjektiv im gegebenen Kontext so flektiert.
Als Grund kommt \textbf{ausschließlich} die Morphologie der dem Adjektiv vorangehenden Wortform infrage:

\begin{enumerate}\Lf
  \item ein Artikel ohne Flexionsendung ($-$)
  \item ein Artikel oder Pronomen mit Flexionsendung ($+$)
  \item kein Artikel oder Pronomen (NP ohne Artikelwort; $\emptyset$)
\end{enumerate}

\begin{center}
  \resizebox{\textwidth}{!}{%
  \begin{tabular}[h]{rllcl}
    \toprule
    & \textbf{Adjektiv im Satzkontext} & \textbf{Klassifikation} && \textbf{Grund} \\
    \midrule
    (1) &  \textit{Ich kaufe den \uline{leckeren} Kaffee aus Rom.} & \Solalt{\XBox}{\Square}~adj\ \ \Solalt{\Square}{\Square}~pron && \Solalt{\Square}{\Square}~$-$\ \ \Solalt{\XBox}{\Square}~$+$\ \ \Solalt{\Square}{\Square}~$\emptyset$ \\
    (2) &  \textit{\uline{Große} Bäume vor dem Fenster spenden Kühle.} & \Solalt{\Square}{\Square}~adj\ \ \Solalt{\XBox}{\Square}~pron && \Solalt{\Square}{\Square}~$-$\ \ \Solalt{\Square}{\Square}~$+$\ \ \Solalt{\XBox}{\Square}~$\emptyset$ \\
    (3) &  \textit{Seine \uline{nervigen} Kinder bleiben zuhause.} & \Solalt{\XBox}{\Square}~adj\ \ \Solalt{\Square}{\Square}~pron && \Solalt{\Square}{\Square}~$-$\ \ \Solalt{\XBox}{\Square}~$+$\ \ \Solalt{\Square}{\Square}~$\emptyset$ \\
    (4) &  \textit{Mit diesen \uline{komischen} Leuten kann ich nichts anfangen.} & \Solalt{\XBox}{\Square}~adj\ \ \Solalt{\Square}{\Square}~pron && \Solalt{\Square}{\Square}~$-$\ \ \Solalt{\XBox}{\Square}~$+$\ \ \Solalt{\Square}{\Square}~$\emptyset$ \\
    (5) &  \textit{Sein \uline{schöner} Volvo Amazon ist im Bestzustand.} & \Solalt{\Square}{\Square}~adj\ \ \Solalt{\XBox}{\Square}~pron && \Solalt{\XBox}{\Square}~$-$\ \ \Solalt{\Square}{\Square}~$+$\ \ \Solalt{\Square}{\Square}~$\emptyset$ \\
    (6) &  \textit{Wir warten die Bremsen des \uline{alten} Rekords.} & \Solalt{\XBox}{\Square}~adj\ \ \Solalt{\Square}{\Square}~pron && \Solalt{\Square}{\Square}~$-$\ \ \Solalt{\XBox}{\Square}~$+$\ \ \Solalt{\Square}{\Square}~$\emptyset$ \\
    (7) &  \textit{Wir besuchen ein \uline{schönes} Schloss.} & \Solalt{\Square}{\Square}~adj\ \ \Solalt{\XBox}{\Square}~pron && \Solalt{\XBox}{\Square}~$-$\ \ \Solalt{\Square}{\Square}~$+$\ \ \Solalt{\Square}{\Square}~$\emptyset$ \\
    (8) &  \textit{Jenes \uline{schöne} Umspannwerk steht in Twistetal.} & \Solalt{\XBox}{\Square}~adj\ \ \Solalt{\Square}{\Square}~pron && \Solalt{\Square}{\Square}~$-$\ \ \Solalt{\XBox}{\Square}~$+$\ \ \Solalt{\Square}{\Square}~$\emptyset$ \\
    (9) &  \textit{Das ist kein \uline{echter} Spitzweg!} & \Solalt{\Square}{\Square}~adj\ \ \Solalt{\XBox}{\Square}~pron && \Solalt{\XBox}{\Square}~$-$\ \ \Solalt{\Square}{\Square}~$+$\ \ \Solalt{\Square}{\Square}~$\emptyset$ \\
    (10) & \textit{Meinen Eltern zeigen wir das \uline{schöne} Schloss.} & \Solalt{\XBox}{\Square}~adj\ \ \Solalt{\Square}{\Square}~pron && \Solalt{\Square}{\Square}~$-$\ \ \Solalt{\XBox}{\Square}~$+$\ \ \Solalt{\Square}{\Square}~$\emptyset$ \\
    \bottomrule
  \end{tabular}}
\end{center}

\Doppelzeile

\section{Flexionstypen der Verben}

Entscheiden Sie für die Verben im nachstehenden Textausschnitt, ob sie mit Stammvokaländerungen (SVÄ$+$; auch: stark; Ablaut, Alternanz usw.), ohne Stammvokaländerungen (SVÄ$-$; auch: schwach) oder wie Modalverben (sog.\ Präteritalpräsentien; PP) flektieren.
Geben Sie eine Form an, die das relativ zur angegebenen Form eindeutig zeigt.
Die aufgeführte und die von Ihnen ergänzte Form sollen also zusammen ein Paar ergeben, an dem man die Flexionsklasse eindeutig ablesen kann.

\newpage

\begin{center}
  \begin{tabular}[h]{rllp{0.25\textwidth}}
    \toprule
    & \textbf{Verbform} & \textbf{Klassifikation} & \textbf{eindeutiges Beispiel} \\
    \midrule
    &&& \\
    (1) & (\textit{sie}) \textit{trank} & \Solalt{\XBox}{\Square}~SVÄ$+$\ \ \Solalt{\Square}{\Square}~SVÄ$-$\ \ \Solalt{\Square}{\Square}~PP &  \Sol{(sie) trinkt} \\\cline{4-4}
    &&& \\
    (2) & (\textit{du}) \textit{darfst} & \Solalt{\Square}{\Square}~SVÄ$+$\ \ \Solalt{\Square}{\Square}~SVÄ$-$\ \ \Solalt{\XBox}{\Square}~PP &  \Sol{(du) durftest} \\\cline{4-4}
    &&& \\
    (3) & (\textit{sie}) \textit{salbten} & \Solalt{\Square}{\Square}~SVÄ$+$\ \ \Solalt{\XBox}{\Square}~SVÄ$-$\ \ \Solalt{\Square}{\Square}~PP &  \Sol{(sie) salben} \\\cline{4-4}
    &&& \\
    (4) & (\textit{ich}) \textit{fülle} & \Solalt{\Square}{\Square}~SVÄ$+$\ \ \Solalt{\XBox}{\Square}~SVÄ$-$\ \ \Solalt{\Square}{\Square}~PP &  \Sol{(ich) füllte} \\\cline{4-4}
    &&& \\
    (5) & (\textit{wir)} \textit{wissen} & \Solalt{\Square}{\Square}~SVÄ$+$\ \ \Solalt{\Square}{\Square}~SVÄ$-$\ \ \Solalt{\XBox}{\Square}~PP &  \Sol{(wir) wussten} \\\cline{4-4}
    &&& \\
    (6) & (\textit{sie haben}) \textit{gestohlen} & \Solalt{\XBox}{\Square}~SVÄ$+$\ \ \Solalt{\Square}{\Square}~SVÄ$-$\ \ \Solalt{\Square}{\Square}~PP &  \Sol{(sie) stahlen} \\\cline{4-4}
  \end{tabular}
\end{center}

\Zeile

\section{Verbformen}

Bilden Sie die genannten Formen der angegebenen Verben.
Wenn nicht Konjunktiv oder Infinitiv angegeben sind, soll der Indikativ gebildet werden.
Wenn nicht Passiv angegeben ist, soll das Aktiv gebildet werden.
Die verwendeten Abkürzungen sind:

\begin{itemize}\Lf
  \item intrinsisch finite Tempora | Präs, Prät, Fut
  \item nicht intrinsisch finites Quasitempus | Perf
  \item Infinitiv | Inf
  \item Modus | (Ind,) Konj
  \item Person | P1, P2, P3
  \item Numerus | Sg, Pl
  \item Diathese | (Akt,) Pass
\end{itemize}

\begin{center}
  \resizebox{\textwidth}{!}{%
  \begin{tabular}[h]{cllp{0.55\textwidth}}
    \toprule
    & \textbf{Verb} & \textbf{vorgegebene Merkmale} & \textbf{Form} \\
    \midrule
    &&& \\
    (1) & \textit{raufen} & Fut Perf P2 Pl &  \Sol{werdet gerauft haben} \\\cline{4-4}
    &&& \\
    (2) & \textit{singen} & Prät P3 Sg &  \Sol{sang} \\\cline{4-4}
    &&& \\
    (3) & \textit{liegen} & Konj Präs P3 Sg &  \Sol{liege} \\\cline{4-4}
    &&& \\
    (4) & \textit{verschenken} & Inf Perf Pass &  \Sol{verschenkt worden sein} \\\cline{4-4}
    &&& \\
    (5) & \textit{rennen} & Inf Perf &  \Sol{gerannt sein} \\\cline{4-4}
    &&& \\
    (6) & \textit{müssen} & Konj Prät P2 Pl &  \Sol{müsstet} \\\cline{4-4}
    &&& \\
    (7) & \textit{begrüßen} & Fut Perf P2 Pl Pass &  \Sol{werdet begrüßt worden sein} \\\cline{4-4}
  \end{tabular}}
\end{center}

\newpage

\section{Konjunktiv}

Versuchen Sie, den nachstehenden Text zunächst in den Konjunktiv~1 und dann in den Konjunktiv~2 zu setzen.
Die Ersetzungsregeln zur Vermeidung von formalen Ähnlichkeiten sind:

\begin{enumerate}\Lf
  \item Wenn die Form des Konj1 (Konj Präs) nicht von der Form des Ind Präs zu unterscheiden ist, wird der Konj2 (Konj Prät) genommen.
  \item Wenn die Form des Konj2 (Konj Prät) nicht von der Form des Ind Prät zu unterscheiden ist, wird die analytische \textit{würde}-Paraphrase genommen.
\end{enumerate}

Diskutieren Sie im Anschluss daran, welche Formen trotz der Ersatzregeln grundsätzlich Probleme machen.

\begin{quote}
  Die Grammatik folgt Regeln, und sie folgte schon immer Regeln.
  Nur das kann der Grund sein, dass wir einander verstehen, wenn wir Sprache benutzen.
  Die Mathematik ist axiomatisch eingeführt worden.
  Sie gehorcht damit ausnahmslosen Regeln, während die Regeln der Grammatik Ausnahmen zulassen.
\end{quote}

\Sol{\textbf{Konjunktiv I = Konjunktiv Präsens}
  \begin{quote}
    Die Grammatik folge Regeln, und sie (i) würde schon immer Regeln folgen \slash\ (ii) sei schon immer Regeln gefolgt.
    Nur das könne der Grund sein, dass wir einander verstünden, wenn wir Sprache (i) benutzten \slash\ (ii) benutzen würden.
    Die Mathematik sei axiomatisch eingeführt worden.
    Sie gehorche damit ausnahmslosen Regeln, während die Regeln der Grammatik Ausnahmen zuließen.
  \end{quote}
}

\Sol{\textbf{Konjunktiv 2 = Konjunktiv Präteritum}
  \begin{quote}
    Die Grammatik würde Regeln folgen, und sie (i) würde schon immer Regeln folgen \slash\ (ii) wäre schon immer Regeln gefolgt.
    Nur das könnte der Grund sein, dass wir einander verstünden, wenn wir Sprache (i) benutzten \slash\ (ii) benutzen würden.
    Die Mathematik wäre axiomatisch eingeführt worden.
    Sie würde damit ausnahmlosen Regeln folgen, während die Regeln der Grammatik Ausnahmen zuließen.
  \end{quote}
}

\Sol{\textbf{Diskussion:} In der Schule wird stets so getan, als sei das Hauptproblem die Formengleichheit bestimmter Indikativ- und Infinitivformen zu Konjunktivformen.
Nur diese Fälle werden durch die üblichen Ersetzungsregeln abgedeckt.
Das eigentliche Problem mit dem deutschen Konjunktiv ist allerdings, dass zwar eine Präsens- und eine Präteritum-Variante existieren, diese aber nicht mehr die Bedeutung von Präsens und Präteritum haben.
(Sie kodieren ein Spektrum von indirekter Rede bis Möglichkeit\slash Irrealität.)
Es ist daher nicht zuverlässig möglich, andere Tempora als das Präsens so in den Konjunktiv zu setzen, dass die Tempusbedeutung erhalten bleibt.
Man muss z.\,B.\ statt des Präteritums auf die Perfektformen ausweichen (\textit{sei schon immer Regeln gefolgt}).
Das Futur mit \textit{werden} im Konjunktiv Präteritum escheint automatisch als \textit{würde}-Paraphrase ohne Futurbedeutung und ohne Ausweichformen (\textit{Ich werde lesen.} zu \textit{Ich würde lesen.}).
Dies erodiert allerdings die Ausdrucksmöglichkeiten für Tempus in Konjunktivformen erheblich.
Andere Sprachen sind in diesem Punkt flexibler.}
