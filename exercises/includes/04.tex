\section{Komposita analysieren}\label{sec:analyse}

Im Textausschnitt \textit{Steuerrecht (Deutschland)} sind einige Komposita unterstrichen.
Analysieren Sie die Komposita in der Tabelle, indem Sie die Glieder des Kompositums mit einem Punkt abtrennen.
Eventuelle Fugenelemente trennen Sie vom vorausgehenden Glied mit einem Bindestrich ab.
Für \textit{Kosakenzipfel} wäre die Analyse also \textit{Kosake-n.zipfel}.
Sollten ein Glied oder mehrere Glieder des Kompositums \textbf{keine} Substantiv sein, bestimmen Sie die Wortklassen dieser Glieder.
In der Tabelle sind bereits die Nennformen der Wörter angegeben, da die Flexionsendungen der Komposita im Text nicht mitanalysiert werden sollen.

\Zeile

\begin{quote}
  \textbf{Steuerrecht (Deutschland)}\\
  {\footnotesize Quelle: \url{https://de.wikipedia.org/wiki/Steuerrecht_(Deutschland)} (modifiziert)}\\

  \textbf{Einleitung}\\
  Das (1)\ul{Steuerrecht} ist das (2)\ul{Spezialgebiet} des öffentlichen Rechts, das die (3)\ul{Festsetzung} und Erhebung von Steuern regelt. Das Verfahren der (4)\ul{Steuerfestsetzung} und -erhebung wird weitgehend durch die (5)\ul{Abgabenordnung} bestimmt, die die wesentlichen Vorschriften des (6)\ul{Steuerverfahrensrechts} enthält, während das materielle Steuerrecht, also die konkreten Bestimmungen zur Höhe der Steuerschuld, in zahlreichen Einzelgesetzen verankert ist. Im weiteren Sinne werden zum Steuerrecht auch die (7)\ul{Rechtsnormen} gerechnet, die sich mit der (8)\ul{Steuerverwaltung} und der (9)\ul{Finanzgerichtsbarkeit} befassen. Üblicherweise nicht zum eigentlichen Steuerrecht gezählt werden hingegen die Vorschriften, die sich mit der Steuergesetzgebung und der Verteilung des Steueraufkommens befassen (Teile des Grundgesetzes und das Zerlegungsgesetz). Dennoch sind diese Rechtsnormen für das Verständnis des Steuerrechts unerlässlich.\\

  \textbf{Autonomie des Steuerrechts}\\
  Das Steuerrecht ist ein eigenständiges Rechtsgebiet. In ihm hat der (10)\ul{Rechtsgeber} alle Rechtsnormen, die das Steuerwesen der(11)\ul{Bundesrepublik} Deutschland regeln, zusammengefasst, insbesondere das Verhältnis zwischen den (12)\ul{Steuerhoheitsträgern} und den (13)\ul{Steuerzahlern}. Steuerrechtliche Tatbestände und Rechtsbegriffe sind eigenständig definiert. Zwar sind Privat- und Steuerrecht dort verbunden, wo das Steuerrecht nicht nur an die gegebenen Lebensverhältnisse und damit auch an ihre (14)\ul{zivilrechtliche} Ordnung anknüpft, sondern den Steuergegenstand prinzipiell nach Rechtsformen des bürgerlichen Rechts bestimmt. Knüpft eine steuerrechtliche Norm an eine zivilrechtliche Gestaltung an, so ist die Auslegung der steuerrechtlichen Bestimmung aber weder zwingend an dem Vertragstyp auszurichten, der der von den Parteien gewählten Bezeichnung entspricht, noch wird sie notwendigerweise von der zivilrechtlichen Qualifikation des Rechtsgeschäfts beeinflusst. Auch gilt keine Vermutung, das dem Zivilrecht entlehnte Tatbestandsmerkmal einer Steuerrechtsnorm im Sinne des zivilrechtlichen Verständnisses zu interpretieren, weil Zivilrecht und Steuerrecht (15)\ul{nebengeordnete}, gleichrangige Rechtsgebiete sind, die denselben Sachverhalt aus einer anderen Perspektive und unter anderen (16)\ul{Wertungsgesichtspunkten} beurteilen. Die Parteien können zwar einen Sachverhalt vertraglich gestalten, nicht aber die steuerrechtlichen Folgen bestimmen, die das Steuergesetz an die vorgegebene Gestaltung knüpft. Insoweit gilt eine Vorherigkeit für die Anwendung des Zivilrechts, jedoch kein Vorrang (sog. Autonomie des Steuerrechts).
\end{quote}

\Zeile

\begin{center}
  \scalebox{0.9}{%
    \begin{tabular}[h]{clp{0.4\textwidth}p{0.005\textwidth}p{0.3\textwidth}}
    \toprule
    & \textbf{Kompositum} & \textbf{Analyse} && \textbf{Wortklassen} \\
    \midrule
    &&&&\\
    (1) & \textit{Steuerrecht} &&& \\\cline{3-3}\cline{5-5}
    &&&&\\
    (2) & \textit{Spezialgebiet} &&& \\\cline{3-3}\cline{5-5}
    &&&&\\
    (3) & \textit{Festsetzung} &&& \\\cline{3-3}\cline{5-5}
    &&&&\\
    (4) & \textit{Steuerfestsetzung} &&& \\\cline{3-3}\cline{5-5}
    &&&&\\
    (5) & \textit{Abgabenordnung} &&& \\\cline{3-3}\cline{5-5}
    &&&&\\
    (6) & \textit{Steuerverfahrensrecht} &&& \\\cline{3-3}\cline{5-5}
    &&&&\\
    (7) & \textit{Rechtsnormen} &&& \\\cline{3-3}\cline{5-5}
    &&&&\\
    (8) & \textit{Steuerverwaltung} &&& \\\cline{3-3}\cline{5-5}
    &&&&\\
    (9) & \textit{Finanzgerichtsbarkeit} &&& \\\cline{3-3}\cline{5-5}
    &&&&\\
    (10) & \textit{Rechtsgeber} &&& \\\cline{3-3}\cline{5-5}
    &&&&\\
    (11) & \textit{Bundesrepublik} &&& \\\cline{3-3}\cline{5-5}
    &&&&\\
    (12) & \textit{Steuerhoheitsträger} &&& \\\cline{3-3}\cline{5-5}
    &&&&\\
    (13) & \textit{Steuerzahler} &&& \\\cline{3-3}\cline{5-5}
    &&&&\\
    (14) & \textit{zivilrechtlich} &&& \\\cline{3-3}\cline{5-5}
    &&&&\\
    (15) & \textit{nebengeordnet} &&& \\\cline{3-3}\cline{5-5}
    &&&&\\
    (16) & \textit{Wertungsgesichtspunkt} &&& \\\cline{3-3}\cline{5-5}
  \end{tabular}}
\end{center}

\newpage

\section{Typen}

Bestimmen Sie für die Komposita aus Aufgabe~\ref{sec:analyse} den Kompositionstyp.
Die beiden Typen von Rektionskomposita unterscheiden sich darin, dass sich beim Typ 2 der Kopf wie ein Subjekt des zugrundeliegenden Verbs verhält, beim Typ 1 nicht.
Lesen Sie das dringend nochmal im Buch nach bzw. sehen Sie das Segment der Vorlesung nochmal, wenn Ihnen das nicht klar ist.

\Zeile

\begin{center}
  \scalebox{0.9}{%
  \begin{tabular}[h]{cll}
    \toprule
    & \textbf{Kompositum} & \textbf{Typ} \\
    \midrule
    &&\\
    (1) & \textit{Steuerrecht} & \Square~Determinativk.\ \ \Square~Rektionsk. 1\ \ \Square~Rektionsk. 2\ \ \Square~Andere \\
    &&\\
    (2) & \textit{Spezialgebiet} & \Square~Determinativk.\ \ \Square~Rektionsk. 1\ \ \Square~Rektionsk. 2\ \ \Square~Andere \\
    &&\\
    (3) & \textit{Festsetzung} & \Square~Determinativk.\ \ \Square~Rektionsk. 1\ \ \Square~Rektionsk. 2\ \ \Square~Andere \\
    &&\\
    (4) & \textit{Steuerfestsetzung} & \Square~Determinativk.\ \ \Square~Rektionsk. 1\ \ \Square~Rektionsk. 2\ \ \Square~Andere \\
    &&\\
    (5) & \textit{Abgabenordnung} & \Square~Determinativk.\ \ \Square~Rektionsk. 1\ \ \Square~Rektionsk. 2\ \ \Square~Andere \\
    &&\\
    (6) & \textit{Steuerverfahrensrecht} & \Square~Determinativk.\ \ \Square~Rektionsk. 1\ \ \Square~Rektionsk. 2\ \ \Square~Andere \\
    &&\\
    (7) & \textit{Rechtsnormen} & \Square~Determinativk.\ \ \Square~Rektionsk. 1\ \ \Square~Rektionsk. 2\ \ \Square~Andere \\
    &&\\
    (8) & \textit{Steuerverwaltung} & \Square~Determinativk.\ \ \Square~Rektionsk. 1\ \ \Square~Rektionsk. 2\ \ \Square~Andere \\
    &&\\
    (9) & \textit{Finanzgerichtsbarkeit} & \Square~Determinativk.\ \ \Square~Rektionsk. 1\ \ \Square~Rektionsk. 2\ \ \Square~Andere \\
    &&\\
    (10) & \textit{Rechtsgeber} & \Square~Determinativk.\ \ \Square~Rektionsk. 1\ \ \Square~Rektionsk. 2\ \ \Square~Andere \\
    &&\\
    (11) & \textit{Bundesrepublik} & \Square~Determinativk.\ \ \Square~Rektionsk. 1\ \ \Square~Rektionsk. 2\ \ \Square~Andere \\
    &&\\
    (12) & \textit{Steuerhoheitsträger} & \Square~Determinativk.\ \ \Square~Rektionsk. 1\ \ \Square~Rektionsk. 2\ \ \Square~Andere \\
    &&\\
    (13) & \textit{Steuerzahler} & \Square~Determinativk.\ \ \Square~Rektionsk. 1\ \ \Square~Rektionsk. 2\ \ \Square~Andere \\
    &&\\
    (14) & \textit{zivilrechtlich} & \Square~Determinativk.\ \ \Square~Rektionsk. 1\ \ \Square~Rektionsk. 2\ \ \Square~Andere \\
    &&\\
    (15) & \textit{nebengeordnet} & \Square~Determinativk.\ \ \Square~Rektionsk. 1\ \ \Square~Rektionsk. 2\ \ \Square~Andere \\
    &&\\
    (16) & \textit{Wertungsgesichtspunkt} & \Square~Determinativk.\ \ \Square~Rektionsk. 1\ \ \Square~Rektionsk. 2\ \ \Square~Andere \\
  \end{tabular}}
\end{center}


