\documentclass[12pt,a4paper,twoside]{article}

\usepackage[margin=2cm]{geometry}

\usepackage[ngerman]{babel}

\usepackage{setspace}
\usepackage{booktabs}
\usepackage{array,graphics}
\usepackage{color}
\usepackage{soul}
\usepackage[linecolor=gray,backgroundcolor=yellow!50,textsize=tiny]{todonotes}
\usepackage[linguistics]{forest}
\usepackage{multirow}
\usepackage{pifont}
\usepackage{wasysym}
\usepackage{langsci-gb4e}
\usepackage{soul}
\usepackage{enumitem}
\usepackage{marginnote}

\usepackage[maxbibnames=99,
  maxcitenames=2,
  uniquelist=false,
  backend=biber,
  doi=false,
  url=false,
  isbn=false,
  bibstyle=biblatex-sp-unified,
  citestyle=sp-authoryear-comp]{biblatex}

\definecolor{rot}{rgb}{0.7,0.2,0.0}
\newcommand{\rot}[1]{\textcolor{rot}{#1}}
\definecolor{blau}{rgb}{0.1,0.2,0.7}
\newcommand{\blau}[1]{\textcolor{blau}{#1}}
\definecolor{gruen}{rgb}{0.0,0.7,0.2}
\newcommand{\gruen}[1]{\textcolor{gruen}{#1}}
\definecolor{grau}{rgb}{0.6,0.6,0.6}
\newcommand{\grau}[1]{\textcolor{grau}{#1}}
\definecolor{orongsch}{RGB}{255,165,0}
\newcommand{\orongsch}[1]{\textcolor{orongsch}{#1}}
\definecolor{tuerkis}{RGB}{63,136,143}
\definecolor{braun}{RGB}{108,71,65}
\newcommand{\tuerkis}[1]{\textcolor{tuerkis}{#1}}
\newcommand{\braun}[1]{\textcolor{braun}{#1}}

\newcommand*\Rot{\rotatebox{75}}

\newcommand{\zB}{z.\,B.\ }
\newcommand{\ZB}{Z.\,B.\ }
\newcommand{\Sub}[1]{\ensuremath{_{\text{#1}}}}
\newcommand{\Up}[1]{\ensuremath{^{\text{#1}}}}
\newcommand{\UpSub}[2]{\ensuremath{^{\text{#1}}_{\text{#2}}}}
\newcommand{\Doppelzeile}{\vspace{2\baselineskip}}
\newcommand{\Zeile}{\vspace{\baselineskip}}
\newcommand{\Halbzeile}{\vspace{0.5\baselineskip}}
\newcommand{\Viertelzeile}{\vspace{0.25\baselineskip}}

\newcommand{\whyte}[1]{\textcolor{white}{#1}}

\newcommand{\Spur}[1]{t\Sub{#1}}
\newcommand{\Ti}{\Spur{1}}
\newcommand{\Tii}{\Spur{2}}
\newcommand{\Tiii}{\Spur{3}}
\newcommand{\Tiv}{\Spur{4}}
\newcommand*{\mybox}[1]{\framebox{#1}}
\newcommand\ol[1]{{\setul{-0.9em}{}\ul{#1}}}

\newcommand{\Lf}{
  \setlength{\itemsep}{1pt}
  \setlength{\parskip}{0pt}
  \setlength{\parsep}{0pt}
}

\forestset{
  Ephr/.style={draw, ellipse, thick, inner sep=2pt},
  Eobl/.style={draw, rounded corners, inner sep=5pt},
  Eopt/.style={draw, rounded corners, densely dashed, inner sep=5pt},
  Erec/.style={draw, rounded corners, double, inner sep=5pt},
  Eoptrec/.style={draw, rounded corners, densely dashed, double, inner sep=5pt},
  Ehd/.style={rounded corners, fill=gray, inner sep=5pt,
    delay={content=\whyte{##1}}
  },
  Emult/.style={for children={no edge}, for tree={l sep=0pt}},
  phrasenschema/.style={for tree={l sep=2em, s sep=2em}},
  sake/.style={tier=preterminal},
  ake/.style={
    tier=preterminal
    },
}

\forestset{
  decide/.style={draw, chamfered rectangle, inner sep=2pt},
  finall/.style={rounded corners, fill=gray, text=white},
  intrme/.style={draw, rounded corners},
  yes/.style={edge label={node[near end, above, sloped, font=\scriptsize]{Ja}}},
  no/.style={edge label={node[near end, above, sloped, font=\scriptsize]{Nein}}}
}

\usepackage{tikz}
\usetikzlibrary{arrows,positioning} 


\author{Prof.\ Dr.\ Roland Schäfer | Germanistische Linguistik FSU Jena}
\title{Morphologie | 02 | Minimale Syntax}
\date{Version von 2023}


\usepackage{fontspec}
\defaultfontfeatures{Ligatures=TeX,Numbers=OldStyle, Scale=MatchLowercase}
\setmainfont{Linux Libertine O}
\setsansfont{Linux Biolinum O}

\setlength{\parindent}{0pt}

\usepackage[headings]{fancyhdr}
\fancyhead[E,O]{}
\fancyfoot[E,O]{}
\renewcommand{\headrulewidth}{0pt}
\pagestyle{fancy}
\setlength{\headsep}{50pt}
\setlength{\textheight}{\textheight-25pt}


\begin{document}

\maketitle

\section{Satzglieder}\label{sec:satzglieder}

Bestimmen Sie für die unterstrichenen Teile in den folgenden Sätzen, ob sie Satzglieder sind, indem Sie den \textbf{Vorfeld-Test} anwenden.
Zur Erinnerung: Beim Vorfeld-Test versuchen Sie, den Satz so umzustellen, dass das potentielle Satzglied vor dem finiten Verb zu stehen kommt.
Finite Verben sind diejenigen Verben, die nach Tempus (Präsens\slash Präteritum), Numerus (Singular\slash Plural) und Person (1\slash 2\slash 3) flektiert sind.

\begin{enumerate}
  \item \ul{Der Winter} ist vorbei.
  \item Otje schickt \ul{seinen Kindern aus} dem Urlaub \ul{eine Karte}.
  \item Wir kaufen öfter \ul{Produkte}, die Regional gefertigt wurden.
  \item K. R. Popper ist \ul{der Philosoph, auf dessen Werken alle falsifikationistischen Wissenschaftstheorien basieren}.
  \item Zu dieser Jahreszeit gibt es keine \ul{klimaneutral produzierten} Erdbeeren \ul{in Deutschland}.
  \item Ich glaube \ul{überhaupt nicht}, \ul{dass ein solcher Unsinn überhaupt ernstgenommen wird}.
  \item Alle Wissenschaftler möchten gerne \ul{einen großen Erfolg für sich verbuchen} können.
  \item Man darf \ul{seinen Hund} \ul{beim Einkaufen} nicht \ul{im Auto} zurücklassen.
  \item \ul{Heute} hat es keinen Zweck, \ul{rudern zu gehen}.
  \item Der abgewählte Präsident goss bei einer Wahlkampfveranstaltung \ul{Öl ins Feuer}.
  \item \ul{Der Hund unter dem Tisch} will endlich \ul{sein Fressen}.
  \item \ul{Dass es heute} regnet, ist \ul{so gut wie sicher}.
  \item \ul{Eine Entscheidung für den Frieden} ist \ul{nicht} \ul{generell} \ul{unvereinbar} mit \ul{einer Entscheidung für militärische Aufrüstung}.
\end{enumerate}

\section{Nominal- und Präpositionalphrasen}

Welche der unterstrichenen Teile aus Aufgabe~\ref{sec:satzglieder} sind NPs und PPs?
Gibt es andere nicht unterstrichene NPs und PPs in den Sätzen?

\end{document}
