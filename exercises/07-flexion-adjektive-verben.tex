\documentclass[12pt,a4paper,twoside]{article}

\usepackage[margin=2cm]{geometry}

\usepackage[ngerman]{babel}

\usepackage{setspace}
\usepackage{array,graphics}
\usepackage{color}
\usepackage{soul}
\usepackage[linecolor=gray,backgroundcolor=yellow!50,textsize=tiny]{todonotes}
\usepackage[linguistics]{forest}
\usepackage{multirow}
\usepackage{pifont}
\usepackage{wasysym}
\usepackage{langsci-gb4e}
\usepackage{soul}
\usepackage{enumitem}
\usepackage{marginnote}
\usepackage{ulem}
\usepackage{longtable,booktabs}

\usepackage[maxbibnames=99,
  maxcitenames=2,
  uniquelist=false,
  backend=biber,
  doi=false,
  url=false,
  isbn=false,
  bibstyle=biblatex-sp-unified,
  citestyle=sp-authoryear-comp]{biblatex}

\definecolor{rot}{rgb}{0.7,0.2,0.0}
\newcommand{\rot}[1]{\textcolor{rot}{#1}}
\definecolor{blau}{rgb}{0.1,0.2,0.7}
\newcommand{\blau}[1]{\textcolor{blau}{#1}}
\definecolor{gruen}{rgb}{0.0,0.7,0.2}
\newcommand{\gruen}[1]{\textcolor{gruen}{#1}}
\definecolor{grau}{rgb}{0.6,0.6,0.6}
\newcommand{\grau}[1]{\textcolor{grau}{#1}}
\definecolor{orongsch}{RGB}{255,165,0}
\newcommand{\orongsch}[1]{\textcolor{orongsch}{#1}}
\definecolor{tuerkis}{RGB}{63,136,143}
\definecolor{braun}{RGB}{108,71,65}
\newcommand{\tuerkis}[1]{\textcolor{tuerkis}{#1}}
\newcommand{\braun}[1]{\textcolor{braun}{#1}}

\newcommand*\Rot{\rotatebox{75}}
\newcommand*\RotRec{\rotatebox{90}}

\newcommand{\zB}{z.\,B.\ }
\newcommand{\ZB}{Z.\,B.\ }
\newcommand{\Sub}[1]{\ensuremath{_{\text{#1}}}}
\newcommand{\Up}[1]{\ensuremath{^{\text{#1}}}}
\newcommand{\UpSub}[2]{\ensuremath{^{\text{#1}}_{\text{#2}}}}
\newcommand{\Doppelzeile}{\vspace{2\baselineskip}}
\newcommand{\Zeile}{\vspace{\baselineskip}}
\newcommand{\Halbzeile}{\vspace{0.5\baselineskip}}
\newcommand{\Viertelzeile}{\vspace{0.25\baselineskip}}

\newcommand{\whyte}[1]{\textcolor{white}{#1}}

\newcommand{\Spur}[1]{t\Sub{#1}}
\newcommand{\Ti}{\Spur{1}}
\newcommand{\Tii}{\Spur{2}}
\newcommand{\Tiii}{\Spur{3}}
\newcommand{\Tiv}{\Spur{4}}
\newcommand*{\mybox}[1]{\framebox{#1}}
\newcommand\ol[1]{{\setul{-0.9em}{}\ul{#1}}}

\newenvironment{nohyphens}{%
  \par
  \hyphenpenalty=10000
  \exhyphenpenalty=10000
  \sloppy
}{\par}

\newcommand{\Lf}{
  \setlength{\itemsep}{1pt}
  \setlength{\parskip}{0pt}
  \setlength{\parsep}{0pt}
}

\forestset{
  Ephr/.style={draw, ellipse, thick, inner sep=2pt},
  Eobl/.style={draw, rounded corners, inner sep=5pt},
  Eopt/.style={draw, rounded corners, densely dashed, inner sep=5pt},
  Erec/.style={draw, rounded corners, double, inner sep=5pt},
  Eoptrec/.style={draw, rounded corners, densely dashed, double, inner sep=5pt},
  Ehd/.style={rounded corners, fill=gray, inner sep=5pt,
    delay={content=\whyte{##1}}
  },
  Emult/.style={for children={no edge}, for tree={l sep=0pt}},
  phrasenschema/.style={for tree={l sep=2em, s sep=2em}},
  sake/.style={tier=preterminal},
  ake/.style={
    tier=preterminal
    },
}

\forestset{
  decide/.style={draw, chamfered rectangle, inner sep=2pt},
  finall/.style={rounded corners, fill=gray, text=white},
  intrme/.style={draw, rounded corners},
  yes/.style={edge label={node[near end, above, sloped, font=\scriptsize]{Ja}}},
  no/.style={edge label={node[near end, above, sloped, font=\scriptsize]{Nein}}}
}

\usepackage{tikz}
\usetikzlibrary{arrows,positioning} 


\author{Prof.\ Dr.\ Roland Schäfer | Germanistische Linguistik FSU Jena}
\title{Morphologie | 06 | Adjektive und Verben}
\date{Version Sommer 2023 (\today)}


\usepackage{fontspec}
\defaultfontfeatures{Ligatures=TeX,Numbers=OldStyle, Scale=MatchLowercase}
\setmainfont{Linux Libertine O}
\setsansfont{Linux Biolinum O}

\setlength{\parindent}{0pt}

% \usepackage[headings]{fancyhdr}
% \fancyhead[E,O]{}
% \fancyfoot[E,O]{}
% \renewcommand{\headrulewidth}{0pt}
% \pagestyle{fancy}
% \setlength{\headsep}{50pt}
% \setlength{\textheight}{\textheight-25pt}


\newenvironment{spread}
{%
  \newdimen\origiwspc%
  \newdimen\origiwstr%
  \origiwspc=\fontdimen2\font%
  \origiwstr=\fontdimen3\font%
  \fontdimen2\font=1em%
  \doublespacing%
}{%
  \fontdimen2\font=\origiwspc%
  \fontdimen3\font=\origiwstr%
}



\begin{document}

\maketitle

\section{Adjektivflexion}

Entscheiden Sie, ob die unterstrichenen Adjektive adjektivisch (schwach) oder pronominal (stark) flektieren.

\section{Flexionstypen der Verben}

Entscheiden Sie für die Verben im nachstehenden Textausschnit, ob sie ohne Vokalveränderungen (schwach), mit Vokalveränderungen (stark; Ablaut und ähnliche Phänomene) oder wie Modalverben (Präteritalpräsentien) flektieren.

\section{Analytische Verbformen}

Bilden Sie die genannten Formen der angegebenen Verben.
Segmentieren Sie sie Formen dabei mit Bindestrichen nach der Konvention aus EGBD3.
Wenn nicht Konj oder Inf angegeben ist, soll der Indikativ gebildet werden.
Wenn nicht Pass angegeben ist, soll der Aktiv gebildet werden.
Die Abkürzungen sind:

\begin{itemize}\Lf
  \item Tempus | Präs, Prät
  \item Quasitempus | Perf
  \item Infinitiv | Inf
  \item Modus | Konj
  \item Person | P1, P2, P3
  \item Numerus | Sg, Pl
  \item Diathese | Pass
\end{itemize}

\begin{center}
  \begin{tabular}[h]{cllp{0.55\textwidth}}
    \toprule
    & \textbf{Verb} & \textbf{zu bildende Form} & \textbf{Form} \\
    \midrule
    &&& \\
    (1) & \textit{raufen} & Fut Perf P2 Pl & \\\cline{4-4}
    &&& \\
    (2) & \textit{singen} & Prät P3 Sg & \\\cline{4-4}
    &&& \\
    (3) & \textit{liegen} & Konj Präs P3 Sg & \\\cline{4-4}
    &&& \\
    (4) & \textit{verschenken} & Inf Perf Pass & \\\cline{4-4}
    &&& \\
    (5) & \textit{rennen} & Inf Perf & \\\cline{4-4}
    &&& \\
    (6) & \textit{müssen} & Konj Prät P2 Pl & \\\cline{4-4}
    &&& \\
    (7) & \textit{begrüßen} & Fut Perf P2 Pl Pass & \\\cline{4-4}
  \end{tabular}
\end{center}

\section{Konjunktiv}

Versuchen Sie, den nachstehenden Text zunächst in den Konjunktiv 1 und dann in den Konjunktiv 2 zu setzen.
Die Ersetzungsregeln zur Vermeidung von Formähnlichkeiten sind:

\begin{enumerate}\Lf
  \item Wenn die Form des Konj 1 nicht von der Form des Ind Präs zu unterscheiden ist, wird der Konj 2 genommen.
  \item Wenn die Form des Konj 2 nicht von der Form des Ind Prät zu unterscheiden ist, wird die analytische \textit{würde}-Form genommen.
\end{enumerate}

Diskutieren Sie, welche Formen trotz der Ersatzregeln grundsätzlich Probleme machen.

\begin{quote}
  Die Grammatik folgt Regeln, und sie folgte schon immer Regeln.
  Nur das kann der Grund sein, dass wir einander verstehen, wenn wir Sprache benutzen.
  Die Mathematik ist axiomatisch eingeführt worden.
  Sie gehorcht damit ausnahmslosen Regeln, während die Regeln der Grammatik Ausnahmen zulassen.
\end{quote}

% Die Grammatik folge Regeln, und sie !sei schon immer Regeln !gefolgt.
% Nur das könne der Grund sein, dass einender #verstünden, wenn wir Sprache #benutzten.
% Die Mathematik sei axiomatisch eingeführt worden.
% Sie gehorche damit ausnahmslosen Regeln, während die Regeln der Grammatik Ausnahmen zuließen.

% Die Grammatik §würde Regeln §folgen, und sie !wäre schon immer Regeln !gefolgt.
% Nur das könnte der Grund sein, dass wir einander verstünden, wenn wir Sprache §benutzen §würden.
% Die Mathematik wäre axiomatisch eingeführt worden.
% Sie würde damit ausnahmlosen Regeln folgen, währemd die Regeln der Grammatik Ausnahmen zuließen.

\end{document}
