\section{Funktion in der Flexion}

\subsection{Nominalflexion}

\begin{frame}
  {Was heißt Funktion?}
  \pause
  Rückgriff auf Kapitel 3:
  \pause
  \Halbzeile
  \begin{itemize}[<+->]
    \item \alert{externe} Funktion: kommunikativ, pragmatisch, textuell, kulturell, \dots
    \item \alert{interne} Funktion: innerhalb der Grammatik Relationen kennzeichnend,
      Rekonstruktion der Struktur ermöglichend, Schnittstelle zur Semantik: \rot{Kompositionalität}
    \item nicht immer trennbar
      \Halbzeile
    \item Paradebeispiel für interne Funktion: \alert{Kasussystem}
  \end{itemize}
\end{frame}

\begin{frame}
  {Numerus}
  \pause
  \begin{exe}
    \ex
    \begin{xlist}
      \ex[ ]{Die Trainerin beobachtet [einen guten Wettkampf].}
      \pause
      \ex[*]{Die Trainerin beobachtet [einen guten \rot{Wettkämpfe}].}
    \end{xlist}
    \pause
    \ex
    \begin{xlist}
      \ex[ ]{Die Trainerin beobachtet [einige gute Wettkämpfe].}
      \pause
      \ex[*]{Die Trainerin beobachtet [einige gute \rot{Wettkampf}].}
    \end{xlist}
  \end{exe}
  \pause
  \Halbzeile
  \begin{itemize}[<+->]
    \item \alert{Anzahl von Objekten ("`Gegenständen"')}: konzeptuell beim Subst motiviert
    \item notwendigerweise volatiles Merkmal beim Subst
    \item Pluraliatantum wie \textit{Ferien} oder Singulariatantum wie \textit{Gesundheit}
  \end{itemize}
\end{frame}

\begin{frame}
  {Kasus}
  \pause
  Was ist Kasus? Haben die Kasus an sich eine Bedeutung?
  \Halbzeile
  \pause
  \begin{exe}
    \ex
    \begin{xlist}
      \ex{Wir sehen \rot{den Rasen}.}
      \pause
      \ex{Wir begehen \rot{den Rasen}.}
      \pause
      \ex{Wir säen \rot{den Rasen}.}
      \pause
      \ex{Wir fürchten \rot{uns}.}
    \end{xlist}
    \pause
    \ex
    \begin{xlist}
      \ex \rot{Nächsten März} fahre ich zum Bergwandern in die Tatra.
      \ex Es waren \rot{den ganzen Tag} Menschen zum Gipfel unterwegs.
    \end{xlist}
    \pause
    \ex
    \begin{xlist}
      \ex{Sarah backt \rot{ihrer Freundin} einen Marmorkuchen.}
      \pause
      \ex{Wir kaufen \rot{dir} ein Kilo Rohrzucker.}
      \pause
      \ex{Die Mannschaft spielt \rot{mir} zu drucklos.}
      \pause
      \ex{Der Marmorkuchen schmeckt \rot{den Freundinnen} gut.}
    \end{xlist}
  \end{exe}
\end{frame}


\begin{frame}
  {Kasus: Eigenschaften}
  \pause
  \centering

  \Large
  Kasus stellt \alert{Relationen zwischen\\
  den kasustragenden Nomina und anderen Wörtern}\\
  (Verben, Präpositionen, anderen Nomina) her.\\
\end{frame}

\begin{frame}
  {Person: Deixis}
  \pause
  Was ist die grammatische Person?

  \Halbzeile
  \pause
  \begin{exe}
    \ex
    \begin{xlist}
      \ex{\alert{Ich} unterstütze den FCR Duisburg.}
      \pause
      \ex{\alert{Ihr} unterstützt den FCR Duisburg.}
      \pause
      \ex{\alert{Sie/Diese/Jene/Eine/Man\ldots} unterstützt den FCR Duisburg.}
      \pause
      \ex{\alert{Sie/Diese/Jene/Einige/\ldots} unterstützen den FCR Duisburg.}
    \end{xlist}
  \end{exe}
  \pause
  \Halbzeile
  \begin{itemize}[<+->]
    \item prototypisch beim \alert{Pronomen} funktional motiviert
    \item Substantive: statisch dritte Person
      \Halbzeile
    \item hier: \rot{deiktische Pronomina}
      \begin{itemize}[<+->]
        \item in einer Situation verweisend
        \item nur relativ zu einer Situation interpretierbar
      \end{itemize} 
  \end{itemize}
\end{frame}

\begin{frame}
  {Person: Anaphorik}
  \pause
  \begin{exe}
    \ex \alert{Sarah$_{\textnormal{1}}$} backt \rot{[ihrer Freundin]$_{\textnormal{2}}$} \gruen{[einen Kuchen]$_{\textnormal{3}}$}.\\
      \alert{Sie$_{\textnormal{1}}$} verwendet nur fair gehandelten unraffinierten Rohrzucker.
    \pause
      \ex \alert{Sarah$_{\textnormal{1}}$} backt \rot{[ihrer Freundin]$_{\textnormal{2}}$} \gruen{[einen Kuchen]$_{\textnormal{3}}$}.\\
      \gruen{Er$_{\textnormal{3}}$} besteht nur aus fair gehandelten Zutaten.
    \pause
      \ex \alert{Sarah$_{\textnormal{1}}$} backt \rot{[ihrer Freundin]$_{\textnormal{2}}$} \gruen{[einen Kuchen]$_{\textnormal{3}}$}.\\
      \rot{Sie$_{\textnormal{2}}$} soll \gruen{ihn$_{\textnormal{3}}$} zum Geburtstag geschenkt bekommen.
  \end{exe}
  \Halbzeile
  \pause
  \begin{itemize}[<+->]
    \item anaphorische Pronomina
    \item Rückverweis im Text, Satz, Diskurs
    \item gleiche Indizes zeigen Bedeutungsidentität: Korreferenz
  \end{itemize}
\end{frame}

\begin{frame}
  {Genus, Geschlecht, Gender?}
  \pause
  \begin{exe}
    \ex \label{ex:genus039}
    \begin{xlist}
      \ex \alert{Die Petunie} ist \orongsch{eine Blume}.
      \ex \rot{Der Enzian} ist \orongsch{eine Blume}.
      \ex \gruen{Das Veilchen} ist \orongsch{eine Blume}.
    \end{xlist}
  \end{exe}
  \pause
  \Halbzeile
  \begin{itemize}[<+->]
    \item reine Subklassenbildung beim Substantiv
    \item nicht in Geschlecht oder Gender motiviert
    \item tendentiell Korrespondenz von maskulin und männlich\\
      sowie feminin und weiblich bei Menschen bzw.\ Lebewesen
  \end{itemize}
\end{frame}

\subsection{Verbalflexion}

\begin{frame}
  {Numerus und Person bei Verben}
  \pause
  \begin{itemize}[<+->]
    \item wie gezeigt wurde: \alert{Numerus} und \alert{Person}\\
      im Bereich der Nomina motiviert
    \item Numerus und Person bei Verben: Subjekt-Verb-Kongruenz 
      \Halbzeile
    \item Kongruenz:
      \begin{itemize}[<+->]
        \item reine \alert{Übereinstimmung von Werten}
        \item \rot{beide Einheiten} haben das Merkmal
        \item \alert{Kongruenz zwischen Nomina}: \textit{der schöne Kaftan}
        \item \alert{Subjekt-Verb-Kongruenz}: \textit{Ich schwafle.}
      \end{itemize}
  \end{itemize}
\end{frame}

\begin{frame}
  {Tempus: synthetisch vs.\ analytisch}
  \pause
  Die klassischen "`Tempusformen"' des Deutschen:\\
  \Halbzeile
  \pause
  \begin{center}
    \scalebox{0.75}{\begin{tabular}{ll}
      \toprule
      \textbf{Tempus} & \textbf{Beispiel 3.~Person}\\
      \midrule
      Präsens & \alert<4->{lacht} \\
      Präteritum & \alert<4->{lachte} \\
      Perfekt & hat gelacht \\
      Plusquamperfekt & hatte gelacht \\
      Futur & wird lachen \\
      Futurperfekt & wird gelacht haben \\
      \bottomrule
    \end{tabular}}
  \end{center}
  \pause
  \Halbzeile
  \begin{itemize}[<+->]
    \item \alert{Ganz offensichtlich hat das Deutsche nur zwei Tempusformen\\
      im morphologischen Sinn.}
  \end{itemize} 
\end{frame}

\begin{frame}
  {Funktion: einfache Tempora}
  \pause
  \alert{Präsens: Ereignis- und Sprechzeitpunkt unabhängig}
  \pause
  \begin{exe}\ex\begin{xlist}
      \ex Im Jahr 1961 \alert{beginnt} die DDR mit dem Bau der Mauer.
      \pause
      \ex Morgen \alert{esse} ich Maronen.
      \pause
      \ex Heute \alert{ist} Mittwoch, und donnerstags \alert{kommt} die Müllabfuhr.
  \end{xlist}\end{exe}
  \pause
  \Halbzeile
  \alert{Präteritum: Ereignis- vor Sprechzeitpunkt}
  \pause
  \begin{exe}\ex\begin{xlist}
      \ex Es \alert{klingelte} an der Tür.
      \pause
    \ex Jetzt \alert{klingelte} es an der Tür.
      \pause
    \ex Die Hethiter \alert{wurden} aus Anatolien vertrieben.
  \end{xlist}\end{exe}
  \pause
  \Halbzeile
  \alert{Futur: Sprech- vor Ereigniszeitpunkt}
  \pause
  \begin{exe}\ex\begin{xlist}
      \ex Ich \alert{werde} einen Rottweiler \alert{adoptieren}.
      \pause
      \ex Viele Verstärker \rot{werden} von mir noch \alert{repariert} \rot{werden}.
  \end{xlist}\end{exe}
\end{frame}

\begin{frame}
  {Funktion: komplexe Tempora}
  \pause
  Zusätzlicher Bezug auf einen Referenzzeitpunkt!\\
  \Zeile
  \pause
  \alert{Futurperfekt: Sprech- und Ereigniszeit vor Referenzzeit}
  \pause
  \begin{exe}
    \ex In zwei Jahren \alert{wird} Merkel \alert{abgedankt haben}.
    \pause
    \ex Im Jahr 2010 \alert{wird} Helmut Schmidt \alert{abgedankt haben}.
  \end{exe}
  \pause
  \Zeile
  \alert{Plusquamperfekt: Referenz- vor Sprechzeit, Ereignis- vor Referenzzeit}
  \pause
  \begin{exe}
    \ex Frida nahm das Buch in die Hand. Sie \alert{hatte} es bereits \alert{gelesen}.
      \pause
    \ex Frida legte das Buch weg, nachdem sie es \alert{gelesen hatte}.
  \end{exe}
\end{frame}

\begin{frame}
  {Modus: Grade der Faktizität}
  \pause
  \alert{Indikativ}, \orongsch{Konjunktiv I}, \rot{Konjunktiv II}:
  \pause
  \small
  \begin{exe}
    \ex
    \begin{xlist}
      \ex[]{Sie sagte, der Kuchen \alert{schmeckt} lecker.}
      \ex[]{Sie sagte, der Kuchen \orongsch{schmecke} lecker.}
      \ex[]{Sie sagte, dass der Kuchen lecker \alert{schmeckt}.}
      \ex[]{Sie sagte, dass der Kuchen lecker \orongsch{schmecke}.}
    \end{xlist}
    \pause
    \ex
    \begin{xlist}
      \ex[]{Wenn das \alert{geschieht}, \alert{laufe} ich weg.}
      \ex[]{Immer, wenn das \alert{geschieht}, \alert{laufe} ich weg.}
      \ex[]{Wenn das \rot{geschähe}, \rot{liefe} ich weg.}
      \ex[*]{Immer, wenn das \rot{geschähe}, \rot{liefe} ich weg.}
    \end{xlist}
    \pause
    \ex
    \begin{xlist}
      \ex[]{Ohne Schnee \alert{sind} die Ferien diesmal nicht so schön.}
      \ex[]{Ohne Schnee \rot{wären} die Ferien diesmal nicht so schön.}
    \end{xlist}
    \pause
    \ex
    \begin{xlist}
      \ex[]{Im Urlaub \alert{hat} kein Schnee gelegen.}
      \ex[]{Ach, \rot{hätte} im Urlaub doch Schnee gelegen.}
    \end{xlist}
  \end{exe}
\end{frame}

\begin{frame}
  {Warum gehört Genus Verbi hier nicht hin?}
  \pause
  \begin{exe}
    \ex
    \begin{xlist}
      \ex \rot{Frida} \alert{isst} \orongsch{den Kuchen}.
      \pause
      \ex \orongsch{Der Kuchen} \alert{wird} \alert{gegessen}.
      \pause
      \ex \orongsch{Der Kuchen} \alert{wird} \rot{von Frida} \alert{gegessen}.
    \end{xlist}
  \end{exe}
  \pause
  \Zeile
  \begin{itemize}[<+->]
    \item \rot{keine Flexion} (wie analytische Tempora)
    \item eigentlich eine \alert{lexikalische} Änderung am Verb\\
      (Valenzänderung und Partizipform, s.\ ca.\ Woche 11)
  \end{itemize}
\end{frame}


