\section{Überblick}

\begin{frame}
  {Morphologie | Flexion und Wortbildung}
  \pause
  \begin{itemize}[<+->]
    \item \alert{Formveränderungen} und \alert{Merkmalsänderungen}
      \begin{itemize}[<+->]
        \item Veränderungen von Werten
        \item Veränderungen von Merkmalsaustattungen
      \end{itemize}
      \Halbzeile
    \item Morphe (= Wortbestandteile) und ihre Funktionen
    \item Morphe | alle Stämme und alle nicht-lexikalischen Morphe
      \Halbzeile
    \item statische und volatile Merkmale
    \item Wortbildung vs.\ Flexion, definiert anhand von Merkmalen
      \Halbzeile
    \item Syntax und Morphologie
    \item Phrasenbestimmung
    \item Köpfe
    \item Nominalphrasen und Präpositionalphrasen
  \end{itemize}
\end{frame}


\section{Stämme und Affixe}

\begin{frame}
  {Form und Funktion | Flexion}
  \pause
  \begin{exe}
    \ex
    \begin{xlist}
      \ex \alert{Den Präsidenten} begrüßte \alert{der Dekan} äußerst respektlos.
      \pause
      \ex \alert{Der Dekan} begrüßte \alert{den Präsidenten} äußerst respektlos.
    \end{xlist}
    \pause
    \ex
    \begin{xlist}
      \ex \alert{Die Präsidentin} begrüßte \alert{die Dekanin} äußerst respektlos.
      \pause
      \ex \alert{Die Dekanin} begrüßte \alert{die Präsidentin} äußerst respektlos.
    \end{xlist}
  \end{exe}
  \pause
  \Zeile
  Formveränderungen lexikalischer Wörter \alert{schränken ihre möglichen grammatischen Funktionen und Relationen im Satz ein}\dots\\
  \pause
  \Halbzeile
  \dots und sie haben semantische und systemexterne Folgen.

\end{frame}

\begin{frame}
  {Form und Funktion | Wortbildung}
  \pause
  \begin{exe}
    \ex grün\alert{lich}, röt\alert{lich}, gelb\alert{lich}
    \pause
    \ex Neu\alert{igkeit}, Blöd\alert{heit}, Tauch\alert{er}, Heb\alert{ung}
    \pause
    \ex Fenster\alert{rahmen}, Tücher\alert{spender}, Glas\alert{korken}, Unter\alert{schrank}
  \end{exe}
  \pause
  \Zeile
  Formveränderungen von einem zu einem anderen lexikalischen Wort führen zu Bedeutungs- und kategorialen Veränderungen.
\end{frame}

\begin{frame}
  {Markierungsfunktionen von Morphen I}
  \pause
  \begin{exe}
    \ex
    \begin{xlist}
      \ex{(der) \alert<4>{Berg}}
      \ex{(den) \alert<4>{Berg}}
      \ex{(dem) \alert<4>{Berg}}
      \ex{(des) \alert<5>{Berg}\rot<5>{-es}}
      \ex{(die) \alert<6>{Berg}\rot<6>{-e}}
      \ex{(der) \alert<6>{Berg}\rot<6>{-e}}
    \end{xlist}
    \pause
    \ex
    \begin{xlist}
      \ex{(der) \alert<8>{Mensch}}
      \ex{(den) \alert<9>{Mensch}\rot<9>{-en}}
      \ex{(dem) \alert<9>{Mensch}\rot<9>{-en}}
      \ex{(des) \alert<9>{Mensch}\rot<9>{-en}}
      \ex{(die) \alert<9>{Mensch}\rot<9>{-en}}
      \ex{(der) \alert<9>{Mensch}\rot<9>{-en}}
    \end{xlist}
  \end{exe}
\end{frame}

\begin{frame}
  {Markierungsfunktionen von Morphen II}
  \pause
  \begin{exe}
    \ex
    \begin{xlist}
      \ex{(ich) \alert<3>{kauf}\rot<3>{-e}}
      \ex{(du) \alert<4>{kauf}\rot<4>{-st}}
      \ex{(wir) \alert<5>{kauf}\rot<5>{-en}}
      \ex{(sie) \alert<5>{kauf}\rot<5>{-en}}
    \end{xlist}
  \end{exe}
\end{frame}

\begin{frame}
  {Morphe und Markierungsfunktionen}
  \pause
  \begin{itemize}[<+->]
    \item Formveränderungen
      \begin{itemize}[<+->]
        \item oft nicht \alert{eine} Funktion
        \item \rot{Einschränkung} der möglichen Funktionen
      \end{itemize}
   \Halbzeile 
    \item \alert{Markierungsfunktion} | eine \alert{Einschränkung}\\
      der möglichen Merkmale oder Werte einer Wortform
    \item zum Beispiel \textit{-en} bei schw.\ Maskulina | \rot{nicht} Nominativ Singular
    \item oder \textit{-en} bei Verben im Präsens | Plural und nicht adressatbezogen
      \Halbzeile
    \item \alert{Morphe | alle segmentalen Einheiten mit Markierungsfunktion}
    \item \alert{Stämme} und \alert{Affixe}
  \end{itemize}
\end{frame}

\begin{frame}
  {Stämme I}
  \pause
  \begin{exe}
    \ex
    \begin{xlist}
      \ex{(ich) \alert<5->{kauf}-e\\
        (du) \alert<5->{kauf}-st\\
        (ihr) \alert<5->{kauf}-t }
        \pause
        \ex{(ich) \alert<6->{kauf}-te\\
        (du) \alert<6->{kauf}-test\\
        (ihr) \alert<6->{kauf}-tet}
        \pause
        \ex{(ich habe) ge-\alert<7->{kauf}-t\\
        (du hast) ge-\alert<7->{kauf}-t\\
        (ihr habt) ge-\alert<7->{kauf}-t}
    \end{xlist}
  \end{exe}
\end{frame}

\begin{frame}
  {Stämme II}
  \begin{exe}
    \ex
    \begin{xlist}
      \ex{(ich) \alert<4->{nehm}-e\\
        (du) \rot<5->{nimm}-st\\
          (es) \rot<5->{nimm}-t\\
          (ihr) \alert<4->{nehm}-t}
        \pause
        \ex{(ich) \orongsch<6->{nahm}\\
        (du) \orongsch<6->{nahm}-st\\
          (ihr) \orongsch<6->{nahm}-t}
        \pause
        \ex{(ich habe) ge-\gruen<7->{nomm}-en\\
        (du hast) ge-\gruen<7->{nomm}-en\\
        (ihr habt) ge-\gruen<7->{nomm}-en}
    \end{xlist}
  \end{exe}
  \pause
  \pause
  \pause
  \pause
  \pause
  Der \alert{Stamm} kann nicht "`der unveränderliche Wortbestandteil"'\\
  eines lexikalischen Wortes (in einem Paradigma) sein.\\
  \Zeile
  \pause
  \alert{\dots aber der mit der Bedeutung, also der lexikalischen Markierungsfunktion}!
\end{frame}

\begin{frame}
  {Affixe}
  \pause
  \begin{exe}
    \ex
    \begin{xlist}
      \ex (ich) nehm\alert<6->{-e}
      \pause
      \ex (des) Berg\alert<7->{-es}
      \pause
      \ex Schön\alert<8->{-heit}
      \pause
      \ex \alert<9->{Un-}ding
    \end{xlist}
  \end{exe}
  \Zeile
  \pause
  \pause
  \pause
  \pause
  \pause
  \begin{itemize}[<+->]
    \item \alert{keine lexikalische Markierungsfunktion} (= keine eigene Bedeutung)
    \item \alert{nicht wortfähig} = nicht ohne Stamm verwendbar
  \end{itemize}
\end{frame}



\section{Merkmale in Flexion und Wortbildung}

\begin{frame}
  {Statische und volatile Merkmale}
  \pause
  \begin{itemize}[<+->]
    \item Eigenschaften | "`Rotsein"' (Erdbeere), "`325m hoch"' (Eiffelturm) usw.
    \item Merkmale | \alert{\textsc{Farbe}}, \alert{\textsc{Länge}} usw.
    \item Werte
      \begin{itemize}[<+->]
        \item \alert{\textsc{Farbe}}: \rot{\textit{rot}}, \rot{\textit{grau}}, \ldots
        \item \alert{\textsc{Länge}}: \rot{\textit{3cm}}, \rot{\textit{325m}}, \ldots
      \end{itemize}
  \end{itemize}
  \pause
  \Halbzeile 
  \begin{exe}
    \ex
    \begin{xlist}
      \ex{Haus = [\textsc{Bed}: \gruen<12->{\textbf{\textit{haus}}}, \textsc{Klasse}: \gruen<12->{\textbf{\textit{subst}}}, \textsc{Gen}: \gruen<12->{\textbf{\textit{neut}}}, \textsc{Kas}: \orongsch<13->{\textit{nom}}, \textsc{Num}: \orongsch<13->{\textit{sg}}]}
      \pause
      \ex{Haus-es = [\textsc{Bed}: \gruen<12->{\textbf{\textit{haus}}}, \textsc{Klasse}: \gruen<12->{\textbf{\textit{subst}}}, \textsc{Gen}: \gruen<12->{\textbf{\textit{neut}}}, \textsc{Kas}: \orongsch<13->{\textit{gen}}, \textsc{Num}: \orongsch<13->{\textit{sg}}]}
      \pause
      \ex{Häus-er = [\textsc{Bed}: \gruen<12->{\textbf{\textit{haus}}}, \textsc{Klasse}: \gruen<12->{\textbf{\textit{subst}}}, \textsc{Gen}: \gruen<12->{\textbf{\textit{neut}}}, \textsc{Kas}: \orongsch<13->{\textit{nom}}, \textsc{Num}: \orongsch<13->{\textit{pl}}]}
    \end{xlist}
  \end{exe}
  \Halbzeile
  \pause
  \begin{itemize}[<+->]
    \item bei einem lexikalischen Wort
      \begin{itemize}
        \item \gruen{statische Merkmale} wertestabil
        \item \orongsch{volatile Merkmale} werteverändernd im Paradigma
      \end{itemize}
  \end{itemize}
\end{frame}

\begin{frame}
  {Wortbildung in Abgrenzung zur Flexion}
  \pause
  \begin{exe}
    \ex
    \begin{xlist}
      \ex trocken (Adj) → \alert{Trocken}\rot{-heit} (Subst)\label{ex:trocken}
      \ex Kauf (Subst), Rausch (Subst) → \alert{Kauf}\rot{-rausch} (Subst)\label{ex:kauf}
      \ex gehen (V) → \alert{be}\rot{-gehen} (V)\label{ex:gehen}
    \end{xlist}
    \pause
    \ex
    \begin{xlist}
      \ex \alert{lauf}\rot{-en} (1\slash 3 Pl Prs Ind) → \alert{lauf}\rot{-e} (1 Sg Prs Ind)\label{ex:lauf}
      \ex \alert{Münze} (Sg) → \alert{Münze}\rot{-n} (Pl)\label{ex:muenze}
    \end{xlist}
  \end{exe}
  \pause
  \Halbzeile
  \begin{itemize}[<+->]
    \item Wortbildung
      \begin{itemize}[<+->]
        \item statische Merkmale geändert | Wortklasse, Bedeutung \alert{(\ref{ex:trocken})}
        \item \ldots oder gelöscht | alles außer der Bedeutung des Erstglieds bei Komposition \alert{(\ref{ex:kauf})}
        \item \ldots oder umgebaut | Valenz von Verben beim Applikativ \alert{(\ref{ex:gehen})}
        \item \orongsch{produktives Erschaffen neuer lexikalischer Wörter}
      \end{itemize}
  \Halbzeile
    \item Flexion
      \begin{itemize}
        \item Änderung der Werte volatiler Merkmale \alert{(\ref{ex:lauf},\ref{ex:muenze})}
        \item \alert{oft Anpassung an syntaktischen Kontext}
      \end{itemize}
  \end{itemize}
\end{frame}


\section{Konstituenten}


\begin{frame}
  {Es gibt keine \orongsch{reine Morphologie}}
  \onslide<+->
  \onslide<+->
  Ebenen der Grammatik\\
  \Viertelzeile
  \begin{itemize}[<+->]
    \item \alert{Phonologie} | Kombinatorik von Lauten, Silben, Betonung (Akzent) usw.
    \item \alert{Morphologie} | Kombinatorik von Wortbestandteilen und deren Funktionen
    \item \alert{Syntax} | Kombinatorik von Wörtern, Wortgruppen und Sätzen
    \item \alert{Semantik} | Ableitung von Bedeutungen aus der formalen Kombinatorik
  \end{itemize}
  \onslide<+->
  \Zeile
  Einige Interaktionen und Schnittstellen\\
  \Viertelzeile
  \begin{itemize}[<+->]
    \item \orongsch{Lexik} | Klassifikation von Wörtern nach \alert{grammatischen Merkmalen}
    \item \orongsch{Morphophonologie} | Beschränkungen der \alert{Morphologie} aufgrund der \alert{Phonologie} 
    \item \orongsch{Morphosyntax} | Schnittstelle von Morphologie und Syntax (Kasus, Numerus, Valenz)
    \item \orongsch{Syntax-Semantik-Morphologie-Lexik-Schnittstelle} | Passive, Infinitivsyntax usw.
  \end{itemize}
  \onslide<+->
  \Halbzeile
  \rot{→ Wir brauchen ein minimales (Schul-)Wissen über Syntax in der Morphologie.}
\end{frame}

\begin{frame}
  {Sprachliche Einheiten und ihre Bestandteile}
  \onslide<+->
  \onslide<+->
  Wichtig vor allem für die Syntax | \alert{Strukturbildung}\\
  \Zeile
  \begin{itemize}[<+->]
    \item\footnotesize \alert{Satz} \\
      {Nadezhda reißt die Hantel souveräner als andere Gewichtheberinnen.}
      \Halbzeile

    \item\footnotesize \alert{Satzteile} \\
      {Nadezhda | reißt | die Hantel | souveräner als andere Gewichtheberinnen}
      \Halbzeile

    \item\footnotesize \alert{Wörter} \\
      {Nadezhda | reißt | die | Hantel | souveräner | als | andere | Gewichtheberinnen}
      \Halbzeile

    \item\footnotesize \alert{Wortteile} \\
      {Nadezhda | reiß | t | d | ie | Hantel | souverän | er | als | ander | e | Gewicht | heb | er | inn | en}
      \Halbzeile

    \item\footnotesize \alert{Laute\slash Buchstaben} \\
      {N | a | d | e | z | h | d | a \ldots}
  \end{itemize}
\end{frame}


\begin{frame}
  {Syntaktische Strukturen und morphologische Merkmale}
  \onslide<+->
  \onslide<+->
  \begin{center}
  \resizebox{0.8\textwidth}{!}{\begin{forest}
    [Nadezhda reißt die Hantel souveräner als andere Gewichtheberinnen
      [Nadezhda]
      [reißt]
      [die Hantel, alt=<3->{orongsch}{}
        [die, alt=<4->{orongsch}{}]
        [Hantel, alt=<5->{orongsch}{}]
      ]
      [souveräner als andere Gewichtheberinnen
        [souveräner]
        [als andere Gewichtheberinnen
          [als]
          [andere Gewichtheberinnen, alt=<6->{gruen}{}
            [andere, alt=<7->{gruen}{}]
            [Gewichtheberinnen, alt=<8->{gruen}{}]
          ]
        ]
      ]
    ]
  \end{forest}}
  \end{center}

  \Zeile
  \onslide<9->{Übereinstimmung von Merkmalen in syntaktischen Gruppen\\}
  \onslide<10->{\orongsch{Akkusativ Femininum Singular}} \onslide<11->{| \gruen{Nominativ Plural}}
\end{frame}

\begin{frame}
  {Morphologie und Syntax | I}
  \onslide<+->
  \onslide<+->
  \alert{Kongruenz} | Merkmalübereinstimmung in Nominalphrasen\\
  \Zeile
  \Zeile
  \centering 
  \onslide<+->
  \begin{tikzpicture}[node distance=1.5cm, auto]
    \node (context) {Wir möchten};
    \node [right=of context] (diesen) {\alert<5->{diesen}};
    \node[right=of diesen] (schönen) {\alert<4->{schönen}};
    \node[right=of schönen] (Sportwagen) {\alert<4->{Sportwagen}};
    \onslide<4->{\path[<->, trueblue, draw, bend left=30] (schönen) edge node {\scriptsize Akk Mask Sg} (Sportwagen);}
    \onslide<5->{\path[<->, trueblue, draw, bend left=60] (diesen) edge node[above] {\scriptsize Akk Mask Sg} (Sportwagen);}
    \onslide<6->{\path[<->, trueblue, draw, bend left=30] (diesen) edge node[above] {\scriptsize Akk Mask Sg} (schönen);}
  \end{tikzpicture}
\end{frame}

\begin{frame}
  {Morphologie und Syntax | II}
  \onslide<+->
  \onslide<+->
  \alert{Kongruenz} | Merkmalübereinstimmung zwischen Subjekt und finitem Verb\\
  \Zeile
  \Zeile
  \centering 
  \onslide<+->
  \begin{tikzpicture}[node distance=1cm, auto]
   \node                      (context)    {Ich glaube, dass};
   \node[right=of context]    (ihr)        {\alert<4->{ihr}};
   \node[right=of ihr]        (den)        {den};
   \node[right=of den]        (Wagen)      {Wagen};
   \node[right=of Wagen]      (anschieben) {anschieben};
   \node[right=of anschieben] (müsst)      {\alert<4->{müsst}};
   \onslide<4->{\path[<->, trueblue, draw, bend left=30]  (ihr) edge node {\small 2.~Per Pl} (müsst);}
  \end{tikzpicture}
\end{frame}

\begin{frame}
  {Morphologie und Syntax | III}
  \onslide<+->
  \onslide<+->
  \gruen{Rektion} | Präpositionen bestimmen den Kasus von ganzen \alert{Nominalphrasen}\\
  \Zeile
  \Zeile
  \centering 
  \onslide<+->
  \begin{tikzpicture}[node distance=1cm, auto]
   \node                      (context)    {Wir fahren};
   \node[right=of context]    (mit)        {\gruen<4->{mit}};
   \node[right=of mit]        (dem)        {\alert<6->{dem}};
   \node[right=of dem]        (neuen)      {\alert<5->{neuen}};
   \node[right=of neuen]      (Wagen)      {\alert<4->{Wagen}};
   \node[right=of Wagen]      (rest)       {nach hause};
   \onslide<4->{\path[->, gruen, draw, bend right=30]  (mit) edge node[below] {Dat} (Wagen);}
   \onslide<5->{\path[<->, trueblue, draw, bend left=30]  (neuen) edge node {\footnotesize Dat Mask Sg} (Wagen);}
   \onslide<6->{\path[<->, trueblue, draw, bend left=30]  (dem) edge node {\footnotesize Dat Mask Sg} (neuen);}
  \end{tikzpicture}
\end{frame}

\begin{frame}
  {Morphologie und Syntax | IV}
  \onslide<+->
  \onslide<+->
  \gruen{Rektion} | Verben bestimmen den Kasus von ganzen \alert{Nominalphrasen}\\
  \Zeile
  \Zeile
  \centering 
  \onslide<+->
  \begin{tikzpicture}[node distance=1cm, auto]
    \node                      (Nom)        {\alert<4->{Ich}};
    \node[right=of Nom]        (V)          {\gruen<4->{gab}};
    \node[right=of V]          (Dat)        {\alert<5->{dem netten Kollegen}};
    \node[right=of Dat]        (Akk)        {\alert<6->{den Stift}};
    \node[right=of Akk]        (rest)       {zurück};
    \onslide<4->{\path[->, gruen, draw, bend right=-30]  (V) edge node[below] {Nom} (Nom);}
    \onslide<5->{\path[->, gruen, draw, bend right=30]  (V) edge node[below] {Dat} (Dat);}
    \onslide<6->{\path[->, gruen, draw, bend right=60]  (V) edge node[below] {Akk} (Akk);}
  \end{tikzpicture}
\end{frame}

\begin{frame}
  {Phrasenbestimmung}
  \onslide<+->
  \onslide<+->
  \alert{Konstituenten} | Bestandteile irgendeiner Struktur\\
  \Halbzeile
  \onslide<+->
  \alert{Phrasen} | syntaktische Konstituenten mit bestimmten Eigenschaften\\
  \Zeile
  \begin{itemize}[<+->]
    \item Phrasenbestimmung | ähnlich \alert{Satzgliedanalyse} aus der Schule
    \item \alert{Tests} auf Phrasenstatus
    \item Unsicherheiten trotz Tests
  \end{itemize}
\end{frame}

\begin{frame}
  {Pronominalisierungstest}
  \pause
  \begin{exe}
    \ex Mausi isst \alert<3->{den leckeren Marmorkuchen}.\\
    \pause
      \KTArr{PronTest} Mausi isst \alert{ihn}.
    \pause
    \ex{\label{ex:konstituententests025} \rot<5->{Mausi isst} den Marmorkuchen.\\
    \pause
      \KTArr{PronTest} \Ast \rot{Sie} den Marmorkuchen.}
    \pause
    \ex{\label{ex:konstituententests026} Mausi isst \alert<7->{den Marmorkuchen und das Eis mit Multebeeren}.\\
    \pause
    \KTArr{PronTest} Mausi isst \alert{sie}.}
  \end{exe}
  \pause
  \Halbzeile
  Pronominalausdrücke i.\,w.\,S.
  \begin{exe}
    \ex{\label{ex:konstituententests027} Ich treffe euch \alert<9->{am Montag} \gruen<10->{in der Mensa}.\\
    \pause
    \KTArr{PronTest} Ich treffe euch \alert{dann} \gruen<10->{dort}.}
      \pause
      \pause
      \ex{\label{ex:konstituententests028} Er liest den Text \alert<12->{auf eine Art, die ich nicht ausstehen kann}.\\
      \pause
      \KTArr{PronTest} Er liest den Text \alert{so}.}
  \end{exe}
\end{frame}

\begin{frame}
  {Vorfeldtest\slash Bewegungstest}
  \pause
  \begin{exe}
    \ex
    \begin{xlist}
      \ex{Sarah sieht den Kuchen \alert<3->{durch das Fenster}.\\
        \pause
        \KTArr{VfTest} \alert{Durch das Fenster} sieht Sarah den Kuchen.}
      \pause
      \ex{Er versucht \alert{zu essen}.\\
        \pause
        \KTArr{VfTest} \alert<5->{Zu essen} versucht er.}
      \pause
      \ex{Sarah möchte gerne \alert{einen Kuchen backen}.\\
        \pause
        \KTArr{VfTest} \alert<7->{Einen Kuchen backen} möchte Sarah gerne.}
      \pause
      \ex{Sarah möchte \rot<9->{gerne einen} Kuchen backen.\\
        \pause
        \KTArr{VfTest} \Ast \rot{Gerne einen} möchte Sarah Kuchen backen.}
    \end{xlist}
  \end{exe}
  \pause
  \Halbzeile
  verallgemeinerter "`Bewegungstest"'\\
  \begin{exe}
    \ex\label{ex:konstituententests037}
    \begin{xlist}
      \ex{\label{ex:konstituententests038} Gestern hat \alert<11->{Elena} \gruen<11->{im Turmspringen} \orongsch<11->{eine Medaille} gewonnen.}
      \pause
      \ex{\label{ex:konstituententests039} Gestern hat \gruen{im Turmspringen} \alert{Elena} \orongsch{eine Medaille} gewonnen.}
      \pause
      \ex{\label{ex:konstituententests040} Gestern hat \gruen{im Turmspringen} \orongsch{eine Medaille} \alert{Elena} gewonnen.}
    \end{xlist}
  \end{exe}
\end{frame}

\begin{frame}
  {Koordinationstest}
  \pause
  \begin{exe}
    \ex\label{ex:konstituententests041}
    \begin{xlist}
      \ex Wir essen \alert<3->{einen Kuchen}.\\
      \pause
        \KTArr{KoorTest} Wir essen \alert{einen Kuchen} \gruen{und} \alert{ein Eis}.
      \pause
      \ex Wir \alert<5->{essen einen Kuchen}.\\
      \pause
        \KTArr{KoorTest} Wir \alert{essen einen Kuchen} \gruen{und} \alert{lesen ein Buch}.
      \pause
      \ex Sarah hat versucht, \alert<7->{einen Kuchen zu backen}.\\
      \pause
        \KTArr{KoorTest} Sarah hat versucht, \alert{einen Kuchen zu backen} \gruen{und} \\{}\alert{heimlich das Eis aufzuessen}.
      \pause
      \ex Wir sehen, dass \alert<9->{die Sonne scheint}.\\
      \pause
        \KTArr{KoorTest} Wir sehen, dass \alert{die Sonne scheint} \gruen{und} \\{}\alert{Mausi den Rasen mäht}.
    \end{xlist}
  \end{exe}
  \pause
  \begin{exe}
    \ex{\label{ex:konstituententests047} Der Kellner notiert, dass \rot<11->{meine Kollegin einen Salat} möchte.\\
    \pause
    \KTArr{KoorTest} Der Kellner notiert, dass \rot{meine Kollegin einen Salat}\\
    \gruen{und} \rot{mein Kollege einen Sojaburger} möchte.}
    \end{exe}
\end{frame}

\begin{frame}
  {Jede Phrase hat einen Kopf!}
  \onslide<+->
  \onslide<+->
  \orongsch{Der Kopf bestimmt \alert{allein} über die relevanten grammatischen Eigenschaften\\
  der Phrase und kann nie weggelassen werden.}\\
  \onslide<+->
  \Halbzeile
  Phrasen werden daher nach der Kategorie des Kopfes benannt.\\
  \Zeile
  \begin{itemize}[<+->]
    \item \alert{Nominalphrasen} (NPs) haben \orongsch{Nomina} als Köpfe
      \begin{itemize}[<+->]
        \item \alert{[der schöne \orongsch{Baum} vor dem Fenster]}
        \item \grau{Ich kenne} \alert{keinerlei \orongsch{Blumen}, die jetzt schon blühen würden}.
      \end{itemize}
      \Halbzeile
    \item \alert{Adjektivphrasen} (APs) haben \orongsch{Adjektive} als Köpfe
      \begin{itemize}[<+->]
        \item \grau{der} \alert{[überaus \orongsch{schöne}]} \grau{Baum vor dem Fenster}
        \item \grau{Die Kollegin ist} \alert{[\orongsch{stolz} auf ihre Tochter]}.
      \end{itemize}
      \Halbzeile
    \item \alert{Präpositionalphrasen} (PPs) haben \orongsch{Präpositionen} als Köpfe
      \begin{itemize}[<+->]
        \item \grau{der Baum} \alert{[\orongsch{vor} dem Fenster]}
        \item \grau{Der Baum steht} \alert{[\orongsch{vor} dem Fenster]}.
      \end{itemize}
  \end{itemize}
\end{frame}


\begin{frame}
  {Einige typische Muster von Nominalphrasen (NPs)}
  \onslide<+->
  \onslide<+->
  Je nach Art des Kopfs -- Eigenname (Name), Substantiv (Subst), Pronomen (Pro) --\\
  sind die Positionen links vom Kopf nicht besetzbar.\\
  \onslide<+->
 \Halbzeile 
  \begin{center}
    \scalebox{0.8}{\begin{tabular}[h]{lllll}
      \toprule
      \grau{Artikel oder}    & \grau{AP}         & \alert{nominaler}     & \grau{PPs, Adverben}   & \grau{Relativsätze und}  \\
      \grau{Genitiv-NP}      & \grau{} & \alert{Kopf}          & \grau{usw.} & \grau{Komplementsätze}   \\
      \midrule
      &&&& \\
      \textit{die}             & \textit{drei}       & \alert{\textit{Tische}}\Sub{Subst} & \textit{vor der Tafel}    & \textit{die heute fehlen}              \\
      &&&& \\
      \textit{Otjes}           & \textit{intelligente} & \alert{\textit{Kinder}}\Sub{Subst} & & \\
      &&&& \\
      && \alert{\textit{Orangensaft}}\Sub{Subst} && \\
      &&&& \\
      \Dim                     & \Dim                & \alert{\textit{Lemmy}}\Sub{Name} & \textit{von Motörhead}      &                               \\
      &&&& \\
      \Dim                     & \Dim               & \alert{\textit{jener}}\Sub{Pro}  & \textit{dort drüben} & \\
      &&&& \\
      \Dim                     & \Dim               & \alert{\textit{alle}}\Sub{Pro}   & & \textit{die einen Kaffe möchten} \\
    \end{tabular}}
  \end{center}
\end{frame}

\begin{frame}
  {Einige typische Muster von Nominalphrasen (PPs)}
  \onslide<+->
  \onslide<+->
  Die NP rechts ist obligatorisch, ihr Kasus wird von der Präposition bestimmt.
  \onslide<+->
  \Halbzeile
  \begin{center}
    \scalebox{0.8}{\begin{tabular}[h]{lll}
      \toprule
      \grau{Modifizierer} & \alert{Präposition} & \alert{NP (Kasus von} \\
       & \alert{(Kopf)}      & \alert{Präposition bestimmt)} \\
      \midrule
      && \\
      & \textit{\alert{mit}} & \textit{den drei Tischen vor der Tafel, die heute fehlen} \\
      && \\
      & \textit{\alert{von}} & \textit{Otjes intelligenten Kindern} \\
      && \\
      \textit{ganz} & \textit{\alert{ohne}} & \textit{Orangensaft} \\
      && \\
      & \textit{\alert{dank}} & \textit{Lemmy von Motörhead} \\
      && \\
      \textit{genau} & \textit{\alert{neben}} & \textit{jenem dort drüben} \\
      && \\
      & \textit{\alert{für}}   & \textit{alle, die Kaffee möchten} \\
    \end{tabular}}
  \end{center}
\end{frame}

\section{Zur nächsten Woche | Überblick}

\begin{frame}
  {Morphologie und Lexikon des Deutschen | Plan}
  \rot{Alle} angegebenen Kapitel\slash Abschnitte aus \rot{\citet{Schaefer2018b}} sind \rot{Klausurstoff}!\\
  \Halbzeile
  \begin{enumerate}
    \item Grammatik und Grammatik im Lehramt (Kapitel 1 und 3)
    \item Morphologie und Grundbegriffe (Kapitel 2, Kapitel 7 und Abschnitte 11.1--11.2)
    \item \rot{Wortklassen als Grundlage der Grammatik (Kapitel 6)}
    \item Wortbildung | Komposition (Abschnitt 8.1)
    \item Wortbildung | Derivation und Konversion (Abschnitte 8.2 und 8.3)
    \item Flexion | Nomina außer Adjektiven (Abschnitte 9.1--9.3)
    \item Flexion | Adjektive und Verben (Abschnitt 9.4 und Kapitel 10)
    \item Valenz (Abschnitte 2.3, 14.1 und 14.3)
    \item Verbtypen als Valenztypen (Abschnitte 14.4, 14.5, 14.7--14.9) 
    \item Kernwortschatz und Fremdwort (vorwiegend Folien)
  \end{enumerate}
  \Halbzeile
  \centering 
  \url{https://langsci-press.org/catalog/book/224}
\end{frame}


