\section{Überblick}

\begin{frame}
  {Andere Wortbildungsmuster}
  \onslide<+->
  \begin{itemize}[<+->]
    \item \alert{Konversion} | Stamm\Sub{1} → Stamm\Sub{2} \\ 
      \textit{laufen} → (\textit{der}) \textit{Lauf}
      \Zeile
    \item \alert{Derivation} | Stamm\Sub{1} + Affix → Stamm\Sub{2}\\
      \textit{schön} → (\textit{die}) \textit{Schönheit}
      \Halbzeile
    \item Typische Anwendungsbereiche für \alert{Präfigierung} und \alert{Suffigierung} im Deutschen
  \end{itemize}
\end{frame}

\section{Konversion}

\begin{frame}
  {Beispiele für Konversion}
  \pause
  Konversion | \orongsch{Stamm\Sub{1} \slash\ Wortform} → \alert{Stamm\Sub{2}}
  \Halbzeile
  \pause
  \begin{exe}
    \ex[ ]{\orongsch{einkauf-en} → \alert{Einkauf}}
    \pause
    \ex[ ]{\orongsch{einkauf-en} → \alert{Einkaufen}}
    \pause
    \ex[ ]{\orongsch{ernst} → \alert{Ernst}}
    \pause
    \ex[ ]{\orongsch{schwarz} → \alert{Schwarz}}
    \pause
    \ex[ ]{\orongsch{gestrichen} → \alert{gestrichen}}
    \pause
    \ex[!]{\orongsch{schwarz} → \alert{schwärzen}}
    \pause
    \ex[!]{\orongsch{schieß-en} → \alert{Schuss}}
    \pause
    \ex[?]{\orongsch{stech-en} → \alert{Stich}}
  \end{exe}
\end{frame}

\begin{frame}
  {Stammkonversion}
  \pause
  \begin{itemize}[<+->]
    \item \orongsch{Stamm} → \alert{Stamm} \alert{(mit Wortklassenwechsel)}
      \Halbzeile
    \item produktiv vor allem 
      \Halbzeile
      \begin{itemize}[<+->]
        \item \alert{Verbstammnominalisierung} | \textit{\orongsch{einkauf-en}} → \textit{der \alert{Einkauf}}\\
          \grau{Flexion wie ein normales maskulines Substantiv}
          \Halbzeile

        \item \alert{(Farb-)Adjektivnominalisierung} | \textit{das Kleid ist \orongsch{rot}} → \textit{das \alert{Rot} des Kleids}\\
          \grau{Flexion wie ein normales neutrales Substantiv}
          \Halbzeile

        \item \alert{metasprachliche Nominalisierung} | \textit{saturiert, \orongsch{aber} unzufrieden} → \textit{das ständige \alert{Aber}}\\
          \grau{Flexion wir ein normales neutrales Substantiv}

      \end{itemize}
  \end{itemize}
\end{frame}

\begin{frame}
  {Wortformenkonversion}
  \pause
  \begin{itemize}[<+->]
    \item \orongsch{flektierte Wortform} → \alert{Stamm \slash\ Wortform} (mit Wortklassenwechsel)
      \Halbzeile
    \item produktiv vor allem
      \Halbzeile
      \begin{itemize}[<+->]
        \item \alert{Infinitivnominalisierung} | \textit{Ich gehe \orongsch{einkaufen}.} → \textit{Das \alert{Einkaufen} macht Spaß.}\\
          \grau{Flexion wie ein normales neutrales Substantiv}
          \Halbzeile

        \item \alert{Adjektivnominalisierung} | \textit{Zwei \orongsch{doppelte} Brötchen bitte.} → \textit{Zwei \alert{Doppelte} bitte.}\\
          \grau{Flexion wie ein Adjektiv | daher Konversion Wortform → Wortform}
          \Halbzeile

        \item \alert{Adjektiadverbialisierung} | \textit{Das Auto ist \orongsch{schnell}.} → \textit{Das Auto fährt \alert{schnell}.}\\
          \grau{keine Flexion außer Komparativ}
      \end{itemize}
  \end{itemize}
\end{frame}

\section{Derivation}

\begin{frame}
  {Beispiele für Derivation}
  \pause
  Derivation | \orongsch{Stamm\Sub{1}} + \alert{Affix} → Stamm\Sub{2}
  \Halbzeile
  \pause
  \begin{exe}
    \ex
    \begin{xlist}
      \ex \orongsch{Scherz} → \orongsch{scherz}\alert{:haft}
      \pause
      \ex \orongsch{brenn}-en → \orongsch{brenn}\alert{:bar}
      \pause
      \ex \orongsch{grün} → \orongsch{grün}\alert{:lich}
    \end{xlist}
    \pause
    \Halbzeile
    \ex
    \begin{xlist}
      \ex \orongsch{doof} → \orongsch{Doof}\alert{:heit}
      \pause
      \ex \orongsch{Fahrer} → \orongsch{Fahrer}\alert{:in}
      \pause
      \ex \orongsch{Kunde} → \orongsch{Kund}\alert{:schaft}
      \pause
      \ex \orongsch{Hund} → \orongsch{Hünd}\alert{:chen}
    \end{xlist}
    \pause
    \Halbzeile
    \ex
    \begin{xlist}
      \ex \orongsch{Schlange} → \orongsch{schläng}\alert{:el}-n
      \pause
      \ex \orongsch{Ruck} → \orongsch{ruck}\alert{:el}-n
    \end{xlist}
  \end{exe}
\end{frame}

\begin{frame}
  {Mit und ohne Wortklassenwechsel}
  \pause
  \begin{itemize}[<+->]
    \item \alert{mit} Wortklassenwechsel | Wortart ändert sich (\textit{Hand} → \textit{händ:isch})
    \item \alert{ohne} Wortklassenwechsel | Wortart bleibt gleich (\textit{rot} → röt:lich)
      \Zeile
    \item ohne Wortklassenwechsel | geänderte statische Merkmale?
      \begin{itemize}[<+->]
        \item in jedem Fall \alert{Bedeutung}
        \item prototypisch \textit{Dank → Un:dank}, \textit{bedeutend → un:bedeutend}
      \end{itemize}
  \end{itemize}
\end{frame}

\begin{frame}
  {Etwas schwierigere Fälle}
  \pause
  \begin{exe}
    \ex
    \begin{xlist}
      \ex{bebeispielen, bestuhlen, bevölkern}
      \ex{entvölkern, entgräten, entwanzen}
      \ex{verholzen, vernageln, verwanzen, verzinnen}
    \end{xlist}
    \pause
    \ex
    \begin{xlist}
      \ex{ergrauen, ermüden, erneuern}
      \ex{befreien, beengen, begrünen}
    \end{xlist}
  \end{exe}
  \pause
  \Halbzeile
  \begin{itemize}[<+->]
    \item entweder \alert{Stammkonversion + Präfigierung}
      \begin{itemize}[<+->]
        \item \textit{grau} (Adjektiv)
        \item[→] \textit{grau-en} (Stammkonversion zum Verb)
        \item[→] \textit{er:grau-en} (Präfigierung ohne Wortklassenwechsel)
      \end{itemize}
    \item oder \alert{wortartenverändernde Präfixe}
      \begin{itemize}[<+->]
        \item \textit{grau} (Adjektiv)
        \item[→] \textit{er:grau-en} (Präfigierung mit Wortklassenwechsel zum Verb)
      \end{itemize}
  \end{itemize}
\end{frame}

\begin{frame}
  {Im Bereich welcher Wortklassen wird vor allem suffigiert?}
  \pause
  \begin{center}
    \scalebox{0.5}{
      \begin{tabular}{llll}
        \toprule
        \textbf{Ausgangsklasse} & \textbf{Substantiv-Affix} & \textbf{Adjektiv-Affix} & \textbf{Verb-Affix} \\
       \midrule
       \multirow{8}{*}{\textbf{Substantiv}} & :chen & :haft & \\
       & \textit{Äst:chen} & \textit{schreck:haft} & \\
       \cmidrule{2-4}
       
       & :in & :ig & \\
       & \textit{Arbeiter:in} & \textit{fisch:ig} & \\
       \cmidrule{2-4}
       
       & :ler & :isch & \\
       & \textit{Volkskund:ler} & \textit{händ:isch} & \\
       \cmidrule{2-4}
       
       & :schaft & :lich & \\
       & \textit{Wissen:schaft} & \textit{häus:lich} & \\
       
       \midrule
       \multirow{6}{*}{\textbf{Adjektiv}} & :heit & :lich & \\
       & \textit{Schön:heit} & \textit{röt:lich} & \\
       \cmidrule{2-4}
       
       & :keit && \\
       & \textit{Heiter:keit} & & \\
       \cmidrule{2-4}
       
       & :igkeit && \\
       & \textit{Neu:igkeit} & & \\
       
       \midrule
       \multirow{6}{*}{\textbf{Verb}} & :er & :bar & :el \\
       & \textit{Arbeit:er} & \textit{bieg:bar} & \textit{kreis:el-n} \\
       \cmidrule{2-4}
       
       & :erei && \\
       & \textit{Arbeit:erei} & & \\
       \cmidrule{2-4}
       
       & :ung && \\
       & \textit{Les:ung} & & \\
       
       \bottomrule
      \end{tabular}
    }\\
    \Zeile
    \pause
    \alert{\large \ldots\ von Nomina und Verben zu Nomen | vor allem zum Substantivderivation}\\
  \end{center}
\end{frame}

\begin{frame}
  {In welchem Bereich wird prototypisch präfigiert?}
  \onslide<+->
  \onslide<+->
  Verbpräfixe | \alert{Präfix} + \orongsch{Verb} → Verb\\
  \Halbzeile
  \begin{itemize}[<+->]
    \item \orongsch{kauf}-en → \alert{ver:}\orongsch{kauf}-en
    \item \orongsch{hol}-en → \alert{über:}\orongsch{hol}-en
    \item \orongsch{stell}-en → \alert{unter:}\orongsch{stell}-en
  \end{itemize}
  \Zeile
  \onslide<+->
  Verpartikeln | \gruen{Partikel} + \alert{Verb} → Verb\\
  \Halbzeile
  \begin{itemize}[<+->]
    \item \orongsch{leg}-en → \gruen{um=}\orongsch{leg}-en
    \item \orongsch{geh}-en → \gruen{entlang=}\orongsch{geh}-en
    \item \orongsch{trenn}-en → \gruen{ab=}\orongsch{trenn}-en
  \end{itemize}
\end{frame}

\begin{frame}
  {Unterschiede zwischen Verbpräfixen und Verbpartikeln}
  \onslide<+->
  \begin{itemize}[<+->]
    \item bei der Trennbarkeit
      \begin{itemize}[<+->]
        \item \ldots\ weil wir es \alert{ver:}\orongsch{kaufen} | Wir \alert{ver:}\orongsch{kaufen} es.
        \item \ldots\ weil wir es \gruen{ab:}\orongsch{trennen} | Wir \orongsch{trennen} es \gruen{ab}.
      \end{itemize}
      \Zeile
    \item bei Partizipbildung
      \begin{itemize}[<+->]
        \item \alert{ver:}\orongsch{kauf}-en → \alert{ver:}\orongsch{kauf}\rot{-t}
        \item \gruen{ab=}\orongsch{trenn}-en → \gruen{ab=}\rot{ge-}\orongsch{trenn}\rot{-t}
      \end{itemize}
  \end{itemize}
  \onslide<+->

  \Zeile
  \centering 
  \grau{Wir kommen auf die Formen später nochmal kurz zurück.}
\end{frame}

\begin{frame}
  {Notationskonvention in EGBD}
  \pause
  \begin{itemize}[<+->]
    \item \alert{Flexionsendungen und Fugen} mit Bindestrich: \textit{Tisch-es}, \textit{Fäng-e}
    \item \alert{Kompositumsglieder} mit Punkt | \textit{Tasche-n.tuch}
    \item \alert{Derivationsaffixe} mit Doppelpunkt | \textit{Läuf:er}, \textit{ver:blüh-en}
    \item \alert{Verbpartikeln} mit Gleichheitszeichen | \textit{ab=trenn-en}, \textit{auf=schieb-en}
    \Halbzeile
  \item Markierung für umlautauslösende Affixe aus EGBD3 \rot{entfällt}
      \begin{itemize}[<+->]
        \item \grau{\char`~ bei Flexion (Plural \textit{\char`~er}, \textit{Männ-er})}
        \item \grau{\~: bei Derivation (wie bei \textit{\~:lich}, töd:lich)}
      \end{itemize}
    \Halbzeile
  \item spezifisch EGBD, keine allgemeine Konvention
  \item \rot{Die Notation muss für die Klausur sicher beherrscht werden!}
  \end{itemize}
\end{frame}


\ifdefined\TITLE
  \section{Zur nächsten Woche | Überblick}

  \begin{frame}
    {Morphologie und Lexikon des Deutschen | Plan}
    \rot{Alle} angegebenen Kapitel\slash Abschnitte aus \rot{\citet{Schaefer2018b}} sind \rot{Klausurstoff}!\\
    \Halbzeile
    \begin{enumerate}
      \item Grammatik und Grammatik im Lehramt (Kapitel 1 und 3)
      \item Morphologie und Grundbegriffe (Kapitel 2, Kapitel 7 und Abschnitte 11.1--11.2)
      \item Wortklassen als Grundlage der Grammatik (Kapitel 6)
      \item Wortbildung | Komposition (Abschnitt 8.1)
      \item Wortbildung | Derivation und Konversion (Abschnitte 8.2--8.3)
      \item \rot{Flexion | Nomina außer Adjektiven (Abschnitte 9.1--9.3)}
      \item Flexion | Adjektive und Verben (Abschnitt 9.4 und Kapitel 10)
      \item Valenz (Abschnitte 2.3, 14.1 und 14.3)
      \item Verbtypen als Valenztypen (Abschnitte 14.4--14.5, 14.7--14.9) 
      \item Kernwortschatz und Fremdwort (vorwiegend Folien)
    \end{enumerate}
    \Halbzeile
    \centering 
    \url{https://langsci-press.org/catalog/book/224}
  \end{frame}
\fi
