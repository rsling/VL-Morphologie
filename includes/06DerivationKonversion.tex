\section{Überblick}

\begin{frame}
  {Andere Wortbildungsmuster}
  \onslide<+->
  \begin{itemize}[<+->]
    \item \alert{Konversion} | Stamm\Sub{1} → Stamm\Sub{2} \\ 
      \textit{laufen} → (\textit{der}) \textit{Lauf}
      \Zeile
    \item \alert{Derivation} | Stamm\Sub{1} + Affix → Stamm\Sub{2}\\
      \textit{schön} → (\textit{die}) \textit{Schönheit}
      \Halbzeile
    \item Typische Anwendungsbereiche für \alert{Präfigierung} und\\
      \alert{Suffigierung} im Deutschen
      \Zeile
    \item \citet[8.2,8.3]{Schaefer2018b}
  \end{itemize}
\end{frame}

\section{Konversion}

\begin{frame}
  {Beispiele für Konversion}
  \pause
  Konversion: \alert{Stamm\Sub{1} oder Wortform → neuer Stamm\Sub{2}}
  \Halbzeile
  \pause
  \begin{exe}
    \ex[ ]{einkauf-en → Einkauf}
    \pause
    \ex[ ]{einkauf-en → Einkaufen}
    \pause
    \ex[ ]{ernst → Ernst}
    \pause
    \ex[ ]{schwarz → Schwarz}
    \pause
    \ex[ ]{gestrichen → gestrichen}
    \pause
    \ex[!]{schwarz → schwärzen}
    \pause
    \ex[!]{schieß-en → Schuss}
    \pause
    \ex[?]{stech-en → Stich}
  \end{exe}
\end{frame}

\begin{frame}
  {Stammkonversion}
  \pause
  \begin{itemize}[<+->]
    \item Ausgangswort: \rot{Stamm}
    \item[→] Zielwort: Stamm \alert{(mit Wortklassenwechsel)}
      \Halbzeile
    \item also \textit{Einkauf}, \textit{Schwarz}, \textit{Ernst}
      \Halbzeile
    \item Zielwort: andere Flexion, gemäß Zielwortklasse
      \begin{itemize}[<+->]
        \item \textit{kaufst}; \textit{des Kaufs}
        \item \textit{dem schwarzen Schal}; \textit{dem Schwarz der Nacht}
      \end{itemize}
  \end{itemize}
\end{frame}

\begin{frame}
  {Wortformenkonversion}
  \pause
  \begin{itemize}[<+->]
    \item Ausgangswort: \rot{flektierte Wortform}
    \item[→] Zielwort: Stamm \alert{(mit Wortklassenwechsel)}
      \Halbzeile
    \item also (\textit{das}) \textit{Einkaufen}, (\textit{das}) \textit{Gemahlene} usw.
  \end{itemize}
\end{frame}

\section{Derivation}

\begin{frame}
  {Beispiele für Derivation}
  \pause
  Derivation: \alert{Stamm\Sub{1} + Affix → neuer Stamm\Sub{2}}
  \Halbzeile
  \pause
  \begin{exe}
    \ex
    \begin{xlist}
      \ex Scherz → scherz\alert{:haft}
      \pause
      \ex brenn-en → brenn\alert{:bar}
      \pause
      \ex grün → grün\alert{:lich}
    \end{xlist}
    \pause
    \Halbzeile
    \ex
    \begin{xlist}
      \ex doof → Doof\alert{:heit}
      \pause
      \ex Fahrer → Fahrer\alert{:in}
      \pause
      \ex Kunde → Kund\alert{:schaft}
      \pause
      \ex Hund → Hünd\alert{:chen}
    \end{xlist}
    \pause
    \Halbzeile
    \ex
    \begin{xlist}
      \ex Schlange → schläng\alert{:el}-n
      \pause
      \ex Ruck → ruck\alert{:el}-n
    \end{xlist}
  \end{exe}
\end{frame}

\begin{frame}
  {Mit und ohne Wortklassenwechsel}
  \pause
  \begin{itemize}[<+->]
    \item \alert{mit} Wortklassenwechsel: Wortart ändert sich (\textit{Hand} → \textit{händ:isch})
    \item \alert{ohne} Wortklassenwechsel: Wortart bleibt gleich (\textit{rot} → röt:lich)
      \Zeile
    \item ohne Wortklassenwechsel: geänderte statische Merkmale?
      \begin{itemize}[<+->]
        \item in jedem Fall \alert{Bedeutung}
        \item prototypisch: \textit{Dank → Un:dank}, \textit{bedeutend → un:bedeutend}
      \end{itemize}
  \end{itemize}
\end{frame}

\begin{frame}
  {Etwas schwierigere Fälle}
  \pause
  \begin{exe}
    \ex
    \begin{xlist}
      \ex{bebeispielen, bestuhlen, bevölkern}
      \ex{entvölkern, entgräten, entwanzen}
      \ex{verholzen, vernageln, verwanzen, verzinnen}
    \end{xlist}
    \pause
    \ex
    \begin{xlist}
      \ex{ergrauen, ermüden, erneuern}
      \ex{befreien, beengen, begrünen}
    \end{xlist}
  \end{exe}
  \pause
  \Halbzeile
  \begin{itemize}[<+->]
    \item entweder \alert{Stammkonversion + Präfigierung}
      \begin{itemize}[<+->]
        \item \textit{grau} (Adjektiv)
        \item[→] \textit{grau-en} (Stammkonversion zum Verb)
        \item[→] \textit{er:grau-en} (Präfigierung ohne Wortklassenwechsel)
      \end{itemize}
    \item oder \alert{wortartenverändernde Präfixe}
      \begin{itemize}[<+->]
        \item \textit{grau} (Adjektiv)
        \item[→] \textit{er:grau-en} (Präfigierung mit Wortklassenwechsel zum Verb)
      \end{itemize}
  \end{itemize}
\end{frame}

\begin{frame}
  {In welchem Bereich wird vor allem suffigiert?}
  \pause
  \begin{center}
    \scalebox{0.5}{
      \begin{tabular}{llll}
        \toprule
        \textbf{Ausgangsklasse} & \textbf{Substantiv-Affix} & \textbf{Adjektiv-Affix} & \textbf{Verb-Affix} \\
       \midrule
       \multirow{8}{*}{\textbf{Substantiv}} & \~:chen & :haft & \\
       & \textit{Äst:chen} & \textit{schreck:haft} & \\
       \cmidrule{2-4}
       
       & :in & :ig & \\
       & \textit{Arbeiter:in} & \textit{fisch:ig} & \\
       \cmidrule{2-4}
       
       & :ler & \~:isch & \\
       & \textit{Volkskund:ler} & \textit{händ:isch} & \\
       \cmidrule{2-4}
       
       & :schaft & \~:lich & \\
       & \textit{Wissen:schaft} & \textit{häus:lich} & \\
       
       \midrule
       \multirow{6}{*}{\textbf{Adjektiv}} & :heit & \~:lich & \\
       & \textit{Schön:heit} & \textit{röt:lich} & \\
       \cmidrule{2-4}
       
       & :keit && \\
       & \textit{Heiter:keit} & & \\
       \cmidrule{2-4}
       
       & :igkeit && \\
       & \textit{Neu:igkeit} & & \\
       
       \midrule
       \multirow{6}{*}{\textbf{Verb}} & :er & :bar & \~:el \\
       & \textit{Arbeit:er} & \textit{bieg:bar} & \textit{kreis:el-n} \\
       \cmidrule{2-4}
       
       & :erei && \\
       & \textit{Arbeit:erei} & & \\
       \cmidrule{2-4}
       
       & :ung && \\
       & \textit{Les:ung} & & \\
       
       \bottomrule
      \end{tabular}
    }\\
    \Zeile
    \pause
    \alert{\large \ldots zum Nomen hin, vor allem zum Substantiv.}\\
    \pause
    \rot{\large In welchem Bereich wird prototypisch präfigiert?}
  \end{center}
\end{frame}

\begin{frame}
  {Notationskonvention im Buch}
  \pause
  \begin{itemize}[<+->]
    \item \alert{Flexion (und Fuge)} mit Bindestrich: \textit{Tisch-es}, \textit{Fäng-e}
    \item \alert{Komposition} mit Punkt: \textit{Tasche-n.tuch}
    \item \alert{Derivation} mit Doppelpunkt: \textit{Läuf:er}, \textit{ver:blühen}
    \item \alert{Verbpartikeln} mit Gleichheitszeichen: \textit{ab=trenn-en}, \textit{auf=schieb-en}
    \Halbzeile
    \item bei Angabe der einzelnen Affixe, wenn sie Umlaut auslösen:
      \begin{itemize}[<+->]
	      \item \char`~ bei Flexion (Plural \textit{\char`~er}, \textit{Männ-er})
        \item \~: bei Derivation (wie bei \textit{\~:lich}, töd:lich)
      \end{itemize}
    \Halbzeile
  \item spezifisch EGBD, keine allgemeine Konvention
  \end{itemize}
\end{frame}

\section{Übung}

\begin{frame}
  {Wortbildung analysieren}
  \onslide<+->
  \begin{itemize}[<+->]
    \item Suchen Sie im gegebenen Text nach Derivationen und Konversionen.
    \item Analysieren Sie sie mit der Notationskonvention aus EGBD3.
    \item Überlegen Sie, wie produktiv die Bildungen sind.
      \Zeile
    \item Überlegen Sie anhand der Derivartionsanalyse in verschiedenen Wortklassen,\\
      in welchem Bereich im Deutschen typischerweise präfigiert wird.
  \end{itemize}
\end{frame}

\section{Ausblick}

\begin{frame}
  {Nominalflexion}
  \onslide<+->
  \begin{itemize}[<+->]
    \item Funktion in der Nominalflexion
    \item Flexion(sklassen) der Substantive
    \item Flexion der Pronomina und Artikel
    \item Flexion der Adjektive
      \Zeile
    \item \citet[Kapitel~9]{Schaefer2018b}
  \end{itemize}
\end{frame}
