\section{Überblick}

\begin{frame}
  {Funktionale Wortschatzgliederung bei Verben}
  \onslide<+->
  \begin{itemize}[<+->]
    \item passivierbare und nicht passivierbare Verben
    \item valenzgebundene und nicht valenzgebundene Dative
    \item Präpositionalobjekte
  \end{itemize}
\end{frame}

\section{Valenz}

\section{Rollen}

\section{Passive}

\begin{frame}
  {\textit{werden}-Passiv oder Vorgangspassiv}
  \pause
  "`Nur transitive Verben können passiviert werden."'\pause\rot{--- Nein!}
  \pause
    \begin{exe}
    \addtolength\itemsep{-0.25\baselineskip}
      \ex\label{ex:werdenpassivundverbtypen110}
      \begin{xlist}\addtolength\itemsep{-0.5\baselineskip}
          \ex[ ]{\label{ex:werdenpassivundverbtypen111} \alert{Johan} wäscht \orongsch{den Wagen}.}
          \ex[ ]{\label{ex:werdenpassivundverbtypen112} \orongsch{Der Wagen} wird \alert{(von Johan)} gewaschen.}
      \end{xlist}
      \pause
      \ex\label{ex:werdenpassivundverbtypen113}
      \begin{xlist}\addtolength\itemsep{-0.5\baselineskip}
          \ex[ ]{\label{ex:werdenpassivundverbtypen114} \alert{Alma} schenkt \gruen{dem Schlossherrn} \orongsch{den Roman}.}
          \ex[ ]{\label{ex:werdenpassivundverbtypen115} \orongsch{Der Roman} wird \gruen{dem Schlossherrn} \alert{(von Alma)} geschenkt.}
      \end{xlist}
      \pause
      \ex\label{ex:werdenpassivundverbtypen116}
      \begin{xlist}\addtolength\itemsep{-0.5\baselineskip}
          \ex[ ]{\label{ex:werdenpassivundverbtypen117} \alert{Johan} bringt \orongsch{den Brief} zur Post.}
          \ex[ ]{\label{ex:werdenpassivundverbtypen118} \orongsch{Der Brief} wird \alert{(von Johan)} zur Post gebracht.}
      \end{xlist}
      \pause
      \ex\label{ex:werdenpassivundverbtypen119}
      \begin{xlist}\addtolength\itemsep{-0.5\baselineskip}
          \ex[ ]{\label{ex:werdenpassivundverbtypen120} \alert{Der Maler} dankt \gruen{den Fremden}.}
          \ex[ ]{\label{ex:werdenpassivundverbtypen121} \gruen{Den Fremden} wird \alert{(vom Maler)} gedankt.}
      \end{xlist}
      \pause
      \ex\label{ex:werdenpassivundverbtypen122}
      \begin{xlist}\addtolength\itemsep{-0.5\baselineskip}
          \ex[ ]{\label{ex:werdenpassivundverbtypen123} \alert{Johan} arbeitet hier immer montags.}
          \ex[ ]{\label{ex:werdenpassivundverbtypen124} Montags wird hier \alert{(von Johan)} immer gearbeitet.}
      \end{xlist}
      \pause
      \ex\label{ex:werdenpassivundverbtypen125}
      \begin{xlist}\addtolength\itemsep{-0.5\baselineskip}
          \ex[ ]{\label{ex:werdenpassivundverbtypen126} \alert{Der Ball} platzt bei zu hohem Druck.}
          \ex[*]{\label{ex:werdenpassivundverbtypen127} Bei zu hohem Druck wird \rot{(vom Ball)} geplatzt.}
      \end{xlist}
      \pause
      \ex\label{ex:werdenpassivundverbtypen128}
      \begin{xlist}\addtolength\itemsep{-0.5\baselineskip}
          \ex[ ]{\label{ex:werdenpassivundverbtypen129} \alert{Der Rottweiler} fällt \gruen{Michelle} auf.}
          \ex[*]{\label{ex:werdenpassivundverbtypen130} \alert{Michelle} wird \rot{(von dem Rottweiler)} aufgefallen.}
      \end{xlist}
    \end{exe}
\end{frame}

\begin{frame}
  {Was passiert beim Vorgangspassiv?}
  \pause
  \begin{itemize}[<+->]
    \item Auxiliar: \textit{werden}, Verbform: Partizip
    \item für Passivierbarkeit relevant: \alert{die Nominativ-Ergänzung!}
      \Halbzeile
    \item \alert{Passivierung = Valenzänderung}:
      \begin{itemize}[<+->]
        \item Nominativ-Ergänzung → optionale \textit{von}-PP-Angabe
        \item eventuelle Akkusativ-Ergänzung → obligatorische Nominativ-Ergänzung
        \item kein Akkusativ: kein "`Subjekt"' = keine Nom-Erg (\textit{es} ist positional)
        \item \grau{Dativ-Ergänzung → Dativ-Ergänzung (usw.)}
        \item \grau{Angaben: keine Änderung}
      \end{itemize}
    \Halbzeile
  \item \alert{nicht passivierbare Verben}?
    \begin{itemize}[<+->]
      \item {ohne }\rot{agentivische}\alert{ Nominativ-Ergänzung}
      \item Achtung! Gilt nur mit prototypischem Charakter\ldots
      \item Siehe Vertiefung 14.2 auf S.~439!
    \end{itemize}
  \end{itemize}
\end{frame}

\begin{frame}
  {Feinere Klassifikation von Verben}
  \pause
  \begin{itemize}[<+->]
    \item Neuklassifikation vor dem Hintergrund des Vorgangspassivs
    \item Wenn so eine Klassifikation einen Wert haben soll:\\
      \alert{Berücksichtigung der semantischen Rollen unabdinglich!}
    \item Bedingung für Vorgangs-Passiv: \alert{Nom\_Ag}
  \end{itemize} 
  \pause
  \Zeile
  \centering
  \scalebox{0.9}{\begin{tabular}{lllll}
    \toprule
    \textbf{Valenz} & \textbf{Passiv} & \textbf{Name} & \textbf{Beispiel} \\
    \midrule
    \alert{Nom\_Ag} & ja & Unergative & \textit{arbeiten} \\
    Nom & nein & Unakkusative & \textit{platzen} \\
    \alert{Nom\_Ag}, Akk & ja & Transitive & \textit{waschen} \\
    \alert{Nom\_Ag}, Dat & ja & unergative Dativverben & \textit{danken} \\
    Nom, Dat & nein & unakkusative Dativverben & \textit{auf"|fallen} \\
    \alert{Nom\_Ag}, Dat, Akk & ja & Ditransitive & \textit{geben} \\
    \bottomrule
  \end{tabular}}\\
  \raggedright
  \Zeile
  \pause
  Immer noch nichts als eine reine Bequemlichkeitsterminologie,\\
  um bestimmte (durchaus wichtige) Valenzmuster hervorzuheben.
\end{frame}


\begin{frame}
  {\textit{bekommen}-Passiv oder Rezipientenpassiv}
  \pause
  Es gibt nicht "`das Passiv im Deutschen"'.\\
  \Halbzeile
  \pause
  \begin{exe}
    \ex\label{ex:bekommenpassiv138}
    \begin{xlist}
      \ex[ ]{\small\label{ex:bekommenpassiv139} \gruen{Mein Kollege} bekommt \orongsch{den Wagen} \alert{(von Johan)} gewaschen.}
      \pause
      \ex[ ]{\small\label{ex:bekommenpassiv140} \gruen{Der Schlossherr} bekommt \orongsch{den Roman} \alert{(von Alma)} geschenkt.}
      \pause
      \ex[ ]{\small\label{ex:bekommenpassiv141} \gruen{Mein Kollege} bekommt \orongsch{den Brief} \alert{(von Johan)} zur Post gebracht.}
      \pause
      \ex[ ]{\small\label{ex:bekommenpassiv142} \gruen{Die Fremden} bekommen \alert{(von dem Maler)} gedankt.}
      \pause
      \ex[?]{\small\label{ex:bekommenpassiv143} \gruen{Mein Kollege} bekommt hier immer montags \alert{(von Johan)} gearbeitet.}
      \pause
      \ex[*]{\small\label{ex:bekommenpassiv144} \gruen{Mein Kollege} bekommt bei zu hohem Druck \rot{(von dem Ball)} geplatzt.}
      \pause
      \ex[*]{\small\label{ex:bekommenpassiv145} \gruen{Michelle} bekommt \rot{(von dem Rottweiler)} aufgefallen.}
    \end{xlist}
  \end{exe}
  \pause\Halbzeile
  \alert{Das ist eine Passivbildung, die genauso den Nom\_Ag betrifft\\
  wie das Vorgangspassiv.}
\end{frame}

\begin{frame}
  {Was passiert beim Rezipientenpassiv?}
  \pause
  Alles, was sich verglichen mit Vorgangspassiv nicht unterscheidet, grau.\\
  \Halbzeile
  \pause
  \begin{itemize}[<+->]
    \item Auxiliar: \textit{bekommen} (evtl.\ \textit{kriegen}), \grau{Verbform: Partizip}
    \item \grau{für Passivierbarkeit relevant: die Nominativ-Ergänzung!}
      \Halbzeile
    \item \grau{Passivierung = Valenzänderung}:
      \begin{itemize}[<+->]
        \item \grau{Nominativ-Ergänzung → optionale \textit{von}-PP-Angabe}
        \item eventuelle Akkusativ-Ergänzung: → Akkusativ-Ergänzung
        \item \alert{Dativ-Ergänzung → Nominativ-Ergänzung}
        \item \rot{kein Dativ: kein Rezipientenpassiv}
        \item \grau{Angaben: keine Änderung}
      \end{itemize}
    \Halbzeile
  \item \grau{nicht passivierbare Verben?}
    \begin{itemize}[<+->]
      \item \grau{ohne agentivische Nominativ-Ergänzung}
      \item \grau{Achtung! Gilt nur mit prototypischem Charakter\ldots}
      \item \grau{Siehe Vertiefung 14.2 auf S.~439!}
    \end{itemize}
  \end{itemize}
\end{frame}

\begin{frame}
  {Rezipientenpassiv bei unergativen Verben}
  \pause
  Warum war dieser Satz zweifelhaft?\\
  \begin{exe}
    \ex[?]{\small \gruen{Mein Kollege} bekommt hier immer montags \alert{(von Johan)} gearbeitet.}
  \end{exe}
  \pause
  \Halbzeile
  Ist der zugehörige Aktivsatz besser?\\
  \pause
  \begin{exe}
    \ex[?]{\small Montags arbeitet \alert{Johan} \gruen{meinem Kollegen} hier immer.}
  \end{exe}
  \pause
  \begin{itemize}[<+->]
    \item Nein.
    \item \alert{keine Frage des Rezipientenpassivs}
    \item bei diesen Verben: eher \textit{für}-PP
  \end{itemize}
\end{frame}

\section{Übung}

\begin{frame}
  {Passivierbarkeit}
\end{frame}

\section{Ausblick}

\begin{frame}
  {Verbtypen}
\end{frame}
