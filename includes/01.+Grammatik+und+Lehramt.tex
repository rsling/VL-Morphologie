\section{Überblick}

\begin{frame}
  {Grammatik und Grammatikunterricht}
  \onslide<+->
  \begin{itemize}[<+->]
    \item \blau{Grammatik}
      \begin{itemize}
        \item Grammatik und Grammatikalität
        \item Häufigkeiten und typische Muster
        \item Sprachrichtigkeit
      \end{itemize}
      \Zeile
    \item \blau{Lehramt}
      \begin{itemize}
        \item Wozu Deutschunterricht?
        \item Bildungssprache und Sprachbetrachtung
        \item Aufgaben von Lehrpersonal in Deutsch
      \end{itemize}
  \end{itemize}
\end{frame}


\section{Grammatik}

\begin{frame}
  {Deutsche Sätze erkennen und interpretieren}
  \onslide<+->
  \onslide<+->
  \begin{exe}
    \ex[ ]{Dies ist ein Satz.}
    \onslide<+->
    \ex[*]{Satz dies ein ist.}
    \onslide<+->
    \ex[*]{Kno kna knu.}
    \onslide<+->
    \ex[*]{This is a sentence.}
    \onslide<+->
    \Zeile
    \ex[*]{Dies ist ein Satz}
  \end{exe}
\end{frame}


\begin{frame}
  {Form und Bedeutung: Kompositionalität}
  \onslide<+->
  \onslide<+->
  \begin{exe}
    \ex Das ist ein Kneck.
    \onslide<+->
    \Zeile
  \ex Jede Farbe ist ein Kurzwellenradio.
  \ex Der dichte Tank leckt.
\end{exe}
    \Zeile
  \onslide<+->

  \Large\begin{block}{Kompositionalität}
    Die Bedeutung komplexer sprachlicher Ausdrücke ergibt sich aus der Bedeutung ihrer Teile und der Art ihrer grammatischen Kombination. 
    Diese Eigenschaft von Sprache nennt man Kompositionalität.
  \end{block}
\end{frame}

\begin{frame}
  {Grammatik als System -- und  -- Grammatikalität}
  \onslide<+->
  \onslide<+->
  \Large\begin{block}{Grammatik}
    Eine Grammatik ist ein \alert{System von Regularitäten}, nach denen aus einfachen Einheiten komplexe Einheiten einer Sprache gebildet werden.
  \end{block}
  \Zeile

  \onslide<+->

  \begin{block}{Grammatikalität}
    Jede von einer bestimmten Grammatik beschriebene Symbolfolge ist \alert{grammatisch} relativ zu dieser Grammatik, alle anderen sind \alert{ungrammatisch}.
  \end{block}
\end{frame}

\begin{frame}
  {(Un)grammatisch ist nicht gleich (in)akzeptabel}
  \onslide<+->
  \onslide<+->
  \begin{exe}
    \ex\begin{xlist}
      \ex Bäume wachsen werden hier so schnell nicht wieder.
      \onslide<+->
      \ex Touristen übernachten sollen dort schon im nächsten Sommer.
      \onslide<+->
      \ex Schweine sterben müssen hier nicht.
      \onslide<+->
      \ex Der letzte Zug vorbeigekommen ist hier 1957.
      \onslide<+->
      \ex Das Telefon geklingelt hat hier schon lange nicht mehr.
      \onslide<+->
      \ex Häuser gestanden haben hier schon immer.
      \onslide<+->
      \ex Ein Abstiegskandidat gewinnen konnte hier noch kein einziges Mal.
      \onslide<+->
      \ex Ein Außenseiter gewonnen hat hier erst letzte Woche.
      \onslide<+->
      \ex Die Heimmannschaft zu gewinnen scheint dort fast jedes Mal.
      \onslide<+->
      \ex Ein Außenseiter gewonnen zu haben scheint hier noch nie.
      \onslide<+->
      \ex Ein Außenseiter zu gewinnen versucht hat dort schon oft.
      \onslide<+->
      \ex Einige Außenseiter gewonnen haben dort schon im Laufe der Jahre.
    \end{xlist}
  \end{exe}
\end{frame}

\begin{frame}
  {Grammatikalität und Inakzeptabilität}
  \onslide<+->
  \onslide<+->
  \alert{Grammatikalität}\\
  \Halbzeile
  \begin{itemize}[<+->]
    \item grammatisch | Strukturen, die von einer Grammatik beschrieben werden
    \item ungrammatische Strukturen markiert mit Asterisk *
  \end{itemize}
  \Zeile
  \alert{Akzeptabilität}
  \Halbzeile
  \begin{itemize}[<+->]
    \item akzeptabel | Strukturen, die Menschen als ihre Sprache akzeptieren
    \item mögliche Gründe für Unterschiede zwischen Grammatikalität und Akzeptabilität
      \begin{itemize}[<+->]
        \item kognitive Grammatik | nicht unbedingt eindeutig kodiert (probabilistisch)
        \item Performanz | Störeinflüsse \slash\ eingeschränkte kognitive Verarbeitungsfähigkeit
        \item Individualgrammatik | unterschiedliche Grammatiken auf Basis individuellen Inputs
      \end{itemize}
  \end{itemize}
\end{frame}

\begin{frame}
  {Kern und Peripherie}
  \onslide<+->
  \onslide<+->
  Manche grammatischen Strukturen sind \alert{typischer} als \rot{andere}.\\
  \Zeile
  \onslide<+->
  \begin{exe}
    \ex\label{ex:kernundperipherie022}
      \begin{xlist}
        \ex \alert{Baum, Haus, Matte, Döner, Angst, Öl, Kutsche, \ldots}
        \ex \rot{System, Kapuze, Bovist, Schlamassel, Marmelade, Melodie, \ldots}
      \end{xlist}
      \onslide<+->
      \ex
      \begin{xlist}
        \ex \alert{geht, läuft, lacht, schwimmt, liest, \ldots}
        \ex \rot{kann, muss, will, darf, soll, mag}
      \end{xlist}
      \onslide<+->
      \ex
      \begin{xlist}
        \ex \alert{des Hundes, des Geistes, des Tisches, des Fußes, \ldots}
        \ex \rot{des Schweden, des Bären, des Prokuristen, des Phantasten, \ldots}
      \end{xlist}
  \end{exe}
  \onslide<+->
  \Zeile
  \Large
  \centering
  \alert{Hohe Typenhäufigkeit} vs.\ \rot{niedrige Typenhäufigkeit}.  
\end{frame}

\begin{frame}
  {Zwei verschiedene Häufigkeiten}
  \onslide<+->
  \onslide<+->
  \Large\begin{block}{Typenhäufigkeit}
    Wie viele \alert{verschiedene} Realisierungen (=~Typen)\\
    einer Sorte linguistischer Einheiten gibt es?
  \end{block}

  \onslide<+->
  \Zeile
  
  \begin{block}{Tokenhäufigkeit}
    Wie häufig sind die \alert{ggf.\ identischen} Realisierungen\\
    (=~Tokens) einer Sorte linguistischer Einheiten?
  \end{block}
\end{frame}

\begin{frame}
  {Regel vs.\ Regularität bzw.\ Generalisierung}
  \onslide<+->
  \begin{itemize}[<+->]
    \item Relativsätze und eingebettete \textit{w}-Sätze werden nicht\\
      durch Komplementierer (\textit{dass}) eingeleitet.
      \Viertelzeile
    \item \textit{fragen} ist ein schwaches Verb.
      \Viertelzeile
    \item \textit{zurückschrecken} bildet das Perfekt mit dem Hilfsverb \textit{sein}.
      \Viertelzeile
    \item Im Aussagesatz steht vor dem finiten Verb genau ein Satzglied.
      \Viertelzeile
    \item In Kausalsätzen mit \textit{weil} steht das finite Verb an letzter Stelle.
      \Viertelzeile
    \item \textit{zwecks} ist eine Präposition, die den Genitiv regiert und\\
      nur mit Ereignissubstantiven kombiniert werden kann.
  \end{itemize}
\end{frame}


\begin{frame}
  {Normkorm? Regularitätenkonform?}
  \onslide<+->
  \onslide<+->
  \begin{exe}
    \ex
    \begin{xlist}
      \ex Dann sieht man auf der ersten Seite \alert{wer} \rot{dass} kommt.
      \onslide<+->
      \Halbzeile
      \ex Er \rot{frägt} nach der Uhrzeit.
      \onslide<+->
      \Halbzeile
      \ex Man \rot{habe} zu jener Zeit nicht vor Morden \alert{zurückgeschreckt}.
      \onslide<+->
      \Halbzeile
      \ex \rot{Der Universität} \alert{zum Jubiläum} gratulierte auch Bundesministerin Wilms.
      \onslide<+->
      \Halbzeile
      \ex Er ist noch im Büro, \alert{weil} das Licht \rot{brennt} noch.
      \onslide<+->
      \Halbzeile
      \ex Ich schreibe Ihnen \alert{zwecks} \rot{Platz} im Seminar.
    \end{xlist}
  \end{exe}
\end{frame}


\begin{frame}
  {Regel und Regularität}
  \onslide<+->
  \onslide<+->
  \begin{block}{Regularität}
    Eine grammatische Regularität innerhalb eines Sprachsystems liegt dann vor, wenn sich Klassen von Symbolen unter vergleichbaren Bedingungen gleich (und damit vorhersagbar) verhalten.
  \end{block}

  \onslide<+->
  \Halbzeile

  \begin{block}{Regel}
    Eine grammatische Regel ist die Beschreibung einer Regularität, die in einem normativen Kontext geäußert wird.
  \end{block}

  \onslide<+->
  \Halbzeile
  
  \begin{block}{Generalisierung}
    Eine grammatische Generalisierung ist eine durch Beobachtung zustandegekommene Beschreibung einer Regularität.
  \end{block}
\end{frame}

\begin{frame}
  {Regel vs.\ Regularität bzw.\ Generalisierung}
  \onslide<+->
  \onslide<+->
  Was ist dann der Status dieser Aussagen?\\
  \Zeile 
  \onslide<+->
  \begin{itemize}
    \item[?] \grau{Relativsätze und eingebettete \textit{w}-Sätze werden nicht\\
      durch Komplementierer (\textit{dass}) eingeleitet.}
    \item[?] \grau{\textit{fragen} ist ein schwaches Verb.}
    \item[?] \grau{\textit{zurückschrecken} bildet das Perfekt mit dem Hilfsverb \textit{sein}.}
    \item[?] \grau{Im Aussagesatz steht vor dem finiten Verb genau ein Satzglied.}
    \item[?] \grau{In Kausalsätzen mit \textit{weil} steht das finite Verb an letzter Stelle.}
    \item[?] \grau{\textit{zwecks} ist eine Präposition, die den Genitiv regiert und\\
    nur mit Ereignissubstantiven kombiniert werden kann.}
  \end{itemize}
  \Zeile
  \onslide<+->
  \ding{222} Entweder \alert{Generalisierungen} über die Grammatik von \alert{Varietäten des Deutschen} \\
  oder \rot{normative Regeln}, die die gegebenen Sätze als \rot{falsch} kennzeichnen.
\end{frame}

\begin{frame}
  {Norm ist Beschreibung}
  \onslide<+->
  \begin{itemize}[<+->]
    \item Norm als Grundkonsens
    \item Sprache und Norm im Wandel
    \item Norm und Situation (Register, Stil, \dots)
    \item Variation in der Norm
      \Zeile
    \item \alert{Wichtigkeit der Norm, insbesondere im schulischen Deutschunterricht}
    \item \alert{Normabweichungen erklären} | Warum passt der Fehler nicht ins System?
    \item \alert{das System erklären} | Wie hängt "`richtig"' mit Generalisierungen zusammen?
    \item \rot{schwarze Grammatikdidaktik} | "`Das ist falsch, merk dir das!"'
  \end{itemize}
\end{frame}


\section{Grammatik im Lehramtsstudium}

\begin{frame}
  {Bildungssprache in der siebten Jahrgangsstufe}
  \onslide<+->
  \onslide<+->
  \alert{Gib in eigenen Worten die Aufgabenstellung wieder.} \grau{(\citealt{GogolinLange2011,Feilke2012})}\\
  \Halbzeile
  \onslide<+->
  \begin{quote}\footnotesize
    \orongsch{(Textaufgabe)}\\
    Im Salzbergwerk Bad Friedrichshall wird Steinsalz abgebaut. Das Salz lagert 40 m unter Meereshöhe, während Bad Friedrichshall 155 m über Meereshöhe liegt. Welche Strecke legt der Förderkorb zurück? \grau{(aus: mathe live, 7. Sj, 2000, S. 19)}\\
    \Halbzeile
    \onslide<+->
    \alert{(Schülerantwort A)}\\
    es steht also m m h- die wollen Steinsalz abbauen und das ist zwar in Salzbergwerk Bad Frieshalle -- oder wie das hier steht -- Friedrichshall -- ja und mmh das das liegt aber vier\slash vierzig Millimeter unter des Meeres -- ja vierzig Meter unter Meereshöhe -- und aber die wollen während ähm aber die wollen bei Fried\slash Friedrichshall 155 Meter über das Meereshöhe Meereshöhe liegt -- obwohldas da ober liegt und jetzt wissen sie nicht welche Strecke sie nehmen sollen undjetzt wollen sie wissen -- wie viel Strecken Strecken es eigentlich ist -- m m h weil so ein För\slash Förderkorb bis zur Erdoberfläche zurück\\
    \Halbzeile
    \onslide<+->
    \alert{(Schülerantwort B)}\\
    also -- ähm (\ldots) -- da das\slash der\slash das Bergwerk Bergwerk 40 Meter unter der Meereshöhe liegt und und Friedrichshall 155 über der Meereshöhe--  muss man 155 plus 40 machen -- weil- dieser -- ähmähm (\ldots) Förderkorb muss ja von 40 Meter 40 Meter unter Meeres\slash unter der Meereshöhe nach oben -- das alles transportieren
  \end{quote}
\end{frame}

\begin{frame}
  {Sprachbetrachtung und Literatur im Deutsch-Abitur I}
  \onslide<+->
  \onslide<+->
  Sprachlich-grammatische Betrachtung zur Literatur in Abiturarbeiten \grau{\citep{Haecker2009}}.\\
  \Zeile
  \onslide<+->
  \begin{quote}
    Bsp. 4: Diese Verknüpfung durch Kommas oder Gedankenstriche zeigen (G), dass Ferdinand und sein Vater eine gehobene Sprache sprechen.\\
    \Zeile
    \onslide<+->
    Bsp. :5 Die (\ldots) rhetorischen Fragen deuten darauf hin, dass sich der Präsident irgendwo versucht für sein Handeln zu rechtfertigen und seinem Sohn weiterhin Vorwürfe zu machen (Sb).\\
    \Zeile
    \onslide<+->
    Bsp. 6: Ferdinands und Luisens Persönlichkeiten wurden sehr durch Sprache und die szenische Gestaltung der Szene unterstützt. Ferdinand, der Stürmer und Dränger, bedient sich einer sehr bildhaften Sprache durch Metaphern, Personifikationen und Vergleiche. Luises Sprache ist dagegen durch viele Pausen und Satzstücken (G) geprägt, was ihre Verzweiflung und Unruhe deutlich macht.
  \end{quote}
\end{frame}

\begin{frame}
  {Sprachbetrachtung und Literatur im Deutsch-Abitur II}
  \onslide<+->
  \onslide<+->
  Sprachlich-grammatische Betrachtung zur Literatur in Abiturarbeiten \grau{\citep{Haecker2009}}.\\
  \Zeile
  \onslide<+->
  \begin{quote}
    Bsp. 10: <Kirsch> \ldots durch den Wegfall des Verbs <soll> nur das Wesentliche, in diesem Fall die Landschaft in ihrer Schönheit, beachtet werden \ldots Die Konjunktion \textit{und} (V.\ 16) führt alles zusammen. Das Adverb \textit{ganz} (V.\ 17) verstärkt das Ideal: Ruhe und Schönheit. Der Konsekutivsatz \textit{dass man weiß} (V.\ 19), eingeleitet durch \textit{so} (V.\ 18) stellt den Zusammenhang der Idylle mit der lyrischen Person her. Dieser wird erweitert durch den Kausalsatz \textit{weil man glauben kann} (V.\ 21). Der Zusammenhang wird weiter auch betont mit dem Demonstrativpronomen \textit{dieses} (V.\ 20) und dem bestimmten Artikel \textit{das} (V.\ 22). 
  \end{quote}
\end{frame}


\begin{frame}
  {Bildungssprache}
  \onslide<+->
  \onslide<+->
  \alert{\textit{Der Deutschunterricht führt zu einem kompletten Umbau\\
  der Grammatik des Kindes.}} \grau{\citep{Bredel2013,Eisenberg2004}}\\
  \Zeile
  \begin{itemize}[<+->]
    \item Anforderungen:
    \begin{itemize}[<+->]
      \item Darstellung komplexer Sachverhalte
      \item \dots\ und nicht-faktischer (z.\,B.\ hypothetischer) Sachverhalte
      \item Intensionalität (Abstraktion statt Beispielen)
      \item Registerbewusstsein
    \end{itemize}
       \Halbzeile 
      \item Eigenschaften:
    \begin{itemize}[<+->]
      \item dekontextualisiert
      \item schriftorientiert
      \item normorientiert
    \end{itemize}
        \Halbzeile
      \item \alert{Das alles ist verknüpft mit spezifischen grammatischen Formen!}
      \item[\ding{222}] \alert{Bildungssprache}
  \end{itemize}
\end{frame}

\begin{frame}
  {Sprachbetrachtung}
  \onslide<+->
  \begin{itemize}[<+->]
    \item \alert{Sprachbetrachtung \ding{222}\ Bildungssprache}
     \Zeile 
    \item Bewusstsein über richtige und angemessene Form
     \Zeile 
    \item explizite Sprachbetrachtung im Alltag:
      \Halbzeile
      \begin{itemize}[<+->]
        \item Selbst- oder Fremdkorrektur
        \item Suche nach dem richtigen Ausdruck
        \item Orthographie optimieren
        \item Texte optimieren
        \item Begriffe definieren
        \item Grammatikalität beurteilen
      \end{itemize}
  \end{itemize}
\end{frame}

\begin{frame}
  {Ausgangsbasis: vorliterate Kinder und Sprachbetrachtung}
  \onslide<+->
  \onslide<+->
  Klassische Studien \grau{(\citealt{Bredel2013}, auch \citealt[57--58]{Schaefer2018})}\\
  \Halbzeile
  \onslide<+->
  \begin{itemize}[<+->]
    \item \alert{bedeutungsbezogene} bzw.\ \alert{holistische} Betrachtung
      \Viertelzeile
      \begin{itemize}
        \item \textit{Welches Wort ist länger: Haus oder Streichholzschächtelchen?} \ding{222}\ \textit{Haus.}
          \Viertelzeile
        \item Assoziationen zu Substantiven wie \textit{Bett} \ding{222}\ \alert{Ereignisse} wie \textit{Schlafengehen} usw.\\
          Erwachsene assoziieren \alert{taxonomisch verwandte Gegenstände} (Nachttisch, Sofa usw.)
          \Viertelzeile
        \item \textit{Warum heißt der Geburtstag  "`Geburtstag"'?} \ding{222}\ \textit{Weil es Geschenke und Kuchen gibt.}
          \Viertelzeile
        \item \textit{Wieviele Wörter enthält der Satz "`Im alten Haus lebt eine junge Frau."'} \ding{222}\ \textit{Zwei.}
      \end{itemize}
      \Halbzeile
    \item Aber \alert{erfolgreich}: \textit{Benenne das letzte Wort des Satzes.}
      \Halbzeile
    \item[\ding{222}] Die mentale Grammatik basiert auf Wörtern,\\
      der sprachbetrachtende Zugriff allerdings noch nicht.
  \end{itemize}
\end{frame}

\begin{frame}
  {Schulunterricht}
  \onslide<+->
  \begin{itemize}[<+->]
    \item \alert{systematisch}
      \begin{itemize}
        \item in knapper Zeit das Ganze im Blick
      \end{itemize}
      \Zeile
    \item funktional im Sinn von \alert{Form-Funktion-Beziehung}
      \begin{itemize}
        \item Formen systematisieren
        \item erst dann auf Funktionen beziehen
      \end{itemize}
      \Zeile
    \item \alert{induktiv}
      \begin{itemize}
        \item keine rein deduktive Anwendung vorgegebener Begriffe
        \item Erkenntnisprozesse über sprachliche Formen und Funktionen
        \item \alert{\textit{Grammatik machen}} (Eisenberg)
      \end{itemize}
  \end{itemize}
\end{frame}

\begin{frame}
  {Aufgaben von Lehrpersonen}
  \onslide<+->
  \onslide<+->
  \alert{\textit{Lehrkräften wird die Sprache der Lernenden anvertraut.}} \grau{\citep{Eisenberg2004}}\\
  \Zeile
  \begin{itemize}[<+->]
    \item Unterrichten der Schrift, Orthographie und Schreibung
    \item Unterweisung in Bildungssprache\slash Sprachbetrachtung
    \item Erkennen und \alert{Einordnen} von \alert{sprachlichen Defiziten}
    \item Erkennen von \alert{Interferenz mit Dialekt bzw.\ anderen Erstsprachen}
    \item \alert{Bewerten} von sprachlichen Leistungen
    \item \alert{Erklären} der Bewertung (auch gegenüber Eltern)
      \Zeile
    \item[\ding{222}] Anforderung: vertieftes Wissen über Sprache, vor allem Grammatik
    \item[\ding{222}] Methode der sprachlichen Analyse über Faktenwissen hinaus
    \item[\ding{222}] \rot{Studierende des Lehramts müssen ein erheblich tieferes Grammatikwissen\\
    als die Schulkinder und Jugendlichen haben, die sie später unterrichten!}
  \end{itemize}
\end{frame}


\begin{frame}
  {Wie war das?}
  Ich wiederhole zur Sicherheit nochmal\ldots\\
  \Zeile
  \onslide<+->
  \begin{center}
    \Large\rot{Studierende des Lehramts müssen\\
               ein erheblich tieferes Grammatikwissen\\
               als die Schulkinder und Jugendlichen haben,\\
               die sie später unterrichten!}
  \end{center}
\end{frame}

\begin{frame}
  {"`Wozu brauchen wir das denn?"'}
  \onslide<+->
  \begin{itemize}[<+->]
    \item Diese Frage gilt hiermit als nachhaltig beantwortet.
      \Zeile
    \item \rot{Linguistik: keine praktische Anleitungen für erfolgreiche Schulstunden}
    \item Grundausbildung im \alert{Umgang mit Sprache} 
      \Zeile
    \item Minimalforderung: \alert{Examinierte Lehrkräfte müssen\\
      die Aufgaben für die späteren Lernenden selber lösen und\\
      in den Gesamtkontext einordnen können.}
  \end{itemize}
\end{frame}

\section{Zur nächsten Woche | Überblick}

\begin{frame}
  {Morphologie und Lexikon des Deutschen | Plan}
  \rot{Alle} angegebenen Kapitel\slash Abschnitte aus \rot{\citet{Schaefer2018b}} sind \rot{Klausurstoff}!\\
  \Halbzeile
  \begin{enumerate}
    \item Grammatik und Grammatik im Lehramt \rot{(Kapitel 1 und 3)}
    \item \alert{Morphologie und Grundbegriffe} \rot{(Kapitel 2, Kapitel 7 und Abschnitte 11.1--11.2)}
    \item Wortklassen als Grundlage der Grammatik \rot{(Kapitel 6)}
    \item Wortbildung | Komposition \rot{(Abschnitt 8.1)}
    \item Wortbildung | Derivation und Konversion \rot{(Abschnitte 8.2 und 8.3)}
    \item Flexion | Nomina außer Adjektiven \rot{(Abschnitte 9.1--9.3)}
    \item Flexion | Adjektive und Verben \rot{(Abschnitt 9.4 und Kapitel 10)}
    \item Valenz \rot{(Abschnitte 2.3, 14.1 und 14.3)}
    \item Verbtypen als Valenztypen \rot{(Abschnitte 14.4, 14.5, 14.7--14.9)} 
    \item Kernwortschatz und Fremdwort \grau{(vorwiegend Folien)}
  \end{enumerate}
  \Halbzeile
  \centering 
  \url{https://langsci-press.org/catalog/book/224}
\end{frame}
