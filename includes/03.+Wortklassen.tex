\section{Überblick}

\begin{frame}
  {Wörter und Wortklassen}
  \pause
  \begin{itemize}[<+->]
    \item Was sind Wörter?
    \item Lexikalisches vs.\ syntaktisches Wort
      \Halbzeile
    \item Wozu Wortklassen?
    \item \rot{Bedeutungsklassen} und Wortklassen
    \item \alert{Morphologie} von Wortklassen
%    \item \alert{Syntax} von Wortklassen
      \Halbzeile
    \item wichtige Wortklassen
      \begin{itemize}[<+->]
        \item Nomen
        \item Verb
        \item Präposition
        \item Adverb
        \item \ldots
      \end{itemize}
  \end{itemize}
\end{frame}


\section{Wörter}

\begin{frame}
  {Ebenen und Einheiten}
  \pause
  Kombinatorik von Wortbestandteilen und von Wörtern
  \pause
  \Zeile
  \begin{exe}
    \ex
    \begin{xlist}
      \ex[]{Staat-es}
      \pause
      \ex[*]{Tür-es}
    \end{xlist}
    \pause
    \Zeile
    \ex
    \begin{xlist}
      \ex[]{Der Satz ist eine grammatische Einheit.}
      \pause
      \ex[*]{Die Satz ist eine grammatische Einheit.}
    \end{xlist}
  \end{exe}
\end{frame}

\begin{frame}
  {Alle Wörter haben eine Bedeutung?}
  \pause
  \begin{exe}
    \ex \alert{Es} \alert{wird} schon wieder früh dunkel.
    \pause
    \ex Kristine denkt, \alert{dass} \alert{es} bald regnen \alert{wird}.
    \pause
    \ex Adrianna \alert{hat} gestern \alert{den} Keller inspiziert.
    \pause
    \ex Camilla \alert{und} Emma sehen \alert{sich} \alert{die} Fotos \alert{an}.
  \end{exe}
  \Zeile
  \pause
  \large
  \alert{Bedeutungstragende} Wörter und \alert{Funktionswörter}
\end{frame}

\begin{frame}
  {Morphologie und Syntax}
  \pause
  \begin{itemize}[<+->]
    \item Kombinatorik für \alert{Wortbestandteile} | Morphologie
      \begin{itemize}[<+->]
        \item Wortbestandteile \zB mit \alert{Umlaut} | \textit{rot} -- \textit{röter}
        \item oder \alert{Ablaut} | \textit{heben} -- \textit{hob}
      \end{itemize}
    \item Kombinatorik für \alert{Wörter} | Syntax
      \Zeile
    \item \alert{Zirkuläre oder leere Definitionen?}
    \item \rot{Nein!} | eigene Regularität → eigene Struktur
      \Zeile
    \item Wortbestandteile (bis auf bizarre Grenzfälle) \alert{nicht trennbar}
      \begin{itemize}
        \item \textit{heb-t}\\
          *\textit{heb mit Mühe t}
        \item \textit{Ge-hob-en-heit} \\
          *\textit{Gehoben anspruchsvolle heit}
        \item \textit{Sie geht schnell heim.}\\
          \textit{Schnell geht sie heim.}
      \end{itemize}
  \end{itemize}
\end{frame}

\begin{frame}
  {Wort und Wortform I}
  \pause
  \begin{exe}
    \ex
    \begin{xlist}
      \ex (der) Tisch
      \pause
      \ex (den) Tisch
      \pause
      \ex (dem) Tisch\alert{e}
      \pause
      \ex (des) Tisch\alert{es}
      \pause
      \ex (die) Tisch\alert{e}
      \pause
      \ex (den) Tisch\alert{en}
    \end{xlist}
  \end{exe}
  \pause
  \begin{exe}
    \ex
    \begin{xlist}
      \ex Der \_\_\_\ ist voll hässlich.
      \pause
      \ex Ich kaufe den \_\_\_ nicht.
      \pause
      \ex Wir speisten am \_\_\_\ des Bundespräsidenten.
      \pause
      \ex Der Preis des \_\_\_\ ist eine Unverschämtheit.
      \pause
      \ex Die \_\_\_\ kosten nur noch die Hälfte.
      \pause
      \ex Mit den \_\_\_\ können wir nichts mehr anfangen.
    \end{xlist}
  \end{exe}
\end{frame}

\begin{frame}
  {Wort und Wortform II}
  \pause
  \begin{block}{Wortform}
    Eine Wortform ist eine in syntaktischen Strukturen auftretende und in diesen Strukturen nicht weiter zu unterteilende Einheit.
    [\ldots]
  \end{block}
  \Zeile
  \pause
  \begin{block}{Lexikalisches Wort}
    Das (\alert{lexikalische}) \alert{Wort} ist eine Repräsentation von paradigmatisch zusammengehörenden Wortformen.
    Für das lexikalische Wort sind die Werte nur für diejenigen Merkmale spezifiziert, die in allen Wortformen des Paradigmas dieselben Werte haben.
    [\ldots]
  \end{block}
\end{frame}

\section{Methode}

\begin{frame}
  {Klassische Grundschul-Wortarten}
  \pause
  \scalebox{0.85}{%
  \begin{minipage}{\textwidth}
    \begin{itemize}[<+->]
      \item Dingwort
      \item Tuwort, Tätigkeitswort
      \item Wiewort, Eigenschaftswort
      \item Umstandswort
      \Halbzeile
      \item Dazu die Vermittlungsversuche
        \begin{itemize}[<+->]
          \item \alert{Dingwörter} kann man anfassen. \onslide<8->{\rot{Nein!}}
            \pause
          \item \textit{Die ontologischen Referenten von Substantiven sind physikalische Objekte.} \rot{Nein!}
            \Halbzeile
          \item \alert{Wiewort} | Wie ist die Kanzlerin? -- Katatonisch.
          \item \alert{Tuwort} | Was macht\slash tut Johanna? -- Laufen.
          \item \alert{Umstandswort} | Wie, wo oder warum schläft Johanna? -- Ruhig.
        \end{itemize}
      \Halbzeile
      \item Wieso auch nicht?
        \begin{itemize}[<+->]
          \item Anfassen? Wolken, Ideen, Steckdosen, Rasierklingen, \dots
          \item *Die Kanzlerin ist ehemalig.
          \item Was macht Johanna? -- Hausaufgaben.
          \item Was tut Johanna? -- *Verlaufen. \slash *Sich verlaufen. \slash *Unterliegen.
          \item *Was macht\slash tut das Yoghurt? -- Verschimmeln.
          \item Wie schläft Johanna? -- *Erstaunlicherweise.
        \end{itemize}
    \end{itemize}
  \end{minipage}
  }
\end{frame}

\begin{frame}
  {Ein paar neue Wortarten nach Bedeutungen I}
  \pause
  \begin{itemize}[<+->]
    \item "`Wie, wo, warum?"' \onslide<3->{--- Warum eigentlich nicht drei Wortarten?}
      \Halbzeile
      \pause
    \item \alert{Bewegungsverben} | \textit{laufen}, \textit{springen}, \textit{fahren}, \dots
    \item \alert{Zustandsverben} | \textit{duften}, \textit{wohnen}, \textit{liegen}, \dots
      \Halbzeile
    \item \alert{Konkreta} | \textit{Haus}, \textit{Buch}, \textit{Blume}, \textit{Stier}, \dots
    \item \alert{Abstrakta} | \textit{Konzept}, \textit{Glaube}, \textit{Wunder}, \textit{Kausalität}, \dots
      \Halbzeile
    \item \alert{Zählsubstantive} | \textit{Kumquat}, \textit{Studentin}, \textit{Mikrobe}, \textit{Kneipe}, \dots
    \item \alert{Stoffsubstantive} | \textit{Wasser}, \textit{Wein}, \textit{Zement}, \textit{Mehl}, \dots
  \end{itemize}
\end{frame}

\begin{frame}
  {Ein paar neue Wortarten nach Bedeutungen II}
  \pause
  Aber Moment mal\dots\\
  \pause
  \Zeile
  \begin{exe}
    \ex
    \begin{xlist}
      \ex \alert{Wein} kann lecker sein.
      \ex \alert{Eine Kumquat kann} lecker sein.
      \ex \alert{Kumquats können} lecker sein.
    \end{xlist}
     \pause
     \ex
     \begin{xlist}
       \ex Ein Glas \alert{guter Wein}\slash\alert{guten Weins} kostet 10€.
       \ex Ein Glas \alert{?gute Kumquats}\slash\alert{guter Kumquats} kostet 4€.
     \end{xlist}
     \pause
     \ex
     \begin{xlist}
       \ex Johanna hätte gerne \alert{eine Kumquat}.
       \ex Johanna hätter gerne \alert{einen Wein}.
     \end{xlist}
   \end{exe}
   \pause
   \Zeile
   Es gibt hier durchaus auch \alert{formale} Unterschiede.
\end{frame}

\begin{frame}
  {Morphologische Klassifikation}
  \pause
  \begin{exe}
    \ex
    \begin{xlist}
      \ex{Ich pfeif\alert{e}.\\
      Du pfeif\alert{st}.\\
      Die Schiedsrichterin pfeif\alert{t}.}
        \pause
        \ex{Ich schlaf\alert{e}.\\
        {Du schl\rot{ä}f\alert{st}.}\\
        Die Schiedsrichterin schl\rot{ä}f\alert{t}.}
    \end{xlist}
        \pause
    \ex
    \begin{xlist}
      \ex{der Berg\\
        des Berg\alert{es}\\
        die Berg\alert{e}}
        \pause
      \ex{der Mensch\\
        des Mensch\alert{en}\\
        die Mensch\alert{en}}
        \pause
      \ex{der Staat\\
        des Staat\alert{es}\\
        die Staat\alert{en}}
    \end{xlist}
  \end{exe}
\end{frame}

\begin{frame}
  {Morphologische Klassifikation}
  \pause
  \begin{center}
    \Large Wörter lassen sich in Kategorien einordnen,\\
    je nachdem \alert{welche Merkmale und Formen sie haben}.
  \end{center}
  \Zeile
  \pause
  \begin{itemize}[<+->]
    \item Verben | \textsc{Numerus}, \textsc{Person}, \textsc{Tempus}, \ldots
    \item Substantive | \textsc{Numerus}, \textsc{Genus}, \rot{\textsc{Person} ?}, \ldots
  \end{itemize}
\end{frame}

\begin{frame}
  {Achtung!}
  \pause
  \alert{Änderung der Wortklassenzugehörigkeit} eines Wortes
  \pause
  \Zeile
  \begin{exe}
    \ex\label{ex:paradigmatischeklassifikation017}\begin{xlist}
      \ex{Wir sind des \alert{Wanderns} müde.}
      \pause
      \ex{Wir \alert{wandern}.}
    \end{xlist}
  \end{exe}
  \pause
  \Zeile
  → \rot{zwei verschiedene} lexikalische Wörter
  \pause
  \begin{itemize}[<+->]
    \item \textit{Wandern} | \textsc{Numerus}, \textsc{Genus}, \ldots
    \item \textit{wandern} | \textsc{Numerus}, \textsc{Person}, \textsc{Tempus}, \ldots
  \end{itemize}
\end{frame}


\begin{frame}[fragile]
  {Filter}
  \begin{itemize}
    \item<2-> \alert{Kategorien} definiert über Merkmale und Werte.
      \begin{itemize}[<+->]
        \item<3-> Hat \textsc{Numerus} oder nicht?
        \item<4-> Hat \textsc{Genus} oder nicht?
      \end{itemize}
  \end{itemize}
  \Zeile
  \centering 
  \hspace{0.25\textwidth}\scalebox{0.6}{
    \begin{minipage}{0.5\textwidth}  
      \centering 
    \begin{forest}
      /tikz/every node/.append style={font=\footnotesize},
      for tree={l sep=2em, s sep=2.5em},
      [\textit{Wort}, intrme, {visible on=<5->}, for children={visible on=<6->}
        [{Hat  Numerus?}, decide, for children={visible on=<7->}
          [\textit{flektierbar}, intrme, yes, {visible on=<7->}, for children={visible on=<9->}
            [{Ist finit  flektierbar?}, decide, {visible on=<9->}, for children={visible on=<10->}
              [\textbf{Verb}, finall, yes, {visible on=<10->}]
              [\textit{Nomen}, intrme, no, {visible on=<11->}]
            ]
          ]
          [\textit{nicht flektierbar}, intrme, no, {visible on=<8->}, for children={visible on=<12->}
            [{Hat Valenz-\slash  Kasusrektion?}, decide, {visible on=<12->}, for children={visible on=<13->}
              [\textbf{Präposition}, finall, yes, {visible on=<13->}]
              [\textit{andere}, intrme, no, {visible on=<14->}]
            ]
          ]
        ]
      ]
    \end{forest}
   \end{minipage}
   }
\end{frame}


\section{Einige Wortklassen}

\begin{frame}
  {Flektierbare Wörter | Numerus}
  \pause
  \begin{exe}
    \ex
    \begin{xlist}
      \ex Tüte, Tüten
      \pause
      \ex Baum, Bäume
    \end{xlist}
    \pause
    \ex
    \begin{xlist}
      \ex (ich) gehe, (wir) gehen
      \pause
      \ex (du) gehst, (ihr) geht
    \end{xlist}
    \Zeile
    \pause
    \ex
    \begin{xlist}
      \ex \orongsch<12->{Ein} \orongsch<13->{roter} \alert<8->{Apfel} \rot<9->{hängt} am Baum.
      \pause
      \ex \orongsch<14->{Rote} \alert<10->{Äpfel} \rot<11->{hängen} am Baum.
    \end{xlist}
  \end{exe}
  \Zeile
  \pause
  \pause
  \pause
  \pause
  \pause
  \pause
  \pause
  \pause
  Als \alert{Kongruenzmerkmal} ist Numerus in der Definition\\
  der flektierbaren Wortklassen \alert{strukturell motiviert}.
\end{frame}

\begin{frame}
  {Substantive vs.\ Nomina}
  \pause
  \begin{exe}
    \ex \alert<5->{Die stärkste} \rot<5->{Gewichtheberin} wurde Weltmeisterin.
    \pause
    \ex \alert<5->{Der stärkste} \rot<5->{Versuch} war der zweite.
    \pause
    \ex \alert<5->{Das höchste} \rot<5->{Gewicht} wurde von Tatjana gerissen.
  \end{exe}
  \Zeile
  \pause
  \pause
  \begin{itemize}[<+->]
    \item \rot{Substantive} | festes Genus
    \item \alert{andere Nomina} (Artikel\slash Pronomen, Adjektiv) | Genuskongruenz mit dem Substantiv
  \end{itemize}
\end{frame}

\begin{frame}
  {Adjektive}
  \pause
  \begin{exe}
    \ex
    \begin{xlist}
      \ex Gestern wurde \alert<6->{kein} \rot<3->{großer} \alert<6->{Ball} gespielt.
      \ex Gestern wurde \alert<6->{der} \rot<3->{große} \alert<6->{Ball} gespielt.
    \end{xlist}
    \pause
    \pause
    \ex
    \begin{xlist}
      \ex Gestern wurden \alert<6->{keine} \rot<5->{großen} \alert<6->{Bälle} gespielt.
      \ex Gestern wurden \alert<6->{die} \rot<5->{großen} \alert<6->{Bälle} gespielt.
      \ex Gestern wurden \onslide<6->{\alert{\_}}\ \rot<5->{große} \alert<6->{Bälle} gespielt.
    \end{xlist}
  \end{exe}
  \Zeile
  \pause
  \pause
  \pause
  \centering
  \resizebox{0.4\textwidth}{!}{
    \begin{tabular}{lllllll}
      \toprule
      \multicolumn{3}{l}{} & \textbf{Mask} & \textbf{Neut} & \textbf{Fem} & \textbf{Pl} \\
      \midrule
      \multirow{4}{*}{\textbf{stark}} & \textbf{Nom} & \multirow{4}{*}{heiß-} & er & es & e & e \\
      & \textbf{Akk} && en & es & e & e \\
      & \textbf{Dat} && em & em & er & en \\
      & \textbf{Gen} && en & en & er & er \\
      \midrule
      \multirow{4}{*}{\textbf{schwach}} & \textbf{Nom} & \multirow{4}{*}{(der) heiß-} & e & e & e & en \\
      & \textbf{Akk} && en & e & e & en \\
      & \textbf{Dat} && en & en & en & en \\
      & \textbf{Gen} && en & en & en & en \\
      \midrule
      \multirow{4}{*}{\textbf{gemischt}} & \textbf{Nom} & \multirow{4}{*}{(kein) heiß-} & \Dim er & \Dim es & e & en \\
      & \textbf{Akk} && en & \Dim es & e & en \\
      & \textbf{Dat} && en & en & en & en \\
      & \textbf{Gen} && en & en & en & en \\
      \bottomrule
    \end{tabular}
  }
\end{frame}

\begin{frame}
  {Präpositionen flektieren nicht und regieren Kasus}
  \pause
  \begin{exe}
    \ex
    \begin{xlist}
      \ex{\alert<3->{Mit} \rot<4->{dem kaputten Rasen} ist nichts mehr anzufangen.}
      \pause
      \pause
      \pause
      \ex{\alert<6->{Angesichts} \rot<7->{des kaputten Rasens} wurde das Spiel abgesagt.}
    \end{xlist}
  \end{exe}
  \pause
  \pause
  \pause
  \Zeile
  \begin{block}{Rektion}
    In einer Rektionsrelation werden durch die regierende Einheit (das \alert{Regens}) Werte für bestimmte Merkmale\slash Werte (und damit ggf.\ auch die Form) beim regierten Element (dem \alert{Rectum}) verlangt.\\
  \end{block}
  \Zeile
  \pause
  \begin{block}{Präposition}
    Präpositionen kasusregieren eine obligatorische Nominalphrase.
  \end{block}
\end{frame}

\begin{frame}
  {Komplementierer}
  \pause
  \begin{exe}
    \ex
    \begin{xlist}
      \ex[]{Ich glaube, [\alert<3->{dass} dieser Nebensatz ein Verb \alert<4->{enthält}].}
      \ex[]{[\alert<6->{Während} die Spielzeit \alert<7->{läuft}], zählt jedes Tor.}
      \ex[]{Es fällt ihnen schwer [\rot<8->{zu laufen}].}
      \ex[\rot<11->{*}]{[\alert<9->{Obwohl} kein Tor \alert<10->{fiel}].}
    \end{xlist}
  \end{exe}
  \Zeile
  \pause
  \pause
  \pause
  \pause
  \pause
  \pause
  \pause
  \pause
  \pause
  \pause
  \begin{block}{Komplementierer}
    Komplementierer leiten Nebensätze ein.\\
    Die Rede von der \textit{unterordnenden Konjunktion} ist ungeschickt.
  \end{block}
\end{frame}

\begin{frame}
  {Nicht-flektierbare Wörter im "`Vorfeld"'}
  \pause
  Was steht im unabhängigen Aussagesatz am Satzanfang?\\
  \pause
  {\rot{Antworten Sie nie mehr mit "`das Subjekt"'!}}
  \pause
  \begin{exe}
    \ex\label{ex:adverbenadkopulasundpartikeln038}
    \begin{xlist}
      \ex[ ]{\alert<5->{Gestern} hat der FCR Duisburg gewonnen.}
      \pause
      \pause
      \ex[ ]{\alert<7->{Erfreulicherweise} hat der FCR Duisburg gestern gewonnen.}
      \pause
      \pause
      \ex[ ]{\alert<9->{Oben} finden wir andere Beispiele.}
      \pause
      \pause
      \ex[*]{\alert<11->{Doch} ist das aber nicht das Ende der Saison.}
      \pause
      \pause
      \ex[*]{\alert<13->{Und} ist die Saison zuende.}
      \pause
      \pause
    \end{xlist}
    \ex\label{ex:adverbenadkopulasundpartikeln044} Das ist aber \alert{doch} nicht das Ende der Saison.
  \end{exe}
  \pause
  \Viertelzeile
  \begin{block}{Adverb}
    Adverben sind die übriggebliebenen nicht-flektierbaren Wörter,\\
    die im Vorfeld stehen können.
  \end{block}
\end{frame}


\begin{frame}[fragile]
  {"`Alle Wortklassen"'}
  \pause
  \begin{center}
    \scalebox{0.35}{
    \begin{minipage}{\textwidth}
    \centering
    \begin{forest}
      /tikz/every node/.append style={font=\footnotesize},
      for tree={l sep=2em, s sep=2.5em, align=center},
      [\textit{Wort}, intrme
        [{Hat\\Numerus?}, decide
          [\textit{flektierbar}, intrme, yes
            [{Ist finit\\flektierbar?}, decide
              [\textbf{Verb}, finall, yes]
              [\textit{Nomen}, intrme, no
                [{Hat festes\\Genus?}, decide
                  [\textbf{Substantiv}, finall, yes]
                  [{\textit{anderes}\\\textit{Nomen}}, intrme, no
                    [{Hat Stärke-\\flexion?}, decide
                      [\textbf{Adjektiv}, finall, yes]
                      [{\textit{Artikel\slash}\\\textit{Pronomen}}, intrme, no]
                    ]
                  ]
                ]
              ]
            ]
          ]
          [\textit{nicht flektierbar}, intrme, no
          [{Hat Valenz-\slash\\Kasusrektion?}, decide
              [\textbf{Präposition}, finall, yes]
              [\textit{andere}, intrme, no
                [{Leitet Neben-\\Sätze ein?}, decide
                  [\textbf{Komplementierer}, finall, yes]
                  [{\textit{Partikel\slash}\\\textit{Adverb}}, intrme, no
                    [{Kann das Vor-\\feld besetzen?}, decide
                      [{\textit{Adverb\slash}\\\textit{Adkopula}}, intrme, yes
                        [{Wird typisch mit\\Kopula verwendet?}, decide
                          [\textbf{Adkopula}, finall, yes]
                          [\textbf{Adverb}, finall, no]
                        ]
                      ]
                      [\textit{Partikel}, intrme, no
                        [{Kann Sätze\\ersetzen?}, decide
                          [\textbf{Satzäquivalent}, finall, yes]
                          [\textit{andere}, intrme, no
                            [{Kann Konsti-\\tuenten verbinden?}, decide
                              [\textbf{Konjunktion}, finall, yes]
                              [\textit{Rest}, intrme, no]
                            ]
                          ]
                        ]
                      ]
                    ]
                  ]
                ]
              ]
            ]
          ]
        ]
      ]
    \end{forest}
    \end{minipage}
    }
  \end{center}
\end{frame}


\begin{frame}
  {Wie viele Wortklassen gibt es?}
  \pause
  \begin{itemize}[<+->]
    \item Alle Wörter sind \alert{Wörter}.
    \item Also gibt es \rot{eine Wortklasse}.
      \Zeile
    \item Jedes Wort hat \alert{individuelle Eigenschaften}.
    \item Also gibt es \rot{so viele Wortklassen wie Wörter}.
      \Zeile
    \item Wozu brauchen wir überhaupt Wortklassen? Sie \ldots\\
      \Viertelzeile
      \begin{itemize}[<+->]
        \item \dots sind \alert{die Ausgangsbasis der Morphologie und der Syntax}.
        \item \dots erlauben die Formulierung von \alert{Generalisierungen}.
        \item \dots sind so fein unterteilt, wie es unsere Beschreibung erfordert.
        \item \dots sind \rot{nicht universell}!
        \item \dots sind \alert{Einheiten unserer Theorie bzw.\ Grammatik}.
      \end{itemize}
  \end{itemize}
\end{frame}


\section{Schulaufgaben}

\begin{frame}
  {Ein Beispiel aus \textit{Alles klar!} 7\slash 8}
  Hier soll der Gebrauch von \alert{Adjektiven} geübt werden\ldots\\
  \Zeile
  \begin{center}
    \begin{minipage}{0.15\textwidth}\footnotesize
      \orongsch{\it\textbf{traumhaft}\\
      unvergesslich\\
      besten\\
      bunt\\
      spannend\\
      atemberaubend\\
      toll\\
      gemütlich\\
      riesig\\
      beheizt\\
      nächtlich\\
      groß\\
      interessant}
    \end{minipage}\hspace{0.05\textwidth}\begin{minipage}{0.7\textwidth}\footnotesize\setstretch{1.25}
      \alert{Lies die Anzeige eines Veranstalters für Jugendreisen. Überlege, wohin die Wörter aus der Randspalte passen könnten, und setze sie mit der richtigen Endung ein.}\\

      \textbf{\ul{\textit{Traumhafte}} Reisen mit den \ul{\ \ \ \ \ \ \ \ \ \ \ } Freunden!}\\

      \setstretch{1.25}In der \ul{\ \ \ \ \ \ \ \ \ \ \ }
      Natur der Alpen erwartet euch ein \ul{\ \ \ \ \ \ \ \ \ \ \ }
      Freizeitprogramm: \ul{\ \ \ \ \ \ \ \ \ \ \ }
      Sportturniere, \ul{\ \ \ \ \ \ \ \ \ \ \ }
      Reitausflüge übers Land, \ul{\ \ \ \ \ \ \ \ \ \ \ }
      Wanderungen mit Fackeln, \ul{\ \ \ \ \ \ \ \ \ \ \ }
      Partys in unserer Disko. Wir bieten ein \ul{\ \ \ \ \ \ \ \ \ \ \ }
      Sportgelände mit \ul{\ \ \ \ \ \ \ \ \ \ \ }
      Swimmingpool, einen \ul{\ \ \ \ \ \ \ \ \ \ \ }
      Kletterturm, einen Computerraum und ein eigenes Kino. Das ist doch wesentlich \ul{\ \ \ \ \ \ \ \ \ \ \ }
      , als mit den Eltern in den Urlaub zu fahren, oder? Dieser Urlaub wird bestimmt ein \ul{\ \ \ \ \ \ \ \ \ \ \ }
      Erlebnis!
    \end{minipage}
  \end{center}
  \Viertelzeile
  \tiny \grau{Maempel, Oppenländer \& Scholz. 2012. \textit{Alles klar!} 7\slash 8. Lern- und Übungsheft Grammatik und Zeichensetzung. Berlin: Cornelsen. (Layout ungefähr nachgebaut.)}
\end{frame}


\begin{frame}
  {Warum fehlen hier viele \alert{bildungssprachliche} Arten von Adjektiven?}
  \pause
  \small
  Diese Adjektivklassen fehlen nahezu vollständig in der Aufgabe
  \pause
  \begin{itemize}[<+->]
    \item \alert{temporal} | der \alert{gestrige} Vorfall
    \item \alert{quantifizierend} (relativ, Zählsubstantiv) | \textit{die \alert{zahlreichen} Äpfel}
    \item \alert{quantifizierend} (relativ, Stoffsubstantiv) | \textit{\alert{reichlich} Apfelmus}
    \item \alert{quantifizierend} (absolut) | \textit{die \alert{drei} Bienen}
    \item \alert{intensional} | \textit{der \alert{ehemalige} Präsident}\slash\textit{die \alert{fiktive} Gestalt}
    \item \alert{phorisch} | \textit{die \alert{obigen}}/\textit{\alert{weiteren}}/\textit{\alert{anderen} Ausführungen}
  \end{itemize}
  \pause
  \Halbzeile
  Fällt Ihnen was auf?
  \pause
  \begin{itemize}[<+->]
    \item Das sind im Wesentlichen die, die \rot{nicht prädikativ verwendbar} sind.
    \item Der Wie-Wort-Test basiert aber auf prädikativer Verwendbarkeit.
    \item Aber viele Adjektive sind nicht prädikativ verwendbar. 
  \end{itemize}
\end{frame}

\section{Zur nächsten Woche | Überblick}

\begin{frame}
  {Morphologie und Lexikon des Deutschen | Plan}
  \rot{Alle} angegebenen Kapitel\slash Abschnitte aus \rot{\citet{Schaefer2018b}} sind \rot{Klausurstoff}!\\
  \Halbzeile
  \begin{enumerate}
    \item Grammatik und Grammatik im Lehramt (Kapitel 1 und 3)
    \item Morphologie und Grundbegriffe (Kapitel 2, Kapitel 7 und Abschnitte 11.1--11.2)
    \item Wortklassen als Grundlage der Grammatik (Kapitel 6)
    \item \rot{Wortbildung | Komposition (Abschnitt 8.1)}
    \item Wortbildung | Derivation und Konversion (Abschnitte 8.2 und 8.3)
    \item Flexion | Nomina außer Adjektiven (Abschnitte 9.1--9.3)
    \item Flexion | Adjektive und Verben (Abschnitt 9.4 und Kapitel 10)
    \item Valenz (Abschnitte 2.3, 14.1 und 14.3)
    \item Verbtypen als Valenztypen (Abschnitte 14.4, 14.5, 14.7--14.9) 
    \item Kernwortschatz und Fremdwort (vorwiegend Folien)
  \end{enumerate}
  \Halbzeile
  \centering 
  \url{https://langsci-press.org/catalog/book/224}
\end{frame}


