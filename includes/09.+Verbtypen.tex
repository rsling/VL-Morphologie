\section{Überblick}

\begin{frame}
  {Weitere Unterteilung des Verbwortschatzes}
  \onslide<+->
  \begin{itemize}[<+->]
    \item \alert{Doppelakkusative} und Objektstatus
      \Zeile
    \item \alert{Dative} als Ergänzungen (Objekte)
    \item Dativpassiv als Test
      \Zeile
    \item \alert{Statusrektion} | Modalverben, Halbmodalverben, Hilfsverben
  \end{itemize}
\end{frame}

\section{Objekte und Valenz}

\begin{frame}
  {Terminologische Zuordnung}
  \onslide<+->
  \begin{itemize}[<+->]
    \item \alert{Subjekt} | mit Verb kongruierende Nominativ-Ergänzung
    \item \alert{direktes Objekt} | Akkusativ-Ergänzung eines Verbs
    \item \alert{indirektes Objekt} | Dativ-Ergänzung eines Verbs
    \item \alert{Präpositionalobjekt} | Präpositionsgruppe mit Ergänzungsstatus
      \Zeile
    \item \rot{Nichts davon} ist zwangsläufig immer vorhanden!
      \begin{itemize}[<+->]
        \item \textit{Mir graut.} | \rot{kein Subjekt}
        \item \textit{Der Ballon platzt.} | \rot{kein Objekt}
      \end{itemize}
      \Zeile
    \item \alert{adverbiale Bestimmung} | Angabe zum Verb(?)
  \end{itemize}
\end{frame}

\begin{frame}
  {Direkte Objekte und Doppelakkusative}
  \onslide<+->
  \onslide<+->
  Was ist ein direktes Objekt\slash Akkusativobjekt?
  \begin{itemize}[<+->]
    \item \alert{Akkusativ-Ergänzungen zum Verb}
    \item \alert{oder Nebensätze an deren Stelle}
  \end{itemize}
  \onslide<+->
  \Halbzeile
  Und Doppelakkusative?\\
  \onslide<+->
  \begin{exe}
    \ex\label{ex:akkusativeunddirekteobjekte158}
    \begin{xlist}
      \ex[ ]{\label{ex:akkusativeunddirekteobjekte159} Ich lehre \alert{ihn} \orongsch{das Schwimmen}.}
      \onslide<+->
      \ex[*]{\label{ex:akkusativeunddirekteobjekte160} \orongsch{Das Schwimmen} wird \alert{ihn} gelehrt.}
      \onslide<+->
      \ex[*]{\label{ex:akkusativeunddirekteobjekte161} \alert{Er} wird \orongsch{das Schwimmen} gelehrt.}
      \onslide<+->
      \ex[ ]{\label{ex:akkusativeunddirekteobjekte161} Hier wird \orongsch{das Schwimmen} gelehrt.}
    \end{xlist}
  \end{exe}
  \begin{itemize}[<+->]
    \item beide Akkusative im Passiv nicht nominativfähig
    \item \grau{Korrektur zum Buch: Doppelakkusative bilden unpersönliche Passive.}
  \end{itemize} 
\end{frame}

\section{Dative}

\begin{frame}
  {\textit{bekommen}-Passiv oder Rezipientenpassiv}
  \pause
  Es gibt nicht "`das Passiv im Deutschen"'.\\
  \Halbzeile
  \pause
  \begin{exe}
    \ex\label{ex:bekommenpassiv138}
    \begin{xlist}
      \ex[ ]{\small\label{ex:bekommenpassiv139} \gruen{Mein Kollege} bekommt \orongsch{den Wagen} \alert{(von Johan)} gewaschen.}
      \pause
      \ex[ ]{\small\label{ex:bekommenpassiv140} \gruen{Der Schlossherr} bekommt \orongsch{den Roman} \alert{(von Alma)} geschenkt.}
      \pause
      \ex[ ]{\small\label{ex:bekommenpassiv141} \gruen{Mein Kollege} bekommt \orongsch{den Brief} \alert{(von Johan)} zur Post gebracht.}
      \pause
      \ex[ ]{\small\label{ex:bekommenpassiv142} \gruen{Die Fremden} bekommen \alert{(von dem Maler)} gedankt.}
      \pause
      \ex[?]{\small\label{ex:bekommenpassiv143} \gruen{Mein Kollege} bekommt hier immer montags \alert{(von Johan)} gearbeitet.}
      \pause
      \ex[*]{\small\label{ex:bekommenpassiv144} \gruen{Mein Kollege} bekommt bei zu hohem Druck \rot{(von dem Ball)} geplatzt.}
      \pause
      \ex[*]{\small\label{ex:bekommenpassiv145} \gruen{Michelle} bekommt \rot{(von dem Rottweiler)} aufgefallen.}
    \end{xlist}
  \end{exe}
  \pause\Halbzeile
  \alert{Das ist eine Passivbildung, die genauso den Nom\_Ag betrifft\\
  wie das Vorgangspassiv.}
\end{frame}

\begin{frame}
  {Was passiert beim Rezipientenpassiv?}
  \pause
  Alles, was sich verglichen mit Vorgangspassiv nicht unterscheidet, grau.\\
  \Halbzeile
  \pause
  \begin{itemize}[<+->]
    \item Auxiliar: \textit{bekommen} (evtl.\ \textit{kriegen}), \grau{Verbform: Partizip}
    \item \grau{für Passivierbarkeit relevant: die Nominativ-Ergänzung!}
      \Halbzeile
    \item \grau{Passivierung = Valenzänderung}:
      \begin{itemize}[<+->]
        \item \grau{Nominativ-Ergänzung → optionale \textit{von}-PP-Angabe}
        \item eventuelle Akkusativ-Ergänzung: → Akkusativ-Ergänzung
        \item \alert{Dativ-Ergänzung → Nominativ-Ergänzung}
        \item \rot{kein Dativ: kein Rezipientenpassiv}
        \item \grau{Angaben: keine Änderung}
      \end{itemize}
    \Halbzeile
  \item \grau{nicht passivierbare Verben?}
    \begin{itemize}[<+->]
      \item \grau{ohne agentivische Nominativ-Ergänzung}
      \item \grau{Achtung! Gilt nur mit prototypischem Charakter\ldots}
      \item \grau{Siehe Vertiefung 14.2 auf S.~439!}
    \end{itemize}
  \end{itemize}
\end{frame}

\begin{frame}
  {Rezipientenpassiv bei unergativen Verben}
  \onslide<+->
  \onslide<+->
  Warum war dieser Satz zweifelhaft?\\
  \onslide<+->
  \begin{exe}
    \ex[?]{\small \gruen{Mein Kollege} bekommt hier immer montags \alert{(von Johan)} gearbeitet.}
  \end{exe}
  \onslide<+->
  \Halbzeile
  Ist der zugehörige Aktivsatz besser?\\
  \onslide<+->
  \begin{exe}
    \ex[?]{\small Montags arbeitet \alert{Johan} \gruen{meinem Kollegen} hier immer.}
  \end{exe}
  \begin{itemize}[<+->]
    \item Nein.
    \item \alert{keine Frage des Rezipientenpassivs}
    \item bei diesen Verben: eher \textit{für}-PP
  \end{itemize}
\end{frame}


\begin{frame}
  {Indirekte Objekte}
  \pause
  Welche Dative sind Ergänzungen (= Teil der Valenz)?\\
  \pause
  \Halbzeile
  \begin{exe}
    \ex\label{ex:dativeundindirekteobjekte166}
    \begin{xlist}
      \ex[ ]{\label{ex:dativeundindirekteobjekte167} \alert{Alma} gibt \gruen{ihm} heute ein Buch.}
      \pause
      \ex[ ]{\label{ex:dativeundindirekteobjekte168} \alert{Alma} fährt \gruen{mir} heute aber wieder schnell.}
      \pause
      \ex[ ]{\label{ex:dativeundindirekteobjekte169} \alert{Alma} mäht \gruen{mir} heute den Rasen.}
      \pause
      \ex[ ]{\label{ex:dativeundindirekteobjekte170} \alert{Alma} klopft \gruen{mir} heute auf die Schulter.}
    \end{xlist}
  \end{exe}
  \Halbzeile
  \pause
  Recht einfache Entscheidung, da wir Passiv\\
  als \alert{Valenzänderung} beschreiben:\\
  \pause
  \begin{exe}
    \ex\label{ex:dativeundindirekteobjekte171}
    \begin{xlist}
      \ex[ ]{\label{ex:dativeundindirekteobjekte172} \gruen{Er} bekommt \alert{von Alma} heute ein Buch gegeben.}
      \ex[*]{\label{ex:dativeundindirekteobjekte173} \rot{Ich} bekomme \alert{von Alma} heute aber wieder schnell gefahren.}
      \ex[ ]{\label{ex:dativeundindirekteobjekte174} \gruen{Ich} bekomme \alert{von Alma} heute den Rasen gemäht.}
      \ex[ ]{\label{ex:dativeundindirekteobjekte175} \gruen{Ich} bekomme \alert{von Alma} heute auf die Schulter geklopft.}
    \end{xlist}
  \end{exe}
\end{frame}

\begin{frame}
  {Die vier wichtigen verbabhängigen Dative}
  \pause
  \begin{exe}
    \ex\label{ex:dativeundindirekteobjekte166x}
    \begin{xlist}
      \ex{\label{ex:dativeundindirekteobjekte167x} Alma gibt \gruen{ihm} heute ein Buch.}
      \pause
      \ex{\label{ex:dativeundindirekteobjekte168x} Alma fährt \orongsch{mir} heute aber wieder schnell.}
      \pause
      \ex{\label{ex:dativeundindirekteobjekte169x} Alma mäht \alert{mir} heute den Rasen.}
      \pause
      \ex{\label{ex:dativeundindirekteobjekte170x} Alma klopft \alert{mir} heute auf die Schulter.}
    \end{xlist}
  \end{exe}
  \Halbzeile
  \pause
  \begin{itemize}[<+->]
    \item (\ref{ex:dativeundindirekteobjekte167x}) = \gruen{Ergänzung} bei ditransitivem Verb
    \item (\ref{ex:dativeundindirekteobjekte168x}) = \orongsch{Bewertungsdativ} (Angabe, im Vorfeld\slash direkt nach finitem Verb)
    \item (\ref{ex:dativeundindirekteobjekte169x}) = \alert{Nutznießerdativ} (\alert{Ergänzung per Valenzerweiterung})
    \item (\ref{ex:dativeundindirekteobjekte170x}) = \alert{Pertinenzdativ} (\alert{Ergänzung per Valenzerweiterung})
      \Halbzeile
    \item Bewertungsdativ, Nutznießerdativ und Pertinenzdativ\\
      nennt man auch \textit{freie Dative}.
  \end{itemize}
\end{frame}

\begin{frame}
  {Valenzveränderungen im Beispiel}
  \pause
  1.~Wir beginnen mit einem Verb mit \alert{Nom\_Ag} und einem \orongsch{Akk}:\\
  \pause
  \Halbzeile
  \begin{exe}
    \ex \alert{Alma} mäht \orongsch{den Rasen}.
  \end{exe}
  \Zeile
  \pause
  2.~Der \gruen{Nutznießerdativ} wird als Valenzerweiterung hinzugefügt:\\
  \pause
  \Halbzeile
  \begin{exe}
    \ex \alert{Alma} mäht \gruen{meinem Kollegen} \orongsch{den Rasen}.
  \end{exe}
  \Zeile
  \pause
  3.~Das Rezipientenpassiv (Valenzänderung) kann jetzt gebildet werden:
  \pause
  \Halbzeile
  \begin{exe}
    \ex \gruen{Mein Kollege} bekommt \alert{(von Alma)} \orongsch{den Rasen} gemäht.
  \end{exe}
\end{frame}

\section{Statusrektion}

\begin{frame}
  {Statusrektion | Verben regieren Verben}
  \onslide<+->
  \begin{itemize}[<+->]
    \item bisher | \alert{nominale} und \alert{präpositionale} Objekte
    \item andere Verben | \alert{Statusrektion}, valenzgebundene infinite Verben
      \Zeile
    \item die drei Status des infiniten Verbs
      \begin{itemize}[<+->]
        \item \tuerkis{1.~Status} | reiner Infinitiv (\textit{kaufen})
        \item \gruen{2.~Status} | Infinitiv mit \textit{zu} (\textit{zu kaufen})
        \item \rot{3.~Status} | Partizip
      \end{itemize}
      \Zeile
    \item Die folgende Zusammenfassung ist nicht exhaustiv!
  \end{itemize}
\end{frame}

\begin{frame}
  {Valenzgebundener 3. Status}
  \onslide<+->
  \onslide<+->
  \begin{exe}
    \ex Nadezhda \alert{hat} meine Hantel \rot{signiert}.
    \ex Nadezhda \alert{ist} zur Siegerehrung \rot{gegangen}.
    \Halbzeile
    \onslide<+->
    \ex Nadezhda \orongsch{wurde} mit meiner Hantel \rot{fotografiert}.
  \end{exe}
  \Zeile
  \begin{itemize}[<+->]
    \item \alert{Perfekt-Hilfsverben (\textit{haben}\slash \textit{sein})} regieren \rot{3.~Status}.
    \item Das \orongsch{Passiv-Hilfsverb (\textit{werden})} regiert ebenfalls \rot{3.~Status}.
  \end{itemize}
\end{frame}


\begin{frame}
  {Valenzgebundener 2.~Status}
  \onslide<+->
  \onslide<+->
  \begin{exe}
    \ex Der Hufschmied \alert{beschließt} die Pferde \gruen{zu behufen}.
    \ex Der Hufschmied \alert{wünscht} die Pferde \gruen{zu behufen}.
    \Halbzeile
    \onslide<+->
    \ex Der Hufschmied \orongsch{scheint} die Pferde \gruen{zu behufen}.
  \end{exe}
  \begin{itemize}[<+->]
    \item Sog.\ \alert{Kontrollverben (\textit{beschließen}\slash \textit{wünschen} usw.)} regieren \gruen{2.~Status}.
    \item Sog.\ \orongsch{Halbmodalverben (\textit{scheinen})} regieren ebenfalls \gruen{2.~Status}.
  \end{itemize}
\end{frame}

\begin{frame}
  {Valenzgebundener 1.~Status}
  \onslide<+->
  \onslide<+->
  \begin{exe}
    \ex Der Hufschmied \alert{wird} die Pferde \tuerkis{behufen}.
    \Halbzeile
    \onslide<+->
    \ex Der Hufschmied \orongsch{möchte} die Pferde \tuerkis{behufen}.
    \ex Der Hufschmied \orongsch{kann} die Pferde \tuerkis{behufen}.
  \end{exe}
  \Zeile
  \begin{itemize}[<+->]
    \item Das \alert{Futur-Hilfsverb (\textit{werden})} regiert \tuerkis{1.~Status}.
    \item \orongsch{Modalverben (\textit{dürfen}, \textit{können}, \textit{mögen}, \textit{müssen}, \textit{sollen}, \textit{wollen})}\\
      regieren ebenfalls \tuerkis{1.~Status}.
  \end{itemize}
\end{frame}

\section{Verbklassen}

\begin{frame}
  {Gliederung des verbalen Lexikons I}
  \onslide<+->
  \onslide<+->
  Nominale\slash präpositionale Valenz:\\
  \begin{itemize}[<+->]
    \item \alert{Nominativ-Ergänzung} (Subjekt) oder nicht
    \item \alert{agentivischer Nominativ} oder nicht-agentivisches 
    \item erste \alert{Akkusativergänzung} (Objekt) oder nicht
    \item zweite Akkusativergänzung (Objekt)
    \item \alert{Dativergänzung} (Objekt) oder nicht
    \item \alert{Präpositionalergänzung} (Objekt) oder nicht
  \end{itemize}
\end{frame}

\begin{frame}
  {Gliederung des verbalen Lexikons II}
  \onslide<+->
  \onslide<+->
  Verben auf der Valenzliste\slash Statusrektion:\\
  \begin{itemize}[<+->]
    \item \tuerkis{1.~Status} (Hilfsverben, Modalverben)
    \item \gruen{2.~Status} (Kontrollverben, Halbmodalverben)
    \item \rot{3.~Status} (Hilfsverben)
  \end{itemize}
\end{frame}

\section{Zur nächsten Woche | Überblick}

\begin{frame}
  {Morphologie und Lexikon des Deutschen | Plan}
  \rot{Alle} angegebenen Kapitel\slash Abschnitte aus \rot{\citet{Schaefer2018b}} sind \rot{Klausurstoff}!\\
  \Halbzeile
  \begin{enumerate}
    \item Grammatik und Grammatik im Lehramt (Kapitel 1 und 3)
    \item Morphologie und Grundbegriffe (Kapitel 2, Kapitel 7 und Abschnitte 11.1--11.2)
    \item Wortklassen als Grundlage der Grammatik (Kapitel 6)
    \item Wortbildung | Komposition (Abschnitt 8.1)
    \item Wortbildung | Derivation und Konversion (Abschnitte 8.2 und 8.3)
    \item Flexion | Nomina außer Adjektiven (Abschnitte 9.1--9.3)
    \item Flexion | Adjektive und Verben (Abschnitt 9.4 und Kapitel 10)
    \item Valenz (Abschnitte 2.3, 14.1 und 14.3)
    \item Verbtypen als Valenztypen (Abschnitte 14.4, 14.5, 14.7--14.9) 
    \item \rot{Kernwortschatz und Fremdwort (vorwiegend Folien)}
  \end{enumerate}
  \Halbzeile
  \centering 
  \url{https://langsci-press.org/catalog/book/224}
\end{frame}

