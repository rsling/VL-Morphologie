\section{Überblick}

\begin{frame}
  {Fremdwort und\slash oder Erbwort}
  \onslide<+->
  \begin{itemize}[<+->]
    \item Entlehnung aus anderen Sprachen
    \item Fremdheit ungleich Entlehnung
    \item Definition Kernwortschatz
      \Zeile
    \item \citet{Eisenberg2018}, \citet{Schaefer2018b}\\
      \grau{Die meisten Beispiele hier entnommen aus \citet{Eisenberg2018}.}
      \Zeile
    \item Das Wichtigste für mich ist, dass Sie hier etwas\\
      über den \alert{Kernwortschatz} lernen -- im Kontrast zu den Fremdwörtern.
  \end{itemize}
\end{frame}

\section{Fremdwort}

\begin{frame}
  {Was kommt uns \textit{fremd} vor?}
  \onslide<+->
  \onslide<+->
  \begin{exe}
    \ex Herzmuskelentzündung, Säurebindungsmittel, Nebennierenschwäche
    \onslide<+->
    \Halbzeile
    \ex Hypolyseninsuffizienz, Thyroxintherapie, Osteoporoseminimierung
    \onslide<+->
    \Halbzeile
    \ex Herzrhythmusstörung, Plasmaeiweißbindung, Schilddrüsenunterfunktion
  \end{exe}
  \onslide<+->
  \Zeile
  \alert{Entlehnung} | Das Wort ist im überblickbaren historischen Rahmen\\
  nicht schon immer im Wortschatz, sondern wurde\\
  aus einer Gebersprache übernommen.\\
  \onslide<+->
  \Zeile
  Spielt das wirklich eine Rolle für den Eindruck von \alert{Fremdheit}?
\end{frame}

\begin{frame}
  {Lehn-\slash und Fremdwörter | Welche Wortklassen?}
  \onslide<+->
  \onslide<+->
  Welche Wortklassen…\\
  \Halbzeile
  \begin{itemize}[<+->]
    \item \ldots sind überhaupt \alert{aufnahmefähig}?
    \item \ldots sind mächtig genug für Prototyp und Abweichung?
    \item \ldots haben starke formale Prototypen?
  \end{itemize}
  \onslide<+->
  \Zeile
  \alert{Substantive} > \alert{Adjektive} > \alert{Verben} > \grau{Adverben} > Rest
\end{frame}

\begin{frame}
  {Vorab | Simplicia}
  \onslide<+->
  \onslide<+->
  Das \alert{einfache} Wort…\\
  \Halbzeile
  \begin{itemize}[<+->]
    \item keine erkennbare Ableitung (\alert{Haus}, \rot{häuslich})
    \item keine Komposition (\alert{Tür}, \rot{Türschloss})
    \item bei Verben | ohne Präfix? (\alert{laufen}, \orongsch{verlaufen})
      \Halbzeile
    \item \alert{Wir betrachten hier erstmal nur Simplizia.}
  \end{itemize}
      \Zeile
      \onslide<+->
  \rot{Achtung!} Terminologie!\\
  \Halbzeile
  \begin{itemize}[<+->]
    \item \alert{Simplex} (Singular)
    \item \alert{Simplicia} oder \alert{Simplizia} (Plural)
      \Halbzeile
    \item niemals \rot{*Simplicium} (Singular)
  \end{itemize}
\end{frame}

\section{Kernwortschatz}

\begin{frame}
  {Kernwortschatz | Substantive}
  \onslide<+->
  \onslide<+->
  \begin{exe}
    \ex Baum, Mensch, Strich, Hand, Frist, Buch, Kind
    \Halbzeile
    \onslide<+->
    \ex \alert{Maskulin} | Hase, Falke, Anker, Krater, Hobel, Igel, Graben, Faden
    \onslide<+->
    \ex \alert{Feminin} | Farbe, Hose, Elster, Kelter, Amsel, Sichel
    \onslide<+->
    \ex \alert{Neutral} | Auge, Erbe, Leder, Wasser, Kabel, Rudel, Becken, Wappen
  \end{exe}
  \onslide<+->
  \Zeile
  \begin{itemize}[<+->]
    \item im Singular einsilbig oder
    \item zweisilbige Trochäen, zweite Silbe enthält \alert{Schwa} (<e> bzw.\ [ə])
      \Halbzeile
    \item im Plural immer zweisilbig
  \end{itemize}
\end{frame}

\begin{frame}
  {Kernwortschatz | Adjektive}
  \onslide<+->
  \onslide<+->
  \begin{exe}
    \ex blau, heiß, klein, lang, nackt, schön, stolz, wild
    \onslide<+->
    \ex lose, müde, heiter, mager, edel, nobel, eben, offen
  \end{exe}
  \onslide<+->
  \Zeile
  Eigenschaften?\\
  \onslide<+->
  \Halbzeile
  Und in anderen Formen?
\end{frame}

\begin{frame}
  {Kernwortschatz | Verben}
  \onslide<+->
  \onslide<+->
  \begin{exe}
  \ex baden, denken, leben, schieben, stehen, tragen, wohnen
  \ex rudern, hadern, zetern, bügeln, jubeln, segeln
  \ex atmen, ordnen, öffnen, regnen, zeichnen
  \end{exe}
  \onslide<+->
  \Zeile
  Eigenschaften?\\
  \onslide<+->
  \Halbzeile
  Und in anderen Formen?
\end{frame}

\begin{frame}
  {Kernwortschatz | Lehnwörter, nicht fremd}
  \onslide<+->
  \onslide<+->
  \begin{exe}
    \ex \alert{Englisch} | Akte, Boss, Film, grillen, Lift, Rocker, sponsern, starten, streiken, Stress, tippen, Toner, Tunnel
    \onslide<+->
    \ex \alert{Französisch} | Bluse, Dame, Lärm, Möbel, Mode, nett, nobel, Onkel, Plüsch, Puder, Robe, Soße, Suppe, Tante, Tasse, Torte, Weste
    \onslide<+->
    \ex \alert{Italienisch} | Bank, Barke, Bratsche, Fuge, Kasse, Kurs, Kuppel, Lanze, Liste, Mole, Null, Oper, Paste, Posten, Putte, Reis, Rest
    \onslide<+->
    \ex \alert{Griechisch} | Arzt, Ball, Engel, Fieber, Leier, Ketzer, Kirche, Lesbe, Meter, Pfarrer, Pflaster, Sarg, taufen, Teufel, Tisch, Zone
    \onslide<+->
    \ex \alert{Lateinisch} | Eimer, Esel, Fenster, Kerker, krass, Kreuz, Küche, Mauer, Meile, Mühle, Schule, Straße, Wanne, Wein, Ziegel
    \onslide<+->
    \ex \alert{Hebräisch\slash Jiddisch} | Bammel, dufte, Jubel, Kaff, kotzen, koscher, Nepp, petzen, Ramsch, Zoff
  \end{exe}
\end{frame}

\begin{frame}
  {Fremdwort}
  \onslide<+->
  \onslide<+->
  \centering 
  \alert{Fremdwort} | Fremdwörter sind \alert{nicht im Kern des Systems}.\\
  Sie weichen von den (proto)typischen phonologischen, morphologischen\\
  oder graphematischen Mustern ab, denen die \alert{meisten Wörter} folgen.\\
  \onslide<+->
  \Zeile
  Fremdwörter sind oft intuitiv als \alert{fremd} erkennbar.\\
  \onslide<+->
  \Zeile
  Es gibt \alert{fremde Erbwörter} und \alert{nicht-fremde Lehnwörter}.
\end{frame}

\section{Gradueller Kern}

\begin{frame}
  {Genauer hingeschaut | \textit{Ramsch} usw.}
  \onslide<+->
  \onslide<+->
  Die folgenden Wörter sind nicht im ganz engen Kernwortschatz. Warum?\\
  \Zeile
  \begin{itemize}[<+->]
    \item Bratsche
    \item Bronze
    \item Arzt
    \item Fenster
    \item Ramsch
  \end{itemize}
  \onslide<+->
  \Zeile
  Es kommen jeweils \alert{extrem seltene Konsonantenverbindungen} vor.\\
  \onslide<+->
  Vergleiche \alert{\textit{Mensch}}.
\end{frame}

\begin{frame}
  {Nahe Fremd-\slash Lehnwörter | \textit{quasseln}, \textit{Bagger} usw.}
  \onslide<+->
  \onslide<+->
  Die folgenden Wörter sind Kernwortschatz nach der einfachen Definition.\\
  Wieso sind sie trotzdem ungewöhnlich bzw. vom Kern entfernt?\\
  \Zeile
  \onslide<+->
  \begin{exe}
    \ex Ebbe, Krabbe, kribbeln, Robbe, sabbern, schrubben
    \ex Buddel, Kladde, paddeln, Pudding, Widder
    \ex Bagger, Dogge, Egge, Flagge, Roggen
    \Halbzeile
    \ex quasseln (kontrastiere \textit{prasseln})
  \end{exe}
  \onslide<+->
  \Zeile
  \alert{Stimmhafte Obstruenten am Silbengelenk} sollte es nicht geben.\\
  Siehe Graphematik | Warum \textit{quasseln} besonders schwierig ist.
\end{frame}

\begin{frame}
  {Kern und Peripherie | Abstufungen}
  \onslide<+->
  \onslide<+->
  Was ist an diesen Wörtern etwas fremder als am innersten Kern?\\
  \Halbzeile
  \begin{exe}
    \ex Arbeit, Bischof, Echo, Efeu, Gulasch, Heimat, Oma, Pfirsich, Uhu
    \onslide<+->
    \ex Forelle, Holunder, Hornisse, Kaliber, Kamille, Marone, Maschine
    \onslide<+->
    \ex Ameise, Abenteuer, Akelei, Kehricht, Kleinod, Kobold, Nachtigall\\
    \onslide<+->
    \ex Azur, Bovist, Delfin, Granit, Kanal, Hermelin, Humor, Taifun, Topas
  \end{exe}
  \onslide<+->
  \Zeile
  \alert{Vollvokale} in Nebensilben, \alert{mehr als zwei Silben}, \alert{Pseudokomposita}, \alert{Endsilbenbetonung}.\\
  \Halbzeile
  \onslide<+->
  Welche von diesen Wörtern sind entlehnt?
\end{frame}

\begin{frame}
  {Sind Lehn-\slash Fremdwörter kein Deutsch?}
  \onslide<+->
  \onslide<+->
  Eine Anekdote aus meinem Japanologie-Studium (1998 Bochum):\\
  \Viertelzeile
  \textit{\rot{"`Diphthong ist ein griechisches Wort!} Es wird nach dem Präfix Di- getrennt!"'}\\
  \onslide<+->
  \Viertelzeile
  → \rot{Unsinn!} \grau{Auch wenn die Trennung nach \textit{Di-} bildungssprachlich zu empfehlen ist.}\\
  \onslide<+->
  \Zeile
  Sprechen wir \ldots
  \begin{itemize}[<+->]
    \item \ldots\ Japanisch beim \alert{Sushi}?
    \item \ldots\ Italienisch beim \alert{Cappuccino}?
    \item \ldots\ Französisch beim \alert{Soufflet}?
    \item \ldots\ Englisch beim \alert{Burger}?
  \end{itemize}
  \Halbzeile
  \onslide<+->
  Natürlich nicht. Die Wörter wurden \alert{ins Deutsche entlehnt und sind Deutsch}.\\
  \Viertelzeile
  Auch \alert{Kern und Peripherie} sind nicht mehr oder weniger Deutsch.
\end{frame}

\section{Fremde Wortbildung}

\begin{frame}
  {Lehnwortbildung und Stämme}
  \onslide<+->
  \onslide<+->
  Besonders bei Lehnwortbildungen | Der \alert{Stamm} ist oft selber \alert{nicht wortfähig}.\\
  \Zeile
  \onslide<+->
  \alert{Provider} ist ein deutsches Wort. \onslide<+-> Aber \rot{*provide(n)} ist es nicht.\\
  \Viertelzeile
  \onslide<+->
  Ähnlich ist es bei \alert{Clearing} und \rot{*clear(en)}.\\
  \onslide<+->
  \Zeile
  Inwiefern solche Bildungen als Wortbildungen wahrgenommen werden,\\
  ist schwer und ggf.\ nur im Einzelfall zu entscheiden.
\end{frame}

\begin{frame}
  {Anglizistische Wortbildung | \orongsch{-er}}
  \onslide<+->
  \onslide<+->
  \begin{exe}
    \ex \alert{Kernwörter} | Denker, Fälscher, Leser, Schläger, Turner
    \onslide<+->
    \Zeile
    \ex \alert{Anglizismen} | Beater, Camper, Carrier, Catcher, Dealer, Globetrotter, Hacker, Hitchhiker, Jazzer, Jobber, Jogger, Keeper, Killer, Manager, Producer, Promoter, Provider, Pusher, Surfer, Swinger, User, Walker
  \end{exe}
  \onslide<+->
  \Zeile
  \begin{itemize}[<+->]
    \item Sind die Bildungen \alert{fremd} im Sinn des Nicht-Kerns?
    \item Beziehen Sie sich für Einzelwörter auch auf einzelne der vorkommenden Laute.
  \end{itemize}
\end{frame}

\begin{frame}
  {Anglizistische Wortbildung | \orongsch{-ing}}
  \onslide<+->
  \onslide<+->
  \begin{exe}
    \ex Boarding, Clearing, Coaching, Dumping, Jogging, Mailing, Recycling,
Scratching, Skimming, Shopping, Surfing
    \onslide<+->
    \Halbzeile
    \ex Bodybuilding, Canyoning, Dribbling, Forechecking, Nordic Walking, Slacklining, Tackling, Trekking
  \end{exe}
  \onslide<+->
  \Zeile
  \begin{itemize}[<+->]
    \item Was unterscheidet die erste von der zweiten Gruppe?
    \item Welche Stämme sind \alert{wortfähig}?
    \item Bei wortfähigen Stämmen | Können Sie sich vorstellen,\\
      dass \alert{zuerst das abgeleitete Wort entlehnt wurde} und\\
      der Stamm nachträglich abgetrennt wurde?
  \end{itemize}
\end{frame}

\begin{frame}
  {Einige gallizistische Wortbildungsmuster I}
  \onslide<+->
  \onslide<+->
  \begin{exe}
    \ex \alert{Adjektive auf esk}
    \begin{xlist}
      \ex \small arabesk, balladesk, burlesk, clownesk, gigantesk, karnevalesk, karrikaturesk, pittoresk, romanesk
      \ex \small chaplinesk, dantesk, donjuanesk, godardesk, goyaesk, hoffmannesk, kafkaesk, zappaesk
    \end{xlist}
    \onslide<+->
    \ex \alert{Adjektive auf ös}
    \begin{xlist}
      \ex \small bravourös, desaströs, fibrös, medikamentös, monströs, nervös, pompös, porös, ruinös, schikanös, skandalös, venös, virös
      \ex \small graziös, infektiös, minutiös, sentenziös, tendenziös
      \ex \small bituminös, libidinös, mirakulös, muskulös, nebulös, tuberkulös, voluminös
      \ex \small leprös, kariös, dubiös, ingeniös, kapriziös, luxuriös, melodiös, mysteriös
    \end{xlist}
  \end{exe}
  \onslide<+->
  \Viertelzeile
  Siehe auch Adjektive auf \alert{är}.
\end{frame}

% \begin{frame}
%   {Einige gallizistische Wortbildungsmuster II}
%   \begin{exe}
%   \ex {Adjektive auf är}
%   \begin{xlist}
%     \ex doktrinär, familiär, legendär, reaktionär, sekundär, singulär, stationär, visionär
%     \ex intermediär, konträr, radiär, sekulär
%     \ex muskulär, regulär, zirkulär, zellulär
%     \ex arbiträr, binär, ordinär, pekuniär, sanitär, solitär, subsidiär, temporär
%   \end{xlist}
%     \ex
%   \end{exe}
% \end{frame}

\begin{frame}
  {Einige gallizistische Wortbildungsmuster II}
  \onslide<+->
  \onslide<+->
  \begin{exe}
    \ex \alert{Substantive auf age}
    \begin{xlist}
      \ex \small Blamage, Karambolage, Massage, Montage, Passage, Reportage,\\
      Sabotage, Spionage
      \ex \small Bandage, Collage, Dränage, Etage, Garage, Passage, Plantage,\\
      Reportage, Trikotage
    \end{xlist}
    \onslide<+->
    \ex \alert{Substantive auf eur}
    \begin{xlist}
      \ex \small Akteur, Bankrotteur, Charmeur, Kontrolleur, Parfümeur, Rechercheur
      \ex \small Arrangeur, Chauffeur, Deserteur, Flaneur, Friseur, Hasardeur, Hypnotiseur, Jongleur, Kommandeur, Masseur, Monteur, Saboteur, Souffleur
      \ex \small Installateur, Konstrukteur, Operateur, Provokateur, Redakteur, Restaurateur, Spediteur
    \end{xlist}
  \end{exe}
  \onslide<+->
  \Viertelzeile
  Siehe auch Nomina auf \alert{ee}
\end{frame}

\section{Zur nächsten Woche | Überblick}

\begin{frame}
  {Morphologie und Lexikon des Deutschen | Plan}
  \rot{Alle} angegebenen Kapitel\slash Abschnitte aus \rot{\citet{Schaefer2018b}} sind \rot{Klausurstoff}!\\
  \Halbzeile
  \begin{enumerate}
    \item \rot{Grammatik und Grammatik im Lehramt (Kapitel 1 und 3)}
    \item \rot{Morphologie und Grundbegriffe (Kapitel 2, Kapitel 7 und Abschnitte 11.1--11.2)}
    \item \rot{Wortklassen als Grundlage der Grammatik (Kapitel 6)}
    \item \rot{Wortbildung | Komposition (Abschnitt 8.1)}
    \item \rot{Wortbildung | Derivation und Konversion (Abschnitte 8.2 und 8.3)}
    \item \rot{Flexion | Nomina außer Adjektiven (Abschnitte 9.1--9.3)}
    \item \rot{Flexion | Adjektive und Verben (Abschnitt 9.4 und Kapitel 10)}
    \item \rot{Valenz (Abschnitte 2.3, 14.1 und 14.3)}
    \item \rot{Verbtypen als Valenztypen (Abschnitte 14.4, 14.5, 14.7--14.9)}
    \item \rot{Kernwortschatz und Fremdwort (vorwiegend Folien)}
  \end{enumerate}
  \Halbzeile
  \centering 
  \url{https://langsci-press.org/catalog/book/224}
\end{frame}



