\section{Überblick}

\begin{frame}
  {Wortbildung | Komposition}
  \onslide<+->
  \begin{itemize}[<+->]
    \item Wiederholung | statische und volatile Merkmale
    \item Wiederholung | Wortbildung und Flexion
      \Zeile
    \item Produktivität und Transparenz
    \item Köpfe und Typen von Komposita
    \item Kompositionsfugen
  \end{itemize}
\end{frame}

\section{Wortbildung}


\begin{frame}
  {Wiederholung | Statische und volatile Merkmale}
  \pause
  \begin{itemize}[<+->]
    \item Eigenschaften | "`Rotsein"' (Erdbeere), "`325m hoch"' (Eiffelturm) usw.
    \item Merkmale | \alert{\textsc{Farbe}}, \alert{\textsc{Länge}} usw.
    \item Werte
      \begin{itemize}[<+->]
        \item \alert{\textsc{Farbe}}: \rot{\textit{rot}}, \rot{\textit{grau}}, \ldots
        \item \alert{\textsc{Länge}}: \rot{\textit{3cm}}, \rot{\textit{325m}}, \ldots
      \end{itemize}
  \end{itemize}
  \pause
  \Halbzeile 
  \begin{exe}
    \ex
    \begin{xlist}
      \ex{Haus = [\textsc{Bed}: \gruen<12->{\textbf{\textit{haus}}}, \textsc{Klasse}: \gruen<12->{\textbf{\textit{subst}}}, \textsc{Gen}: \gruen<12->{\textbf{\textit{neut}}}, \textsc{Kas}: \orongsch<13->{\textit{nom}}, \textsc{Num}: \orongsch<13->{\textit{sg}}]}
      \pause
      \ex{Haus-es = [\textsc{Bed}: \gruen<12->{\textbf{\textit{haus}}}, \textsc{Klasse}: \gruen<12->{\textbf{\textit{subst}}}, \textsc{Gen}: \gruen<12->{\textbf{\textit{neut}}}, \textsc{Kas}: \orongsch<13->{\textit{gen}}, \textsc{Num}: \orongsch<13->{\textit{sg}}]}
      \pause
      \ex{Häus-er = [\textsc{Bed}: \gruen<12->{\textbf{\textit{haus}}}, \textsc{Klasse}: \gruen<12->{\textbf{\textit{subst}}}, \textsc{Gen}: \gruen<12->{\textbf{\textit{neut}}}, \textsc{Kas}: \orongsch<13->{\textit{nom}}, \textsc{Num}: \orongsch<13->{\textit{pl}}]}
    \end{xlist}
  \end{exe}
  \Halbzeile
  \pause
  \begin{itemize}[<+->]
    \item bei einem lexikalischen Wort
      \begin{itemize}
        \item \gruen{statische Merkmale} wertestabil
        \item \orongsch{volatile Merkmale} werteverändernd im Paradigma
      \end{itemize}
  \end{itemize}
\end{frame}

\begin{frame}
  {Wiederholung | Wortbildung in Abgrenzung zur Flexion}
  \pause
  \begin{exe}
    \ex
    \begin{xlist}
      \ex trocken (Adj) → \alert{Trocken}\rot{-heit} (Subst)\label{ex:trocken}
      \ex Kauf (Subst), Rausch (Subst) → \alert{Kauf}\rot{-rausch} (Subst)\label{ex:kauf}
      \ex gehen (V) → \alert{be}\rot{-gehen} (V)\label{ex:gehen}
    \end{xlist}
    \pause
    \ex
    \begin{xlist}
      \ex \alert{lauf}\rot{-en} (1\slash 3 Pl Prs Ind) → \alert{lauf}\rot{-e} (1 Sg Prs Ind)\label{ex:lauf}
      \ex \alert{Münze} (Sg) → \alert{Münze}\rot{-n} (Pl)\label{ex:muenze}
    \end{xlist}
  \end{exe}
  \pause
  \Halbzeile
  \begin{itemize}[<+->]
    \item Wortbildung
      \begin{itemize}[<+->]
        \item statische Merkmale geändert | Wortklasse, Bedeutung \alert{(\ref{ex:trocken})}
        \item \ldots oder gelöscht | alles außer der Bedeutung des Erstglieds bei Komposition \alert{(\ref{ex:kauf})}
        \item \ldots oder umgebaut | Valenz von Verben beim Applikativ \alert{(\ref{ex:gehen})}
        \item \orongsch{produktives Erschaffen neuer lexikalischer Wörter}
      \end{itemize}
  \Halbzeile
    \item Flexion
      \begin{itemize}
        \item Änderung der Werte volatiler Merkmale \alert{(\ref{ex:lauf},\ref{ex:muenze})}
        \item \alert{oft Anpassung an syntaktischen Kontext}
      \end{itemize}
  \end{itemize}
\end{frame}



\begin{frame}
  {Wortbildung}
  \onslide<+->
  \begin{itemize}[<+->]
    \item virtuell unbegrenzter Wortschatz
      \Zeile
    \item gut durchschaubares und \alert{gut lernbares} System\\
      \grau{trotz vieler Probleme und Einschränkungen im Detail}
      \Zeile
    \item Funktionen der Wortbildung
      \begin{itemize}
        \item Komposition | \alert{komplexe Konzepte} (\textit{Lötzinnschmelztemperatur})
        \item Konversion | \alert{Reifizierung} (z.B.\ eines Ereignisses als Objekt, \textit{der Lauf})
        \item Derivation | \alert{Modifikation von Bedeutungen} (\textit{\alert{un}schön}),\\
          \alert{Bezug auf Teilaspekte von Konzepten} (z.\,B.\ Ereigniskonzepten, \textit{Fahr\alert{er}})
      \end{itemize}
      \Halbzeile
    \item Hauptproblem der Wortbildung\\
      \rot{Welche Bildungen sind wirklich produktiv?}
  \end{itemize}
\end{frame}


\begin{frame}
  {Wortbildung in der Bildungssprache}
  \pause
  \begin{itemize}[<+->]
    \item Wortbildung als einer der Kerne der Bildungssprache
    \item kann sowohl \alert{verdichten} als auch \alert{präzisieren}
    \Halbzeile
    \item komplexe Sachverhalte \alert{optimiert} formulieren
      \begin{itemize}[<+->]
        \item möglichst kurz
        \item maximal verständlich | Wortbildung hochgradig etabliert\\
          im Deutschen → problemlose Verarbeitung durch Hörer
      \end{itemize}
      \Halbzeile
    \item Aber \rot{das Unterrichten externer Funktionsregularitäten ist besonders\\
      im Fall der Wortbildung extrem schwierig.}
      \Halbzeile
      \begin{itemize}[<+->]
        \item "`Wenn du kommunikativ X erreichen willst, nimm eine Derivation auf \textit{-igkeit}."'
        \item So funktioniert das wohl eher nicht.
        \item Eine allgemeine souveräne \alert{Beherrschung des formalen Systems}\\
          führt zu einer globalen \alert{Optimierung der Schrift- und Bildungssprache}
      \end{itemize}
  \end{itemize}
\end{frame}


\section{Komposition}

\begin{frame}
  {Beispiele für Komposition}
  \onslide<+->
  Komposition | \alert{Stamm\Sub{1} + Stamm\Sub{2} → neuer Stamm\Sub{3}}
  \Halbzeile
  \onslide<+->
  \begin{exe}
    \ex
    \begin{xlist}
      \ex{Kopf.\alert{hörer}}
      \onslide<+->
      \ex{Laut.\alert{sprecher}}
      \onslide<+->
      \ex{Kraft.\alert{werk}}
      \onslide<+->
      \ex{Lehr.\alert{veranstaltung}}
      \onslide<+->
      \ex{Rot.\alert{eiche}}
      \onslide<+->
      \ex{Lauf.\alert{schuhe}}
      \onslide<+->
      \ex{Ess.\alert{besteck}}
      \onslide<+->
      \ex{Fertig.\alert{gericht}}
      \onslide<+->
      \ex{feuer.\alert{rot}}
    \end{xlist}
  \end{exe}
\end{frame}

\begin{frame}
  {Produktivität und Transparenz}
  \onslide<+->
  \begin{itemize}[<+->]
    \item \alert{alle} Beispiele auf der vorherigen Folie \alert{lexikalisiert}
      \begin{itemize}[<+->]
        \item vergleichsweise häufig vorkommende Wörter
        \item überwiegend spezifischere Bedeutung, als Bestandteile vermuten lassen
        \item aber Art der Bildung erkennbar
        \item zumindest für erwachsene Sprecher auch bewusst
      \end{itemize}
      \Halbzeile
    \item \alert{transparent} | Rekonstruierbarkeit der Bildung\\
      (auch bei abweichender Gesamtbedeutung)
      \Halbzeile
    \item \alert{produktiv gebildet} | Neubildung durch Sprecher\\
      in einer gegebenen Situation
    \item Produktivität ist \rot{graduell} aufzufassen!
    \item \orongsch{\textit{Buchbutter}} > \textit{Batterieschublade} > \textit{Laufschuhe} > \gruen{\textit{Hundstage}}
  \end{itemize}
\end{frame}

\begin{frame}[fragile,label=hierarchie]
  {Rekursion}
  \onslide<+->
  \onslide<+->
  \begin{center}
    \scalebox{0.7}{
      \begin{forest}
        [Bushaltestellenunterstandsreparatur
          [Bushaltestellenunterstand
            [Bushaltestelle
              [Bus]
              [Haltestelle
                [halten]
                [Stelle]
              ]
            ]
            [Unterstand
              [unter]
              [Stand]
            ]
          ]
          [Reparatur]
        ]
      \end{forest}
    }
  \end{center}
  \begin{itemize}[<+->]
    \item Wortbildung | immer \alert{binär}, also \alert{Wort+Wort} (nicht \rot{Wort+Wort+Wort} usw.)
      \Viertelzeile
    \item \alert{hierarchische Strukturbildung} durch wiederholte lineare Anfügung
      \Viertelzeile
    \item Rekursion allgemein | \alert{Eine Verknüpfung hat als Ergebnis\\
      eine Einheit, die wieder auf dieselbe Art verknüpft werden kann.}
    \item Rekursion in Linguistik | immer eingeschränkt, nicht "`endlos"'
  \end{itemize}
\end{frame}

\begin{frame}
  {Köpfe}
  \onslide<2->
  \begin{exe}
    \ex
    \begin{xlist}
      \ex \orongsch<19->{Laut}.\alert<10->{sprecher} \onslide<19->{\orongsch{(\textit{laut} verliert Wortklasse, \dots)}}
      \onslide<3->
      \ex \orongsch<20->{Kraft}.\alert<11->{werk} \onslide<20->{\orongsch{(\textit{Kraft} verliert Wortklasse, Genus, \dots)}}
      \onslide<4->
      \ex \orongsch<21->{Lauf}.\alert<12->{schuhe} \onslide<21->{\orongsch{(\textit{laufen} verliert Wortklasse? Genus? \dots)}}
      \onslide<5->
      \ex \orongsch<22->{Ess}.\alert<13->{besteck} \onslide<22->{\orongsch{(\textit{essen} verliert Wortklasse, \dots)}}
      \onslide<6->
      \ex \orongsch<23->{feuer}.\alert<14->{rot} \onslide<23->{\orongsch{(\textit{Feuer} verliert Wortklasse, \dots)}}
    \end{xlist}
  \end{exe}
  \onslide<7->
  \begin{itemize}
    \item \alert{Kopf}
      \begin{itemize}
          \onslide<8->
        \item steht immer rechts
          \onslide<9->
        \item bestimmt alle grammatischen Merkmale des Kompositums
      \end{itemize}
      \Halbzeile
      \onslide<15->
    \item \orongsch{Nicht-Kopf}
      \begin{itemize}
          \onslide<16->
        \item immer links
          \onslide<17->
        \item verliert alle grammatischen Merkmale
          \onslide<18->
        \item Bedeutung geht in Gesamtbedeutung ein
      \end{itemize}
  \end{itemize}
\end{frame}

\begin{frame}
  {Relevante Kompositionstypen | Determinativkomposita}
  \onslide<+->
  Determinativkomposita | \textit{Schulheft}, \textit{Regalbrett} usw.
  \Halbzeile
  \begin{itemize}[<+->]
    \item Kopf--Kern-Test
      \begin{itemize}[<+->]
        \item Ein Schulheft ist ein Heft. \gruen{\Ck}
        \item Ein Regalbrett ist ein Brett. \gruen{\Ck}
      \end{itemize}
    \item Nicht-Kopf--Kern-Test
      \begin{itemize}[<+->]
        \item Ein Schulheft ist eine Schule. \rot{\Fl}
        \item Ein Regalbrett ist ein Regal. \rot{\Fl}
      \end{itemize}
      \Halbzeile
    \item Rektionstest
      \begin{itemize}[<+->]
        \item Bei einem Schulheft heftet\slash verheftet\slash beheftet\ldots jemand eine Schule \rot{\Fl}
        \item Bei einem Regalbrett brettert\slash verbrettert\dots jemand ein Regal \rot{\Fl}
      \end{itemize}
  \end{itemize}
\end{frame}


\begin{frame}
  {Relevante Kompositionstypen | Rektionskomposita}
  \onslide<+->
  Objekt-Rektionskomposita | \textit{Hemdenwäsche}, \textit{Geldfälschung} usw.
  \Halbzeile
  \begin{itemize}[<+->]
    \item Kopf--Kern-Test
      \begin{itemize}[<+->]
        \item Eine Hemdenwäsche ist eine Wäsche. \rot{\Ck}
        \item Eine Geldfälschung ist eine Fälschung. \rot{\Ck}
      \end{itemize}
    \item Nicht-Kopf--Kern-Test
      \begin{itemize}[<+->]
        \item Eine Hemdenwäsche ist ein Hemd. \rot{\Fl}
        \item Eine Geldfälschung ist Geld. \rot{\Fl}
      \end{itemize}
      \Halbzeile
    \item Objekt-Rektionstest
      \begin{itemize}[<+->]
        \item Bei einer Hemdenwäsche werden Hemden gewaschen. \gruen{\Ck}
        \item Bei einer Geldfälschung wird Geld gefälscht. \gruen{\Ck}
      \end{itemize}
      \Halbzeile
    \item Kopf | oft mit \alert{-ung} usw. von einem Verb abgeleitet
    \item Nicht-Kopf zu Kopf wie \alert{Objekt} zu Verb
  \end{itemize}
\end{frame}


\begin{frame}
  {Relevante Kompositionstypen | Rektionskomposita}
  \onslide<+->
  Subjekt-Rektionskomposita | \textit{Hemdenwäscher}, \textit{Geldfälscher} usw.
  \Halbzeile
  \begin{itemize}[<+->]
    \item Kopf--Kern-Test
      \begin{itemize}[<+->]
        \item Ein Hemdenwäscher ist eine Wäsche. \gruen{\Ck}
        \item Ein Geldfälscher ist eine Fälschung. \gruen{\Ck}
      \end{itemize}
    \item Nicht-Kopf--Kern-Test
      \begin{itemize}[<+->]
        \item Ein Hemdenwäscher ist ein Hemd. \rot{\Fl}
        \item Ein Geldfälscher ist Geld. \rot{\Fl}
      \end{itemize}
      \Halbzeile
    \item Subjekt-Rektionstest
      \begin{itemize}[<+->]
        \item Ein Hemdenwäscher wäscht Hemden. \gruen{\Ck}
        \item Ein Geldfälscher fälscht Geld. \gruen{\Ck}
      \end{itemize}
      \Halbzeile
    \item Kopf | meistens mit \alert{\textit{-er}} von einem Verb abgeleitet
    \item Nicht-Kopf zu Kopf wie \alert{Subjekt} zu Verb
  \end{itemize}
\end{frame}

\begin{frame}
  {Kompositionsfugen bei Substantiv-Substantiv-Komposita}
  \onslide<+->
  \onslide<+->
  \begin{center}
    \scalebox{0.9}{
      \begin{tabular}{llrr}
        \toprule
        Fuge          & Beispiel                        & Komposita \% & Erstglieder \% \\
        \midrule                                                                                                    
        \alert{$\varnothing$} & \textit{Garten.tür}             & \alert{60.25}        & \alert{41.77}  \\ 
        \alert{-(e)s}         & \textit{Gelegenheit-s.dieb}     & \alert{23.69}        & \alert{45.74}  \\ 
        \alert{-n}            & \textit{Katze-n.pfote}          & \alert{10.38}        &  \rot{5.29}  \\ 
        \rot{-en}             & \textit{Frau-en.stimme}         &  \rot{3.02}          &  \rot{4.19}  \\ 
        \rot{*e}              & \textit{Kirsch.kuchen}          &  \rot{0.78}          &  \rot{0.20}  \\ 
        \rot{-e}              & \textit{Geschenk-e.laden}       &  \rot{0.71}          &  \rot{1.90}  \\ 
        \rot{-er}             & \textit{Kind-er.buch}           &  \rot{0.38}          &  \rot{0.07}  \\ 
        \rot{\char`~er}       & \textit{Büch-er.regal}          &  \rot{0.37}          &  \rot{0.11}  \\ 
        \rot{\char`~e}        & \textit{Händ-e.druck}           &  \rot{0.22}          &  \rot{0.63}  \\ 
        \rot{-ns}             & \textit{Name-ns.schutz}         &  \rot{0.13}          &  \rot{0.04}  \\ 
        \rot{\char`~}         & \textit{Mütter.zentrum}         &  \rot{0.05}          &  \rot{0.06}  \\ 
        \rot{-ens}            & \textit{Herz-ens.angelegenheit} &  \rot{0.03}          &  \rot{0.01}  \\ 
        \bottomrule
      \end{tabular}
    }\\
    \Halbzeile
    \grau{\tiny{(aus \citealt{SchaeferPankratz2018})}}
  \end{center}
\end{frame}

\begin{frame}
  {Steuerung der Fugen durch Erstglied}
  \onslide<+->
  \begin{itemize}[<+->]
    \item Wörter mit s-Plural (\textit{Kaffees}, \textit{Kameras}) \rot{niemals mit s-Fuge}
      \Halbzeile
    \item \alert{derivierte} Substantive (meist Abstrakta) (\textit{-heit}, \textit{-keit}, \textit{-tum}) \alert{prototypisch s-Fuge}
      \begin{itemize}[<+->]
        \item sehr viele Feminina, Fuge nicht paradigmatisch (= keine Flexionsform)
      \end{itemize}
      \Halbzeile
    \item starke\slash gemischte Maskulina | manchmal -(\textit{e})\textit{s}
      \begin{itemize}[<+->]
        \item Genitiv? Welche Funktion sollte ein Genitiv im Kompositum haben?
        \item Lassen sich die Komposita mit s-Fuge mit Genitiv umformulieren?
        \item \textit{Freundeskreis → \rot{*Kreis des Freundes}}
        \item \textit{Geschlechtsverkehr → \rot{*Verkehr des Geschlechts}}
        \item \textit{Berufstätigkeit → \rot{*Tätigkeit des Berufs}}
        \item \textit{Auslandsaufenthalt → \rot{*Aufenthalt des Auslands}}
      \end{itemize}
    \Halbzeile
  \item die s-Fugen an \alert{Feminina} sowieso nicht als Genitiv möglich
      \begin{itemize}
        \item \textit{Gelegenheitsdieb} → \rot{*\textit{Dieb der Gelegenheits}}
      \end{itemize}
  \end{itemize}
\end{frame}

\section{Zur nächsten Woche | Überblick}

\begin{frame}
  {Morphologie und Lexikon des Deutschen | Plan}
  \rot{Alle} angegebenen Kapitel\slash Abschnitte aus \rot{\citet{Schaefer2018b}} sind \rot{Klausurstoff}!\\
  \Halbzeile
  \begin{enumerate}
    \item Grammatik und Grammatik im Lehramt (Kapitel 1 und 3)
    \item Morphologie und Grundbegriffe (Kapitel 2, Kapitel 7 und Abschnitte 11.1--11.2)
    \item Wortklassen als Grundlage der Grammatik (Kapitel 6)
    \item Wortbildung | Komposition (Abschnitt 8.1)
    \item \rot{Wortbildung | Derivation und Konversion (Abschnitte 8.2--8.3)}
    \item Flexion | Nomina außer Adjektiven (Abschnitte 9.1--9.3)
    \item Flexion | Adjektive und Verben (Abschnitt 9.4 und Kapitel 10)
    \item Valenz (Abschnitte 2.3, 14.1 und 14.3)
    \item Verbtypen als Valenztypen (Abschnitte 14.4--14.5, 14.7--14.9) 
    \item Kernwortschatz und Fremdwort (vorwiegend Folien)
  \end{enumerate}
  \Halbzeile
  \centering 
  \url{https://langsci-press.org/catalog/book/224}
\end{frame}


