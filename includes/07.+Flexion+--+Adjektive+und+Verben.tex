\section{Überblick}

\begin{frame}
  {Flexion | Verben}
  \onslide<+->
  \begin{itemize}[<+->]
    \item Adjektivflexion | stark, schwach, gemischt?
      \Zeile
    \item Funktion in der Flexion der Verben
    \item Flexion stark\slash schwach
      \begin{itemize}[<+->]
        \item Ablaut
        \item Person\slash Numerus
        \item Tempus
        \item Modus
      \end{itemize}
  \end{itemize}
\end{frame}

\section{Adjektive}

\begin{frame}
  {Adjektive | Das traditionelle Chaos}
  \pause
  \begin{center}
    \resizebox{0.5\textwidth}{!}{
    \begin{tabular}{lllllll}
      \toprule
      \multicolumn{3}{l}{} & \textbf{Mask} & \textbf{Neut} & \textbf{Fem} & \textbf{Pl} \\
      \midrule
      \multirow{4}{*}{\rot<4->{\textbf{stark}}} & \textbf{Nom} & \multirow{4}{*}{\rot<4->{$\emptyset$} heiß-} & \rot<4->{er} & \rot<4->{es} & \rot<4->{e} & \rot<4->{e} \\
      & \textbf{Akk} && \rot<4->{en} & \rot<4->{es} & \rot<4->{e} & \rot<4->{e} \\
      & \textbf{Dat} && \rot<4->{em} & \rot<4->{em} & \rot<4->{er} & \rot<4->{en} \\
      & \textbf{Gen} && \rot<4->{en} & \rot<4->{en} & \rot<4->{er} & \rot<4->{er} \\
      \midrule
      \multirow{4}{*}{\alert<5->{\textbf{schwach}}} & \textbf{Nom} & \multirow{4}{*}{\alert<5->{der} heiß-} & \alert<5->{e} & \alert<5->{e} & \alert<5->{e} & \alert<5->{en} \\
      & \textbf{Akk} && \alert<5->{en} & \alert<5->{e} & \alert<5->{e} & \alert<5->{en} \\
      & \textbf{Dat} && \alert<5->{en} & \alert<5->{en} & \alert<5->{en} & \alert<5->{en} \\
      & \textbf{Gen} && \alert<5->{en} & \alert<5->{en} & \alert<5->{en} & \alert<5->{en} \\
      \midrule
      \multirow{4}{*}{\gruen<6->{\textbf{gemischt}}} & \textbf{Nom} & \multirow{4}{*}{\gruen<6->{kein} heiß-} & \gruen<6->{er} & \gruen<6->{es} & \gruen<6->{e} & \gruen<6->{en} \\
      & \textbf{Akk} && \gruen<6->{en} & \gruen<6->{es} & \gruen<6->{e} & \gruen<6->{en} \\
      & \textbf{Dat} && \gruen<6->{en} & \gruen<6->{en} & \gruen<6->{en} & \gruen<6->{en} \\
      & \textbf{Gen} && \gruen<6->{en} & \gruen<6->{en} & \gruen<6->{en} & \gruen<6->{en} \\
      \bottomrule
    \end{tabular}
  }
  \end{center}
  \pause
  \Halbzeile
  \begin{itemize}[<+->]
    \item "`Merke"' (oder vielleicht auch nicht)
      \begin{itemize}[<+->]
        \item \rot{ohne} Artikel | \rot{starkes} Adjektiv
        \item mit \alert{definitem} Artikel | \alert{schwaches} Adjektiv
        \item mit \gruen{indefinitem} Artikel | \gruen{gemischtes} Adjektiv
      \end{itemize} 
  \end{itemize} 
\end{frame}


\begin{frame}
  {Ohne Artikelwort | Adjektive flektieren fast wie Artikelwort}
  \pause
  \centering 
    \scalebox{0.9}{\begin{tabular}{llp{2em}ll}
      \toprule
      dies\alert{-er} & Kaffee  && heiß\alert{-er}   & Kaffee  \\
      dies\alert{-en} & Kaffee  && heiß\alert{-en}   & Kaffee  \\
      dies\alert{-em} & Kaffee  && heiß\alert{-em}   & Kaffee  \\
      dies\rot{-es}   & Kaffees && heiß\rot{-en}     & Kaffee\textbf<3->{\orongsch<3->{s}} \\
      \midrule
      dies\alert{-es} & Dessert && heiß\alert{-es}   & Dessert \\
      dies\alert{-em} & Dessert && heiß\alert{-em}   & Dessert \\
      dies\rot{-es}   & Desserts&& heiß\rot{-en}     & Dessert\textbf<3->{\orongsch<3->{s}} \\
      \midrule
      dies\alert{-e}  & Brühe   && lecker\alert{-e}  & Brühe \\
      dies\alert{-er} & Brühe   && lecker\alert{-er} & Brühe \\
      \midrule
      dies\alert{-e}  & Kekse   && heiß\alert{-e}    & Kekse \\
      dies\alert{-en} & Keksen  && heiß\alert{-en}   & Keksen\\
      dies\alert{-er} & Kekse   && heiß\alert{-er}   & Kekse \\
      \bottomrule
    \end{tabular}}\\
  \pause
  \Zeile 
  \alert{Fällt Ihnen was auf?}
\end{frame}

\begin{frame}
  {Artikelwort mit normalen Affixen | "`adjektivische"' Flexion}
  \pause
  \begin{center}
    \begin{tabular}{lll}
      \toprule
      dies\alert{-er} & lecker\grau{-e}  & Kaffee   \\
      dies\alert{-en} & lecker\rot{-en} & Kaffee   \\
      dies\alert{-em} & lecker\grau{-en} & Kaffee   \\
      dies\alert{-es} & lecker\grau{-en} & Kaffees  \\
      \midrule
      dies\alert{-es} & lecker\grau{-e}  & Dessert  \\
      dies\alert{-em} & lecker\grau{-en} & Dessert  \\
      dies\alert{-es} & lecker\grau{-en} & Desserts \\
      \midrule
      dies\alert{-e}  & lecker\grau{-e}  & Brühe    \\
      dies\alert{-er} & lecker\grau{-en} & Brühe    \\
      \midrule
      dies\alert{-e}  & lecker\grau{-en} & Kekse    \\
      dies\alert{-en} & lecker\grau{-en} & Kekse    \\
      dies\alert{-er} & lecker\grau{-en} & Kekse    \\
      \bottomrule
    \end{tabular}
  \end{center}
\end{frame}

\begin{frame}
  {Die adjektivische Flexion}
  \pause
  Fast perfekte systeminterne Funktionsoptimierung\\
  \Zeile
  \pause
  \begin{center}
    \begin{tabular}{|l|llll|}
      \cline{2-5}
      \multicolumn{1}{c|}{}& \textbf{Mask} & \textbf{Neut} & \textbf{Fem} & \textbf{Pl} \\
      \hline
      \textbf{Nom} && \multirow{2}{*}{-e} & \multicolumn{1}{c|}{} & \\ \cline{2-2}
      \textbf{Akk} & \multicolumn{1}{c|}{-en} && \multicolumn{1}{c|}{} & \\ \cline{2-4}
      \textbf{Dat} &&& \multirow{2}{*}{-en} & \\
      \textbf{Gen} &&&& \\
      \hline
    \end{tabular}
  \end{center}
  \pause
  "`Zielsystem"'\\
  \begin{center}
    \begin{tabular}{|l|c|c|}
      \cline{2-3}
      \multicolumn{1}{c|}{} & \multicolumn{1}{c|}{\textbf{Singular}} & \multicolumn{1}{c|}{\textbf{Plural}} \\
      \hline
      \multicolumn{1}{|c|}{\textbf{strukturell}} & \multirow{2}{*}{-e} &  \\
      \multicolumn{1}{|c|}{\textbf{$-$ Akk Mask}} &  &  \\
      \cline{1-2}
      \multicolumn{1}{|c|}{\textbf{oblique}} & \multicolumn{1}{c}{} & \multirow{2}{*}{-en} \\
      \multicolumn{1}{|c|}{\textbf{$+$ Akk Mask}} & \multicolumn{1}{c}{} & \\
      \hline
    \end{tabular}
  \end{center}
\end{frame}


\begin{frame}
  {Gemischt?}
  \pause
  Die Besonderheiten des Indefinit- und Possessivartikels treffen\\
  auf die Regularitäten der Adjektivflexion!
  \pause
  \begin{center}
    \scalebox{0.9}{
      \begin{tabular}{lll}
        \toprule
        mein-$\emptyset$\hspace{2em}\onslide<4->{\rot{\HandCuffRight}} & lecker\rot{-er}   & Kaffee   \\
        mein\alert{-en} & lecker\grau{-en} & Kaffee   \\
        mein\alert{-em} & lecker\grau{-en} & Kaffee   \\
        mein\alert{-es} & lecker\grau{-en} & Kaffees  \\
        \midrule
        mein-$\emptyset$\hspace{2em}\onslide<4->{\rot{\HandCuffRight}} & lecker\rot{-es}   & Dessert  \\
        mein\alert{-em} & lecker\grau{-en} & Dessert  \\
        mein\alert{-es} & lecker\grau{-en} & Desserts \\
        \midrule
        mein\alert{-e}  & lecker\grau{-e}  & Brühe    \\
        mein\alert{-er} & lecker\grau{-en} & Brühe    \\
        \midrule
        mein\alert{-e}  & lecker\grau{-en} & Kekse    \\
        mein\alert{-en} & lecker\grau{-en} & Kekse    \\
        mein\alert{-er} & lecker\grau{-en} & Kekse    \\
        \bottomrule
      \end{tabular}
    }
  \end{center}
  \pause
\end{frame}

\begin{frame}[fragile]
  {Das System}
  \pause
  \begin{center}
    \begin{tikzpicture}[every text node part/.style={align=center}]
      \node [draw, chamfered rectangle] (FlexAVoran) at (3,6) {Geht ein Artikelwort\\mit Flexionsendung voraus?};

      \node [rounded corners, fill=gray] (SchwaFlexi) at (0,3) {\whyte{struktureller Singular \textit{-e}}\\\whyte{Rest \textit{-en}}};
      \node [rounded corners, fill= gray] (pronomAffi) at (6,3) {\whyte{pronominale}\\\whyte{Affixe}};

      \node [draw, rounded corners, inner sep=6pt] (AdFlexAusn) at (3,0) {\textit{-en}};

      \draw (FlexAVoran) -- node [above, sloped] {\footnotesize Ja}       (SchwaFlexi);
      \draw (FlexAVoran) -- node [above, sloped] {\footnotesize Nein}     (pronomAffi);
      \draw [dashed] (SchwaFlexi) -- node [above, sloped] {\footnotesize Ausnahme} node [below, sloped] {\footnotesize Akkusativ Maskulinum} (AdFlexAusn);
      \draw [dashed] (pronomAffi) -- node [above, sloped] {\footnotesize Ausnahme Genitiv} node [below, sloped] {\footnotesize Maskulinum\slash Neutrum} (AdFlexAusn);
    \end{tikzpicture}
  \end{center}
\end{frame}





\section{Verben}

\begin{frame}
  {Flexionsklassen der Verben}
  \onslide<+->
  \onslide<+->
  Welche Klassen von Verben haben eigene Flexionsmuster?
  \begin{itemize}[<+->]
    \item \alert{schwache} Verben (die meisten)
    \item \alert{starke} Verben (\alert{Vokalstufen}, nicht nur Ablaut)
    \item "`gemischte"' Verben (wenn es sein muss)
      \Halbzeile
    \item Modalverben (Präteritalpräsentien)
    \item Hilfsverben und Kopulaverben (suppletiv oder idiosynkratisch)
  \end{itemize}
  \onslide<+->
  \Zeile
  Was sind die Markierungsfunktionen der Affixe in der Verbalflexion?
  \begin{itemize}[<+->]
    \item Person und Numerus
    \item Tempus
    \item Modus
      \Halbzeile
    \item Infinitheit (verschiedene Sorten)
  \end{itemize}
\end{frame}

\begin{frame}
  {Flexionstypen von Vollverben}
  \pause
  \begin{center}
    \resizebox{0.9\textwidth}{!}{
      \begin{tabular}{llllll}
        \toprule
         & \textbf{2-stufig} & \textbf{3-stufig} & \textbf{U3-stufig} & \textbf{4-stufig} & \textbf{schwach} \\
        \midrule
        \textbf{1 Pers Präs} & h\alert{e}b-e  & spr\alert{i}ng-e     & l\alert{au}f-e     & br\alert{e}ch-e     & l\alert{a}ch-e \\
        \textbf{2 Pers Präs} & h\alert{e}b-st & spr\alert{i}ng-st    & l\orongsch{äu}f-st  & br\orongsch{i}ch-st & l\alert{a}ch-st \\
        \textbf{1 Pers Prät} & h\rot{o}b      & spr\rot{a}ng         & l\rot{ie}f         & br\rot{a}ch         & l\alert{a}ch-te \\
        \textbf{Partizip} & ge-h\rot{o}b-en   & ge-spr\gruen{u}ng-en & ge-l\alert{au}f-en & ge-br\gruen{o}ch-en & ge-l\alert{a}ch-t \\
        \bottomrule
      \end{tabular}
    }
  \end{center}
\end{frame}


\begin{frame}
  {Flexion in den beiden Tempora}
  \pause
  \begin{center}
    \resizebox{0.65\textwidth}{!}{
      \begin{tabular}{llll}
        \toprule
        && \multicolumn{2}{c}{\textbf{schwach}} \\
        \multicolumn{2}{c}{} & \textbf{Präsens} & \textbf{Präteritum} \\
        \midrule
        \multirow{3}{*}{\textbf{Singular}} & \textbf{1} & lach\orongsch<4->{-(e)} & lach\alert<8->{-te} \\
        & \textbf{2} & lach-st & lach\alert<8->{-te}-st \\
        &\textbf{3} & lach\orongsch<5->{-t} & lach\alert<8->{-te}\orongsch<5->{-$\emptyset$}\\
        \midrule
        \multirow{3}{*}{\textbf{Plural}} & \textbf{1} & lach-en & lach\alert<8->{-te}-n \\
        & \textbf{2} & lach-t & lach\alert<8->{-te}-t \\
        & \textbf{3} & lach-en & lach\alert<8->{-te}-n \\
        \bottomrule
      \end{tabular}~\begin{tabular}{ll}
        \toprule
        \multicolumn{2}{c}{\textbf{stark}} \\
        \textbf{Präsens} & \textbf{Präteritum} \\
        \midrule
        br\gruen<7->{e}ch\orongsch<4->{-(e)} & br\gruen<7->{a}ch \\
        br\gruen<7->{i}ch-st & br\gruen<7->{a}ch-st \\
        br\gruen<7->{i}ch\orongsch<5->{-t} & br\gruen<7->{a}ch\orongsch<5->{-$\emptyset$} \\
        \midrule
        br\gruen<7->{e}ch-en & br\gruen<7->{a}ch-en \\
        br\gruen<7->{e}ch-t & br\gruen<7->{a}ch-t \\
        br\gruen<7->{e}ch-en & br\gruen<7->{a}ch-en \\
        \bottomrule
      \end{tabular}
    }
  \end{center}
  \pause
  \Halbzeile
  \begin{itemize}[<+->]
    \item Person-Numerus
      \begin{itemize}[<+->]
        \item erste Singular \orongsch{\textit{-(e)}} nur im Präsens
        \item dritte Singular \orongsch{\textit{-t}} nur im Präsens
      \end{itemize}
    \item Präteritum
      \begin{itemize}[<+->]
        \item mit \gruen{Vokalstufe} (stark)
        \item mit Affix \alert{\textit{-te}} (schwach)
      \end{itemize} 
  \end{itemize}
\end{frame}



\begin{frame}
  {Person-Numerus-Affixe}
  \onslide<+->
  \onslide<+->
  \begin{center}
    \begin{tabular}{llcc}
      \toprule
      \multicolumn{2}{c}{} & \textbf{PN1} & \textbf{PN2} \\
      \midrule
      \multirow{3}{*}{\textbf{Singular}} & \textbf{1} & -(e) & \Dim \\
        & \textbf{2} & \multicolumn{2}{c}{-st} \\
        & \textbf{3} & -t & \Dim \\
      \midrule
      \multirow{2}{*}{\textbf{Plural}} & \textbf{1/3} & \multicolumn{2}{c}{-en} \\
        & \textbf{2} & \multicolumn{2}{c}{-t} \\
      \bottomrule
    \end{tabular}
  \end{center}
  \onslide<+->
  \Zeile
  Mehr gibt es im ganzen System nicht.
\end{frame}

\begin{frame}
  {Konjunktiv}
  \pause
  \begin{center}
   \resizebox{0.65\textwidth}{!}{
      \begin{tabular}{llll}
        \toprule
        && \multicolumn{2}{c}{\textbf{schwach}} \\
        \multicolumn{2}{c}{} & \textbf{Präsens} & \textbf{Präteritum} \\
        \midrule
        \multirow{3}{*}{\textbf{Singular}} & \textbf{1} & lach\alert<6->{-e} & lach-t\alert<6->{-e} \\
        & \textbf{2} & lach\alert<6->{-e}\orongsch<4->{-st} & lach-t\alert<6->{-e}\orongsch<4->{-st} \\
        & \textbf{3} & lach\alert<6->{-e} & lach-t\alert<6->{-e} \\
        \midrule
        \multirow{3}{*}{\textbf{Plural}} & \textbf{1} & lach\alert<6->{-e}\orongsch<4->{-n} & lach-t\alert<6->{-e}\orongsch<4->{-n} \\
        & \textbf{2} & lach\alert<6->{-e}\orongsch<4->{-t} & lach-t\alert<6->{-e}\orongsch<4->{-t} \\
        & \textbf{3} & lach\alert<6->{-e}\orongsch<4->{-n} & lach-t\alert<6->{-e}\orongsch<4->{-n} \\
        \bottomrule
      \end{tabular}~\begin{tabular}{ll}
        \toprule
        \multicolumn{2}{c}{\textbf{stark}} \\
        \textbf{Präsens} & \textbf{Präteritum} \\
        \midrule
        brech\alert<6->{-e} & br\rot<5->{ä}ch\alert<6->{-e} \\
        brech\alert<6->{-e}\orongsch<4->{-st} & br\rot<5->{ä}ch\alert<6->{-e}\orongsch<4->{-st} \\
        brech\alert<6->{-e} & br\rot<5->{ä}ch\alert<6->{-e} \\
        \midrule
        brech\alert<6->{-e}\orongsch<4->{-n} & br\rot<5->{ä}ch\alert<6->{-e}\orongsch<4->{-n} \\
        brech\alert<6->{-e}\orongsch<4->{-t} & br\rot<5->{ä}ch\alert<6->{-e}\orongsch<4->{-t} \\
        brech\alert<6->{-e}\orongsch<4->{-n} & br\rot<5->{ä}ch\alert<6->{-e}\orongsch<4->{-n} \\
        \bottomrule
      \end{tabular}
    }
  \end{center}
  \pause
  \begin{itemize}[<+->]
    \item unabhängig von Funktion | Präsens und Präteritum
    \item immer \orongsch<4->{PN2}
    \item wenn möglich \rot<5->{Umlaut} bei starken Verben
    \item immer \alert<6->{-e} nach Stamm bzw.\ Stamm\textit{-t}(\textit{e})
  \end{itemize}
\end{frame}


\begin{frame}
  {Infinite Formen}
  \onslide<+->
  \onslide<+->
  Kein Tempus, keine Person, keinen Numerus, keinen Modus \ldots\\\
  \gruen{werden aber von anderen Verben (z.\,B.\ Modalverben, Hilfsverben) gefordert}.\\

  \onslide<+->
  \Halbzeile
  \begin{center}
    \scalebox{0.7}{
      \begin{tabular}{lp{13em}p{13em}}
%        \toprule
        & \textbf{Infinitiv} & \textbf{Partizip} \\
%        \midrule
        \textbf{schwach} & lach\alert{-en} & ge-lach\alert{-t} \\
        \textbf{stark} & brech\alert{-en} & ge-broch\alert{-en} \\
%        \bottomrule
      \end{tabular}
    }

    \onslide<+->
    \Zeile
    \scalebox{0.7}{
      \begin{tabular}{lp{13em}p{13em}}
%        \toprule
        & \textbf{Infinitiv} & \textbf{Partizip} \\
%        \midrule
        \textbf{schwach} & Stamm + \textit{en} & (\textit{ge}) + Stamm + \textit{t} \\
        \textbf{stark} & Präsensstamm + \textit{en} & (\textit{ge}) + Partizipstamm + \textit{en} \\
%        \bottomrule
      \end{tabular}
    }

    \onslide<+->
    \vspace{2.5\baselineskip}
    \raggedright
    Partizipien bei Präfixverben und Partikelverben\\
    \onslide<+->
    \Halbzeile
    \centering
    \scalebox{0.7}{
      \begin{tabular}{lp{13em}p{13em}}
%        \toprule
        & \textbf{Präfixverb} & \textbf{Partikelverb} \\
%        \midrule
        \textbf{schwach} & \hspace{0em} \rot{\textbf{ver:}}lach\alert{-t} & \hspace{0em} \rot{\textbf{aus=ge-}}lach\alert{-t} \\
        \textbf{stark} & \hspace{0em} \rot{\textbf{unter:}}broch\alert{-en} & \hspace{0em} \rot{\textbf{ab=ge-}}broch\alert{-en} \\
%        \bottomrule
      \end{tabular}
    }
  \end{center}
\end{frame}

\begin{frame}
  {Weitere Arten von Verben}
  \onslide<+->
  \onslide<+->
  Hilfs- und Modalverben mit besonderer Syntax und besonderer Formenbildung
  \onslide<+->
  \Halbzeile
  \begin{exe}
    \ex\label{ex:unterklassen072}
    \begin{xlist}
      \ex{\label{ex:unterklassen073} Frida \alert<9->{isst} den Marmorkuchen.}
      \onslide<+->
      \ex{\label{ex:unterklassen074} Frida \orongsch<10->{hat} den Marmorkuchen \alert<9->{gegessen}.}
      \onslide<+->
      \ex{\label{ex:unterklassen075} Der Marmorkuchen \orongsch<10->{wird} \alert<9->{gegessen}.}
      \onslide<+->
      \ex{\label{ex:unterklassen076} Frida \rot<11->{soll} den Marmorkuchen \alert<9->{essen}.}
      \onslide<+->
      \ex{\label{ex:unterklassen077} Dies hier \gruen<12->{ist} der leckere Marmorkuchen.}
      \onslide<+->
      \ex{\label{ex:unterklassen078} Der Marmorkuchen \gruen<12->{wird} lecker.}
    \end{xlist}
  \end{exe}
  \onslide<+->
  \Halbzeile
  \centering 
  \onslide<9->{\alert{Vollverben\slash lexikalische Verben}}\onslide<10->{, \orongsch{Hilfsverben}}\onslide<11->{, \rot{Modalverben}}\onslide<12->{, \gruen{Kopulaverben}}
\end{frame}

\begin{frame}
  {Modalverben}
  \onslide<+->
  \onslide<+->
  \alert{Modalverben} | verlangen ein weiteres Verb im Infinitiv, flektieren anders\\
  \onslide<+->
  \Zeile
  \centering 
  \begin{tabular}{llllllll}
    \toprule
    \multirow{2}{*}{\textbf{Sg}} & \textbf{1/\rot<5->{3}} & darf & kann & mag & muss & soll & will \\
    & \textbf{2} & darf-st & kann-st & mag-st & muss-t & soll-st & will-st \\
    \midrule
    \multirow{2}{*}{\textbf{Pl}} & \textbf{1/3} & d\gruen<4->{ü}rf-en & k\gruen<4->{ö}nn-en & m\gruen<4->{ö}g-en & m\gruen<4->{ü}ss-en & soll-en & w\gruen<4->{o}ll-en \\
    & \textbf{2} & d\gruen<4->{ü}rf-t & k\gruen<4->{ö}nn-t & m\gruen<4->{ö}g-t & m\gruen<4->{ü}ss-t & soll-t & w\gruen<4->{o}ll-t \\
    \bottomrule
  \end{tabular}
  \Zeile
  \begin{itemize}[<+->]
    \item \gruen{Ablautstufe mit Umlaut für Präsens Plural}
    \item \rot{kein Affix für 3.~Person Singular Präsens, daher 1.~Person gleich 3.~Person}
    \item historisch Präteritalformen reinterpretiert | \alert{Präteritalpräsentien}
    \item neues Präteritum, schwach gebildet (\textit{durf-te}, \textit{konn-te} usw.)
  \end{itemize}
\end{frame}

\begin{frame}
  {Und was war eigentlich mit den anderen Tempora?}
  \onslide<+->
  \onslide<+->
  Die Schulgrammatik lehrt \alert{sechs Tempusformen}, wir nur \rot{zwei}.\\
  \onslide<+->
  \Zeile
  \begin{center}
    \begin{tabular}[h]{lll}
      \textbf{Präsens}         & \textit{es \alert{geht}}                                     & \onslide<4->{\alert{synthetisch }} \\
      \textbf{Präteritum}      & \textit{es \alert{ging}}                                     & \onslide<4->{\alert{synthetisch }} \\
      && \\
      \textbf{Futur}         & \textit{es \orongsch{wird} \alert{gehen}}                    & \onslide<5->{\orongsch{analytisch }} \\
      && \\
      \textbf{Perfekt}         & \textit{es \orongsch{ist} \alert{gegangen}}                  & \onslide<5->{\orongsch{analytisch }} \\
      \textbf{Plusquamperfekt} & \textit{es \orongsch{war} \alert{gegangen}}                  & \onslide<5->{\orongsch{analytisch }} \\
      \textbf{Futurperfekt}         & \textit{es \orongsch{wird} \alert{gegangen} \orongsch{sein}} & \onslide<5->{\orongsch{analytisch }} \\
    \end{tabular}
  \end{center}
  \Zeile
  \begin{itemize}[<+->]
    \item Nur zwei werden als Form (\alert{synthetisch}) gebildet.
    \item Der Rest wird mit \orongsch{Hilfsverben} und \alert{infiniten Verbformen} (\orongsch{analytisch}) gebildet.
  \end{itemize}
\end{frame}

\begin{frame}
  {Präsens, Präteritum, Futur}
  \onslide<+->
  \begin{itemize}[<+->]
    \item Präsens
      \begin{itemize}[<+->]
        \item kein spezifischer Zeitbezug
        \item synthetische finite Form
      \end{itemize}
      \Viertelzeile
    \item Präteritum
      \begin{itemize}[<+->]
        \item Vergangenheitsbezug
        \item synthetische finite Form
      \end{itemize}
     \Viertelzeile 
    \item Futur
      \begin{itemize}[<+->]
        \item Zukunftsbezug oder Absichtserklärung
        \item analytische Form mit \rot{stets finitem} Hilfsverb
      \end{itemize}
  \end{itemize}
  \onslide<+->
  \Halbzeile
  \hspace{3em}\scalebox{0.8}{\begin{minipage}{\textwidth}
    \begin{exe}
      \onslide<11->{\ex[ ]{\ldots\ dass ich \alert{gehen werde}.}}
      \onslide<12->{\ex[*]{\ldots\ dass ich \rot{gehen werden} möchte.}}
      \onslide<13->{\ex[*]{\ldots\ dass ich \rot{gehen geworden} habe\slash bin.}}
      \onslide<14->{\ex[*]{\ldots\ dass ich \rot{gehen zu werden} habe.}}
    \end{exe}
  \end{minipage}}
\end{frame}

\begin{frame}
  {Perfekt}
  \onslide<+->
  \onslide<+->
  Form\\
  \Viertelzeile
  \begin{itemize}[<+->]
    \item Hilfsverb \orongsch{sein} oder \orongsch{haben} + \alert{Partizip} des anderen Verbs
      \Halbzeile
    \item Infinitiv des Perfekts | \alert{gegangen} (Partizip) \orongsch{sein} (Inf des HVs)
    \item Präsens des Perfekts | \alert{gegangen} (Partizip) \orongsch{bin\slash bist\slash ist\slash\ldots} (Präs des HVs)
    \item Präteritum des Perfekts | \alert{gegangen} (Partizip) \orongsch{war\slash warst\slash\ldots} (Prät des HVs)
    \item Futur des Perfekts | \alert{gegangen} (Partizip) \orongsch{sein werde\slash wirst\slash wird\slash\ldots} (Futur des HVs)
  \end{itemize}
  \onslide<+->
  \Halbzeile
  Funktion\\
  \Viertelzeile
  \begin{itemize}[<+->]
    \item Vergangenheitsbezug | Präsensperfekt oft austauschbar mit Präteritum
    \item bei Austauschbarkeit oft umgangssprachlich verglichen mit Präteritum
    \item Zusatzbedeutung der Abgeschlossenheit bei bestimmten semantischen Verbtypen
      \begin{itemize}[<+->]
        \item \textit{Im Jahr 1993 \alert{zerstörte} der Kommerz den Techno.} \grau{| nicht doppeldeutig}
        \item \textit{Im Jahr 1993 \orongsch{hat} der Kommerz den Techno \alert{zerstört}.} \grau{| doppeldeutig}
      \end{itemize}
  \end{itemize}
\end{frame}

\begin{frame}
  {Zusammenfassung | Finite Tempora und Perfekt}
  \onslide<+->
  \onslide<+->
  Klare Beziehungen zwischen den finiten Tempora und dem Perfekt\\
  \Zeile
  \begin{itemize}[<+->]
    \item Finite Tempora
      \begin{itemize}[<+->]
        \item Präsens | finite synthetische Form
        \item Präteritum | finite synthetische Form
        \item Futur (= Futur 1) | analytisch mit stets finitem Hilfsverb
      \end{itemize}
     \Zeile 
    \item \alert{Perfekta mit finiten Tempusformen des Hilfsverbs}
      \begin{itemize}[<+->]
        \item Präsensperfekt (= Perfekt) | Präsensform des Perfekts
        \item Präteritumsperfekt (= Plusquamperfekt) | Präteritalform des Perfekts
        \item Futurperfekt (= Futur 2) | Futur des Perfekts
      \end{itemize}
  \end{itemize}
  
\end{frame}

\ifdefined\TITLE
  \section{Zur nächsten Woche | Überblick}

  \begin{frame}
    {Morphologie und Lexikon des Deutschen | Plan}
    \rot{Alle} angegebenen Kapitel\slash Abschnitte aus \rot{\citet{Schaefer2018b}} sind \rot{Klausurstoff}!\\
    \Halbzeile
    \begin{enumerate}
      \item Grammatik und Grammatik im Lehramt (Kapitel 1 und 3)
      \item Morphologie und Grundbegriffe (Kapitel 2, Kapitel 7 und Abschnitte 11.1--11.2)
      \item Wortklassen als Grundlage der Grammatik (Kapitel 6)
      \item Wortbildung | Komposition (Abschnitt 8.1)
      \item Wortbildung | Derivation und Konversion (Abschnitte 8.2--8.3)
      \item Flexion | Nomina außer Adjektiven (Abschnitte 9.1--9.3)
      \item Flexion | Adjektive und Verben (Abschnitt 9.4 und Kapitel 10)
      \item \rot{Valenz (Abschnitte 2.3, 14.1 und 14.3)}
      \item Verbtypen als Valenztypen (Abschnitte 14.4--14.5, 14.7--14.9) 
      \item Kernwortschatz und Fremdwort (vorwiegend Folien)
    \end{enumerate}
    \Halbzeile
    \centering 
    \url{https://langsci-press.org/catalog/book/224}
  \end{frame}
\fi

