\section{Überblick}

\begin{frame}
  {Funktionale Wortschatzgliederung bei Verben}
  \onslide<+->
  \begin{itemize}[<+->]
    \item bisher | \alert{morphologisch motivierte} Gliederung des Lexikons
    \item \zB\ Pluralklassen bei Substantiven
      \Zeile
    \item weitere Gliederung | \alert{morphosyntaktisch-funktional}
    \item inbesondere \alert{Verbklassen}
      \Halbzeile
      \begin{itemize}[<+->]
        \item \alert{passivierbare} Verben
        \item \alert{Valenzklassen} (transitiv, intransitiv etc.)
        \item Verben mit Präpositionalobjekten
        \item \ldots\ nur ein Ausschnitt der möglichen Klassen
      \end{itemize}
  \end{itemize}
\end{frame}

\section{Valenz}

\begin{frame}
  {Ergänzungen und Angaben}
  \onslide<+->
  \onslide<+->
  \begin{exe}
    \ex\label{ex:valenz034}
    \begin{xlist}
      \ex{Gabriele malt \alert{[ein Bild]}.}
      \onslide<+->
      \ex{Gabriele malt \orongsch{[gerne]}.}
      \onslide<+->
      \ex{Gabriele malt \orongsch{[den ganzen Tag]}.}
      \onslide<+->
      \ex{Gabriele malt \orongsch{[ihrem Mann]} \rot{[zu figürlich]}.}
    \end{xlist}
  \end{exe}
  \Halbzeile
  \begin{itemize}[<+->]
    \item \alert{[ein Bild]} mit besonderer Relation zum Verb | \alert{Objekt\slash Ergänzung}
    \item keine solche Relation bei den anderen | \orongsch{Adverbial\slash Angaben}
    \item "`Weglassbarkeit"' (Optionalität) nicht entscheidend
  \end{itemize}
\end{frame}

\begin{frame}
  {Lizenzierung}
  \pause
  \begin{exe}
    \ex 
    \begin{xlist}
      \ex[ ]{Gabriele isst \orongsch{[den ganzen Tag]} Walnüsse.}
    \pause
      \ex[ ]{Gabriele läuft \orongsch{[den ganzen Tag]}.}
      \pause
      \ex[ ]{Gabriele backt ihrer Schwester \orongsch{[den ganzen Tag]} Stollen.}
      \pause
      \ex[ ]{Gabriele litt \orongsch{[den ganzen Tag]} unter Sonnenbrand.}
    \end{xlist}
    \pause\Halbzeile
    \ex 
    \begin{xlist}
      \ex[*]{Gabriele isst \alert{[ein Bild]} Walnüsse.}
      \pause
      \ex[*]{Gabriele läuft \alert{[ein Bild]}.}
      \pause
      \ex[*]{Gabriele backt ihrer Schwester \alert{[ein Bild]} Stollen.}
      \pause
      \ex[*]{Gabriele litt \alert{[ein Bild]} unter Sonnenbrand. }
      \pause
    \end{xlist}
  \end{exe}
  \pause\Halbzeile
  \begin{itemize}[<+->]
    \item \orongsch{Angaben} sind verb-unspezifisch lizenziert
    \item \alert{Ergänzungen} sind verb(klassen)spezifisch lizenziert
    \item \gruen{Valenz = Liste der Ergänzungen eines lexikalischen Worts}
  \end{itemize}
\end{frame}


\begin{frame}
  {Weitere Eigenschaften von Ergänzungen und Angaben}
  \pause
  \alert{Iterierbarkeit} (= Wiederholbarkeit) von Angaben, nicht Ergänzungen\\
  \pause
  \Halbzeile
  \begin{exe}
    \ex[ ]{Wir müssen den Wagen \orongsch{[jetzt]} \orongsch{[mit aller Kraft]} \orongsch{[vorsichtig]} anschieben.}
    \pause
    \ex[ ]{Wir essen \orongsch{[schnell]} \orongsch{[mit Appetit]} \orongsch{[an einem Tisch]}\\
      \orongsch{[mit der Gabel]} \alert{[einen Salat]}.}
    \pause
    \ex[*]{Wir essen \orongsch{[schnell]} \rot{[ein Tofugericht]} \orongsch{[mit Appetit]} \orongsch{[an einem Tisch]}\\
      \orongsch{[mit der Gabel]} \alert{[einen Salat]}.}
  \end{exe}
\end{frame}


\begin{frame}
  {Ergänzungen | Schnittstelle von Syntax und Semantik}
  \onslide<+->
  \onslide<+->
  Verbsemantik | Welche \alert{Rolle} spielen die von den Satzgliedern bezeichneten Dinge\\
  in der vom Verb beschriebenen Situation?\\
  \Zeile
  \onslide<+->
  Semantik von \alert{Ergänzungen} | \alert{abhängig} vom Verb\\
  \onslide<+->
  \Viertelzeile
  Semantik von \gruen{Angaben} | \gruen{unabhängig} vom Verb\\
  \Halbzeile
  \pause
  \begin{exe}
    \ex\label{ex:valenz071}
    \begin{xlist}
      \ex{\label{ex:valenz072}Ich lösche \alert{[den Ordner]} \gruen{[während der Hausdurchsuchung]}.}
      \pause
      \ex{\label{ex:valenz073}Ich mähe \alert{[den Rasen]} \gruen{[während der Ferien]}.}
      \pause
      \ex{\label{ex:valenz074}Ich fürchte \alert{[den Sturm]} \gruen{[während des Sommers]}.}
    \end{xlist}
  \end{exe}
\end{frame}

\begin{frame}
  {Valenz}
  \onslide<+->
  \onslide<+->
  \begin{block}{Angaben}
    \alert{Angaben} sind grammatisch immer lizenziert und bringen\\
    ihre eigene semantische Rolle mit.\\
    \Halbzeile
    \grau{Sie können aber semantisch\slash pragmatisch inkompatibel sein.}
  \end{block}
  \Zeile
  \onslide<+->
  \begin{block}{Ergänzungen}
    \gruen{Ergänzungen} werden spezifisch vom Verb lizenziert und in ihrer semantischen Rolle\\
    vom Verb festgelegt. Jede dieser Rollen kann nur einmal vergeben werden.
  \end{block}
\end{frame}


\section{Rollen}

\begin{frame}
  {Was sind "`Rollen"'}
  \pause
  \begin{exe}
    \ex
    \begin{xlist}
      \ex{\alert{Michelle} kauft einen Rottweiler.}
      \pause
      \ex{\alert{Der Rottweiler} schläft.}
      \pause
      \ex{\alert{Der Rottweiler} erfreut Marina.}
    \end{xlist}
  \end{exe}
  \pause
  \Halbzeile
  \begin{itemize}[<+->]
    \item semantische Generalisierung über \alert{Käuferin}, \alert{Schläfer}, \alert{Erfreuer}?
    \item \rot{"`Das Subjekt drückt aus, wer oder was im Satz handelt."' --- Unsinn!}
    \item Nur die \alert{Käuferin} handelt!
      \Halbzeile
    \item Verben als Kodierung eines \alert{Situationstyps} 
    \item Situationstypen mit charakteristischen \alert{Mitspielern}
    \item Handelnde, Betroffene, Veränderte, Emotionen Erfahrende, \ldots
    \item "`Mitspieler"' im weiteren Sinn, auch Gegenstände, Zeitpunkte usw.
      \Halbzeile
    \item Gleichsetzung von Rollen mit Kasus \rot{absoluter Unsinn}
  \end{itemize}
\end{frame}

\begin{frame}
  {Agens und Experiencer}
  \pause
  \begin{exe}
    \ex
    \begin{xlist}
      \ex{\alert{Michelle} kauft \orongsch{einen Rottweiler}.}
      \ex{\orongsch{Der Rottweiler} schläft.}
      \ex{\orongsch{Der Rottweiler} erfreut \rot{Marina}.}
    \end{xlist}
  \end{exe}
  \pause
  \Halbzeile
  \begin{itemize}[<+->]
    \item Rollen in den Beispielen
      \begin{itemize}[<+->]
        \item \alert{Michelle} → Handelnde = \alert{Agens}
        \item \rot{Marina} → psychischen Zustand Erfahrende: \rot{Experiencer}
        \item \orongsch{Rottweiler} → andere Rollen, hier nicht weiter analysiert (Rx)
      \end{itemize}
  \end{itemize}
\end{frame}

\begin{frame}
  {Rollenzuweisung\ldots\ und Ergänzungen und Angaben}
  \pause
  \begin{itemize}[<+->]
    \item für einen Situationstyp charakteristische Rollen?
      \Viertelzeile
    \item (fast) \alert{immer} \zB
      \begin{itemize}[<+->]
        \item \alert{Zeitpunkt}
        \item \alert{Ort}
        \item \alert{Dauer}
      \end{itemize}
      \Viertelzeile
    \item \rot{nicht immer} \zB
      \begin{itemize}[<+->]
        \item \rot{Handelnde} (\textit{schlafen}, \textit{fallen}, \textit{gefallen}, \ldots)
        \item \rot{psychischen Zustand Erfahrende} (\textit{laufen}, \textit{reparieren}, \textit{häkeln}, \ldots)
        \item \rot{physisch Veränderte} (\textit{betrachten}, \textit{belassen}, \textit{verkaufen}, \ldots)
      \end{itemize}
      \Viertelzeile
    \item Auch wenn Kaufen, Fallen usw.\ Emotionen auslöst:\\
      \alert{Das jeweilige Verb (\textit{kaufen}, \textit{fallen} usw.) sagt darüber nichts aus!}
      \Viertelzeile
    \item \rot{Ergänzung}: gekoppelt an \alert{verbspezifische} Rolle 
    \item \alert{Angabe}: gekoppelt an \alert{verbunspezifische} Rolle
%    \item Ergänzung = subklassenspezifische Lizenzierung
  \end{itemize}
\end{frame}

\begin{frame}
  {Das Prinzip der Rollenzuweisung}
  \pause
  \begin{itemize}[<+->]
    \item situationsspezifische Rollen: \alert{nur einmal vergebbar}\\
    = Prinzip der Rollenzuweisung
      \Halbzeile
    \item semantische Motivation für:
      \begin{itemize}[<+->]
        \item Angaben sind iterierbar,
        \item Ergänzungen nicht.
      \end{itemize}
      \Halbzeile
    \item und \alert{Koordinationen}?
  \end{itemize}
  \pause
  \begin{exe}
    \ex \alert{Marina und Michelle} kaufen bei \rot{einer seriösen Züchterin\\
    und ihrer Freundin} einen \orongsch{Dobermann und einen Rottweiler}.
  \end{exe}
  \pause
  \begin{itemize}[<+->]
    \item semantisch: Summenindividuen o.\,ä.
    \item \alert{Grammatik und Semantik untrennbar, gegenseitig bedingend}
  \end{itemize}
\end{frame}

\section{Passive}

\begin{frame}
  {Valenzänderungen | Vorbemerkung}
  \onslide<+->
  \onslide<+->
  \alert{Wir beschreiben Passivbildung als Valenzänderung\ldots}\\
  \Halbzeile
  \begin{itemize}[<+->]
    \item im Prinzip eine Art von \alert{Wortbildung}
    \item Valenz von \textit{kaufen} \{\alert{Nominativ-NP\Sub{1}}, \orongsch{Akkusativ-NP\Sub{2}}\}\\
      → Valenz des Passivs von \textit{kaufen} \{\orongsch{Nominativ-NP\Sub{2}}\}
      \Halbzeile
    \item andere Wortbildungsprozesse mit Valenzänderungen
      \begin{itemize}[<+->]
        \item Valenzanreicherung beim Applikativ \textit{be:}
        \item \textit{geh-en} → \textit{be:geh-en}
        \item Valenzänderung \{Nominativ-NP\Sub{1}\} → \{Nominativ-NP\Sub{1}, Akkusativ-NP\Sub{2}\}
        \item \textit{Ich gehe auf der Straße.} → \textit{Ich begehe die Straße.}
      \end{itemize}
  \end{itemize}
\end{frame}

\begin{frame}
  {\textit{werden}-Passiv oder Vorgangspassiv}
  \pause
  "`Nur transitive Verben können passiviert werden."'\pause\rot{--- Nein!}
  \pause
    \begin{exe}
    \addtolength\itemsep{-0.25\baselineskip}
      \ex\label{ex:werdenpassivundverbtypen110}
      \begin{xlist}\addtolength\itemsep{-0.5\baselineskip}
          \ex[ ]{\label{ex:werdenpassivundverbtypen111} \alert{Johan} wäscht \orongsch{den Wagen}.}
          \ex[ ]{\label{ex:werdenpassivundverbtypen112} \orongsch{Der Wagen} wird \alert{(von Johan)} gewaschen.}
      \end{xlist}
      \pause
      \ex\label{ex:werdenpassivundverbtypen113}
      \begin{xlist}\addtolength\itemsep{-0.5\baselineskip}
          \ex[ ]{\label{ex:werdenpassivundverbtypen114} \alert{Alma} schenkt \gruen{dem Schlossherrn} \orongsch{den Roman}.}
          \ex[ ]{\label{ex:werdenpassivundverbtypen115} \orongsch{Der Roman} wird \gruen{dem Schlossherrn} \alert{(von Alma)} geschenkt.}
      \end{xlist}
      \pause
      \ex\label{ex:werdenpassivundverbtypen116}
      \begin{xlist}\addtolength\itemsep{-0.5\baselineskip}
          \ex[ ]{\label{ex:werdenpassivundverbtypen117} \alert{Johan} bringt \orongsch{den Brief} zur Post.}
          \ex[ ]{\label{ex:werdenpassivundverbtypen118} \orongsch{Der Brief} wird \alert{(von Johan)} zur Post gebracht.}
      \end{xlist}
      \pause
      \ex\label{ex:werdenpassivundverbtypen119}
      \begin{xlist}\addtolength\itemsep{-0.5\baselineskip}
          \ex[ ]{\label{ex:werdenpassivundverbtypen120} \alert{Der Maler} dankt \gruen{den Fremden}.}
          \ex[ ]{\label{ex:werdenpassivundverbtypen121} \gruen{Den Fremden} wird \alert{(vom Maler)} gedankt.}
      \end{xlist}
      \pause
      \ex\label{ex:werdenpassivundverbtypen122}
      \begin{xlist}\addtolength\itemsep{-0.5\baselineskip}
          \ex[ ]{\label{ex:werdenpassivundverbtypen123} \alert{Johan} arbeitet hier immer montags.}
          \ex[ ]{\label{ex:werdenpassivundverbtypen124} Montags wird hier \alert{(von Johan)} immer gearbeitet.}
      \end{xlist}
      \pause
      \ex\label{ex:werdenpassivundverbtypen125}
      \begin{xlist}\addtolength\itemsep{-0.5\baselineskip}
          \ex[ ]{\label{ex:werdenpassivundverbtypen126} \alert{Der Ball} platzt bei zu hohem Druck.}
          \ex[*]{\label{ex:werdenpassivundverbtypen127} Bei zu hohem Druck wird \rot{(vom Ball)} geplatzt.}
      \end{xlist}
      \pause
      \ex\label{ex:werdenpassivundverbtypen128}
      \begin{xlist}\addtolength\itemsep{-0.5\baselineskip}
          \ex[ ]{\label{ex:werdenpassivundverbtypen129} \alert{Der Rottweiler} fällt \gruen{Michelle} auf.}
          \ex[*]{\label{ex:werdenpassivundverbtypen130} \alert{Michelle} wird \rot{(von dem Rottweiler)} aufgefallen.}
      \end{xlist}
    \end{exe}
\end{frame}

\begin{frame}
  {Was passiert beim Vorgangspassiv?}
  \pause
  \begin{itemize}[<+->]
    \item Auxiliar: \textit{werden}, Verbform: Partizip
    \item für Passivierbarkeit relevant: \alert{die Nominativ-Ergänzung!}
      \Halbzeile
    \item \alert{Passivierung als Valenzänderung}:
      \begin{itemize}[<+->]
        \item Nominativ-Ergänzung → optionale \textit{von}-PP-Angabe
        \item eventuelle Akkusativ-Ergänzung → obligatorische Nominativ-Ergänzung
        \item kein Akkusativ: kein "`Subjekt"' = keine Nom-Erg (\textit{es} ist positional)
        \item \grau{Dativ-Ergänzung → Dativ-Ergänzung (usw.)}
        \item \grau{Angaben: keine Änderung}
      \end{itemize}
    \Halbzeile
  \item \alert{nicht passivierbare Verben}?
    \begin{itemize}[<+->]
      \item {ohne }\rot{agentivische}\alert{ Nominativ-Ergänzung}
      \item Achtung! Gilt nur mit prototypischem Charakter\ldots
      \item Siehe Vertiefung 14.2 auf S.~439!
    \end{itemize}
  \end{itemize}
\end{frame}

\begin{frame}
  {Feinere Klassifikation von Verben}
  \pause
  \begin{itemize}[<+->]
    \item Neuklassifikation vor dem Hintergrund des Vorgangspassivs
    \item Wenn so eine Klassifikation einen Wert haben soll:\\
      \alert{Berücksichtigung der semantischen Rollen unabdinglich!}
    \item Bedingung für Vorgangs-Passiv: \alert{Nom\_Ag}
  \end{itemize} 
  \pause
  \Zeile
  \centering
  \scalebox{0.9}{\begin{tabular}{lllll}
    \toprule
    \textbf{Valenz} & \textbf{Passiv} & \textbf{Name} & \textbf{Beispiel} \\
    \midrule
    \alert{Nom\_Ag} & ja & Unergative & \textit{arbeiten} \\
    Nom & nein & Unakkusative & \textit{platzen} \\
    \alert{Nom\_Ag}, Akk & ja & Transitive & \textit{waschen} \\
    \alert{Nom\_Ag}, Dat & ja & unergative Dativverben & \textit{danken} \\
    Nom, Dat & nein & unakkusative Dativverben & \textit{auf"|fallen} \\
    \alert{Nom\_Ag}, Dat, Akk & ja & Ditransitive & \textit{geben} \\
    \bottomrule
  \end{tabular}}\\
  \raggedright
  \Zeile
  \pause
  Immer noch nichts als eine reine Bequemlichkeitsterminologie,\\
  um bestimmte (durchaus wichtige) Valenzmuster hervorzuheben.
\end{frame}

\section{Verben mit Präpositionalobjekten}

\begin{frame}
  {Präpositionalobjekte}
  \pause
  PP-Angabe vs.\ PP-Ergänzung: oft schwierig zu entscheiden.\\
  \Viertelzeile
  \pause
  \begin{exe}
    \ex\label{ex:ppergaenzungenundppangaben189}
    \begin{xlist}
      \ex{\label{ex:ppergaenzungenundppangaben190} Viele Menschen leiden \alert{unter Vorurteilen}.}
      \pause
      \ex{\label{ex:ppergaenzungenundppangaben191} Viele Menschen schwitzen \orongsch{unter Sonnenschirmen}.}
    \end{xlist}
  \end{exe}
  \Viertelzeile
  \pause
  \begin{itemize}[<+->]
    \item \alert{Ergänzungen}:
      \begin{itemize}[<+->]
        \item Semantik der PP nur verbgebunden interpretierbar
        \item = semantische Rolle der PP vom Verb zugewiesen
      \end{itemize}
    \item \orongsch{Angaben}:
      \begin{itemize}[<+->]
        \item Semantik der PP selbständig erschließbar (lokal unter)
        \item = "`semantische Rolle"' der PP von der Präposition zugewiesen
      \end{itemize}
      \Viertelzeile
    \item \alert{Sehen Sie, wie schnell man in der (Grund-)Schulgrammatik\\
      in gefährliche linguistische Fahrwasser gerät?}
    \item \rot{Wenn Sie dieses Wissen nicht haben, unterrichten Sie sehr leicht\\
      komplett Falsches, zumal wenn es im Lehrbuch falsch steht.}
  \end{itemize}
\end{frame}


\begin{frame}
  {Der umstrittene PP-Angaben-Test}
  \pause
  Die PP mit \textit{"`Dies geschieht PP."'} aus dem Satz auskoppeln.\\
  \Halbzeile
  \pause
  \begin{exe}
    \ex\label{ex:ppergaenzungenundppangaben192}
    \begin{xlist}
      \ex[*]{\label{ex:ppergaenzungenundppangaben193} Viele Menschen leiden.
      \rot{Dies geschieht unter Vorurteilen.}}
        \pause
      \ex[ ]{\label{ex:ppergaenzungenundppangaben194} Viele Menschen schwitzen.
      \alert{Dies geschieht unter Sonnenschirmen.}}
        \pause
      \ex[*]{\label{ex:ppergaenzungenundppangaben195} Mausi schickt einen Brief.
      \rot{Dies geschieht an ihre Mutter.}}
        \pause
      \ex[*]{\label{ex:ppergaenzungenundppangaben196} Mausi befindet sich.
      \rot{Dies geschieht in Hamburg.}}
        \pause
      \ex[?]{\label{ex:ppergaenzungenundppangaben197} Mausi liegt.
      \orongsch{Dies geschieht auf dem Bett.}}
    \end{xlist}
  \end{exe}
  \Halbzeile
  \pause
  \begin{itemize}[<+->]
    \item der beste Test, den es gibt
    \item trotz Problemen
    \item \rot{Verlangen Sie von Schülern keine Entscheidungen,\\
    die Sie selber nicht operationalisieren können!}
  \end{itemize}
\end{frame}

\section{Zur nächsten Woche | Überblick}

\begin{frame}
  {Morphologie und Lexikon des Deutschen | Plan}
  \rot{Alle} angegebenen Kapitel\slash Abschnitte aus \rot{\citet{Schaefer2018b}} sind \rot{Klausurstoff}!\\
  \Halbzeile
  \begin{enumerate}
    \item Grammatik und Grammatik im Lehramt (Kapitel 1 und 3)
    \item Morphologie und Grundbegriffe (Kapitel 2, Kapitel 7 und Abschnitte 11.1--11.2)
    \item Wortklassen als Grundlage der Grammatik (Kapitel 6)
    \item Wortbildung | Komposition (Abschnitt 8.1)
    \item Wortbildung | Derivation und Konversion (Abschnitte 8.2 und 8.3)
    \item Flexion | Nomina außer Adjektiven (Abschnitte 9.1--9.3)
    \item Flexion | Adjektive und Verben (Abschnitt 9.4 und Kapitel 10)
    \item Valenz (Abschnitte 2.3, 14.1 und 14.3)
    \item \rot{Verbtypen als Valenztypen (Abschnitte 14.4, 14.5, 14.7--14.9)}
    \item Kernwortschatz und Fremdwort (vorwiegend Folien)
  \end{enumerate}
  \Halbzeile
  \centering 
  \url{https://langsci-press.org/catalog/book/224}
\end{frame}



