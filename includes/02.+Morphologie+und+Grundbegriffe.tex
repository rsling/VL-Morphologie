\section{Überblick}

\begin{frame}
  {Morphologie: Flexion und Wortbildung}
  \pause
  \begin{itemize}[<+->]
    \item \alert{Formveränderungen} und \alert{Merkmalsänderungen}
      \begin{itemize}[<+->]
        \item Veränderungen von Werten
        \item Veränderungen von Merkmalsaustattungen
      \end{itemize}
      \Halbzeile
    \item Morphe (= Wortbestandteile) und ihre Funktionen
    \item Morphe: alle Stämme und alle nicht-lexikalischen Morphe
      \Halbzeile
    \item statische und volatile Merkmale
    \item Wortbildung vs.\ Flexion, definiert anhand von Merkmalen
      \Zeile
    \item \citet[7.1]{Schaefer2018b}
  \end{itemize}
\end{frame}


\section{Stämme und Affixe}

\begin{frame}
  {Form und Funktion: Flexion}
  \pause
  \begin{exe}
    \ex
    \begin{xlist}
      \ex \alert{Den Präsidenten} begrüßte \alert{der Dekan} äußerst respektlos.
      \pause
      \ex \alert{Der Dekan} begrüßte \alert{den Präsidenten} äußerst respektlos.
    \end{xlist}
    \pause
    \ex
    \begin{xlist}
      \ex \alert{Die Präsidentin} begrüßte \alert{die Dekanin} äußerst respektlos.
      \pause
      \ex \alert{Die Dekanin} begrüßte \alert{die Präsidentin} äußerst respektlos.
    \end{xlist}
  \end{exe}
  \pause
  \Zeile
  Formveränderungen lexikalischer Wörter \alert{schränken ihre möglichen grammatischen Funktionen und Relationen im Satz ein}\dots\\
  \pause
  \Halbzeile
  \dots und sie haben semantische und systemexterne Folgen.

\end{frame}

\begin{frame}
  {Form und Funktion: Wortbildung}
  \pause
  \begin{exe}
    \ex grün\alert{lich}, röt\alert{lich}, gelb\alert{lich}
    \pause
    \ex Neu\alert{igkeit}, Blöd\alert{heit}, Tauch\alert{er}, Heb\alert{ung}
    \pause
    \ex Fenster\alert{rahmen}, Tücher\alert{spender}, Glas\alert{korken}, Unter\alert{schrank}
  \end{exe}
  \pause
  \Zeile
  Formveränderungen von einem zu einem anderen lexikalischen Wort führen zu Bedeutungs- und kategorialen Veränderungen.
\end{frame}

\begin{frame}
  {Markierungsfunktionen von Morphen I}
  \pause
  \begin{exe}
    \ex
    \begin{xlist}
      \ex{(der) \alert<4>{Berg}}
      \ex{(den) \alert<4>{Berg}}
      \ex{(dem) \alert<4>{Berg}}
      \ex{(des) \alert<5>{Berg}\rot<5>{-es}}
      \ex{(die) \alert<6>{Berg}\rot<6>{-e}}
      \ex{(der) \alert<6>{Berg}\rot<6>{-e}}
    \end{xlist}
    \pause
    \ex
    \begin{xlist}
      \ex{(der) \alert<8>{Mensch}}
      \ex{(den) \alert<9>{Mensch}\rot<9>{-en}}
      \ex{(dem) \alert<9>{Mensch}\rot<9>{-en}}
      \ex{(des) \alert<9>{Mensch}\rot<9>{-en}}
      \ex{(die) \alert<9>{Mensch}\rot<9>{-en}}
      \ex{(der) \alert<9>{Mensch}\rot<9>{-en}}
    \end{xlist}
  \end{exe}
\end{frame}

\begin{frame}
  {Markierungsfunktionen von Morphen II}
  \pause
  \begin{exe}
    \ex
    \begin{xlist}
      \ex{(ich) \alert<3>{kauf}\rot<3>{-e}}
      \ex{(du) \alert<4>{kauf}\rot<4>{-st}}
      \ex{(wir) \alert<5>{kauf}\rot<5>{-en}}
      \ex{(sie) \alert<5>{kauf}\rot<5>{-en}}
    \end{xlist}
  \end{exe}
\end{frame}

\begin{frame}
  {Morphe und Markierungsfunktionen}
  \pause
  \begin{itemize}[<+->]
    \item Formveränderungen:
      \begin{itemize}[<+->]
        \item oft nicht \alert{eine} Funktion
        \item \rot{Einschränkung} der möglichen Funktionen
      \end{itemize}
   \Halbzeile 
    \item \alert{Markierungsfunktion}: eine \alert{Reduktion}\\
      der möglichen Merkmale oder Werte einer Wortform
    \item zum Beispiel \textit{-en} bei schw.\ Maskulina: \rot{nicht} Nominativ Singular
    \item oder \textit{-en} bei Verben im Präsens: Plural und nicht adressatbezogen
      \Halbzeile
    \item \alert{Morphe = alle segmentalen Einheiten mit Markierungsfunktion}
    \item konkret: \alert{Stämme} und \alert{Affixe}
  \end{itemize}
\end{frame}

\begin{frame}
  {Stämme I}
  \pause
  \begin{exe}
    \ex
    \begin{xlist}
      \ex{(ich) \alert<5->{kauf}-e\\
        (du) \alert<5->{kauf}-st\\
        (ihr) \alert<5->{kauf}-t }
        \pause
        \ex{(ich) \alert<6->{kauf}-te\\
        (du) \alert<6->{kauf}-test\\
        (ihr) \alert<6->{kauf}-tet}
        \pause
        \ex{(ich habe) ge-\alert<7->{kauf}-t\\
        (du hast) ge-\alert<7->{kauf}-t\\
        (ihr habt) ge-\alert<7->{kauf}-t}
    \end{xlist}
  \end{exe}
\end{frame}

\begin{frame}
  {Stämme II}
  \begin{exe}
    \ex
    \begin{xlist}
      \ex{(ich) \alert<4->{nehm}-e\\
        (du) \rot<5->{nimm}-st\\
          (es) \rot<5->{nimm}-t\\
          (ihr) \alert<4->{nehm}-t}
        \pause
        \ex{(ich) \orongsch<6->{nahm}\\
        (du) \orongsch<6->{nahm}-st\\
          (ihr) \orongsch<6->{nahm}-t}
        \pause
        \ex{(ich habe) ge-\gruen<7->{nomm}-en\\
        (du hast) ge-\gruen<7->{nomm}-en\\
        (ihr habt) ge-\gruen<7->{nomm}-en}
    \end{xlist}
  \end{exe}
  \pause
  \pause
  \pause
  \pause
  \pause
  Der \alert{Stamm} kann nicht "`der unveränderliche Wortbestandteil"'\\
  eines lexikalischen Wortes (in einem Paradigma) sein.\\
  \Zeile
  \pause
  \alert{\dots aber der mit der Bedeutung, also der lexikalischen Markierungsfunktion}!
\end{frame}

\begin{frame}
  {Affixe}
  \pause
  \begin{exe}
    \ex
    \begin{xlist}
      \ex (ich) nehm\alert<6->{-e}
      \pause
      \ex (des) Berg\alert<7->{-es}
      \pause
      \ex Schön\alert<8->{-heit}
      \pause
      \ex \alert<9->{Un-}ding
    \end{xlist}
  \end{exe}
  \Zeile
  \pause
  \pause
  \pause
  \pause
  \pause
  \begin{itemize}[<+->]
    \item \alert{keine lexikalische Markierungsfunktion} (= keine eigene Bedeutung)
    \item \alert{nicht wortfähig} = nicht ohne Stamm verwendbar
  \end{itemize}
\end{frame}



\section{Merkmale in Flexion und Wortbildung}

\begin{frame}
  {Statische und volatile Merkmale}
  \pause
  \begin{itemize}[<+->]
    \item Eigenschaften: "`Rotsein"' (Erdbeere), "`325m hoch"' (Eiffelturm) usw.
    \item Merkmale: \alert{\textsc{Farbe}}, \alert{\textsc{Länge}} usw.
    \item Werte:
      \begin{itemize}[<+->]
        \item \alert{\textsc{Farbe}}: \rot{\textit{rot}}, \rot{\textit{grau}}, \ldots
        \item \alert{\textsc{Länge}}: \rot{\textit{3cm}}, \rot{\textit{325m}}, \ldots
      \end{itemize}
  \end{itemize}
  \pause
  \Halbzeile 
  \begin{exe}
    \ex
    \begin{xlist}
      \ex{Haus = [\textsc{Bed}: \gruen<12->{\textbf{\textit{haus}}}, \textsc{Klasse}: \gruen<12->{\textbf{\textit{subst}}}, \textsc{Gen}: \gruen<12->{\textbf{\textit{neut}}}, \textsc{Kas}: \orongsch<13->{\textit{nom}}, \textsc{Num}: \orongsch<13->{\textit{sg}}]}
      \pause
      \ex{Haus-es = [\textsc{Bed}: \gruen<12->{\textbf{\textit{haus}}}, \textsc{Klasse}: \gruen<12->{\textbf{\textit{subst}}}, \textsc{Gen}: \gruen<12->{\textbf{\textit{neut}}}, \textsc{Kas}: \orongsch<13->{\textit{gen}}, \textsc{Num}: \orongsch<13->{\textit{sg}}]}
      \pause
      \ex{Häus-er = [\textsc{Bed}: \gruen<12->{\textbf{\textit{haus}}}, \textsc{Klasse}: \gruen<12->{\textbf{\textit{subst}}}, \textsc{Gen}: \gruen<12->{\textbf{\textit{neut}}}, \textsc{Kas}: \orongsch<13->{\textit{nom}}, \textsc{Num}: \orongsch<13->{\textit{pl}}]}
    \end{xlist}
  \end{exe}
  \Halbzeile
  \pause
  \begin{itemize}[<+->]
    \item bei einem lexikalischen Wort:
      \begin{itemize}
        \item \gruen{statische Merkmale} wertestabil
        \item \orongsch{volatile Merkmale} werteverändernd im Paradigma
      \end{itemize}
  \end{itemize}
\end{frame}

\begin{frame}
  {Wortbildung in Abgrenzung zur Flexion}
  \pause
  \begin{exe}
    \ex
    \begin{xlist}
      \ex trocken (Adj) → \alert{Trocken}\rot{-heit} (Subst)
      \ex Kauf (Subst), Rausch (Subst) → \alert{Kauf}\rot{-rausch} (Subst)
      \ex gehen (V) → \alert{be}\rot{-gehen} (V)
    \end{xlist}
    \pause
    \ex
    \begin{xlist}
      \ex \alert{lauf}\rot{-en} (1\slash 3 Pl Prs Ind) → \alert{lauf}\rot{-e} (1 Sg Prs Ind)
      \ex \alert{Münze} (Sg) → \alert{Münze}\rot{-n} (Pl)
    \end{xlist}
  \end{exe}
  \pause
  \Halbzeile
  \begin{itemize}[<+->]
    \item Wortbildung
      \begin{itemize}[<+->]
        \item statische Merkmale geändert (Wortklasse, Bedeutung)
        \item \ldots oder gelöscht (alles außer Bedeutung: Erstglied bei Komposition)
        \item \ldots oder umgebaut (Valenz von Verben beim Applikativ)
        \item \alert{produktives Erschaffen neuer lexikalischer Wörter}
      \end{itemize}
  \Halbzeile
    \item Flexion
      \begin{itemize}
        \item Änderung der Werte volatiler Merkmale
        \item typisch: Anpassung an syntaktischen Kontext
      \end{itemize}
  \end{itemize}
\end{frame}

\section{Übung}

\begin{frame}
  {Stämme und Affixe}
  \Zeile
  \centering 
  Suchen Sie im Text der letzten Woche nach \alert{einfachen Wörtern}\\
  sowie \alert{Wörtern mit Stamm und Affix(en)}.\\
  \Zeile
  Versuchen Sie, die \alert{Markierungsfunktionen}\\
  der Stämme und Affixe zu bestimmen.
\end{frame}

\section{Vorschau}

\begin{frame}
  {Nächste Woche | Wortklassen}
  \Zeile
  \centering 
  Bitte lesen Sie \orongsch{unbedingt} \alert{Kapitel 6 (Wortklassen)} aus EGBD3!
\end{frame}
